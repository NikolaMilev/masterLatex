\chapter{Mašinsko učenje}

Mašinsko učenje počelo je da stiče veliku popularnost devedesetih godina prošlog veka zahvaljujući potrebi da se uči iz ogromne količine dostupnih podataka i uspešnosti ovog pristupa u tome. Za popularizaciju mašinskog učenja početkom 21. veka najzaslužnije su neuronske mreže, u toj meri da je mašinsko učenje često poistovećeno s njima. Ovo, naravno, nije tačno; sem neuronskih mreža, postoje razne druge tehnike, kao što su metod potpornih vektora, genetski algoritmi, itd. \\

Mašinsko učenje nastalo je iz čovekove želje da oponaša prirodne mehanizme učenja kod čoveka i životinja i korišćenja dobijenih rezultata u cilju praktične upotrebe. Termin mašinsko učenje prvi je upotrebio pionir veštačke inteligencije, Artur Semjuel \footnote{\url{https://en.wikipedia.org/wiki/Machine_learning}}. Doprineo je razvoju veštačke inteligencije istražujući igru dame (eng. checkers) i tražeći način da stvori računarski program koji na osnovu iskusva može da savlada ovu igru. \footnote{\url{http://infolab.stanford.edu/pub/voy/museum/samuel.html}} \\

Mašinsko učenje može se definisati kao disciplina koja se bavi izgradnjom prilagodljivih računarskih sistema koji su sposobni da poboljšaju svoje performanse koristeći informacije iz iskustva \footnote{\url{http://poincare.matf.bg.ac.rs/~janicic/courses/vi.pdf}}. No, u biti mašinskog učenja leži generalizacija (dedukcija). Dve vrste zaključivanja, indukcija\footnote{Indukcija -- zaključivanje od pojedinačnog ka opštem} i dedukcija \footnote{Dedukcija -- zaključivanje od opšteg ka konkretnom} imaju svoje odgovarajuće discipline u sklopu veštačke inteligencije: mašinsko učenje i automatsko rezonovanje. Kao što se indukcija i dedukcija razlikuju, i mašinsko učenje i automatsko rezonovanje imaju različite oblasti primene. Automatsko rezonovanje zasnovano je na matematičkoj logici i koristi se kada problem čovek relativno lako može formulisati ali ga, često zbog velikog prostora mogućih rešenja, ne može jednostavno rešiti. U ovoj oblasti, neophodno je dobiti apsolutno tačna rešenja, ne dopuštajući nikakav nivo greške.
Pri mašinskom učenju, teže je formalno definisati problem jer postoji relativno visok nivo apstrakcije. Čovek neke od ovih problema lako rešava a neke ne. Ukoliko je neophodno napraviti sistem koji prepoznaje lica na slikama, kako definisati problem? Od čega se tačno sastoji lice? Metodama automatskog rezonovanja bilo bi nemoguće definisati ovaj problem i rešiti ga. Mašinsko učenje, s druge strane, pokazalo se kao dobar pristup. \\

Ova oblast je kroz manje od 20 godina od popularizacije postala deo svakodnevnice. U sklopu društvene mreže Fejsbuk (eng. Facebook) implementiran je sistem za prepoznavanje lica koji preporučuje profile osoba koje se nalaze na slikama. Razni veb servisi koriste metode mašinskog učenja radi stvaranja sistema za preporuke (artikala u prodavnicama, video sadržaja na platformama za njihovo gledanje, itd.) i sistema za detekciju prevara. Mnoge firme koje se bave trgovinom na berzi imaju sisteme koji automatski trguju deonicama. U medicini, jedna od primena mašinskog učenja jeste za uspostavljanje dijagnoze. Još neke primene su u marketingu, za procesiranje prirodnih jezika, bezbednost, itd.



\section{Vrste mašinskog učenja}

Tri vrste mašinskog učenja su nadgledano učenje (eng. supervised learning), nenadgledano učenje (eng. unsupervised learning) i učenje potkrepljivanjem (eng. reinforcement learning). Iako mnogi autori dele mašinsko učenje na samo nadgledano i nenadgledano, postoji razlika između učenja potkrepljivanjem i preostale dve vrste. U nastavku su dati opisi pristupa nadgledanog i nenadgledanog učenja. Učenju uslovljavanjem, kao centralnoj temi ovog rada, posvećeno je više pažnje u poglavlju [GLAVA O RL]. 