\chapter{Mašinsko učenje}

Mašinsko učenje počelo je da stiče veliku popularnost devedesetih godina prošlog veka zahvaljujući potrebi i mogućnosti da se uči iz ogromne količine dostupnih podataka i uspešnosti ovog pristupa u tome. Za popularizaciju mašinskog učenja početkom 21. veka najzaslužnije su neuronske mreže, u toj meri da je pojam mašinskog učenja često poistovećen sa pojmom neuronskih mreža. Ovo, naravno, nije tačno; sem neuronskih mreža, postoje razne druge tehnike, kao što su metod potpornih vektora, genetski algoritmi, itd. \par

Mašinsko učenje nastalo je iz čovekove želje da oponaša prirodne mehanizme učenja kod čoveka i životinja kao jedne od osnovnih svojstava inteligencije i korišćenja dobijenih rezultata u cilju praktične upotrebe. Termin mašinsko učenje prvi je upotrebio pionir veštačke inteligencije, Artur Semjuel\footnote{\url{https://en.wikipedia.org/wiki/Machine_learning} -- da li da citiram Wiki ili njihov izvor?}, koji je doprineo razvoju veštačke inteligencije istražujući igru dame (eng. checkers) i tražeći način da stvori računarski program koji na osnovu iskusva može da savlada ovu igru\footnote{\url{http://infolab.stanford.edu/pub/voy/museum/samuel.html} -- kako citirati izvor sa veba?}. \par

Mašinsko učenje može se definisati kao disciplina koja se bavi izgradnjom prilagodljivih računarskih sistema koji su sposobni da poboljšaju svoje performanse koristeći informacije iz iskustva \footnote{\url{http://poincare.matf.bg.ac.rs/~janicic/courses/vi.pdf} -- pretpostavljam da stavim referencu ka literaturi gde će knjiga biti navedena?}. No, u biti mašinskog učenja leži generalizacija, tj. indukcija. Dve vrste zaključivanja, indukcija\footnote{Indukcija -- zaključivanje od pojedinačnog ka opštem} i dedukcija\footnote{Dedukcija -- zaključivanje od opšteg ka konkretnom} imaju svoje odgovarajuće discipline u sklopu veštačke inteligencije: mašinsko učenje i automatsko rezonovanje. Kao što se indukcija i dedukcija razlikuju, i mašinsko učenje i automatsko rezonovanje imaju različite oblasti primene. Automatsko rezonovanje zasnovano je na matematičkoj logici i koristi se kada problem čovek relativno lako može formulisati ali ga, često zbog velikog prostora mogućih rešenja, ne može jednostavno rešiti. U ovoj oblasti, neophodno je dobiti apsolutno tačna rešenja, ne dopuštajući nikakav nivo greške.
Pri mašinskom učenju, teže je formalno definisati problem jer postoji relativno visok nivo apstrakcije. Čovek neke od ovih problema lako rešava a neke ne. Ukoliko je neophodno napraviti sistem koji prepoznaje lica na slikama, kako definisati problem? Od čega se tačno sastoji lice? Kako prepoznati elemente lica? Metodama automatskog rezonovanja bilo bi nemoguće definisati ovaj problem i rešiti ga. Mašinsko učenje, s druge strane, pokazalo se kao dobar pristup. Ono što je još karakteristično za mašinsko učenje jeste da rešenje ne mora biti savršeno tačno, iako se tome teži, i nivo prihvatljivog odstupanja zavisi od problema i konteksta primene. \par

Ova oblast je kroz manje od 20 godina od popularizacije postala deo svakodnevnice. U sklopu društvene mreže Fejsbuk (eng. Facebook) implementiran je sistem za prepoznavanje lica koji preporučuje profile osoba koje se nalaze na slikama. Razni veb servisi koriste metode mašinskog učenja radi stvaranja sistema za preporuke (artikala u prodavnicama, video sadržaja na platformama za njihovo gledanje, itd.) i sistema za detekciju prevara. Mnoge firme koje se bave trgovinom na berzi imaju sisteme koji automatski trguju deonicama. U medicini, jedna od primena mašinskog učenja jeste za uspostavljanje dijagnoze. Još neke primene su u marketingu, za procesiranje prirodnih jezika, bezbednost, itd.

%[ODNOS MASINSKOG UCENJA I OSTALIH OBLASTI MATEMATIKE?]

\section{Vrste mašinskog učenja}

Kada se priča o određenoj vrsti mašinskog učenja, podrazumevaju se vrste problema, kao i načini za njihovo rešavanje. Prema problemima koji se rešavaju, mašinsko učenje deli se na tri vrste: nadgledano učenje (eng. supervised learning), nenadgledano učenje (eng. unsupervised learning) i učenje potkrepljivanjem (eng. reinforcement learning). Iako se podela mnogih autora sastoji samo iz nadgledanog i nenadgledanog učenja, postoji razlika između učenja potkrepljivanjem i preostale dve vrste. U nastavku su dati opisi pristupa nadgledanog i nenadgledanog učenja. Učenju uslovljavanjem, kao centralnoj temi ovog rada, posvećeno je više pažnje u poglavlju~ \ref{ch:rl}. 

\subsection{Nadgledano mašinsko učenje}

Pri nadgledanom mašinskom učenju, date su vrednosti ulaza i izlaza koje im odgovaraju za određeni broj slučajeva. Sistem treba na osnovu već datih veza za pojedinačne parove da ustanovi kakva veza postoji između tih parova i izvrši generalizaciju, odnosno, ukoliko ulazne podatke označimo sa $x$ a izlazne sa $y$, sistem treba da odredi funkciju $f$ takvu da 
\begin{center}
	$y \approx f(x)$
\end{center}
Pri uspešno rešenom problemu nadgledanog učenja, funkcija $f$ davaće tačna rešenja i za podatke koji do sada nisu viđeni.
Ulazne vrednosti nazivaju se atributima (eng. features) a izlazne ciljnim promenljivima (eng. target values). Ovim opisom nije određema dimenzionalnost ni za ulazne ni za izlazne promenljive, iako je dimenzija izlazne promenljive uglavnom 1. Funkcija $f$ naziva se modelom. \par
Skup svih mogućih funkcija odgovarajuće dimenzionalnosti bio bi previše veliki za pretragu i zbog toga se uvode pretpostavke o samom modelu. Pretpostavlja da je definisan skup svih dopustivih modela i da je potrebno naći najpogodniji element tog skupa. Najčešće je taj skup određen parametrima, tj. uzima se da funkcija zavisi od nekog parametra $w$ koji je u opštem slušaju višedimenzioni i funkcija se označava kao $f_w(x)$.

\par
Neophodno je uvesti funkciju greške modela (eng. loss function), odnosno funkciju koja opisuje koliko dati model dobro određuje izlaz za dati ulaz. Ova funkcija se najčešće označava sa $L$ i $L(y, f_w(x))$ predstavlja razliku između željene i dobijene vrednosti za pojedinačni par promenljivih. No, nijedan par promenljivih nije dovoljan za opis kvaliteta modela već treba naći funkciju koja globalno ocenjuje odstupanje modela od stvarnih vrednosti. U praksi, podrazumeva se postojanje uzorka:
\begin{center}
	$D=\{(x_i, y_i)|i=1,...,N\}$
\end{center}
i uvodi se empirijski rizik, odnosno sledeća funkcija:
\begin{center}
	$E(w, D) = \frac{1}{N}\sum_{i=1}^{N}L(y_i, f_w(x_i))$
\end{center}
koja se još naziva prosečnom greškom. Takođe se skup $D$ često ne navodi i njegovo postojanje se podrazumeva. Uobičajeno, algoritmi nadgledanog mašinskog učenja zasnivaju se na minimizaciji prosečne greške. Ipak, treba imati u vidu da ovaj pristup nije teorijski zagarantovan i da to zavisi od skupa modela po kom se vrši minimizacija. \par

Postoje dva osnovna tipa nadgledanog mašinskog učenja:
\begin{itemize}
	\item Klasifikacija 
	\item Regresija
\end{itemize}

Klasifikacija (eng. classification) predstavlja oblast mašinskog učenja gde je cilj predvideti klasu u kojoj se ciljna promenljiva nalazi. Neki od primera klasifikacije su svrstavanje slika na one koje sadrže ili ne sadrže lice, označavanje nepoželjne (spam) elektronske pošte i prepoznavanje objekata na slikama. 
Najjednostavniji primer klasifikacije može se videti na slici \ref{fig:bin_klas}, gde su trouglovima označeni podaci iznad prave $y=2x+1$ a krugovima podaci ispod date prave.

\begin{figure}
	\centering
	\resizebox{.8\linewidth}{!}{%% Creator: Matplotlib, PGF backend
%%
%% To include the figure in your LaTeX document, write
%%   \input{<filename>.pgf}
%%
%% Make sure the required packages are loaded in your preamble
%%   \usepackage{pgf}
%%
%% Figures using additional raster images can only be included by \input if
%% they are in the same directory as the main LaTeX file. For loading figures
%% from other directories you can use the `import` package
%%   \usepackage{import}
%% and then include the figures with
%%   \import{<path to file>}{<filename>.pgf}
%%
%% Matplotlib used the following preamble
%%   \usepackage{fontspec}
%%   \setmainfont{DejaVu Serif}
%%   \setsansfont{DejaVu Sans}
%%   \setmonofont{DejaVu Sans Mono}
%%
\begingroup%
\makeatletter%
\begin{pgfpicture}%
\pgfpathrectangle{\pgfpointorigin}{\pgfqpoint{6.400000in}{4.800000in}}%
\pgfusepath{use as bounding box, clip}%
\begin{pgfscope}%
\pgfsetbuttcap%
\pgfsetmiterjoin%
\definecolor{currentfill}{rgb}{1.000000,1.000000,1.000000}%
\pgfsetfillcolor{currentfill}%
\pgfsetlinewidth{0.000000pt}%
\definecolor{currentstroke}{rgb}{1.000000,1.000000,1.000000}%
\pgfsetstrokecolor{currentstroke}%
\pgfsetdash{}{0pt}%
\pgfpathmoveto{\pgfqpoint{0.000000in}{0.000000in}}%
\pgfpathlineto{\pgfqpoint{6.400000in}{0.000000in}}%
\pgfpathlineto{\pgfqpoint{6.400000in}{4.800000in}}%
\pgfpathlineto{\pgfqpoint{0.000000in}{4.800000in}}%
\pgfpathclose%
\pgfusepath{fill}%
\end{pgfscope}%
\begin{pgfscope}%
\pgfsetbuttcap%
\pgfsetmiterjoin%
\definecolor{currentfill}{rgb}{1.000000,1.000000,1.000000}%
\pgfsetfillcolor{currentfill}%
\pgfsetlinewidth{0.000000pt}%
\definecolor{currentstroke}{rgb}{0.000000,0.000000,0.000000}%
\pgfsetstrokecolor{currentstroke}%
\pgfsetstrokeopacity{0.000000}%
\pgfsetdash{}{0pt}%
\pgfpathmoveto{\pgfqpoint{0.800000in}{0.528000in}}%
\pgfpathlineto{\pgfqpoint{5.760000in}{0.528000in}}%
\pgfpathlineto{\pgfqpoint{5.760000in}{4.224000in}}%
\pgfpathlineto{\pgfqpoint{0.800000in}{4.224000in}}%
\pgfpathclose%
\pgfusepath{fill}%
\end{pgfscope}%
\begin{pgfscope}%
\pgfpathrectangle{\pgfqpoint{0.800000in}{0.528000in}}{\pgfqpoint{4.960000in}{3.696000in}} %
\pgfusepath{clip}%
\pgfsetbuttcap%
\pgfsetroundjoin%
\definecolor{currentfill}{rgb}{1.000000,0.647059,0.000000}%
\pgfsetfillcolor{currentfill}%
\pgfsetlinewidth{1.003750pt}%
\definecolor{currentstroke}{rgb}{1.000000,0.647059,0.000000}%
\pgfsetstrokecolor{currentstroke}%
\pgfsetdash{}{0pt}%
\pgfpathmoveto{\pgfqpoint{0.800000in}{1.589805in}}%
\pgfpathlineto{\pgfqpoint{0.841667in}{1.673139in}}%
\pgfpathlineto{\pgfqpoint{0.758333in}{1.673139in}}%
\pgfpathclose%
\pgfusepath{stroke,fill}%
\end{pgfscope}%
\begin{pgfscope}%
\pgfpathrectangle{\pgfqpoint{0.800000in}{0.528000in}}{\pgfqpoint{4.960000in}{3.696000in}} %
\pgfusepath{clip}%
\pgfsetbuttcap%
\pgfsetroundjoin%
\definecolor{currentfill}{rgb}{1.000000,0.647059,0.000000}%
\pgfsetfillcolor{currentfill}%
\pgfsetlinewidth{1.003750pt}%
\definecolor{currentstroke}{rgb}{1.000000,0.647059,0.000000}%
\pgfsetstrokecolor{currentstroke}%
\pgfsetdash{}{0pt}%
\pgfpathmoveto{\pgfqpoint{0.850101in}{4.970687in}}%
\pgfpathlineto{\pgfqpoint{0.891768in}{5.054020in}}%
\pgfpathlineto{\pgfqpoint{0.808434in}{5.054020in}}%
\pgfpathclose%
\pgfusepath{stroke,fill}%
\end{pgfscope}%
\begin{pgfscope}%
\pgfpathrectangle{\pgfqpoint{0.800000in}{0.528000in}}{\pgfqpoint{4.960000in}{3.696000in}} %
\pgfusepath{clip}%
\pgfsetbuttcap%
\pgfsetroundjoin%
\definecolor{currentfill}{rgb}{1.000000,0.647059,0.000000}%
\pgfsetfillcolor{currentfill}%
\pgfsetlinewidth{1.003750pt}%
\definecolor{currentstroke}{rgb}{1.000000,0.647059,0.000000}%
\pgfsetstrokecolor{currentstroke}%
\pgfsetdash{}{0pt}%
\pgfpathmoveto{\pgfqpoint{0.900202in}{5.344999in}}%
\pgfpathlineto{\pgfqpoint{0.941869in}{5.428332in}}%
\pgfpathlineto{\pgfqpoint{0.858535in}{5.428332in}}%
\pgfpathclose%
\pgfusepath{stroke,fill}%
\end{pgfscope}%
\begin{pgfscope}%
\pgfpathrectangle{\pgfqpoint{0.800000in}{0.528000in}}{\pgfqpoint{4.960000in}{3.696000in}} %
\pgfusepath{clip}%
\pgfsetbuttcap%
\pgfsetroundjoin%
\definecolor{currentfill}{rgb}{1.000000,0.647059,0.000000}%
\pgfsetfillcolor{currentfill}%
\pgfsetlinewidth{1.003750pt}%
\definecolor{currentstroke}{rgb}{1.000000,0.647059,0.000000}%
\pgfsetstrokecolor{currentstroke}%
\pgfsetdash{}{0pt}%
\pgfpathmoveto{\pgfqpoint{0.950303in}{2.702496in}}%
\pgfpathlineto{\pgfqpoint{0.991970in}{2.785829in}}%
\pgfpathlineto{\pgfqpoint{0.908636in}{2.785829in}}%
\pgfpathclose%
\pgfusepath{stroke,fill}%
\end{pgfscope}%
\begin{pgfscope}%
\pgfpathrectangle{\pgfqpoint{0.800000in}{0.528000in}}{\pgfqpoint{4.960000in}{3.696000in}} %
\pgfusepath{clip}%
\pgfsetbuttcap%
\pgfsetroundjoin%
\definecolor{currentfill}{rgb}{1.000000,0.647059,0.000000}%
\pgfsetfillcolor{currentfill}%
\pgfsetlinewidth{1.003750pt}%
\definecolor{currentstroke}{rgb}{1.000000,0.647059,0.000000}%
\pgfsetstrokecolor{currentstroke}%
\pgfsetdash{}{0pt}%
\pgfpathmoveto{\pgfqpoint{1.000404in}{2.642758in}}%
\pgfpathlineto{\pgfqpoint{1.042071in}{2.726092in}}%
\pgfpathlineto{\pgfqpoint{0.958737in}{2.726092in}}%
\pgfpathclose%
\pgfusepath{stroke,fill}%
\end{pgfscope}%
\begin{pgfscope}%
\pgfpathrectangle{\pgfqpoint{0.800000in}{0.528000in}}{\pgfqpoint{4.960000in}{3.696000in}} %
\pgfusepath{clip}%
\pgfsetbuttcap%
\pgfsetroundjoin%
\definecolor{currentfill}{rgb}{1.000000,0.647059,0.000000}%
\pgfsetfillcolor{currentfill}%
\pgfsetlinewidth{1.003750pt}%
\definecolor{currentstroke}{rgb}{1.000000,0.647059,0.000000}%
\pgfsetstrokecolor{currentstroke}%
\pgfsetdash{}{0pt}%
\pgfpathmoveto{\pgfqpoint{1.050505in}{5.287553in}}%
\pgfpathlineto{\pgfqpoint{1.092172in}{5.370886in}}%
\pgfpathlineto{\pgfqpoint{1.008838in}{5.370886in}}%
\pgfpathclose%
\pgfusepath{stroke,fill}%
\end{pgfscope}%
\begin{pgfscope}%
\pgfpathrectangle{\pgfqpoint{0.800000in}{0.528000in}}{\pgfqpoint{4.960000in}{3.696000in}} %
\pgfusepath{clip}%
\pgfsetbuttcap%
\pgfsetroundjoin%
\definecolor{currentfill}{rgb}{1.000000,0.647059,0.000000}%
\pgfsetfillcolor{currentfill}%
\pgfsetlinewidth{1.003750pt}%
\definecolor{currentstroke}{rgb}{1.000000,0.647059,0.000000}%
\pgfsetstrokecolor{currentstroke}%
\pgfsetdash{}{0pt}%
\pgfpathmoveto{\pgfqpoint{1.100606in}{1.148267in}}%
\pgfpathlineto{\pgfqpoint{1.142273in}{1.231601in}}%
\pgfpathlineto{\pgfqpoint{1.058939in}{1.231601in}}%
\pgfpathclose%
\pgfusepath{stroke,fill}%
\end{pgfscope}%
\begin{pgfscope}%
\pgfpathrectangle{\pgfqpoint{0.800000in}{0.528000in}}{\pgfqpoint{4.960000in}{3.696000in}} %
\pgfusepath{clip}%
\pgfsetbuttcap%
\pgfsetroundjoin%
\definecolor{currentfill}{rgb}{1.000000,0.647059,0.000000}%
\pgfsetfillcolor{currentfill}%
\pgfsetlinewidth{1.003750pt}%
\definecolor{currentstroke}{rgb}{1.000000,0.647059,0.000000}%
\pgfsetstrokecolor{currentstroke}%
\pgfsetdash{}{0pt}%
\pgfpathmoveto{\pgfqpoint{1.150707in}{2.771214in}}%
\pgfpathlineto{\pgfqpoint{1.192374in}{2.854548in}}%
\pgfpathlineto{\pgfqpoint{1.109040in}{2.854548in}}%
\pgfpathclose%
\pgfusepath{stroke,fill}%
\end{pgfscope}%
\begin{pgfscope}%
\pgfpathrectangle{\pgfqpoint{0.800000in}{0.528000in}}{\pgfqpoint{4.960000in}{3.696000in}} %
\pgfusepath{clip}%
\pgfsetbuttcap%
\pgfsetroundjoin%
\definecolor{currentfill}{rgb}{1.000000,0.647059,0.000000}%
\pgfsetfillcolor{currentfill}%
\pgfsetlinewidth{1.003750pt}%
\definecolor{currentstroke}{rgb}{1.000000,0.647059,0.000000}%
\pgfsetstrokecolor{currentstroke}%
\pgfsetdash{}{0pt}%
\pgfpathmoveto{\pgfqpoint{1.200808in}{1.874288in}}%
\pgfpathlineto{\pgfqpoint{1.242475in}{1.957621in}}%
\pgfpathlineto{\pgfqpoint{1.159141in}{1.957621in}}%
\pgfpathclose%
\pgfusepath{stroke,fill}%
\end{pgfscope}%
\begin{pgfscope}%
\pgfpathrectangle{\pgfqpoint{0.800000in}{0.528000in}}{\pgfqpoint{4.960000in}{3.696000in}} %
\pgfusepath{clip}%
\pgfsetbuttcap%
\pgfsetroundjoin%
\definecolor{currentfill}{rgb}{1.000000,0.647059,0.000000}%
\pgfsetfillcolor{currentfill}%
\pgfsetlinewidth{1.003750pt}%
\definecolor{currentstroke}{rgb}{1.000000,0.647059,0.000000}%
\pgfsetstrokecolor{currentstroke}%
\pgfsetdash{}{0pt}%
\pgfpathmoveto{\pgfqpoint{1.250909in}{2.035079in}}%
\pgfpathlineto{\pgfqpoint{1.292576in}{2.118413in}}%
\pgfpathlineto{\pgfqpoint{1.209242in}{2.118413in}}%
\pgfpathclose%
\pgfusepath{stroke,fill}%
\end{pgfscope}%
\begin{pgfscope}%
\pgfpathrectangle{\pgfqpoint{0.800000in}{0.528000in}}{\pgfqpoint{4.960000in}{3.696000in}} %
\pgfusepath{clip}%
\pgfsetbuttcap%
\pgfsetroundjoin%
\definecolor{currentfill}{rgb}{1.000000,0.647059,0.000000}%
\pgfsetfillcolor{currentfill}%
\pgfsetlinewidth{1.003750pt}%
\definecolor{currentstroke}{rgb}{1.000000,0.647059,0.000000}%
\pgfsetstrokecolor{currentstroke}%
\pgfsetdash{}{0pt}%
\pgfpathmoveto{\pgfqpoint{1.301010in}{2.153348in}}%
\pgfpathlineto{\pgfqpoint{1.342677in}{2.236681in}}%
\pgfpathlineto{\pgfqpoint{1.259343in}{2.236681in}}%
\pgfpathclose%
\pgfusepath{stroke,fill}%
\end{pgfscope}%
\begin{pgfscope}%
\pgfpathrectangle{\pgfqpoint{0.800000in}{0.528000in}}{\pgfqpoint{4.960000in}{3.696000in}} %
\pgfusepath{clip}%
\pgfsetbuttcap%
\pgfsetroundjoin%
\definecolor{currentfill}{rgb}{1.000000,0.647059,0.000000}%
\pgfsetfillcolor{currentfill}%
\pgfsetlinewidth{1.003750pt}%
\definecolor{currentstroke}{rgb}{1.000000,0.647059,0.000000}%
\pgfsetstrokecolor{currentstroke}%
\pgfsetdash{}{0pt}%
\pgfpathmoveto{\pgfqpoint{1.351111in}{1.666477in}}%
\pgfpathlineto{\pgfqpoint{1.392778in}{1.749811in}}%
\pgfpathlineto{\pgfqpoint{1.309444in}{1.749811in}}%
\pgfpathclose%
\pgfusepath{stroke,fill}%
\end{pgfscope}%
\begin{pgfscope}%
\pgfpathrectangle{\pgfqpoint{0.800000in}{0.528000in}}{\pgfqpoint{4.960000in}{3.696000in}} %
\pgfusepath{clip}%
\pgfsetbuttcap%
\pgfsetroundjoin%
\definecolor{currentfill}{rgb}{1.000000,0.647059,0.000000}%
\pgfsetfillcolor{currentfill}%
\pgfsetlinewidth{1.003750pt}%
\definecolor{currentstroke}{rgb}{1.000000,0.647059,0.000000}%
\pgfsetstrokecolor{currentstroke}%
\pgfsetdash{}{0pt}%
\pgfpathmoveto{\pgfqpoint{1.401212in}{2.161171in}}%
\pgfpathlineto{\pgfqpoint{1.442879in}{2.244504in}}%
\pgfpathlineto{\pgfqpoint{1.359545in}{2.244504in}}%
\pgfpathclose%
\pgfusepath{stroke,fill}%
\end{pgfscope}%
\begin{pgfscope}%
\pgfpathrectangle{\pgfqpoint{0.800000in}{0.528000in}}{\pgfqpoint{4.960000in}{3.696000in}} %
\pgfusepath{clip}%
\pgfsetbuttcap%
\pgfsetroundjoin%
\definecolor{currentfill}{rgb}{1.000000,0.647059,0.000000}%
\pgfsetfillcolor{currentfill}%
\pgfsetlinewidth{1.003750pt}%
\definecolor{currentstroke}{rgb}{1.000000,0.647059,0.000000}%
\pgfsetstrokecolor{currentstroke}%
\pgfsetdash{}{0pt}%
\pgfpathmoveto{\pgfqpoint{1.451313in}{5.088047in}}%
\pgfpathlineto{\pgfqpoint{1.492980in}{5.171380in}}%
\pgfpathlineto{\pgfqpoint{1.409646in}{5.171380in}}%
\pgfpathclose%
\pgfusepath{stroke,fill}%
\end{pgfscope}%
\begin{pgfscope}%
\pgfpathrectangle{\pgfqpoint{0.800000in}{0.528000in}}{\pgfqpoint{4.960000in}{3.696000in}} %
\pgfusepath{clip}%
\pgfsetbuttcap%
\pgfsetroundjoin%
\definecolor{currentfill}{rgb}{1.000000,0.647059,0.000000}%
\pgfsetfillcolor{currentfill}%
\pgfsetlinewidth{1.003750pt}%
\definecolor{currentstroke}{rgb}{1.000000,0.647059,0.000000}%
\pgfsetstrokecolor{currentstroke}%
\pgfsetdash{}{0pt}%
\pgfpathmoveto{\pgfqpoint{1.501414in}{4.137191in}}%
\pgfpathlineto{\pgfqpoint{1.543081in}{4.220524in}}%
\pgfpathlineto{\pgfqpoint{1.459747in}{4.220524in}}%
\pgfpathclose%
\pgfusepath{stroke,fill}%
\end{pgfscope}%
\begin{pgfscope}%
\pgfpathrectangle{\pgfqpoint{0.800000in}{0.528000in}}{\pgfqpoint{4.960000in}{3.696000in}} %
\pgfusepath{clip}%
\pgfsetbuttcap%
\pgfsetroundjoin%
\definecolor{currentfill}{rgb}{1.000000,0.647059,0.000000}%
\pgfsetfillcolor{currentfill}%
\pgfsetlinewidth{1.003750pt}%
\definecolor{currentstroke}{rgb}{1.000000,0.647059,0.000000}%
\pgfsetstrokecolor{currentstroke}%
\pgfsetdash{}{0pt}%
\pgfpathmoveto{\pgfqpoint{1.551515in}{1.834261in}}%
\pgfpathlineto{\pgfqpoint{1.593182in}{1.917594in}}%
\pgfpathlineto{\pgfqpoint{1.509848in}{1.917594in}}%
\pgfpathclose%
\pgfusepath{stroke,fill}%
\end{pgfscope}%
\begin{pgfscope}%
\pgfpathrectangle{\pgfqpoint{0.800000in}{0.528000in}}{\pgfqpoint{4.960000in}{3.696000in}} %
\pgfusepath{clip}%
\pgfsetbuttcap%
\pgfsetroundjoin%
\definecolor{currentfill}{rgb}{1.000000,0.647059,0.000000}%
\pgfsetfillcolor{currentfill}%
\pgfsetlinewidth{1.003750pt}%
\definecolor{currentstroke}{rgb}{1.000000,0.647059,0.000000}%
\pgfsetstrokecolor{currentstroke}%
\pgfsetdash{}{0pt}%
\pgfpathmoveto{\pgfqpoint{1.601616in}{4.738554in}}%
\pgfpathlineto{\pgfqpoint{1.643283in}{4.821887in}}%
\pgfpathlineto{\pgfqpoint{1.559949in}{4.821887in}}%
\pgfpathclose%
\pgfusepath{stroke,fill}%
\end{pgfscope}%
\begin{pgfscope}%
\pgfpathrectangle{\pgfqpoint{0.800000in}{0.528000in}}{\pgfqpoint{4.960000in}{3.696000in}} %
\pgfusepath{clip}%
\pgfsetbuttcap%
\pgfsetroundjoin%
\definecolor{currentfill}{rgb}{1.000000,0.647059,0.000000}%
\pgfsetfillcolor{currentfill}%
\pgfsetlinewidth{1.003750pt}%
\definecolor{currentstroke}{rgb}{1.000000,0.647059,0.000000}%
\pgfsetstrokecolor{currentstroke}%
\pgfsetdash{}{0pt}%
\pgfpathmoveto{\pgfqpoint{1.651717in}{3.607466in}}%
\pgfpathlineto{\pgfqpoint{1.693384in}{3.690800in}}%
\pgfpathlineto{\pgfqpoint{1.610051in}{3.690800in}}%
\pgfpathclose%
\pgfusepath{stroke,fill}%
\end{pgfscope}%
\begin{pgfscope}%
\pgfpathrectangle{\pgfqpoint{0.800000in}{0.528000in}}{\pgfqpoint{4.960000in}{3.696000in}} %
\pgfusepath{clip}%
\pgfsetbuttcap%
\pgfsetroundjoin%
\definecolor{currentfill}{rgb}{1.000000,0.647059,0.000000}%
\pgfsetfillcolor{currentfill}%
\pgfsetlinewidth{1.003750pt}%
\definecolor{currentstroke}{rgb}{1.000000,0.647059,0.000000}%
\pgfsetstrokecolor{currentstroke}%
\pgfsetdash{}{0pt}%
\pgfpathmoveto{\pgfqpoint{1.701818in}{2.528560in}}%
\pgfpathlineto{\pgfqpoint{1.743485in}{2.611894in}}%
\pgfpathlineto{\pgfqpoint{1.660152in}{2.611894in}}%
\pgfpathclose%
\pgfusepath{stroke,fill}%
\end{pgfscope}%
\begin{pgfscope}%
\pgfpathrectangle{\pgfqpoint{0.800000in}{0.528000in}}{\pgfqpoint{4.960000in}{3.696000in}} %
\pgfusepath{clip}%
\pgfsetbuttcap%
\pgfsetroundjoin%
\definecolor{currentfill}{rgb}{1.000000,0.647059,0.000000}%
\pgfsetfillcolor{currentfill}%
\pgfsetlinewidth{1.003750pt}%
\definecolor{currentstroke}{rgb}{1.000000,0.647059,0.000000}%
\pgfsetstrokecolor{currentstroke}%
\pgfsetdash{}{0pt}%
\pgfpathmoveto{\pgfqpoint{1.751919in}{1.902015in}}%
\pgfpathlineto{\pgfqpoint{1.793586in}{1.985348in}}%
\pgfpathlineto{\pgfqpoint{1.710253in}{1.985348in}}%
\pgfpathclose%
\pgfusepath{stroke,fill}%
\end{pgfscope}%
\begin{pgfscope}%
\pgfpathrectangle{\pgfqpoint{0.800000in}{0.528000in}}{\pgfqpoint{4.960000in}{3.696000in}} %
\pgfusepath{clip}%
\pgfsetbuttcap%
\pgfsetroundjoin%
\definecolor{currentfill}{rgb}{1.000000,0.647059,0.000000}%
\pgfsetfillcolor{currentfill}%
\pgfsetlinewidth{1.003750pt}%
\definecolor{currentstroke}{rgb}{1.000000,0.647059,0.000000}%
\pgfsetstrokecolor{currentstroke}%
\pgfsetdash{}{0pt}%
\pgfpathmoveto{\pgfqpoint{1.802020in}{2.412656in}}%
\pgfpathlineto{\pgfqpoint{1.843687in}{2.495989in}}%
\pgfpathlineto{\pgfqpoint{1.760354in}{2.495989in}}%
\pgfpathclose%
\pgfusepath{stroke,fill}%
\end{pgfscope}%
\begin{pgfscope}%
\pgfpathrectangle{\pgfqpoint{0.800000in}{0.528000in}}{\pgfqpoint{4.960000in}{3.696000in}} %
\pgfusepath{clip}%
\pgfsetbuttcap%
\pgfsetroundjoin%
\definecolor{currentfill}{rgb}{1.000000,0.647059,0.000000}%
\pgfsetfillcolor{currentfill}%
\pgfsetlinewidth{1.003750pt}%
\definecolor{currentstroke}{rgb}{1.000000,0.647059,0.000000}%
\pgfsetstrokecolor{currentstroke}%
\pgfsetdash{}{0pt}%
\pgfpathmoveto{\pgfqpoint{1.852121in}{2.548203in}}%
\pgfpathlineto{\pgfqpoint{1.893788in}{2.631536in}}%
\pgfpathlineto{\pgfqpoint{1.810455in}{2.631536in}}%
\pgfpathclose%
\pgfusepath{stroke,fill}%
\end{pgfscope}%
\begin{pgfscope}%
\pgfpathrectangle{\pgfqpoint{0.800000in}{0.528000in}}{\pgfqpoint{4.960000in}{3.696000in}} %
\pgfusepath{clip}%
\pgfsetbuttcap%
\pgfsetroundjoin%
\definecolor{currentfill}{rgb}{1.000000,0.647059,0.000000}%
\pgfsetfillcolor{currentfill}%
\pgfsetlinewidth{1.003750pt}%
\definecolor{currentstroke}{rgb}{1.000000,0.647059,0.000000}%
\pgfsetstrokecolor{currentstroke}%
\pgfsetdash{}{0pt}%
\pgfpathmoveto{\pgfqpoint{1.902222in}{1.907558in}}%
\pgfpathlineto{\pgfqpoint{1.943889in}{1.990891in}}%
\pgfpathlineto{\pgfqpoint{1.860556in}{1.990891in}}%
\pgfpathclose%
\pgfusepath{stroke,fill}%
\end{pgfscope}%
\begin{pgfscope}%
\pgfpathrectangle{\pgfqpoint{0.800000in}{0.528000in}}{\pgfqpoint{4.960000in}{3.696000in}} %
\pgfusepath{clip}%
\pgfsetbuttcap%
\pgfsetroundjoin%
\definecolor{currentfill}{rgb}{1.000000,0.647059,0.000000}%
\pgfsetfillcolor{currentfill}%
\pgfsetlinewidth{1.003750pt}%
\definecolor{currentstroke}{rgb}{1.000000,0.647059,0.000000}%
\pgfsetstrokecolor{currentstroke}%
\pgfsetdash{}{0pt}%
\pgfpathmoveto{\pgfqpoint{1.952323in}{3.122247in}}%
\pgfpathlineto{\pgfqpoint{1.993990in}{3.205581in}}%
\pgfpathlineto{\pgfqpoint{1.910657in}{3.205581in}}%
\pgfpathclose%
\pgfusepath{stroke,fill}%
\end{pgfscope}%
\begin{pgfscope}%
\pgfpathrectangle{\pgfqpoint{0.800000in}{0.528000in}}{\pgfqpoint{4.960000in}{3.696000in}} %
\pgfusepath{clip}%
\pgfsetbuttcap%
\pgfsetroundjoin%
\definecolor{currentfill}{rgb}{1.000000,0.647059,0.000000}%
\pgfsetfillcolor{currentfill}%
\pgfsetlinewidth{1.003750pt}%
\definecolor{currentstroke}{rgb}{1.000000,0.647059,0.000000}%
\pgfsetstrokecolor{currentstroke}%
\pgfsetdash{}{0pt}%
\pgfpathmoveto{\pgfqpoint{2.002424in}{6.410703in}}%
\pgfpathlineto{\pgfqpoint{2.044091in}{6.494036in}}%
\pgfpathlineto{\pgfqpoint{1.960758in}{6.494036in}}%
\pgfpathclose%
\pgfusepath{stroke,fill}%
\end{pgfscope}%
\begin{pgfscope}%
\pgfpathrectangle{\pgfqpoint{0.800000in}{0.528000in}}{\pgfqpoint{4.960000in}{3.696000in}} %
\pgfusepath{clip}%
\pgfsetbuttcap%
\pgfsetroundjoin%
\definecolor{currentfill}{rgb}{1.000000,0.647059,0.000000}%
\pgfsetfillcolor{currentfill}%
\pgfsetlinewidth{1.003750pt}%
\definecolor{currentstroke}{rgb}{1.000000,0.647059,0.000000}%
\pgfsetstrokecolor{currentstroke}%
\pgfsetdash{}{0pt}%
\pgfpathmoveto{\pgfqpoint{2.052525in}{5.313310in}}%
\pgfpathlineto{\pgfqpoint{2.094192in}{5.396643in}}%
\pgfpathlineto{\pgfqpoint{2.010859in}{5.396643in}}%
\pgfpathclose%
\pgfusepath{stroke,fill}%
\end{pgfscope}%
\begin{pgfscope}%
\pgfpathrectangle{\pgfqpoint{0.800000in}{0.528000in}}{\pgfqpoint{4.960000in}{3.696000in}} %
\pgfusepath{clip}%
\pgfsetbuttcap%
\pgfsetroundjoin%
\definecolor{currentfill}{rgb}{1.000000,0.647059,0.000000}%
\pgfsetfillcolor{currentfill}%
\pgfsetlinewidth{1.003750pt}%
\definecolor{currentstroke}{rgb}{1.000000,0.647059,0.000000}%
\pgfsetstrokecolor{currentstroke}%
\pgfsetdash{}{0pt}%
\pgfpathmoveto{\pgfqpoint{2.102626in}{3.923920in}}%
\pgfpathlineto{\pgfqpoint{2.144293in}{4.007254in}}%
\pgfpathlineto{\pgfqpoint{2.060960in}{4.007254in}}%
\pgfpathclose%
\pgfusepath{stroke,fill}%
\end{pgfscope}%
\begin{pgfscope}%
\pgfpathrectangle{\pgfqpoint{0.800000in}{0.528000in}}{\pgfqpoint{4.960000in}{3.696000in}} %
\pgfusepath{clip}%
\pgfsetbuttcap%
\pgfsetroundjoin%
\definecolor{currentfill}{rgb}{1.000000,0.647059,0.000000}%
\pgfsetfillcolor{currentfill}%
\pgfsetlinewidth{1.003750pt}%
\definecolor{currentstroke}{rgb}{1.000000,0.647059,0.000000}%
\pgfsetstrokecolor{currentstroke}%
\pgfsetdash{}{0pt}%
\pgfpathmoveto{\pgfqpoint{2.152727in}{6.173705in}}%
\pgfpathlineto{\pgfqpoint{2.194394in}{6.257038in}}%
\pgfpathlineto{\pgfqpoint{2.111061in}{6.257038in}}%
\pgfpathclose%
\pgfusepath{stroke,fill}%
\end{pgfscope}%
\begin{pgfscope}%
\pgfpathrectangle{\pgfqpoint{0.800000in}{0.528000in}}{\pgfqpoint{4.960000in}{3.696000in}} %
\pgfusepath{clip}%
\pgfsetbuttcap%
\pgfsetroundjoin%
\definecolor{currentfill}{rgb}{1.000000,0.647059,0.000000}%
\pgfsetfillcolor{currentfill}%
\pgfsetlinewidth{1.003750pt}%
\definecolor{currentstroke}{rgb}{1.000000,0.647059,0.000000}%
\pgfsetstrokecolor{currentstroke}%
\pgfsetdash{}{0pt}%
\pgfpathmoveto{\pgfqpoint{2.202828in}{3.907119in}}%
\pgfpathlineto{\pgfqpoint{2.244495in}{3.990453in}}%
\pgfpathlineto{\pgfqpoint{2.161162in}{3.990453in}}%
\pgfpathclose%
\pgfusepath{stroke,fill}%
\end{pgfscope}%
\begin{pgfscope}%
\pgfpathrectangle{\pgfqpoint{0.800000in}{0.528000in}}{\pgfqpoint{4.960000in}{3.696000in}} %
\pgfusepath{clip}%
\pgfsetbuttcap%
\pgfsetroundjoin%
\definecolor{currentfill}{rgb}{1.000000,0.647059,0.000000}%
\pgfsetfillcolor{currentfill}%
\pgfsetlinewidth{1.003750pt}%
\definecolor{currentstroke}{rgb}{1.000000,0.647059,0.000000}%
\pgfsetstrokecolor{currentstroke}%
\pgfsetdash{}{0pt}%
\pgfpathmoveto{\pgfqpoint{2.252929in}{6.329591in}}%
\pgfpathlineto{\pgfqpoint{2.294596in}{6.412925in}}%
\pgfpathlineto{\pgfqpoint{2.211263in}{6.412925in}}%
\pgfpathclose%
\pgfusepath{stroke,fill}%
\end{pgfscope}%
\begin{pgfscope}%
\pgfpathrectangle{\pgfqpoint{0.800000in}{0.528000in}}{\pgfqpoint{4.960000in}{3.696000in}} %
\pgfusepath{clip}%
\pgfsetbuttcap%
\pgfsetroundjoin%
\definecolor{currentfill}{rgb}{1.000000,0.647059,0.000000}%
\pgfsetfillcolor{currentfill}%
\pgfsetlinewidth{1.003750pt}%
\definecolor{currentstroke}{rgb}{1.000000,0.647059,0.000000}%
\pgfsetstrokecolor{currentstroke}%
\pgfsetdash{}{0pt}%
\pgfpathmoveto{\pgfqpoint{2.303030in}{5.495913in}}%
\pgfpathlineto{\pgfqpoint{2.344697in}{5.579246in}}%
\pgfpathlineto{\pgfqpoint{2.261364in}{5.579246in}}%
\pgfpathclose%
\pgfusepath{stroke,fill}%
\end{pgfscope}%
\begin{pgfscope}%
\pgfpathrectangle{\pgfqpoint{0.800000in}{0.528000in}}{\pgfqpoint{4.960000in}{3.696000in}} %
\pgfusepath{clip}%
\pgfsetbuttcap%
\pgfsetroundjoin%
\definecolor{currentfill}{rgb}{1.000000,0.647059,0.000000}%
\pgfsetfillcolor{currentfill}%
\pgfsetlinewidth{1.003750pt}%
\definecolor{currentstroke}{rgb}{1.000000,0.647059,0.000000}%
\pgfsetstrokecolor{currentstroke}%
\pgfsetdash{}{0pt}%
\pgfpathmoveto{\pgfqpoint{2.353131in}{5.284425in}}%
\pgfpathlineto{\pgfqpoint{2.394798in}{5.367758in}}%
\pgfpathlineto{\pgfqpoint{2.311465in}{5.367758in}}%
\pgfpathclose%
\pgfusepath{stroke,fill}%
\end{pgfscope}%
\begin{pgfscope}%
\pgfpathrectangle{\pgfqpoint{0.800000in}{0.528000in}}{\pgfqpoint{4.960000in}{3.696000in}} %
\pgfusepath{clip}%
\pgfsetbuttcap%
\pgfsetroundjoin%
\definecolor{currentfill}{rgb}{1.000000,0.647059,0.000000}%
\pgfsetfillcolor{currentfill}%
\pgfsetlinewidth{1.003750pt}%
\definecolor{currentstroke}{rgb}{1.000000,0.647059,0.000000}%
\pgfsetstrokecolor{currentstroke}%
\pgfsetdash{}{0pt}%
\pgfpathmoveto{\pgfqpoint{2.403232in}{4.847724in}}%
\pgfpathlineto{\pgfqpoint{2.444899in}{4.931058in}}%
\pgfpathlineto{\pgfqpoint{2.361566in}{4.931058in}}%
\pgfpathclose%
\pgfusepath{stroke,fill}%
\end{pgfscope}%
\begin{pgfscope}%
\pgfpathrectangle{\pgfqpoint{0.800000in}{0.528000in}}{\pgfqpoint{4.960000in}{3.696000in}} %
\pgfusepath{clip}%
\pgfsetbuttcap%
\pgfsetroundjoin%
\definecolor{currentfill}{rgb}{1.000000,0.647059,0.000000}%
\pgfsetfillcolor{currentfill}%
\pgfsetlinewidth{1.003750pt}%
\definecolor{currentstroke}{rgb}{1.000000,0.647059,0.000000}%
\pgfsetstrokecolor{currentstroke}%
\pgfsetdash{}{0pt}%
\pgfpathmoveto{\pgfqpoint{2.453333in}{4.103277in}}%
\pgfpathlineto{\pgfqpoint{2.495000in}{4.186610in}}%
\pgfpathlineto{\pgfqpoint{2.411667in}{4.186610in}}%
\pgfpathclose%
\pgfusepath{stroke,fill}%
\end{pgfscope}%
\begin{pgfscope}%
\pgfpathrectangle{\pgfqpoint{0.800000in}{0.528000in}}{\pgfqpoint{4.960000in}{3.696000in}} %
\pgfusepath{clip}%
\pgfsetbuttcap%
\pgfsetroundjoin%
\definecolor{currentfill}{rgb}{1.000000,0.647059,0.000000}%
\pgfsetfillcolor{currentfill}%
\pgfsetlinewidth{1.003750pt}%
\definecolor{currentstroke}{rgb}{1.000000,0.647059,0.000000}%
\pgfsetstrokecolor{currentstroke}%
\pgfsetdash{}{0pt}%
\pgfpathmoveto{\pgfqpoint{2.503434in}{5.221209in}}%
\pgfpathlineto{\pgfqpoint{2.545101in}{5.304542in}}%
\pgfpathlineto{\pgfqpoint{2.461768in}{5.304542in}}%
\pgfpathclose%
\pgfusepath{stroke,fill}%
\end{pgfscope}%
\begin{pgfscope}%
\pgfpathrectangle{\pgfqpoint{0.800000in}{0.528000in}}{\pgfqpoint{4.960000in}{3.696000in}} %
\pgfusepath{clip}%
\pgfsetbuttcap%
\pgfsetroundjoin%
\definecolor{currentfill}{rgb}{1.000000,0.647059,0.000000}%
\pgfsetfillcolor{currentfill}%
\pgfsetlinewidth{1.003750pt}%
\definecolor{currentstroke}{rgb}{1.000000,0.647059,0.000000}%
\pgfsetstrokecolor{currentstroke}%
\pgfsetdash{}{0pt}%
\pgfpathmoveto{\pgfqpoint{2.553535in}{2.489146in}}%
\pgfpathlineto{\pgfqpoint{2.595202in}{2.572480in}}%
\pgfpathlineto{\pgfqpoint{2.511869in}{2.572480in}}%
\pgfpathclose%
\pgfusepath{stroke,fill}%
\end{pgfscope}%
\begin{pgfscope}%
\pgfpathrectangle{\pgfqpoint{0.800000in}{0.528000in}}{\pgfqpoint{4.960000in}{3.696000in}} %
\pgfusepath{clip}%
\pgfsetbuttcap%
\pgfsetroundjoin%
\definecolor{currentfill}{rgb}{1.000000,0.647059,0.000000}%
\pgfsetfillcolor{currentfill}%
\pgfsetlinewidth{1.003750pt}%
\definecolor{currentstroke}{rgb}{1.000000,0.647059,0.000000}%
\pgfsetstrokecolor{currentstroke}%
\pgfsetdash{}{0pt}%
\pgfpathmoveto{\pgfqpoint{2.603636in}{3.692620in}}%
\pgfpathlineto{\pgfqpoint{2.645303in}{3.775953in}}%
\pgfpathlineto{\pgfqpoint{2.561970in}{3.775953in}}%
\pgfpathclose%
\pgfusepath{stroke,fill}%
\end{pgfscope}%
\begin{pgfscope}%
\pgfpathrectangle{\pgfqpoint{0.800000in}{0.528000in}}{\pgfqpoint{4.960000in}{3.696000in}} %
\pgfusepath{clip}%
\pgfsetbuttcap%
\pgfsetroundjoin%
\definecolor{currentfill}{rgb}{1.000000,0.647059,0.000000}%
\pgfsetfillcolor{currentfill}%
\pgfsetlinewidth{1.003750pt}%
\definecolor{currentstroke}{rgb}{1.000000,0.647059,0.000000}%
\pgfsetstrokecolor{currentstroke}%
\pgfsetdash{}{0pt}%
\pgfpathmoveto{\pgfqpoint{2.653737in}{1.965013in}}%
\pgfpathlineto{\pgfqpoint{2.695404in}{2.048347in}}%
\pgfpathlineto{\pgfqpoint{2.612071in}{2.048347in}}%
\pgfpathclose%
\pgfusepath{stroke,fill}%
\end{pgfscope}%
\begin{pgfscope}%
\pgfpathrectangle{\pgfqpoint{0.800000in}{0.528000in}}{\pgfqpoint{4.960000in}{3.696000in}} %
\pgfusepath{clip}%
\pgfsetbuttcap%
\pgfsetroundjoin%
\definecolor{currentfill}{rgb}{1.000000,0.647059,0.000000}%
\pgfsetfillcolor{currentfill}%
\pgfsetlinewidth{1.003750pt}%
\definecolor{currentstroke}{rgb}{1.000000,0.647059,0.000000}%
\pgfsetstrokecolor{currentstroke}%
\pgfsetdash{}{0pt}%
\pgfpathmoveto{\pgfqpoint{2.703838in}{2.839286in}}%
\pgfpathlineto{\pgfqpoint{2.745505in}{2.922620in}}%
\pgfpathlineto{\pgfqpoint{2.662172in}{2.922620in}}%
\pgfpathclose%
\pgfusepath{stroke,fill}%
\end{pgfscope}%
\begin{pgfscope}%
\pgfpathrectangle{\pgfqpoint{0.800000in}{0.528000in}}{\pgfqpoint{4.960000in}{3.696000in}} %
\pgfusepath{clip}%
\pgfsetbuttcap%
\pgfsetroundjoin%
\definecolor{currentfill}{rgb}{1.000000,0.647059,0.000000}%
\pgfsetfillcolor{currentfill}%
\pgfsetlinewidth{1.003750pt}%
\definecolor{currentstroke}{rgb}{1.000000,0.647059,0.000000}%
\pgfsetstrokecolor{currentstroke}%
\pgfsetdash{}{0pt}%
\pgfpathmoveto{\pgfqpoint{2.753939in}{4.757648in}}%
\pgfpathlineto{\pgfqpoint{2.795606in}{4.840981in}}%
\pgfpathlineto{\pgfqpoint{2.712273in}{4.840981in}}%
\pgfpathclose%
\pgfusepath{stroke,fill}%
\end{pgfscope}%
\begin{pgfscope}%
\pgfpathrectangle{\pgfqpoint{0.800000in}{0.528000in}}{\pgfqpoint{4.960000in}{3.696000in}} %
\pgfusepath{clip}%
\pgfsetbuttcap%
\pgfsetroundjoin%
\definecolor{currentfill}{rgb}{1.000000,0.647059,0.000000}%
\pgfsetfillcolor{currentfill}%
\pgfsetlinewidth{1.003750pt}%
\definecolor{currentstroke}{rgb}{1.000000,0.647059,0.000000}%
\pgfsetstrokecolor{currentstroke}%
\pgfsetdash{}{0pt}%
\pgfpathmoveto{\pgfqpoint{2.804040in}{2.356106in}}%
\pgfpathlineto{\pgfqpoint{2.845707in}{2.439439in}}%
\pgfpathlineto{\pgfqpoint{2.762374in}{2.439439in}}%
\pgfpathclose%
\pgfusepath{stroke,fill}%
\end{pgfscope}%
\begin{pgfscope}%
\pgfpathrectangle{\pgfqpoint{0.800000in}{0.528000in}}{\pgfqpoint{4.960000in}{3.696000in}} %
\pgfusepath{clip}%
\pgfsetbuttcap%
\pgfsetroundjoin%
\definecolor{currentfill}{rgb}{1.000000,0.647059,0.000000}%
\pgfsetfillcolor{currentfill}%
\pgfsetlinewidth{1.003750pt}%
\definecolor{currentstroke}{rgb}{1.000000,0.647059,0.000000}%
\pgfsetstrokecolor{currentstroke}%
\pgfsetdash{}{0pt}%
\pgfpathmoveto{\pgfqpoint{2.854141in}{2.935956in}}%
\pgfpathlineto{\pgfqpoint{2.895808in}{3.019290in}}%
\pgfpathlineto{\pgfqpoint{2.812475in}{3.019290in}}%
\pgfpathclose%
\pgfusepath{stroke,fill}%
\end{pgfscope}%
\begin{pgfscope}%
\pgfpathrectangle{\pgfqpoint{0.800000in}{0.528000in}}{\pgfqpoint{4.960000in}{3.696000in}} %
\pgfusepath{clip}%
\pgfsetbuttcap%
\pgfsetroundjoin%
\definecolor{currentfill}{rgb}{1.000000,0.647059,0.000000}%
\pgfsetfillcolor{currentfill}%
\pgfsetlinewidth{1.003750pt}%
\definecolor{currentstroke}{rgb}{1.000000,0.647059,0.000000}%
\pgfsetstrokecolor{currentstroke}%
\pgfsetdash{}{0pt}%
\pgfpathmoveto{\pgfqpoint{2.904242in}{3.155769in}}%
\pgfpathlineto{\pgfqpoint{2.945909in}{3.239103in}}%
\pgfpathlineto{\pgfqpoint{2.862576in}{3.239103in}}%
\pgfpathclose%
\pgfusepath{stroke,fill}%
\end{pgfscope}%
\begin{pgfscope}%
\pgfpathrectangle{\pgfqpoint{0.800000in}{0.528000in}}{\pgfqpoint{4.960000in}{3.696000in}} %
\pgfusepath{clip}%
\pgfsetbuttcap%
\pgfsetroundjoin%
\definecolor{currentfill}{rgb}{1.000000,0.647059,0.000000}%
\pgfsetfillcolor{currentfill}%
\pgfsetlinewidth{1.003750pt}%
\definecolor{currentstroke}{rgb}{1.000000,0.647059,0.000000}%
\pgfsetstrokecolor{currentstroke}%
\pgfsetdash{}{0pt}%
\pgfpathmoveto{\pgfqpoint{2.954343in}{5.642572in}}%
\pgfpathlineto{\pgfqpoint{2.996010in}{5.725906in}}%
\pgfpathlineto{\pgfqpoint{2.912677in}{5.725906in}}%
\pgfpathclose%
\pgfusepath{stroke,fill}%
\end{pgfscope}%
\begin{pgfscope}%
\pgfpathrectangle{\pgfqpoint{0.800000in}{0.528000in}}{\pgfqpoint{4.960000in}{3.696000in}} %
\pgfusepath{clip}%
\pgfsetbuttcap%
\pgfsetroundjoin%
\definecolor{currentfill}{rgb}{1.000000,0.647059,0.000000}%
\pgfsetfillcolor{currentfill}%
\pgfsetlinewidth{1.003750pt}%
\definecolor{currentstroke}{rgb}{1.000000,0.647059,0.000000}%
\pgfsetstrokecolor{currentstroke}%
\pgfsetdash{}{0pt}%
\pgfpathmoveto{\pgfqpoint{3.004444in}{3.368145in}}%
\pgfpathlineto{\pgfqpoint{3.046111in}{3.451479in}}%
\pgfpathlineto{\pgfqpoint{2.962778in}{3.451479in}}%
\pgfpathclose%
\pgfusepath{stroke,fill}%
\end{pgfscope}%
\begin{pgfscope}%
\pgfpathrectangle{\pgfqpoint{0.800000in}{0.528000in}}{\pgfqpoint{4.960000in}{3.696000in}} %
\pgfusepath{clip}%
\pgfsetbuttcap%
\pgfsetroundjoin%
\definecolor{currentfill}{rgb}{1.000000,0.647059,0.000000}%
\pgfsetfillcolor{currentfill}%
\pgfsetlinewidth{1.003750pt}%
\definecolor{currentstroke}{rgb}{1.000000,0.647059,0.000000}%
\pgfsetstrokecolor{currentstroke}%
\pgfsetdash{}{0pt}%
\pgfpathmoveto{\pgfqpoint{3.054545in}{4.336455in}}%
\pgfpathlineto{\pgfqpoint{3.096212in}{4.419788in}}%
\pgfpathlineto{\pgfqpoint{3.012879in}{4.419788in}}%
\pgfpathclose%
\pgfusepath{stroke,fill}%
\end{pgfscope}%
\begin{pgfscope}%
\pgfpathrectangle{\pgfqpoint{0.800000in}{0.528000in}}{\pgfqpoint{4.960000in}{3.696000in}} %
\pgfusepath{clip}%
\pgfsetbuttcap%
\pgfsetroundjoin%
\definecolor{currentfill}{rgb}{1.000000,0.647059,0.000000}%
\pgfsetfillcolor{currentfill}%
\pgfsetlinewidth{1.003750pt}%
\definecolor{currentstroke}{rgb}{1.000000,0.647059,0.000000}%
\pgfsetstrokecolor{currentstroke}%
\pgfsetdash{}{0pt}%
\pgfpathmoveto{\pgfqpoint{3.104646in}{5.985662in}}%
\pgfpathlineto{\pgfqpoint{3.146313in}{6.068995in}}%
\pgfpathlineto{\pgfqpoint{3.062980in}{6.068995in}}%
\pgfpathclose%
\pgfusepath{stroke,fill}%
\end{pgfscope}%
\begin{pgfscope}%
\pgfpathrectangle{\pgfqpoint{0.800000in}{0.528000in}}{\pgfqpoint{4.960000in}{3.696000in}} %
\pgfusepath{clip}%
\pgfsetbuttcap%
\pgfsetroundjoin%
\definecolor{currentfill}{rgb}{1.000000,0.647059,0.000000}%
\pgfsetfillcolor{currentfill}%
\pgfsetlinewidth{1.003750pt}%
\definecolor{currentstroke}{rgb}{1.000000,0.647059,0.000000}%
\pgfsetstrokecolor{currentstroke}%
\pgfsetdash{}{0pt}%
\pgfpathmoveto{\pgfqpoint{3.154747in}{5.332747in}}%
\pgfpathlineto{\pgfqpoint{3.196414in}{5.416080in}}%
\pgfpathlineto{\pgfqpoint{3.113081in}{5.416080in}}%
\pgfpathclose%
\pgfusepath{stroke,fill}%
\end{pgfscope}%
\begin{pgfscope}%
\pgfpathrectangle{\pgfqpoint{0.800000in}{0.528000in}}{\pgfqpoint{4.960000in}{3.696000in}} %
\pgfusepath{clip}%
\pgfsetbuttcap%
\pgfsetroundjoin%
\definecolor{currentfill}{rgb}{1.000000,0.647059,0.000000}%
\pgfsetfillcolor{currentfill}%
\pgfsetlinewidth{1.003750pt}%
\definecolor{currentstroke}{rgb}{1.000000,0.647059,0.000000}%
\pgfsetstrokecolor{currentstroke}%
\pgfsetdash{}{0pt}%
\pgfpathmoveto{\pgfqpoint{3.204848in}{6.902353in}}%
\pgfpathlineto{\pgfqpoint{3.246515in}{6.985686in}}%
\pgfpathlineto{\pgfqpoint{3.163182in}{6.985686in}}%
\pgfpathclose%
\pgfusepath{stroke,fill}%
\end{pgfscope}%
\begin{pgfscope}%
\pgfpathrectangle{\pgfqpoint{0.800000in}{0.528000in}}{\pgfqpoint{4.960000in}{3.696000in}} %
\pgfusepath{clip}%
\pgfsetbuttcap%
\pgfsetroundjoin%
\definecolor{currentfill}{rgb}{1.000000,0.647059,0.000000}%
\pgfsetfillcolor{currentfill}%
\pgfsetlinewidth{1.003750pt}%
\definecolor{currentstroke}{rgb}{1.000000,0.647059,0.000000}%
\pgfsetstrokecolor{currentstroke}%
\pgfsetdash{}{0pt}%
\pgfpathmoveto{\pgfqpoint{3.254949in}{5.456591in}}%
\pgfpathlineto{\pgfqpoint{3.296616in}{5.539924in}}%
\pgfpathlineto{\pgfqpoint{3.213283in}{5.539924in}}%
\pgfpathclose%
\pgfusepath{stroke,fill}%
\end{pgfscope}%
\begin{pgfscope}%
\pgfpathrectangle{\pgfqpoint{0.800000in}{0.528000in}}{\pgfqpoint{4.960000in}{3.696000in}} %
\pgfusepath{clip}%
\pgfsetbuttcap%
\pgfsetroundjoin%
\definecolor{currentfill}{rgb}{1.000000,0.647059,0.000000}%
\pgfsetfillcolor{currentfill}%
\pgfsetlinewidth{1.003750pt}%
\definecolor{currentstroke}{rgb}{1.000000,0.647059,0.000000}%
\pgfsetstrokecolor{currentstroke}%
\pgfsetdash{}{0pt}%
\pgfpathmoveto{\pgfqpoint{3.305051in}{4.884177in}}%
\pgfpathlineto{\pgfqpoint{3.346717in}{4.967511in}}%
\pgfpathlineto{\pgfqpoint{3.263384in}{4.967511in}}%
\pgfpathclose%
\pgfusepath{stroke,fill}%
\end{pgfscope}%
\begin{pgfscope}%
\pgfpathrectangle{\pgfqpoint{0.800000in}{0.528000in}}{\pgfqpoint{4.960000in}{3.696000in}} %
\pgfusepath{clip}%
\pgfsetbuttcap%
\pgfsetroundjoin%
\definecolor{currentfill}{rgb}{1.000000,0.647059,0.000000}%
\pgfsetfillcolor{currentfill}%
\pgfsetlinewidth{1.003750pt}%
\definecolor{currentstroke}{rgb}{1.000000,0.647059,0.000000}%
\pgfsetstrokecolor{currentstroke}%
\pgfsetdash{}{0pt}%
\pgfpathmoveto{\pgfqpoint{3.355152in}{5.022157in}}%
\pgfpathlineto{\pgfqpoint{3.396818in}{5.105490in}}%
\pgfpathlineto{\pgfqpoint{3.313485in}{5.105490in}}%
\pgfpathclose%
\pgfusepath{stroke,fill}%
\end{pgfscope}%
\begin{pgfscope}%
\pgfpathrectangle{\pgfqpoint{0.800000in}{0.528000in}}{\pgfqpoint{4.960000in}{3.696000in}} %
\pgfusepath{clip}%
\pgfsetbuttcap%
\pgfsetroundjoin%
\definecolor{currentfill}{rgb}{1.000000,0.647059,0.000000}%
\pgfsetfillcolor{currentfill}%
\pgfsetlinewidth{1.003750pt}%
\definecolor{currentstroke}{rgb}{1.000000,0.647059,0.000000}%
\pgfsetstrokecolor{currentstroke}%
\pgfsetdash{}{0pt}%
\pgfpathmoveto{\pgfqpoint{3.405253in}{3.991922in}}%
\pgfpathlineto{\pgfqpoint{3.446919in}{4.075255in}}%
\pgfpathlineto{\pgfqpoint{3.363586in}{4.075255in}}%
\pgfpathclose%
\pgfusepath{stroke,fill}%
\end{pgfscope}%
\begin{pgfscope}%
\pgfpathrectangle{\pgfqpoint{0.800000in}{0.528000in}}{\pgfqpoint{4.960000in}{3.696000in}} %
\pgfusepath{clip}%
\pgfsetbuttcap%
\pgfsetroundjoin%
\definecolor{currentfill}{rgb}{1.000000,0.647059,0.000000}%
\pgfsetfillcolor{currentfill}%
\pgfsetlinewidth{1.003750pt}%
\definecolor{currentstroke}{rgb}{1.000000,0.647059,0.000000}%
\pgfsetstrokecolor{currentstroke}%
\pgfsetdash{}{0pt}%
\pgfpathmoveto{\pgfqpoint{3.455354in}{6.839616in}}%
\pgfpathlineto{\pgfqpoint{3.497020in}{6.922949in}}%
\pgfpathlineto{\pgfqpoint{3.413687in}{6.922949in}}%
\pgfpathclose%
\pgfusepath{stroke,fill}%
\end{pgfscope}%
\begin{pgfscope}%
\pgfpathrectangle{\pgfqpoint{0.800000in}{0.528000in}}{\pgfqpoint{4.960000in}{3.696000in}} %
\pgfusepath{clip}%
\pgfsetbuttcap%
\pgfsetroundjoin%
\definecolor{currentfill}{rgb}{1.000000,0.647059,0.000000}%
\pgfsetfillcolor{currentfill}%
\pgfsetlinewidth{1.003750pt}%
\definecolor{currentstroke}{rgb}{1.000000,0.647059,0.000000}%
\pgfsetstrokecolor{currentstroke}%
\pgfsetdash{}{0pt}%
\pgfpathmoveto{\pgfqpoint{3.505455in}{5.133087in}}%
\pgfpathlineto{\pgfqpoint{3.547121in}{5.216421in}}%
\pgfpathlineto{\pgfqpoint{3.463788in}{5.216421in}}%
\pgfpathclose%
\pgfusepath{stroke,fill}%
\end{pgfscope}%
\begin{pgfscope}%
\pgfpathrectangle{\pgfqpoint{0.800000in}{0.528000in}}{\pgfqpoint{4.960000in}{3.696000in}} %
\pgfusepath{clip}%
\pgfsetbuttcap%
\pgfsetroundjoin%
\definecolor{currentfill}{rgb}{1.000000,0.647059,0.000000}%
\pgfsetfillcolor{currentfill}%
\pgfsetlinewidth{1.003750pt}%
\definecolor{currentstroke}{rgb}{1.000000,0.647059,0.000000}%
\pgfsetstrokecolor{currentstroke}%
\pgfsetdash{}{0pt}%
\pgfpathmoveto{\pgfqpoint{3.555556in}{7.312538in}}%
\pgfpathlineto{\pgfqpoint{3.597222in}{7.395872in}}%
\pgfpathlineto{\pgfqpoint{3.513889in}{7.395872in}}%
\pgfpathclose%
\pgfusepath{stroke,fill}%
\end{pgfscope}%
\begin{pgfscope}%
\pgfpathrectangle{\pgfqpoint{0.800000in}{0.528000in}}{\pgfqpoint{4.960000in}{3.696000in}} %
\pgfusepath{clip}%
\pgfsetbuttcap%
\pgfsetroundjoin%
\definecolor{currentfill}{rgb}{1.000000,0.647059,0.000000}%
\pgfsetfillcolor{currentfill}%
\pgfsetlinewidth{1.003750pt}%
\definecolor{currentstroke}{rgb}{1.000000,0.647059,0.000000}%
\pgfsetstrokecolor{currentstroke}%
\pgfsetdash{}{0pt}%
\pgfpathmoveto{\pgfqpoint{3.605657in}{7.035221in}}%
\pgfpathlineto{\pgfqpoint{3.647323in}{7.118555in}}%
\pgfpathlineto{\pgfqpoint{3.563990in}{7.118555in}}%
\pgfpathclose%
\pgfusepath{stroke,fill}%
\end{pgfscope}%
\begin{pgfscope}%
\pgfpathrectangle{\pgfqpoint{0.800000in}{0.528000in}}{\pgfqpoint{4.960000in}{3.696000in}} %
\pgfusepath{clip}%
\pgfsetbuttcap%
\pgfsetroundjoin%
\definecolor{currentfill}{rgb}{1.000000,0.647059,0.000000}%
\pgfsetfillcolor{currentfill}%
\pgfsetlinewidth{1.003750pt}%
\definecolor{currentstroke}{rgb}{1.000000,0.647059,0.000000}%
\pgfsetstrokecolor{currentstroke}%
\pgfsetdash{}{0pt}%
\pgfpathmoveto{\pgfqpoint{3.655758in}{2.947918in}}%
\pgfpathlineto{\pgfqpoint{3.697424in}{3.031251in}}%
\pgfpathlineto{\pgfqpoint{3.614091in}{3.031251in}}%
\pgfpathclose%
\pgfusepath{stroke,fill}%
\end{pgfscope}%
\begin{pgfscope}%
\pgfpathrectangle{\pgfqpoint{0.800000in}{0.528000in}}{\pgfqpoint{4.960000in}{3.696000in}} %
\pgfusepath{clip}%
\pgfsetbuttcap%
\pgfsetroundjoin%
\definecolor{currentfill}{rgb}{1.000000,0.647059,0.000000}%
\pgfsetfillcolor{currentfill}%
\pgfsetlinewidth{1.003750pt}%
\definecolor{currentstroke}{rgb}{1.000000,0.647059,0.000000}%
\pgfsetstrokecolor{currentstroke}%
\pgfsetdash{}{0pt}%
\pgfpathmoveto{\pgfqpoint{3.705859in}{6.404804in}}%
\pgfpathlineto{\pgfqpoint{3.747525in}{6.488137in}}%
\pgfpathlineto{\pgfqpoint{3.664192in}{6.488137in}}%
\pgfpathclose%
\pgfusepath{stroke,fill}%
\end{pgfscope}%
\begin{pgfscope}%
\pgfpathrectangle{\pgfqpoint{0.800000in}{0.528000in}}{\pgfqpoint{4.960000in}{3.696000in}} %
\pgfusepath{clip}%
\pgfsetbuttcap%
\pgfsetroundjoin%
\definecolor{currentfill}{rgb}{1.000000,0.647059,0.000000}%
\pgfsetfillcolor{currentfill}%
\pgfsetlinewidth{1.003750pt}%
\definecolor{currentstroke}{rgb}{1.000000,0.647059,0.000000}%
\pgfsetstrokecolor{currentstroke}%
\pgfsetdash{}{0pt}%
\pgfpathmoveto{\pgfqpoint{3.755960in}{3.113258in}}%
\pgfpathlineto{\pgfqpoint{3.797626in}{3.196591in}}%
\pgfpathlineto{\pgfqpoint{3.714293in}{3.196591in}}%
\pgfpathclose%
\pgfusepath{stroke,fill}%
\end{pgfscope}%
\begin{pgfscope}%
\pgfpathrectangle{\pgfqpoint{0.800000in}{0.528000in}}{\pgfqpoint{4.960000in}{3.696000in}} %
\pgfusepath{clip}%
\pgfsetbuttcap%
\pgfsetroundjoin%
\definecolor{currentfill}{rgb}{1.000000,0.647059,0.000000}%
\pgfsetfillcolor{currentfill}%
\pgfsetlinewidth{1.003750pt}%
\definecolor{currentstroke}{rgb}{1.000000,0.647059,0.000000}%
\pgfsetstrokecolor{currentstroke}%
\pgfsetdash{}{0pt}%
\pgfpathmoveto{\pgfqpoint{3.806061in}{6.861077in}}%
\pgfpathlineto{\pgfqpoint{3.847727in}{6.944410in}}%
\pgfpathlineto{\pgfqpoint{3.764394in}{6.944410in}}%
\pgfpathclose%
\pgfusepath{stroke,fill}%
\end{pgfscope}%
\begin{pgfscope}%
\pgfpathrectangle{\pgfqpoint{0.800000in}{0.528000in}}{\pgfqpoint{4.960000in}{3.696000in}} %
\pgfusepath{clip}%
\pgfsetbuttcap%
\pgfsetroundjoin%
\definecolor{currentfill}{rgb}{1.000000,0.647059,0.000000}%
\pgfsetfillcolor{currentfill}%
\pgfsetlinewidth{1.003750pt}%
\definecolor{currentstroke}{rgb}{1.000000,0.647059,0.000000}%
\pgfsetstrokecolor{currentstroke}%
\pgfsetdash{}{0pt}%
\pgfpathmoveto{\pgfqpoint{3.856162in}{4.542422in}}%
\pgfpathlineto{\pgfqpoint{3.897828in}{4.625755in}}%
\pgfpathlineto{\pgfqpoint{3.814495in}{4.625755in}}%
\pgfpathclose%
\pgfusepath{stroke,fill}%
\end{pgfscope}%
\begin{pgfscope}%
\pgfpathrectangle{\pgfqpoint{0.800000in}{0.528000in}}{\pgfqpoint{4.960000in}{3.696000in}} %
\pgfusepath{clip}%
\pgfsetbuttcap%
\pgfsetroundjoin%
\definecolor{currentfill}{rgb}{1.000000,0.647059,0.000000}%
\pgfsetfillcolor{currentfill}%
\pgfsetlinewidth{1.003750pt}%
\definecolor{currentstroke}{rgb}{1.000000,0.647059,0.000000}%
\pgfsetstrokecolor{currentstroke}%
\pgfsetdash{}{0pt}%
\pgfpathmoveto{\pgfqpoint{3.906263in}{4.966856in}}%
\pgfpathlineto{\pgfqpoint{3.947929in}{5.050189in}}%
\pgfpathlineto{\pgfqpoint{3.864596in}{5.050189in}}%
\pgfpathclose%
\pgfusepath{stroke,fill}%
\end{pgfscope}%
\begin{pgfscope}%
\pgfpathrectangle{\pgfqpoint{0.800000in}{0.528000in}}{\pgfqpoint{4.960000in}{3.696000in}} %
\pgfusepath{clip}%
\pgfsetbuttcap%
\pgfsetroundjoin%
\definecolor{currentfill}{rgb}{1.000000,0.647059,0.000000}%
\pgfsetfillcolor{currentfill}%
\pgfsetlinewidth{1.003750pt}%
\definecolor{currentstroke}{rgb}{1.000000,0.647059,0.000000}%
\pgfsetstrokecolor{currentstroke}%
\pgfsetdash{}{0pt}%
\pgfpathmoveto{\pgfqpoint{3.956364in}{4.498762in}}%
\pgfpathlineto{\pgfqpoint{3.998030in}{4.582095in}}%
\pgfpathlineto{\pgfqpoint{3.914697in}{4.582095in}}%
\pgfpathclose%
\pgfusepath{stroke,fill}%
\end{pgfscope}%
\begin{pgfscope}%
\pgfpathrectangle{\pgfqpoint{0.800000in}{0.528000in}}{\pgfqpoint{4.960000in}{3.696000in}} %
\pgfusepath{clip}%
\pgfsetbuttcap%
\pgfsetroundjoin%
\definecolor{currentfill}{rgb}{1.000000,0.647059,0.000000}%
\pgfsetfillcolor{currentfill}%
\pgfsetlinewidth{1.003750pt}%
\definecolor{currentstroke}{rgb}{1.000000,0.647059,0.000000}%
\pgfsetstrokecolor{currentstroke}%
\pgfsetdash{}{0pt}%
\pgfpathmoveto{\pgfqpoint{4.006465in}{7.217320in}}%
\pgfpathlineto{\pgfqpoint{4.048131in}{7.300653in}}%
\pgfpathlineto{\pgfqpoint{3.964798in}{7.300653in}}%
\pgfpathclose%
\pgfusepath{stroke,fill}%
\end{pgfscope}%
\begin{pgfscope}%
\pgfpathrectangle{\pgfqpoint{0.800000in}{0.528000in}}{\pgfqpoint{4.960000in}{3.696000in}} %
\pgfusepath{clip}%
\pgfsetbuttcap%
\pgfsetroundjoin%
\definecolor{currentfill}{rgb}{1.000000,0.647059,0.000000}%
\pgfsetfillcolor{currentfill}%
\pgfsetlinewidth{1.003750pt}%
\definecolor{currentstroke}{rgb}{1.000000,0.647059,0.000000}%
\pgfsetstrokecolor{currentstroke}%
\pgfsetdash{}{0pt}%
\pgfpathmoveto{\pgfqpoint{4.056566in}{6.145431in}}%
\pgfpathlineto{\pgfqpoint{4.098232in}{6.228764in}}%
\pgfpathlineto{\pgfqpoint{4.014899in}{6.228764in}}%
\pgfpathclose%
\pgfusepath{stroke,fill}%
\end{pgfscope}%
\begin{pgfscope}%
\pgfpathrectangle{\pgfqpoint{0.800000in}{0.528000in}}{\pgfqpoint{4.960000in}{3.696000in}} %
\pgfusepath{clip}%
\pgfsetbuttcap%
\pgfsetroundjoin%
\definecolor{currentfill}{rgb}{1.000000,0.647059,0.000000}%
\pgfsetfillcolor{currentfill}%
\pgfsetlinewidth{1.003750pt}%
\definecolor{currentstroke}{rgb}{1.000000,0.647059,0.000000}%
\pgfsetstrokecolor{currentstroke}%
\pgfsetdash{}{0pt}%
\pgfpathmoveto{\pgfqpoint{4.106667in}{3.154392in}}%
\pgfpathlineto{\pgfqpoint{4.148333in}{3.237725in}}%
\pgfpathlineto{\pgfqpoint{4.065000in}{3.237725in}}%
\pgfpathclose%
\pgfusepath{stroke,fill}%
\end{pgfscope}%
\begin{pgfscope}%
\pgfpathrectangle{\pgfqpoint{0.800000in}{0.528000in}}{\pgfqpoint{4.960000in}{3.696000in}} %
\pgfusepath{clip}%
\pgfsetbuttcap%
\pgfsetroundjoin%
\definecolor{currentfill}{rgb}{1.000000,0.647059,0.000000}%
\pgfsetfillcolor{currentfill}%
\pgfsetlinewidth{1.003750pt}%
\definecolor{currentstroke}{rgb}{1.000000,0.647059,0.000000}%
\pgfsetstrokecolor{currentstroke}%
\pgfsetdash{}{0pt}%
\pgfpathmoveto{\pgfqpoint{4.156768in}{5.877524in}}%
\pgfpathlineto{\pgfqpoint{4.198434in}{5.960857in}}%
\pgfpathlineto{\pgfqpoint{4.115101in}{5.960857in}}%
\pgfpathclose%
\pgfusepath{stroke,fill}%
\end{pgfscope}%
\begin{pgfscope}%
\pgfpathrectangle{\pgfqpoint{0.800000in}{0.528000in}}{\pgfqpoint{4.960000in}{3.696000in}} %
\pgfusepath{clip}%
\pgfsetbuttcap%
\pgfsetroundjoin%
\definecolor{currentfill}{rgb}{1.000000,0.647059,0.000000}%
\pgfsetfillcolor{currentfill}%
\pgfsetlinewidth{1.003750pt}%
\definecolor{currentstroke}{rgb}{1.000000,0.647059,0.000000}%
\pgfsetstrokecolor{currentstroke}%
\pgfsetdash{}{0pt}%
\pgfpathmoveto{\pgfqpoint{4.206869in}{4.309612in}}%
\pgfpathlineto{\pgfqpoint{4.248535in}{4.392945in}}%
\pgfpathlineto{\pgfqpoint{4.165202in}{4.392945in}}%
\pgfpathclose%
\pgfusepath{stroke,fill}%
\end{pgfscope}%
\begin{pgfscope}%
\pgfpathrectangle{\pgfqpoint{0.800000in}{0.528000in}}{\pgfqpoint{4.960000in}{3.696000in}} %
\pgfusepath{clip}%
\pgfsetbuttcap%
\pgfsetroundjoin%
\definecolor{currentfill}{rgb}{1.000000,0.647059,0.000000}%
\pgfsetfillcolor{currentfill}%
\pgfsetlinewidth{1.003750pt}%
\definecolor{currentstroke}{rgb}{1.000000,0.647059,0.000000}%
\pgfsetstrokecolor{currentstroke}%
\pgfsetdash{}{0pt}%
\pgfpathmoveto{\pgfqpoint{4.256970in}{7.652998in}}%
\pgfpathlineto{\pgfqpoint{4.298636in}{7.736332in}}%
\pgfpathlineto{\pgfqpoint{4.215303in}{7.736332in}}%
\pgfpathclose%
\pgfusepath{stroke,fill}%
\end{pgfscope}%
\begin{pgfscope}%
\pgfpathrectangle{\pgfqpoint{0.800000in}{0.528000in}}{\pgfqpoint{4.960000in}{3.696000in}} %
\pgfusepath{clip}%
\pgfsetbuttcap%
\pgfsetroundjoin%
\definecolor{currentfill}{rgb}{1.000000,0.647059,0.000000}%
\pgfsetfillcolor{currentfill}%
\pgfsetlinewidth{1.003750pt}%
\definecolor{currentstroke}{rgb}{1.000000,0.647059,0.000000}%
\pgfsetstrokecolor{currentstroke}%
\pgfsetdash{}{0pt}%
\pgfpathmoveto{\pgfqpoint{4.307071in}{6.789426in}}%
\pgfpathlineto{\pgfqpoint{4.348737in}{6.872759in}}%
\pgfpathlineto{\pgfqpoint{4.265404in}{6.872759in}}%
\pgfpathclose%
\pgfusepath{stroke,fill}%
\end{pgfscope}%
\begin{pgfscope}%
\pgfpathrectangle{\pgfqpoint{0.800000in}{0.528000in}}{\pgfqpoint{4.960000in}{3.696000in}} %
\pgfusepath{clip}%
\pgfsetbuttcap%
\pgfsetroundjoin%
\definecolor{currentfill}{rgb}{1.000000,0.647059,0.000000}%
\pgfsetfillcolor{currentfill}%
\pgfsetlinewidth{1.003750pt}%
\definecolor{currentstroke}{rgb}{1.000000,0.647059,0.000000}%
\pgfsetstrokecolor{currentstroke}%
\pgfsetdash{}{0pt}%
\pgfpathmoveto{\pgfqpoint{4.357172in}{4.784806in}}%
\pgfpathlineto{\pgfqpoint{4.398838in}{4.868140in}}%
\pgfpathlineto{\pgfqpoint{4.315505in}{4.868140in}}%
\pgfpathclose%
\pgfusepath{stroke,fill}%
\end{pgfscope}%
\begin{pgfscope}%
\pgfpathrectangle{\pgfqpoint{0.800000in}{0.528000in}}{\pgfqpoint{4.960000in}{3.696000in}} %
\pgfusepath{clip}%
\pgfsetbuttcap%
\pgfsetroundjoin%
\definecolor{currentfill}{rgb}{1.000000,0.647059,0.000000}%
\pgfsetfillcolor{currentfill}%
\pgfsetlinewidth{1.003750pt}%
\definecolor{currentstroke}{rgb}{1.000000,0.647059,0.000000}%
\pgfsetstrokecolor{currentstroke}%
\pgfsetdash{}{0pt}%
\pgfpathmoveto{\pgfqpoint{4.407273in}{7.819725in}}%
\pgfpathlineto{\pgfqpoint{4.448939in}{7.903058in}}%
\pgfpathlineto{\pgfqpoint{4.365606in}{7.903058in}}%
\pgfpathclose%
\pgfusepath{stroke,fill}%
\end{pgfscope}%
\begin{pgfscope}%
\pgfpathrectangle{\pgfqpoint{0.800000in}{0.528000in}}{\pgfqpoint{4.960000in}{3.696000in}} %
\pgfusepath{clip}%
\pgfsetbuttcap%
\pgfsetroundjoin%
\definecolor{currentfill}{rgb}{1.000000,0.647059,0.000000}%
\pgfsetfillcolor{currentfill}%
\pgfsetlinewidth{1.003750pt}%
\definecolor{currentstroke}{rgb}{1.000000,0.647059,0.000000}%
\pgfsetstrokecolor{currentstroke}%
\pgfsetdash{}{0pt}%
\pgfpathmoveto{\pgfqpoint{4.457374in}{7.100948in}}%
\pgfpathlineto{\pgfqpoint{4.499040in}{7.184281in}}%
\pgfpathlineto{\pgfqpoint{4.415707in}{7.184281in}}%
\pgfpathclose%
\pgfusepath{stroke,fill}%
\end{pgfscope}%
\begin{pgfscope}%
\pgfpathrectangle{\pgfqpoint{0.800000in}{0.528000in}}{\pgfqpoint{4.960000in}{3.696000in}} %
\pgfusepath{clip}%
\pgfsetbuttcap%
\pgfsetroundjoin%
\definecolor{currentfill}{rgb}{1.000000,0.647059,0.000000}%
\pgfsetfillcolor{currentfill}%
\pgfsetlinewidth{1.003750pt}%
\definecolor{currentstroke}{rgb}{1.000000,0.647059,0.000000}%
\pgfsetstrokecolor{currentstroke}%
\pgfsetdash{}{0pt}%
\pgfpathmoveto{\pgfqpoint{4.507475in}{4.142787in}}%
\pgfpathlineto{\pgfqpoint{4.549141in}{4.226120in}}%
\pgfpathlineto{\pgfqpoint{4.465808in}{4.226120in}}%
\pgfpathclose%
\pgfusepath{stroke,fill}%
\end{pgfscope}%
\begin{pgfscope}%
\pgfpathrectangle{\pgfqpoint{0.800000in}{0.528000in}}{\pgfqpoint{4.960000in}{3.696000in}} %
\pgfusepath{clip}%
\pgfsetbuttcap%
\pgfsetroundjoin%
\definecolor{currentfill}{rgb}{1.000000,0.647059,0.000000}%
\pgfsetfillcolor{currentfill}%
\pgfsetlinewidth{1.003750pt}%
\definecolor{currentstroke}{rgb}{1.000000,0.647059,0.000000}%
\pgfsetstrokecolor{currentstroke}%
\pgfsetdash{}{0pt}%
\pgfpathmoveto{\pgfqpoint{4.557576in}{4.983101in}}%
\pgfpathlineto{\pgfqpoint{4.599242in}{5.066435in}}%
\pgfpathlineto{\pgfqpoint{4.515909in}{5.066435in}}%
\pgfpathclose%
\pgfusepath{stroke,fill}%
\end{pgfscope}%
\begin{pgfscope}%
\pgfpathrectangle{\pgfqpoint{0.800000in}{0.528000in}}{\pgfqpoint{4.960000in}{3.696000in}} %
\pgfusepath{clip}%
\pgfsetbuttcap%
\pgfsetroundjoin%
\definecolor{currentfill}{rgb}{1.000000,0.647059,0.000000}%
\pgfsetfillcolor{currentfill}%
\pgfsetlinewidth{1.003750pt}%
\definecolor{currentstroke}{rgb}{1.000000,0.647059,0.000000}%
\pgfsetstrokecolor{currentstroke}%
\pgfsetdash{}{0pt}%
\pgfpathmoveto{\pgfqpoint{4.607677in}{5.365968in}}%
\pgfpathlineto{\pgfqpoint{4.649343in}{5.449301in}}%
\pgfpathlineto{\pgfqpoint{4.566010in}{5.449301in}}%
\pgfpathclose%
\pgfusepath{stroke,fill}%
\end{pgfscope}%
\begin{pgfscope}%
\pgfpathrectangle{\pgfqpoint{0.800000in}{0.528000in}}{\pgfqpoint{4.960000in}{3.696000in}} %
\pgfusepath{clip}%
\pgfsetbuttcap%
\pgfsetroundjoin%
\definecolor{currentfill}{rgb}{1.000000,0.647059,0.000000}%
\pgfsetfillcolor{currentfill}%
\pgfsetlinewidth{1.003750pt}%
\definecolor{currentstroke}{rgb}{1.000000,0.647059,0.000000}%
\pgfsetstrokecolor{currentstroke}%
\pgfsetdash{}{0pt}%
\pgfpathmoveto{\pgfqpoint{4.657778in}{7.302378in}}%
\pgfpathlineto{\pgfqpoint{4.699444in}{7.385711in}}%
\pgfpathlineto{\pgfqpoint{4.616111in}{7.385711in}}%
\pgfpathclose%
\pgfusepath{stroke,fill}%
\end{pgfscope}%
\begin{pgfscope}%
\pgfpathrectangle{\pgfqpoint{0.800000in}{0.528000in}}{\pgfqpoint{4.960000in}{3.696000in}} %
\pgfusepath{clip}%
\pgfsetbuttcap%
\pgfsetroundjoin%
\definecolor{currentfill}{rgb}{1.000000,0.647059,0.000000}%
\pgfsetfillcolor{currentfill}%
\pgfsetlinewidth{1.003750pt}%
\definecolor{currentstroke}{rgb}{1.000000,0.647059,0.000000}%
\pgfsetstrokecolor{currentstroke}%
\pgfsetdash{}{0pt}%
\pgfpathmoveto{\pgfqpoint{4.707879in}{6.913270in}}%
\pgfpathlineto{\pgfqpoint{4.749545in}{6.996603in}}%
\pgfpathlineto{\pgfqpoint{4.666212in}{6.996603in}}%
\pgfpathclose%
\pgfusepath{stroke,fill}%
\end{pgfscope}%
\begin{pgfscope}%
\pgfpathrectangle{\pgfqpoint{0.800000in}{0.528000in}}{\pgfqpoint{4.960000in}{3.696000in}} %
\pgfusepath{clip}%
\pgfsetbuttcap%
\pgfsetroundjoin%
\definecolor{currentfill}{rgb}{1.000000,0.647059,0.000000}%
\pgfsetfillcolor{currentfill}%
\pgfsetlinewidth{1.003750pt}%
\definecolor{currentstroke}{rgb}{1.000000,0.647059,0.000000}%
\pgfsetstrokecolor{currentstroke}%
\pgfsetdash{}{0pt}%
\pgfpathmoveto{\pgfqpoint{4.757980in}{3.965022in}}%
\pgfpathlineto{\pgfqpoint{4.799646in}{4.048355in}}%
\pgfpathlineto{\pgfqpoint{4.716313in}{4.048355in}}%
\pgfpathclose%
\pgfusepath{stroke,fill}%
\end{pgfscope}%
\begin{pgfscope}%
\pgfpathrectangle{\pgfqpoint{0.800000in}{0.528000in}}{\pgfqpoint{4.960000in}{3.696000in}} %
\pgfusepath{clip}%
\pgfsetbuttcap%
\pgfsetroundjoin%
\definecolor{currentfill}{rgb}{1.000000,0.647059,0.000000}%
\pgfsetfillcolor{currentfill}%
\pgfsetlinewidth{1.003750pt}%
\definecolor{currentstroke}{rgb}{1.000000,0.647059,0.000000}%
\pgfsetstrokecolor{currentstroke}%
\pgfsetdash{}{0pt}%
\pgfpathmoveto{\pgfqpoint{4.808081in}{4.176958in}}%
\pgfpathlineto{\pgfqpoint{4.849747in}{4.260292in}}%
\pgfpathlineto{\pgfqpoint{4.766414in}{4.260292in}}%
\pgfpathclose%
\pgfusepath{stroke,fill}%
\end{pgfscope}%
\begin{pgfscope}%
\pgfpathrectangle{\pgfqpoint{0.800000in}{0.528000in}}{\pgfqpoint{4.960000in}{3.696000in}} %
\pgfusepath{clip}%
\pgfsetbuttcap%
\pgfsetroundjoin%
\definecolor{currentfill}{rgb}{1.000000,0.647059,0.000000}%
\pgfsetfillcolor{currentfill}%
\pgfsetlinewidth{1.003750pt}%
\definecolor{currentstroke}{rgb}{1.000000,0.647059,0.000000}%
\pgfsetstrokecolor{currentstroke}%
\pgfsetdash{}{0pt}%
\pgfpathmoveto{\pgfqpoint{4.858182in}{5.430264in}}%
\pgfpathlineto{\pgfqpoint{4.899848in}{5.513598in}}%
\pgfpathlineto{\pgfqpoint{4.816515in}{5.513598in}}%
\pgfpathclose%
\pgfusepath{stroke,fill}%
\end{pgfscope}%
\begin{pgfscope}%
\pgfpathrectangle{\pgfqpoint{0.800000in}{0.528000in}}{\pgfqpoint{4.960000in}{3.696000in}} %
\pgfusepath{clip}%
\pgfsetbuttcap%
\pgfsetroundjoin%
\definecolor{currentfill}{rgb}{1.000000,0.647059,0.000000}%
\pgfsetfillcolor{currentfill}%
\pgfsetlinewidth{1.003750pt}%
\definecolor{currentstroke}{rgb}{1.000000,0.647059,0.000000}%
\pgfsetstrokecolor{currentstroke}%
\pgfsetdash{}{0pt}%
\pgfpathmoveto{\pgfqpoint{4.908283in}{8.398532in}}%
\pgfpathlineto{\pgfqpoint{4.949949in}{8.481865in}}%
\pgfpathlineto{\pgfqpoint{4.866616in}{8.481865in}}%
\pgfpathclose%
\pgfusepath{stroke,fill}%
\end{pgfscope}%
\begin{pgfscope}%
\pgfpathrectangle{\pgfqpoint{0.800000in}{0.528000in}}{\pgfqpoint{4.960000in}{3.696000in}} %
\pgfusepath{clip}%
\pgfsetbuttcap%
\pgfsetroundjoin%
\definecolor{currentfill}{rgb}{1.000000,0.647059,0.000000}%
\pgfsetfillcolor{currentfill}%
\pgfsetlinewidth{1.003750pt}%
\definecolor{currentstroke}{rgb}{1.000000,0.647059,0.000000}%
\pgfsetstrokecolor{currentstroke}%
\pgfsetdash{}{0pt}%
\pgfpathmoveto{\pgfqpoint{4.958384in}{5.011813in}}%
\pgfpathlineto{\pgfqpoint{5.000051in}{5.095146in}}%
\pgfpathlineto{\pgfqpoint{4.916717in}{5.095146in}}%
\pgfpathclose%
\pgfusepath{stroke,fill}%
\end{pgfscope}%
\begin{pgfscope}%
\pgfpathrectangle{\pgfqpoint{0.800000in}{0.528000in}}{\pgfqpoint{4.960000in}{3.696000in}} %
\pgfusepath{clip}%
\pgfsetbuttcap%
\pgfsetroundjoin%
\definecolor{currentfill}{rgb}{1.000000,0.647059,0.000000}%
\pgfsetfillcolor{currentfill}%
\pgfsetlinewidth{1.003750pt}%
\definecolor{currentstroke}{rgb}{1.000000,0.647059,0.000000}%
\pgfsetstrokecolor{currentstroke}%
\pgfsetdash{}{0pt}%
\pgfpathmoveto{\pgfqpoint{5.008485in}{6.003844in}}%
\pgfpathlineto{\pgfqpoint{5.050152in}{6.087177in}}%
\pgfpathlineto{\pgfqpoint{4.966818in}{6.087177in}}%
\pgfpathclose%
\pgfusepath{stroke,fill}%
\end{pgfscope}%
\begin{pgfscope}%
\pgfpathrectangle{\pgfqpoint{0.800000in}{0.528000in}}{\pgfqpoint{4.960000in}{3.696000in}} %
\pgfusepath{clip}%
\pgfsetbuttcap%
\pgfsetroundjoin%
\definecolor{currentfill}{rgb}{1.000000,0.647059,0.000000}%
\pgfsetfillcolor{currentfill}%
\pgfsetlinewidth{1.003750pt}%
\definecolor{currentstroke}{rgb}{1.000000,0.647059,0.000000}%
\pgfsetstrokecolor{currentstroke}%
\pgfsetdash{}{0pt}%
\pgfpathmoveto{\pgfqpoint{5.058586in}{7.089155in}}%
\pgfpathlineto{\pgfqpoint{5.100253in}{7.172489in}}%
\pgfpathlineto{\pgfqpoint{5.016919in}{7.172489in}}%
\pgfpathclose%
\pgfusepath{stroke,fill}%
\end{pgfscope}%
\begin{pgfscope}%
\pgfpathrectangle{\pgfqpoint{0.800000in}{0.528000in}}{\pgfqpoint{4.960000in}{3.696000in}} %
\pgfusepath{clip}%
\pgfsetbuttcap%
\pgfsetroundjoin%
\definecolor{currentfill}{rgb}{1.000000,0.647059,0.000000}%
\pgfsetfillcolor{currentfill}%
\pgfsetlinewidth{1.003750pt}%
\definecolor{currentstroke}{rgb}{1.000000,0.647059,0.000000}%
\pgfsetstrokecolor{currentstroke}%
\pgfsetdash{}{0pt}%
\pgfpathmoveto{\pgfqpoint{5.108687in}{5.526080in}}%
\pgfpathlineto{\pgfqpoint{5.150354in}{5.609414in}}%
\pgfpathlineto{\pgfqpoint{5.067020in}{5.609414in}}%
\pgfpathclose%
\pgfusepath{stroke,fill}%
\end{pgfscope}%
\begin{pgfscope}%
\pgfpathrectangle{\pgfqpoint{0.800000in}{0.528000in}}{\pgfqpoint{4.960000in}{3.696000in}} %
\pgfusepath{clip}%
\pgfsetbuttcap%
\pgfsetroundjoin%
\definecolor{currentfill}{rgb}{1.000000,0.647059,0.000000}%
\pgfsetfillcolor{currentfill}%
\pgfsetlinewidth{1.003750pt}%
\definecolor{currentstroke}{rgb}{1.000000,0.647059,0.000000}%
\pgfsetstrokecolor{currentstroke}%
\pgfsetdash{}{0pt}%
\pgfpathmoveto{\pgfqpoint{5.158788in}{7.730654in}}%
\pgfpathlineto{\pgfqpoint{5.200455in}{7.813988in}}%
\pgfpathlineto{\pgfqpoint{5.117121in}{7.813988in}}%
\pgfpathclose%
\pgfusepath{stroke,fill}%
\end{pgfscope}%
\begin{pgfscope}%
\pgfpathrectangle{\pgfqpoint{0.800000in}{0.528000in}}{\pgfqpoint{4.960000in}{3.696000in}} %
\pgfusepath{clip}%
\pgfsetbuttcap%
\pgfsetroundjoin%
\definecolor{currentfill}{rgb}{1.000000,0.647059,0.000000}%
\pgfsetfillcolor{currentfill}%
\pgfsetlinewidth{1.003750pt}%
\definecolor{currentstroke}{rgb}{1.000000,0.647059,0.000000}%
\pgfsetstrokecolor{currentstroke}%
\pgfsetdash{}{0pt}%
\pgfpathmoveto{\pgfqpoint{5.208889in}{5.441664in}}%
\pgfpathlineto{\pgfqpoint{5.250556in}{5.524997in}}%
\pgfpathlineto{\pgfqpoint{5.167222in}{5.524997in}}%
\pgfpathclose%
\pgfusepath{stroke,fill}%
\end{pgfscope}%
\begin{pgfscope}%
\pgfpathrectangle{\pgfqpoint{0.800000in}{0.528000in}}{\pgfqpoint{4.960000in}{3.696000in}} %
\pgfusepath{clip}%
\pgfsetbuttcap%
\pgfsetroundjoin%
\definecolor{currentfill}{rgb}{1.000000,0.647059,0.000000}%
\pgfsetfillcolor{currentfill}%
\pgfsetlinewidth{1.003750pt}%
\definecolor{currentstroke}{rgb}{1.000000,0.647059,0.000000}%
\pgfsetstrokecolor{currentstroke}%
\pgfsetdash{}{0pt}%
\pgfpathmoveto{\pgfqpoint{5.258990in}{7.641787in}}%
\pgfpathlineto{\pgfqpoint{5.300657in}{7.725120in}}%
\pgfpathlineto{\pgfqpoint{5.217323in}{7.725120in}}%
\pgfpathclose%
\pgfusepath{stroke,fill}%
\end{pgfscope}%
\begin{pgfscope}%
\pgfpathrectangle{\pgfqpoint{0.800000in}{0.528000in}}{\pgfqpoint{4.960000in}{3.696000in}} %
\pgfusepath{clip}%
\pgfsetbuttcap%
\pgfsetroundjoin%
\definecolor{currentfill}{rgb}{1.000000,0.647059,0.000000}%
\pgfsetfillcolor{currentfill}%
\pgfsetlinewidth{1.003750pt}%
\definecolor{currentstroke}{rgb}{1.000000,0.647059,0.000000}%
\pgfsetstrokecolor{currentstroke}%
\pgfsetdash{}{0pt}%
\pgfpathmoveto{\pgfqpoint{5.309091in}{4.198569in}}%
\pgfpathlineto{\pgfqpoint{5.350758in}{4.281902in}}%
\pgfpathlineto{\pgfqpoint{5.267424in}{4.281902in}}%
\pgfpathclose%
\pgfusepath{stroke,fill}%
\end{pgfscope}%
\begin{pgfscope}%
\pgfpathrectangle{\pgfqpoint{0.800000in}{0.528000in}}{\pgfqpoint{4.960000in}{3.696000in}} %
\pgfusepath{clip}%
\pgfsetbuttcap%
\pgfsetroundjoin%
\definecolor{currentfill}{rgb}{1.000000,0.647059,0.000000}%
\pgfsetfillcolor{currentfill}%
\pgfsetlinewidth{1.003750pt}%
\definecolor{currentstroke}{rgb}{1.000000,0.647059,0.000000}%
\pgfsetstrokecolor{currentstroke}%
\pgfsetdash{}{0pt}%
\pgfpathmoveto{\pgfqpoint{5.359192in}{6.715156in}}%
\pgfpathlineto{\pgfqpoint{5.400859in}{6.798489in}}%
\pgfpathlineto{\pgfqpoint{5.317525in}{6.798489in}}%
\pgfpathclose%
\pgfusepath{stroke,fill}%
\end{pgfscope}%
\begin{pgfscope}%
\pgfpathrectangle{\pgfqpoint{0.800000in}{0.528000in}}{\pgfqpoint{4.960000in}{3.696000in}} %
\pgfusepath{clip}%
\pgfsetbuttcap%
\pgfsetroundjoin%
\definecolor{currentfill}{rgb}{1.000000,0.647059,0.000000}%
\pgfsetfillcolor{currentfill}%
\pgfsetlinewidth{1.003750pt}%
\definecolor{currentstroke}{rgb}{1.000000,0.647059,0.000000}%
\pgfsetstrokecolor{currentstroke}%
\pgfsetdash{}{0pt}%
\pgfpathmoveto{\pgfqpoint{5.409293in}{4.423664in}}%
\pgfpathlineto{\pgfqpoint{5.450960in}{4.506998in}}%
\pgfpathlineto{\pgfqpoint{5.367626in}{4.506998in}}%
\pgfpathclose%
\pgfusepath{stroke,fill}%
\end{pgfscope}%
\begin{pgfscope}%
\pgfpathrectangle{\pgfqpoint{0.800000in}{0.528000in}}{\pgfqpoint{4.960000in}{3.696000in}} %
\pgfusepath{clip}%
\pgfsetbuttcap%
\pgfsetroundjoin%
\definecolor{currentfill}{rgb}{1.000000,0.647059,0.000000}%
\pgfsetfillcolor{currentfill}%
\pgfsetlinewidth{1.003750pt}%
\definecolor{currentstroke}{rgb}{1.000000,0.647059,0.000000}%
\pgfsetstrokecolor{currentstroke}%
\pgfsetdash{}{0pt}%
\pgfpathmoveto{\pgfqpoint{5.459394in}{4.922161in}}%
\pgfpathlineto{\pgfqpoint{5.501061in}{5.005494in}}%
\pgfpathlineto{\pgfqpoint{5.417727in}{5.005494in}}%
\pgfpathclose%
\pgfusepath{stroke,fill}%
\end{pgfscope}%
\begin{pgfscope}%
\pgfpathrectangle{\pgfqpoint{0.800000in}{0.528000in}}{\pgfqpoint{4.960000in}{3.696000in}} %
\pgfusepath{clip}%
\pgfsetbuttcap%
\pgfsetroundjoin%
\definecolor{currentfill}{rgb}{1.000000,0.647059,0.000000}%
\pgfsetfillcolor{currentfill}%
\pgfsetlinewidth{1.003750pt}%
\definecolor{currentstroke}{rgb}{1.000000,0.647059,0.000000}%
\pgfsetstrokecolor{currentstroke}%
\pgfsetdash{}{0pt}%
\pgfpathmoveto{\pgfqpoint{5.509495in}{8.781688in}}%
\pgfpathlineto{\pgfqpoint{5.551162in}{8.865022in}}%
\pgfpathlineto{\pgfqpoint{5.467828in}{8.865022in}}%
\pgfpathclose%
\pgfusepath{stroke,fill}%
\end{pgfscope}%
\begin{pgfscope}%
\pgfpathrectangle{\pgfqpoint{0.800000in}{0.528000in}}{\pgfqpoint{4.960000in}{3.696000in}} %
\pgfusepath{clip}%
\pgfsetbuttcap%
\pgfsetroundjoin%
\definecolor{currentfill}{rgb}{1.000000,0.647059,0.000000}%
\pgfsetfillcolor{currentfill}%
\pgfsetlinewidth{1.003750pt}%
\definecolor{currentstroke}{rgb}{1.000000,0.647059,0.000000}%
\pgfsetstrokecolor{currentstroke}%
\pgfsetdash{}{0pt}%
\pgfpathmoveto{\pgfqpoint{5.559596in}{5.492128in}}%
\pgfpathlineto{\pgfqpoint{5.601263in}{5.575461in}}%
\pgfpathlineto{\pgfqpoint{5.517929in}{5.575461in}}%
\pgfpathclose%
\pgfusepath{stroke,fill}%
\end{pgfscope}%
\begin{pgfscope}%
\pgfpathrectangle{\pgfqpoint{0.800000in}{0.528000in}}{\pgfqpoint{4.960000in}{3.696000in}} %
\pgfusepath{clip}%
\pgfsetbuttcap%
\pgfsetroundjoin%
\definecolor{currentfill}{rgb}{1.000000,0.647059,0.000000}%
\pgfsetfillcolor{currentfill}%
\pgfsetlinewidth{1.003750pt}%
\definecolor{currentstroke}{rgb}{1.000000,0.647059,0.000000}%
\pgfsetstrokecolor{currentstroke}%
\pgfsetdash{}{0pt}%
\pgfpathmoveto{\pgfqpoint{5.609697in}{5.211182in}}%
\pgfpathlineto{\pgfqpoint{5.651364in}{5.294515in}}%
\pgfpathlineto{\pgfqpoint{5.568030in}{5.294515in}}%
\pgfpathclose%
\pgfusepath{stroke,fill}%
\end{pgfscope}%
\begin{pgfscope}%
\pgfpathrectangle{\pgfqpoint{0.800000in}{0.528000in}}{\pgfqpoint{4.960000in}{3.696000in}} %
\pgfusepath{clip}%
\pgfsetbuttcap%
\pgfsetroundjoin%
\definecolor{currentfill}{rgb}{1.000000,0.647059,0.000000}%
\pgfsetfillcolor{currentfill}%
\pgfsetlinewidth{1.003750pt}%
\definecolor{currentstroke}{rgb}{1.000000,0.647059,0.000000}%
\pgfsetstrokecolor{currentstroke}%
\pgfsetdash{}{0pt}%
\pgfpathmoveto{\pgfqpoint{5.659798in}{6.999419in}}%
\pgfpathlineto{\pgfqpoint{5.701465in}{7.082752in}}%
\pgfpathlineto{\pgfqpoint{5.618131in}{7.082752in}}%
\pgfpathclose%
\pgfusepath{stroke,fill}%
\end{pgfscope}%
\begin{pgfscope}%
\pgfpathrectangle{\pgfqpoint{0.800000in}{0.528000in}}{\pgfqpoint{4.960000in}{3.696000in}} %
\pgfusepath{clip}%
\pgfsetbuttcap%
\pgfsetroundjoin%
\definecolor{currentfill}{rgb}{1.000000,0.647059,0.000000}%
\pgfsetfillcolor{currentfill}%
\pgfsetlinewidth{1.003750pt}%
\definecolor{currentstroke}{rgb}{1.000000,0.647059,0.000000}%
\pgfsetstrokecolor{currentstroke}%
\pgfsetdash{}{0pt}%
\pgfpathmoveto{\pgfqpoint{5.709899in}{6.584428in}}%
\pgfpathlineto{\pgfqpoint{5.751566in}{6.667761in}}%
\pgfpathlineto{\pgfqpoint{5.668232in}{6.667761in}}%
\pgfpathclose%
\pgfusepath{stroke,fill}%
\end{pgfscope}%
\begin{pgfscope}%
\pgfpathrectangle{\pgfqpoint{0.800000in}{0.528000in}}{\pgfqpoint{4.960000in}{3.696000in}} %
\pgfusepath{clip}%
\pgfsetbuttcap%
\pgfsetroundjoin%
\definecolor{currentfill}{rgb}{1.000000,0.647059,0.000000}%
\pgfsetfillcolor{currentfill}%
\pgfsetlinewidth{1.003750pt}%
\definecolor{currentstroke}{rgb}{1.000000,0.647059,0.000000}%
\pgfsetstrokecolor{currentstroke}%
\pgfsetdash{}{0pt}%
\pgfpathmoveto{\pgfqpoint{5.760000in}{7.412436in}}%
\pgfpathlineto{\pgfqpoint{5.801667in}{7.495770in}}%
\pgfpathlineto{\pgfqpoint{5.718333in}{7.495770in}}%
\pgfpathclose%
\pgfusepath{stroke,fill}%
\end{pgfscope}%
\begin{pgfscope}%
\pgfpathrectangle{\pgfqpoint{0.800000in}{0.528000in}}{\pgfqpoint{4.960000in}{3.696000in}} %
\pgfusepath{clip}%
\pgfsetbuttcap%
\pgfsetroundjoin%
\definecolor{currentfill}{rgb}{0.000000,0.500000,0.000000}%
\pgfsetfillcolor{currentfill}%
\pgfsetlinewidth{1.003750pt}%
\definecolor{currentstroke}{rgb}{0.000000,0.500000,0.000000}%
\pgfsetstrokecolor{currentstroke}%
\pgfsetdash{}{0pt}%
\pgfpathmoveto{\pgfqpoint{0.800000in}{-1.187955in}}%
\pgfpathcurveto{\pgfqpoint{0.811050in}{-1.187955in}}{\pgfqpoint{0.821649in}{-1.183564in}}{\pgfqpoint{0.829463in}{-1.175751in}}%
\pgfpathcurveto{\pgfqpoint{0.837276in}{-1.167937in}}{\pgfqpoint{0.841667in}{-1.157338in}}{\pgfqpoint{0.841667in}{-1.146288in}}%
\pgfpathcurveto{\pgfqpoint{0.841667in}{-1.135238in}}{\pgfqpoint{0.837276in}{-1.124639in}}{\pgfqpoint{0.829463in}{-1.116825in}}%
\pgfpathcurveto{\pgfqpoint{0.821649in}{-1.109012in}}{\pgfqpoint{0.811050in}{-1.104621in}}{\pgfqpoint{0.800000in}{-1.104621in}}%
\pgfpathcurveto{\pgfqpoint{0.788950in}{-1.104621in}}{\pgfqpoint{0.778351in}{-1.109012in}}{\pgfqpoint{0.770537in}{-1.116825in}}%
\pgfpathcurveto{\pgfqpoint{0.762724in}{-1.124639in}}{\pgfqpoint{0.758333in}{-1.135238in}}{\pgfqpoint{0.758333in}{-1.146288in}}%
\pgfpathcurveto{\pgfqpoint{0.758333in}{-1.157338in}}{\pgfqpoint{0.762724in}{-1.167937in}}{\pgfqpoint{0.770537in}{-1.175751in}}%
\pgfpathcurveto{\pgfqpoint{0.778351in}{-1.183564in}}{\pgfqpoint{0.788950in}{-1.187955in}}{\pgfqpoint{0.800000in}{-1.187955in}}%
\pgfpathclose%
\pgfusepath{stroke,fill}%
\end{pgfscope}%
\begin{pgfscope}%
\pgfpathrectangle{\pgfqpoint{0.800000in}{0.528000in}}{\pgfqpoint{4.960000in}{3.696000in}} %
\pgfusepath{clip}%
\pgfsetbuttcap%
\pgfsetroundjoin%
\definecolor{currentfill}{rgb}{0.000000,0.500000,0.000000}%
\pgfsetfillcolor{currentfill}%
\pgfsetlinewidth{1.003750pt}%
\definecolor{currentstroke}{rgb}{0.000000,0.500000,0.000000}%
\pgfsetstrokecolor{currentstroke}%
\pgfsetdash{}{0pt}%
\pgfpathmoveto{\pgfqpoint{0.850101in}{-1.652354in}}%
\pgfpathcurveto{\pgfqpoint{0.861151in}{-1.652354in}}{\pgfqpoint{0.871750in}{-1.647964in}}{\pgfqpoint{0.879564in}{-1.640150in}}%
\pgfpathcurveto{\pgfqpoint{0.887377in}{-1.632337in}}{\pgfqpoint{0.891768in}{-1.621738in}}{\pgfqpoint{0.891768in}{-1.610687in}}%
\pgfpathcurveto{\pgfqpoint{0.891768in}{-1.599637in}}{\pgfqpoint{0.887377in}{-1.589038in}}{\pgfqpoint{0.879564in}{-1.581225in}}%
\pgfpathcurveto{\pgfqpoint{0.871750in}{-1.573411in}}{\pgfqpoint{0.861151in}{-1.569021in}}{\pgfqpoint{0.850101in}{-1.569021in}}%
\pgfpathcurveto{\pgfqpoint{0.839051in}{-1.569021in}}{\pgfqpoint{0.828452in}{-1.573411in}}{\pgfqpoint{0.820638in}{-1.581225in}}%
\pgfpathcurveto{\pgfqpoint{0.812825in}{-1.589038in}}{\pgfqpoint{0.808434in}{-1.599637in}}{\pgfqpoint{0.808434in}{-1.610687in}}%
\pgfpathcurveto{\pgfqpoint{0.808434in}{-1.621738in}}{\pgfqpoint{0.812825in}{-1.632337in}}{\pgfqpoint{0.820638in}{-1.640150in}}%
\pgfpathcurveto{\pgfqpoint{0.828452in}{-1.647964in}}{\pgfqpoint{0.839051in}{-1.652354in}}{\pgfqpoint{0.850101in}{-1.652354in}}%
\pgfpathclose%
\pgfusepath{stroke,fill}%
\end{pgfscope}%
\begin{pgfscope}%
\pgfpathrectangle{\pgfqpoint{0.800000in}{0.528000in}}{\pgfqpoint{4.960000in}{3.696000in}} %
\pgfusepath{clip}%
\pgfsetbuttcap%
\pgfsetroundjoin%
\definecolor{currentfill}{rgb}{0.000000,0.500000,0.000000}%
\pgfsetfillcolor{currentfill}%
\pgfsetlinewidth{1.003750pt}%
\definecolor{currentstroke}{rgb}{0.000000,0.500000,0.000000}%
\pgfsetstrokecolor{currentstroke}%
\pgfsetdash{}{0pt}%
\pgfpathmoveto{\pgfqpoint{0.900202in}{-0.741881in}}%
\pgfpathcurveto{\pgfqpoint{0.911252in}{-0.741881in}}{\pgfqpoint{0.921851in}{-0.737491in}}{\pgfqpoint{0.929665in}{-0.729677in}}%
\pgfpathcurveto{\pgfqpoint{0.937478in}{-0.721864in}}{\pgfqpoint{0.941869in}{-0.711265in}}{\pgfqpoint{0.941869in}{-0.700214in}}%
\pgfpathcurveto{\pgfqpoint{0.941869in}{-0.689164in}}{\pgfqpoint{0.937478in}{-0.678565in}}{\pgfqpoint{0.929665in}{-0.670752in}}%
\pgfpathcurveto{\pgfqpoint{0.921851in}{-0.662938in}}{\pgfqpoint{0.911252in}{-0.658548in}}{\pgfqpoint{0.900202in}{-0.658548in}}%
\pgfpathcurveto{\pgfqpoint{0.889152in}{-0.658548in}}{\pgfqpoint{0.878553in}{-0.662938in}}{\pgfqpoint{0.870739in}{-0.670752in}}%
\pgfpathcurveto{\pgfqpoint{0.862926in}{-0.678565in}}{\pgfqpoint{0.858535in}{-0.689164in}}{\pgfqpoint{0.858535in}{-0.700214in}}%
\pgfpathcurveto{\pgfqpoint{0.858535in}{-0.711265in}}{\pgfqpoint{0.862926in}{-0.721864in}}{\pgfqpoint{0.870739in}{-0.729677in}}%
\pgfpathcurveto{\pgfqpoint{0.878553in}{-0.737491in}}{\pgfqpoint{0.889152in}{-0.741881in}}{\pgfqpoint{0.900202in}{-0.741881in}}%
\pgfpathclose%
\pgfusepath{stroke,fill}%
\end{pgfscope}%
\begin{pgfscope}%
\pgfpathrectangle{\pgfqpoint{0.800000in}{0.528000in}}{\pgfqpoint{4.960000in}{3.696000in}} %
\pgfusepath{clip}%
\pgfsetbuttcap%
\pgfsetroundjoin%
\definecolor{currentfill}{rgb}{0.000000,0.500000,0.000000}%
\pgfsetfillcolor{currentfill}%
\pgfsetlinewidth{1.003750pt}%
\definecolor{currentstroke}{rgb}{0.000000,0.500000,0.000000}%
\pgfsetstrokecolor{currentstroke}%
\pgfsetdash{}{0pt}%
\pgfpathmoveto{\pgfqpoint{0.950303in}{-2.603040in}}%
\pgfpathcurveto{\pgfqpoint{0.961353in}{-2.603040in}}{\pgfqpoint{0.971952in}{-2.598649in}}{\pgfqpoint{0.979766in}{-2.590836in}}%
\pgfpathcurveto{\pgfqpoint{0.987579in}{-2.583022in}}{\pgfqpoint{0.991970in}{-2.572423in}}{\pgfqpoint{0.991970in}{-2.561373in}}%
\pgfpathcurveto{\pgfqpoint{0.991970in}{-2.550323in}}{\pgfqpoint{0.987579in}{-2.539724in}}{\pgfqpoint{0.979766in}{-2.531910in}}%
\pgfpathcurveto{\pgfqpoint{0.971952in}{-2.524096in}}{\pgfqpoint{0.961353in}{-2.519706in}}{\pgfqpoint{0.950303in}{-2.519706in}}%
\pgfpathcurveto{\pgfqpoint{0.939253in}{-2.519706in}}{\pgfqpoint{0.928654in}{-2.524096in}}{\pgfqpoint{0.920840in}{-2.531910in}}%
\pgfpathcurveto{\pgfqpoint{0.913027in}{-2.539724in}}{\pgfqpoint{0.908636in}{-2.550323in}}{\pgfqpoint{0.908636in}{-2.561373in}}%
\pgfpathcurveto{\pgfqpoint{0.908636in}{-2.572423in}}{\pgfqpoint{0.913027in}{-2.583022in}}{\pgfqpoint{0.920840in}{-2.590836in}}%
\pgfpathcurveto{\pgfqpoint{0.928654in}{-2.598649in}}{\pgfqpoint{0.939253in}{-2.603040in}}{\pgfqpoint{0.950303in}{-2.603040in}}%
\pgfpathclose%
\pgfusepath{stroke,fill}%
\end{pgfscope}%
\begin{pgfscope}%
\pgfpathrectangle{\pgfqpoint{0.800000in}{0.528000in}}{\pgfqpoint{4.960000in}{3.696000in}} %
\pgfusepath{clip}%
\pgfsetbuttcap%
\pgfsetroundjoin%
\definecolor{currentfill}{rgb}{0.000000,0.500000,0.000000}%
\pgfsetfillcolor{currentfill}%
\pgfsetlinewidth{1.003750pt}%
\definecolor{currentstroke}{rgb}{0.000000,0.500000,0.000000}%
\pgfsetstrokecolor{currentstroke}%
\pgfsetdash{}{0pt}%
\pgfpathmoveto{\pgfqpoint{1.000404in}{-1.063397in}}%
\pgfpathcurveto{\pgfqpoint{1.011454in}{-1.063397in}}{\pgfqpoint{1.022053in}{-1.059007in}}{\pgfqpoint{1.029867in}{-1.051193in}}%
\pgfpathcurveto{\pgfqpoint{1.037680in}{-1.043380in}}{\pgfqpoint{1.042071in}{-1.032781in}}{\pgfqpoint{1.042071in}{-1.021731in}}%
\pgfpathcurveto{\pgfqpoint{1.042071in}{-1.010681in}}{\pgfqpoint{1.037680in}{-1.000081in}}{\pgfqpoint{1.029867in}{-0.992268in}}%
\pgfpathcurveto{\pgfqpoint{1.022053in}{-0.984454in}}{\pgfqpoint{1.011454in}{-0.980064in}}{\pgfqpoint{1.000404in}{-0.980064in}}%
\pgfpathcurveto{\pgfqpoint{0.989354in}{-0.980064in}}{\pgfqpoint{0.978755in}{-0.984454in}}{\pgfqpoint{0.970941in}{-0.992268in}}%
\pgfpathcurveto{\pgfqpoint{0.963128in}{-1.000081in}}{\pgfqpoint{0.958737in}{-1.010681in}}{\pgfqpoint{0.958737in}{-1.021731in}}%
\pgfpathcurveto{\pgfqpoint{0.958737in}{-1.032781in}}{\pgfqpoint{0.963128in}{-1.043380in}}{\pgfqpoint{0.970941in}{-1.051193in}}%
\pgfpathcurveto{\pgfqpoint{0.978755in}{-1.059007in}}{\pgfqpoint{0.989354in}{-1.063397in}}{\pgfqpoint{1.000404in}{-1.063397in}}%
\pgfpathclose%
\pgfusepath{stroke,fill}%
\end{pgfscope}%
\begin{pgfscope}%
\pgfpathrectangle{\pgfqpoint{0.800000in}{0.528000in}}{\pgfqpoint{4.960000in}{3.696000in}} %
\pgfusepath{clip}%
\pgfsetbuttcap%
\pgfsetroundjoin%
\definecolor{currentfill}{rgb}{0.000000,0.500000,0.000000}%
\pgfsetfillcolor{currentfill}%
\pgfsetlinewidth{1.003750pt}%
\definecolor{currentstroke}{rgb}{0.000000,0.500000,0.000000}%
\pgfsetstrokecolor{currentstroke}%
\pgfsetdash{}{0pt}%
\pgfpathmoveto{\pgfqpoint{1.050505in}{-2.106921in}}%
\pgfpathcurveto{\pgfqpoint{1.061555in}{-2.106921in}}{\pgfqpoint{1.072154in}{-2.102531in}}{\pgfqpoint{1.079968in}{-2.094718in}}%
\pgfpathcurveto{\pgfqpoint{1.087781in}{-2.086904in}}{\pgfqpoint{1.092172in}{-2.076305in}}{\pgfqpoint{1.092172in}{-2.065255in}}%
\pgfpathcurveto{\pgfqpoint{1.092172in}{-2.054205in}}{\pgfqpoint{1.087781in}{-2.043606in}}{\pgfqpoint{1.079968in}{-2.035792in}}%
\pgfpathcurveto{\pgfqpoint{1.072154in}{-2.027978in}}{\pgfqpoint{1.061555in}{-2.023588in}}{\pgfqpoint{1.050505in}{-2.023588in}}%
\pgfpathcurveto{\pgfqpoint{1.039455in}{-2.023588in}}{\pgfqpoint{1.028856in}{-2.027978in}}{\pgfqpoint{1.021042in}{-2.035792in}}%
\pgfpathcurveto{\pgfqpoint{1.013229in}{-2.043606in}}{\pgfqpoint{1.008838in}{-2.054205in}}{\pgfqpoint{1.008838in}{-2.065255in}}%
\pgfpathcurveto{\pgfqpoint{1.008838in}{-2.076305in}}{\pgfqpoint{1.013229in}{-2.086904in}}{\pgfqpoint{1.021042in}{-2.094718in}}%
\pgfpathcurveto{\pgfqpoint{1.028856in}{-2.102531in}}{\pgfqpoint{1.039455in}{-2.106921in}}{\pgfqpoint{1.050505in}{-2.106921in}}%
\pgfpathclose%
\pgfusepath{stroke,fill}%
\end{pgfscope}%
\begin{pgfscope}%
\pgfpathrectangle{\pgfqpoint{0.800000in}{0.528000in}}{\pgfqpoint{4.960000in}{3.696000in}} %
\pgfusepath{clip}%
\pgfsetbuttcap%
\pgfsetroundjoin%
\definecolor{currentfill}{rgb}{0.000000,0.500000,0.000000}%
\pgfsetfillcolor{currentfill}%
\pgfsetlinewidth{1.003750pt}%
\definecolor{currentstroke}{rgb}{0.000000,0.500000,0.000000}%
\pgfsetstrokecolor{currentstroke}%
\pgfsetdash{}{0pt}%
\pgfpathmoveto{\pgfqpoint{1.100606in}{-0.291055in}}%
\pgfpathcurveto{\pgfqpoint{1.111656in}{-0.291055in}}{\pgfqpoint{1.122255in}{-0.286665in}}{\pgfqpoint{1.130069in}{-0.278851in}}%
\pgfpathcurveto{\pgfqpoint{1.137882in}{-0.271038in}}{\pgfqpoint{1.142273in}{-0.260439in}}{\pgfqpoint{1.142273in}{-0.249389in}}%
\pgfpathcurveto{\pgfqpoint{1.142273in}{-0.238338in}}{\pgfqpoint{1.137882in}{-0.227739in}}{\pgfqpoint{1.130069in}{-0.219926in}}%
\pgfpathcurveto{\pgfqpoint{1.122255in}{-0.212112in}}{\pgfqpoint{1.111656in}{-0.207722in}}{\pgfqpoint{1.100606in}{-0.207722in}}%
\pgfpathcurveto{\pgfqpoint{1.089556in}{-0.207722in}}{\pgfqpoint{1.078957in}{-0.212112in}}{\pgfqpoint{1.071143in}{-0.219926in}}%
\pgfpathcurveto{\pgfqpoint{1.063330in}{-0.227739in}}{\pgfqpoint{1.058939in}{-0.238338in}}{\pgfqpoint{1.058939in}{-0.249389in}}%
\pgfpathcurveto{\pgfqpoint{1.058939in}{-0.260439in}}{\pgfqpoint{1.063330in}{-0.271038in}}{\pgfqpoint{1.071143in}{-0.278851in}}%
\pgfpathcurveto{\pgfqpoint{1.078957in}{-0.286665in}}{\pgfqpoint{1.089556in}{-0.291055in}}{\pgfqpoint{1.100606in}{-0.291055in}}%
\pgfpathclose%
\pgfusepath{stroke,fill}%
\end{pgfscope}%
\begin{pgfscope}%
\pgfpathrectangle{\pgfqpoint{0.800000in}{0.528000in}}{\pgfqpoint{4.960000in}{3.696000in}} %
\pgfusepath{clip}%
\pgfsetbuttcap%
\pgfsetroundjoin%
\definecolor{currentfill}{rgb}{0.000000,0.500000,0.000000}%
\pgfsetfillcolor{currentfill}%
\pgfsetlinewidth{1.003750pt}%
\definecolor{currentstroke}{rgb}{0.000000,0.500000,0.000000}%
\pgfsetstrokecolor{currentstroke}%
\pgfsetdash{}{0pt}%
\pgfpathmoveto{\pgfqpoint{1.150707in}{-1.030032in}}%
\pgfpathcurveto{\pgfqpoint{1.161757in}{-1.030032in}}{\pgfqpoint{1.172356in}{-1.025642in}}{\pgfqpoint{1.180170in}{-1.017828in}}%
\pgfpathcurveto{\pgfqpoint{1.187983in}{-1.010015in}}{\pgfqpoint{1.192374in}{-0.999416in}}{\pgfqpoint{1.192374in}{-0.988366in}}%
\pgfpathcurveto{\pgfqpoint{1.192374in}{-0.977316in}}{\pgfqpoint{1.187983in}{-0.966716in}}{\pgfqpoint{1.180170in}{-0.958903in}}%
\pgfpathcurveto{\pgfqpoint{1.172356in}{-0.951089in}}{\pgfqpoint{1.161757in}{-0.946699in}}{\pgfqpoint{1.150707in}{-0.946699in}}%
\pgfpathcurveto{\pgfqpoint{1.139657in}{-0.946699in}}{\pgfqpoint{1.129058in}{-0.951089in}}{\pgfqpoint{1.121244in}{-0.958903in}}%
\pgfpathcurveto{\pgfqpoint{1.113431in}{-0.966716in}}{\pgfqpoint{1.109040in}{-0.977316in}}{\pgfqpoint{1.109040in}{-0.988366in}}%
\pgfpathcurveto{\pgfqpoint{1.109040in}{-0.999416in}}{\pgfqpoint{1.113431in}{-1.010015in}}{\pgfqpoint{1.121244in}{-1.017828in}}%
\pgfpathcurveto{\pgfqpoint{1.129058in}{-1.025642in}}{\pgfqpoint{1.139657in}{-1.030032in}}{\pgfqpoint{1.150707in}{-1.030032in}}%
\pgfpathclose%
\pgfusepath{stroke,fill}%
\end{pgfscope}%
\begin{pgfscope}%
\pgfpathrectangle{\pgfqpoint{0.800000in}{0.528000in}}{\pgfqpoint{4.960000in}{3.696000in}} %
\pgfusepath{clip}%
\pgfsetbuttcap%
\pgfsetroundjoin%
\definecolor{currentfill}{rgb}{0.000000,0.500000,0.000000}%
\pgfsetfillcolor{currentfill}%
\pgfsetlinewidth{1.003750pt}%
\definecolor{currentstroke}{rgb}{0.000000,0.500000,0.000000}%
\pgfsetstrokecolor{currentstroke}%
\pgfsetdash{}{0pt}%
\pgfpathmoveto{\pgfqpoint{1.200808in}{-0.041152in}}%
\pgfpathcurveto{\pgfqpoint{1.211858in}{-0.041152in}}{\pgfqpoint{1.222457in}{-0.036761in}}{\pgfqpoint{1.230271in}{-0.028948in}}%
\pgfpathcurveto{\pgfqpoint{1.238084in}{-0.021134in}}{\pgfqpoint{1.242475in}{-0.010535in}}{\pgfqpoint{1.242475in}{0.000515in}}%
\pgfpathcurveto{\pgfqpoint{1.242475in}{0.011565in}}{\pgfqpoint{1.238084in}{0.022164in}}{\pgfqpoint{1.230271in}{0.029978in}}%
\pgfpathcurveto{\pgfqpoint{1.222457in}{0.037791in}}{\pgfqpoint{1.211858in}{0.042182in}}{\pgfqpoint{1.200808in}{0.042182in}}%
\pgfpathcurveto{\pgfqpoint{1.189758in}{0.042182in}}{\pgfqpoint{1.179159in}{0.037791in}}{\pgfqpoint{1.171345in}{0.029978in}}%
\pgfpathcurveto{\pgfqpoint{1.163532in}{0.022164in}}{\pgfqpoint{1.159141in}{0.011565in}}{\pgfqpoint{1.159141in}{0.000515in}}%
\pgfpathcurveto{\pgfqpoint{1.159141in}{-0.010535in}}{\pgfqpoint{1.163532in}{-0.021134in}}{\pgfqpoint{1.171345in}{-0.028948in}}%
\pgfpathcurveto{\pgfqpoint{1.179159in}{-0.036761in}}{\pgfqpoint{1.189758in}{-0.041152in}}{\pgfqpoint{1.200808in}{-0.041152in}}%
\pgfpathclose%
\pgfusepath{stroke,fill}%
\end{pgfscope}%
\begin{pgfscope}%
\pgfpathrectangle{\pgfqpoint{0.800000in}{0.528000in}}{\pgfqpoint{4.960000in}{3.696000in}} %
\pgfusepath{clip}%
\pgfsetbuttcap%
\pgfsetroundjoin%
\definecolor{currentfill}{rgb}{0.000000,0.500000,0.000000}%
\pgfsetfillcolor{currentfill}%
\pgfsetlinewidth{1.003750pt}%
\definecolor{currentstroke}{rgb}{0.000000,0.500000,0.000000}%
\pgfsetstrokecolor{currentstroke}%
\pgfsetdash{}{0pt}%
\pgfpathmoveto{\pgfqpoint{1.250909in}{-1.392512in}}%
\pgfpathcurveto{\pgfqpoint{1.261959in}{-1.392512in}}{\pgfqpoint{1.272558in}{-1.388122in}}{\pgfqpoint{1.280372in}{-1.380308in}}%
\pgfpathcurveto{\pgfqpoint{1.288185in}{-1.372495in}}{\pgfqpoint{1.292576in}{-1.361895in}}{\pgfqpoint{1.292576in}{-1.350845in}}%
\pgfpathcurveto{\pgfqpoint{1.292576in}{-1.339795in}}{\pgfqpoint{1.288185in}{-1.329196in}}{\pgfqpoint{1.280372in}{-1.321383in}}%
\pgfpathcurveto{\pgfqpoint{1.272558in}{-1.313569in}}{\pgfqpoint{1.261959in}{-1.309179in}}{\pgfqpoint{1.250909in}{-1.309179in}}%
\pgfpathcurveto{\pgfqpoint{1.239859in}{-1.309179in}}{\pgfqpoint{1.229260in}{-1.313569in}}{\pgfqpoint{1.221446in}{-1.321383in}}%
\pgfpathcurveto{\pgfqpoint{1.213633in}{-1.329196in}}{\pgfqpoint{1.209242in}{-1.339795in}}{\pgfqpoint{1.209242in}{-1.350845in}}%
\pgfpathcurveto{\pgfqpoint{1.209242in}{-1.361895in}}{\pgfqpoint{1.213633in}{-1.372495in}}{\pgfqpoint{1.221446in}{-1.380308in}}%
\pgfpathcurveto{\pgfqpoint{1.229260in}{-1.388122in}}{\pgfqpoint{1.239859in}{-1.392512in}}{\pgfqpoint{1.250909in}{-1.392512in}}%
\pgfpathclose%
\pgfusepath{stroke,fill}%
\end{pgfscope}%
\begin{pgfscope}%
\pgfpathrectangle{\pgfqpoint{0.800000in}{0.528000in}}{\pgfqpoint{4.960000in}{3.696000in}} %
\pgfusepath{clip}%
\pgfsetbuttcap%
\pgfsetroundjoin%
\definecolor{currentfill}{rgb}{0.000000,0.500000,0.000000}%
\pgfsetfillcolor{currentfill}%
\pgfsetlinewidth{1.003750pt}%
\definecolor{currentstroke}{rgb}{0.000000,0.500000,0.000000}%
\pgfsetstrokecolor{currentstroke}%
\pgfsetdash{}{0pt}%
\pgfpathmoveto{\pgfqpoint{1.301010in}{-2.646803in}}%
\pgfpathcurveto{\pgfqpoint{1.312060in}{-2.646803in}}{\pgfqpoint{1.322659in}{-2.642413in}}{\pgfqpoint{1.330473in}{-2.634599in}}%
\pgfpathcurveto{\pgfqpoint{1.338287in}{-2.626786in}}{\pgfqpoint{1.342677in}{-2.616187in}}{\pgfqpoint{1.342677in}{-2.605137in}}%
\pgfpathcurveto{\pgfqpoint{1.342677in}{-2.594086in}}{\pgfqpoint{1.338287in}{-2.583487in}}{\pgfqpoint{1.330473in}{-2.575674in}}%
\pgfpathcurveto{\pgfqpoint{1.322659in}{-2.567860in}}{\pgfqpoint{1.312060in}{-2.563470in}}{\pgfqpoint{1.301010in}{-2.563470in}}%
\pgfpathcurveto{\pgfqpoint{1.289960in}{-2.563470in}}{\pgfqpoint{1.279361in}{-2.567860in}}{\pgfqpoint{1.271547in}{-2.575674in}}%
\pgfpathcurveto{\pgfqpoint{1.263734in}{-2.583487in}}{\pgfqpoint{1.259343in}{-2.594086in}}{\pgfqpoint{1.259343in}{-2.605137in}}%
\pgfpathcurveto{\pgfqpoint{1.259343in}{-2.616187in}}{\pgfqpoint{1.263734in}{-2.626786in}}{\pgfqpoint{1.271547in}{-2.634599in}}%
\pgfpathcurveto{\pgfqpoint{1.279361in}{-2.642413in}}{\pgfqpoint{1.289960in}{-2.646803in}}{\pgfqpoint{1.301010in}{-2.646803in}}%
\pgfpathclose%
\pgfusepath{stroke,fill}%
\end{pgfscope}%
\begin{pgfscope}%
\pgfpathrectangle{\pgfqpoint{0.800000in}{0.528000in}}{\pgfqpoint{4.960000in}{3.696000in}} %
\pgfusepath{clip}%
\pgfsetbuttcap%
\pgfsetroundjoin%
\definecolor{currentfill}{rgb}{0.000000,0.500000,0.000000}%
\pgfsetfillcolor{currentfill}%
\pgfsetlinewidth{1.003750pt}%
\definecolor{currentstroke}{rgb}{0.000000,0.500000,0.000000}%
\pgfsetstrokecolor{currentstroke}%
\pgfsetdash{}{0pt}%
\pgfpathmoveto{\pgfqpoint{1.351111in}{0.239201in}}%
\pgfpathcurveto{\pgfqpoint{1.362161in}{0.239201in}}{\pgfqpoint{1.372760in}{0.243592in}}{\pgfqpoint{1.380574in}{0.251405in}}%
\pgfpathcurveto{\pgfqpoint{1.388388in}{0.259219in}}{\pgfqpoint{1.392778in}{0.269818in}}{\pgfqpoint{1.392778in}{0.280868in}}%
\pgfpathcurveto{\pgfqpoint{1.392778in}{0.291918in}}{\pgfqpoint{1.388388in}{0.302517in}}{\pgfqpoint{1.380574in}{0.310331in}}%
\pgfpathcurveto{\pgfqpoint{1.372760in}{0.318144in}}{\pgfqpoint{1.362161in}{0.322535in}}{\pgfqpoint{1.351111in}{0.322535in}}%
\pgfpathcurveto{\pgfqpoint{1.340061in}{0.322535in}}{\pgfqpoint{1.329462in}{0.318144in}}{\pgfqpoint{1.321648in}{0.310331in}}%
\pgfpathcurveto{\pgfqpoint{1.313835in}{0.302517in}}{\pgfqpoint{1.309444in}{0.291918in}}{\pgfqpoint{1.309444in}{0.280868in}}%
\pgfpathcurveto{\pgfqpoint{1.309444in}{0.269818in}}{\pgfqpoint{1.313835in}{0.259219in}}{\pgfqpoint{1.321648in}{0.251405in}}%
\pgfpathcurveto{\pgfqpoint{1.329462in}{0.243592in}}{\pgfqpoint{1.340061in}{0.239201in}}{\pgfqpoint{1.351111in}{0.239201in}}%
\pgfpathclose%
\pgfusepath{stroke,fill}%
\end{pgfscope}%
\begin{pgfscope}%
\pgfpathrectangle{\pgfqpoint{0.800000in}{0.528000in}}{\pgfqpoint{4.960000in}{3.696000in}} %
\pgfusepath{clip}%
\pgfsetbuttcap%
\pgfsetroundjoin%
\definecolor{currentfill}{rgb}{0.000000,0.500000,0.000000}%
\pgfsetfillcolor{currentfill}%
\pgfsetlinewidth{1.003750pt}%
\definecolor{currentstroke}{rgb}{0.000000,0.500000,0.000000}%
\pgfsetstrokecolor{currentstroke}%
\pgfsetdash{}{0pt}%
\pgfpathmoveto{\pgfqpoint{1.401212in}{-1.100664in}}%
\pgfpathcurveto{\pgfqpoint{1.412262in}{-1.100664in}}{\pgfqpoint{1.422861in}{-1.096273in}}{\pgfqpoint{1.430675in}{-1.088460in}}%
\pgfpathcurveto{\pgfqpoint{1.438489in}{-1.080646in}}{\pgfqpoint{1.442879in}{-1.070047in}}{\pgfqpoint{1.442879in}{-1.058997in}}%
\pgfpathcurveto{\pgfqpoint{1.442879in}{-1.047947in}}{\pgfqpoint{1.438489in}{-1.037348in}}{\pgfqpoint{1.430675in}{-1.029534in}}%
\pgfpathcurveto{\pgfqpoint{1.422861in}{-1.021720in}}{\pgfqpoint{1.412262in}{-1.017330in}}{\pgfqpoint{1.401212in}{-1.017330in}}%
\pgfpathcurveto{\pgfqpoint{1.390162in}{-1.017330in}}{\pgfqpoint{1.379563in}{-1.021720in}}{\pgfqpoint{1.371749in}{-1.029534in}}%
\pgfpathcurveto{\pgfqpoint{1.363936in}{-1.037348in}}{\pgfqpoint{1.359545in}{-1.047947in}}{\pgfqpoint{1.359545in}{-1.058997in}}%
\pgfpathcurveto{\pgfqpoint{1.359545in}{-1.070047in}}{\pgfqpoint{1.363936in}{-1.080646in}}{\pgfqpoint{1.371749in}{-1.088460in}}%
\pgfpathcurveto{\pgfqpoint{1.379563in}{-1.096273in}}{\pgfqpoint{1.390162in}{-1.100664in}}{\pgfqpoint{1.401212in}{-1.100664in}}%
\pgfpathclose%
\pgfusepath{stroke,fill}%
\end{pgfscope}%
\begin{pgfscope}%
\pgfpathrectangle{\pgfqpoint{0.800000in}{0.528000in}}{\pgfqpoint{4.960000in}{3.696000in}} %
\pgfusepath{clip}%
\pgfsetbuttcap%
\pgfsetroundjoin%
\definecolor{currentfill}{rgb}{0.000000,0.500000,0.000000}%
\pgfsetfillcolor{currentfill}%
\pgfsetlinewidth{1.003750pt}%
\definecolor{currentstroke}{rgb}{0.000000,0.500000,0.000000}%
\pgfsetstrokecolor{currentstroke}%
\pgfsetdash{}{0pt}%
\pgfpathmoveto{\pgfqpoint{1.451313in}{-0.157967in}}%
\pgfpathcurveto{\pgfqpoint{1.462363in}{-0.157967in}}{\pgfqpoint{1.472962in}{-0.153577in}}{\pgfqpoint{1.480776in}{-0.145764in}}%
\pgfpathcurveto{\pgfqpoint{1.488590in}{-0.137950in}}{\pgfqpoint{1.492980in}{-0.127351in}}{\pgfqpoint{1.492980in}{-0.116301in}}%
\pgfpathcurveto{\pgfqpoint{1.492980in}{-0.105251in}}{\pgfqpoint{1.488590in}{-0.094652in}}{\pgfqpoint{1.480776in}{-0.086838in}}%
\pgfpathcurveto{\pgfqpoint{1.472962in}{-0.079024in}}{\pgfqpoint{1.462363in}{-0.074634in}}{\pgfqpoint{1.451313in}{-0.074634in}}%
\pgfpathcurveto{\pgfqpoint{1.440263in}{-0.074634in}}{\pgfqpoint{1.429664in}{-0.079024in}}{\pgfqpoint{1.421850in}{-0.086838in}}%
\pgfpathcurveto{\pgfqpoint{1.414037in}{-0.094652in}}{\pgfqpoint{1.409646in}{-0.105251in}}{\pgfqpoint{1.409646in}{-0.116301in}}%
\pgfpathcurveto{\pgfqpoint{1.409646in}{-0.127351in}}{\pgfqpoint{1.414037in}{-0.137950in}}{\pgfqpoint{1.421850in}{-0.145764in}}%
\pgfpathcurveto{\pgfqpoint{1.429664in}{-0.153577in}}{\pgfqpoint{1.440263in}{-0.157967in}}{\pgfqpoint{1.451313in}{-0.157967in}}%
\pgfpathclose%
\pgfusepath{stroke,fill}%
\end{pgfscope}%
\begin{pgfscope}%
\pgfpathrectangle{\pgfqpoint{0.800000in}{0.528000in}}{\pgfqpoint{4.960000in}{3.696000in}} %
\pgfusepath{clip}%
\pgfsetbuttcap%
\pgfsetroundjoin%
\definecolor{currentfill}{rgb}{0.000000,0.500000,0.000000}%
\pgfsetfillcolor{currentfill}%
\pgfsetlinewidth{1.003750pt}%
\definecolor{currentstroke}{rgb}{0.000000,0.500000,0.000000}%
\pgfsetstrokecolor{currentstroke}%
\pgfsetdash{}{0pt}%
\pgfpathmoveto{\pgfqpoint{1.501414in}{-1.938403in}}%
\pgfpathcurveto{\pgfqpoint{1.512464in}{-1.938403in}}{\pgfqpoint{1.523063in}{-1.934013in}}{\pgfqpoint{1.530877in}{-1.926199in}}%
\pgfpathcurveto{\pgfqpoint{1.538691in}{-1.918385in}}{\pgfqpoint{1.543081in}{-1.907786in}}{\pgfqpoint{1.543081in}{-1.896736in}}%
\pgfpathcurveto{\pgfqpoint{1.543081in}{-1.885686in}}{\pgfqpoint{1.538691in}{-1.875087in}}{\pgfqpoint{1.530877in}{-1.867274in}}%
\pgfpathcurveto{\pgfqpoint{1.523063in}{-1.859460in}}{\pgfqpoint{1.512464in}{-1.855070in}}{\pgfqpoint{1.501414in}{-1.855070in}}%
\pgfpathcurveto{\pgfqpoint{1.490364in}{-1.855070in}}{\pgfqpoint{1.479765in}{-1.859460in}}{\pgfqpoint{1.471951in}{-1.867274in}}%
\pgfpathcurveto{\pgfqpoint{1.464138in}{-1.875087in}}{\pgfqpoint{1.459747in}{-1.885686in}}{\pgfqpoint{1.459747in}{-1.896736in}}%
\pgfpathcurveto{\pgfqpoint{1.459747in}{-1.907786in}}{\pgfqpoint{1.464138in}{-1.918385in}}{\pgfqpoint{1.471951in}{-1.926199in}}%
\pgfpathcurveto{\pgfqpoint{1.479765in}{-1.934013in}}{\pgfqpoint{1.490364in}{-1.938403in}}{\pgfqpoint{1.501414in}{-1.938403in}}%
\pgfpathclose%
\pgfusepath{stroke,fill}%
\end{pgfscope}%
\begin{pgfscope}%
\pgfpathrectangle{\pgfqpoint{0.800000in}{0.528000in}}{\pgfqpoint{4.960000in}{3.696000in}} %
\pgfusepath{clip}%
\pgfsetbuttcap%
\pgfsetroundjoin%
\definecolor{currentfill}{rgb}{0.000000,0.500000,0.000000}%
\pgfsetfillcolor{currentfill}%
\pgfsetlinewidth{1.003750pt}%
\definecolor{currentstroke}{rgb}{0.000000,0.500000,0.000000}%
\pgfsetstrokecolor{currentstroke}%
\pgfsetdash{}{0pt}%
\pgfpathmoveto{\pgfqpoint{1.551515in}{-2.297595in}}%
\pgfpathcurveto{\pgfqpoint{1.562565in}{-2.297595in}}{\pgfqpoint{1.573164in}{-2.293205in}}{\pgfqpoint{1.580978in}{-2.285391in}}%
\pgfpathcurveto{\pgfqpoint{1.588792in}{-2.277577in}}{\pgfqpoint{1.593182in}{-2.266978in}}{\pgfqpoint{1.593182in}{-2.255928in}}%
\pgfpathcurveto{\pgfqpoint{1.593182in}{-2.244878in}}{\pgfqpoint{1.588792in}{-2.234279in}}{\pgfqpoint{1.580978in}{-2.226466in}}%
\pgfpathcurveto{\pgfqpoint{1.573164in}{-2.218652in}}{\pgfqpoint{1.562565in}{-2.214262in}}{\pgfqpoint{1.551515in}{-2.214262in}}%
\pgfpathcurveto{\pgfqpoint{1.540465in}{-2.214262in}}{\pgfqpoint{1.529866in}{-2.218652in}}{\pgfqpoint{1.522052in}{-2.226466in}}%
\pgfpathcurveto{\pgfqpoint{1.514239in}{-2.234279in}}{\pgfqpoint{1.509848in}{-2.244878in}}{\pgfqpoint{1.509848in}{-2.255928in}}%
\pgfpathcurveto{\pgfqpoint{1.509848in}{-2.266978in}}{\pgfqpoint{1.514239in}{-2.277577in}}{\pgfqpoint{1.522052in}{-2.285391in}}%
\pgfpathcurveto{\pgfqpoint{1.529866in}{-2.293205in}}{\pgfqpoint{1.540465in}{-2.297595in}}{\pgfqpoint{1.551515in}{-2.297595in}}%
\pgfpathclose%
\pgfusepath{stroke,fill}%
\end{pgfscope}%
\begin{pgfscope}%
\pgfpathrectangle{\pgfqpoint{0.800000in}{0.528000in}}{\pgfqpoint{4.960000in}{3.696000in}} %
\pgfusepath{clip}%
\pgfsetbuttcap%
\pgfsetroundjoin%
\definecolor{currentfill}{rgb}{0.000000,0.500000,0.000000}%
\pgfsetfillcolor{currentfill}%
\pgfsetlinewidth{1.003750pt}%
\definecolor{currentstroke}{rgb}{0.000000,0.500000,0.000000}%
\pgfsetstrokecolor{currentstroke}%
\pgfsetdash{}{0pt}%
\pgfpathmoveto{\pgfqpoint{1.601616in}{-3.654665in}}%
\pgfpathcurveto{\pgfqpoint{1.612666in}{-3.654665in}}{\pgfqpoint{1.623265in}{-3.650275in}}{\pgfqpoint{1.631079in}{-3.642461in}}%
\pgfpathcurveto{\pgfqpoint{1.638893in}{-3.634648in}}{\pgfqpoint{1.643283in}{-3.624049in}}{\pgfqpoint{1.643283in}{-3.612998in}}%
\pgfpathcurveto{\pgfqpoint{1.643283in}{-3.601948in}}{\pgfqpoint{1.638893in}{-3.591349in}}{\pgfqpoint{1.631079in}{-3.583536in}}%
\pgfpathcurveto{\pgfqpoint{1.623265in}{-3.575722in}}{\pgfqpoint{1.612666in}{-3.571332in}}{\pgfqpoint{1.601616in}{-3.571332in}}%
\pgfpathcurveto{\pgfqpoint{1.590566in}{-3.571332in}}{\pgfqpoint{1.579967in}{-3.575722in}}{\pgfqpoint{1.572153in}{-3.583536in}}%
\pgfpathcurveto{\pgfqpoint{1.564340in}{-3.591349in}}{\pgfqpoint{1.559949in}{-3.601948in}}{\pgfqpoint{1.559949in}{-3.612998in}}%
\pgfpathcurveto{\pgfqpoint{1.559949in}{-3.624049in}}{\pgfqpoint{1.564340in}{-3.634648in}}{\pgfqpoint{1.572153in}{-3.642461in}}%
\pgfpathcurveto{\pgfqpoint{1.579967in}{-3.650275in}}{\pgfqpoint{1.590566in}{-3.654665in}}{\pgfqpoint{1.601616in}{-3.654665in}}%
\pgfpathclose%
\pgfusepath{stroke,fill}%
\end{pgfscope}%
\begin{pgfscope}%
\pgfpathrectangle{\pgfqpoint{0.800000in}{0.528000in}}{\pgfqpoint{4.960000in}{3.696000in}} %
\pgfusepath{clip}%
\pgfsetbuttcap%
\pgfsetroundjoin%
\definecolor{currentfill}{rgb}{0.000000,0.500000,0.000000}%
\pgfsetfillcolor{currentfill}%
\pgfsetlinewidth{1.003750pt}%
\definecolor{currentstroke}{rgb}{0.000000,0.500000,0.000000}%
\pgfsetstrokecolor{currentstroke}%
\pgfsetdash{}{0pt}%
\pgfpathmoveto{\pgfqpoint{1.651717in}{0.080954in}}%
\pgfpathcurveto{\pgfqpoint{1.662767in}{0.080954in}}{\pgfqpoint{1.673366in}{0.085344in}}{\pgfqpoint{1.681180in}{0.093158in}}%
\pgfpathcurveto{\pgfqpoint{1.688994in}{0.100971in}}{\pgfqpoint{1.693384in}{0.111570in}}{\pgfqpoint{1.693384in}{0.122620in}}%
\pgfpathcurveto{\pgfqpoint{1.693384in}{0.133671in}}{\pgfqpoint{1.688994in}{0.144270in}}{\pgfqpoint{1.681180in}{0.152083in}}%
\pgfpathcurveto{\pgfqpoint{1.673366in}{0.159897in}}{\pgfqpoint{1.662767in}{0.164287in}}{\pgfqpoint{1.651717in}{0.164287in}}%
\pgfpathcurveto{\pgfqpoint{1.640667in}{0.164287in}}{\pgfqpoint{1.630068in}{0.159897in}}{\pgfqpoint{1.622254in}{0.152083in}}%
\pgfpathcurveto{\pgfqpoint{1.614441in}{0.144270in}}{\pgfqpoint{1.610051in}{0.133671in}}{\pgfqpoint{1.610051in}{0.122620in}}%
\pgfpathcurveto{\pgfqpoint{1.610051in}{0.111570in}}{\pgfqpoint{1.614441in}{0.100971in}}{\pgfqpoint{1.622254in}{0.093158in}}%
\pgfpathcurveto{\pgfqpoint{1.630068in}{0.085344in}}{\pgfqpoint{1.640667in}{0.080954in}}{\pgfqpoint{1.651717in}{0.080954in}}%
\pgfpathclose%
\pgfusepath{stroke,fill}%
\end{pgfscope}%
\begin{pgfscope}%
\pgfpathrectangle{\pgfqpoint{0.800000in}{0.528000in}}{\pgfqpoint{4.960000in}{3.696000in}} %
\pgfusepath{clip}%
\pgfsetbuttcap%
\pgfsetroundjoin%
\definecolor{currentfill}{rgb}{0.000000,0.500000,0.000000}%
\pgfsetfillcolor{currentfill}%
\pgfsetlinewidth{1.003750pt}%
\definecolor{currentstroke}{rgb}{0.000000,0.500000,0.000000}%
\pgfsetstrokecolor{currentstroke}%
\pgfsetdash{}{0pt}%
\pgfpathmoveto{\pgfqpoint{1.701818in}{0.105126in}}%
\pgfpathcurveto{\pgfqpoint{1.712868in}{0.105126in}}{\pgfqpoint{1.723467in}{0.109517in}}{\pgfqpoint{1.731281in}{0.117330in}}%
\pgfpathcurveto{\pgfqpoint{1.739095in}{0.125144in}}{\pgfqpoint{1.743485in}{0.135743in}}{\pgfqpoint{1.743485in}{0.146793in}}%
\pgfpathcurveto{\pgfqpoint{1.743485in}{0.157843in}}{\pgfqpoint{1.739095in}{0.168442in}}{\pgfqpoint{1.731281in}{0.176256in}}%
\pgfpathcurveto{\pgfqpoint{1.723467in}{0.184069in}}{\pgfqpoint{1.712868in}{0.188460in}}{\pgfqpoint{1.701818in}{0.188460in}}%
\pgfpathcurveto{\pgfqpoint{1.690768in}{0.188460in}}{\pgfqpoint{1.680169in}{0.184069in}}{\pgfqpoint{1.672355in}{0.176256in}}%
\pgfpathcurveto{\pgfqpoint{1.664542in}{0.168442in}}{\pgfqpoint{1.660152in}{0.157843in}}{\pgfqpoint{1.660152in}{0.146793in}}%
\pgfpathcurveto{\pgfqpoint{1.660152in}{0.135743in}}{\pgfqpoint{1.664542in}{0.125144in}}{\pgfqpoint{1.672355in}{0.117330in}}%
\pgfpathcurveto{\pgfqpoint{1.680169in}{0.109517in}}{\pgfqpoint{1.690768in}{0.105126in}}{\pgfqpoint{1.701818in}{0.105126in}}%
\pgfpathclose%
\pgfusepath{stroke,fill}%
\end{pgfscope}%
\begin{pgfscope}%
\pgfpathrectangle{\pgfqpoint{0.800000in}{0.528000in}}{\pgfqpoint{4.960000in}{3.696000in}} %
\pgfusepath{clip}%
\pgfsetbuttcap%
\pgfsetroundjoin%
\definecolor{currentfill}{rgb}{0.000000,0.500000,0.000000}%
\pgfsetfillcolor{currentfill}%
\pgfsetlinewidth{1.003750pt}%
\definecolor{currentstroke}{rgb}{0.000000,0.500000,0.000000}%
\pgfsetstrokecolor{currentstroke}%
\pgfsetdash{}{0pt}%
\pgfpathmoveto{\pgfqpoint{1.751919in}{-3.143259in}}%
\pgfpathcurveto{\pgfqpoint{1.762969in}{-3.143259in}}{\pgfqpoint{1.773568in}{-3.138869in}}{\pgfqpoint{1.781382in}{-3.131055in}}%
\pgfpathcurveto{\pgfqpoint{1.789196in}{-3.123242in}}{\pgfqpoint{1.793586in}{-3.112643in}}{\pgfqpoint{1.793586in}{-3.101592in}}%
\pgfpathcurveto{\pgfqpoint{1.793586in}{-3.090542in}}{\pgfqpoint{1.789196in}{-3.079943in}}{\pgfqpoint{1.781382in}{-3.072130in}}%
\pgfpathcurveto{\pgfqpoint{1.773568in}{-3.064316in}}{\pgfqpoint{1.762969in}{-3.059926in}}{\pgfqpoint{1.751919in}{-3.059926in}}%
\pgfpathcurveto{\pgfqpoint{1.740869in}{-3.059926in}}{\pgfqpoint{1.730270in}{-3.064316in}}{\pgfqpoint{1.722456in}{-3.072130in}}%
\pgfpathcurveto{\pgfqpoint{1.714643in}{-3.079943in}}{\pgfqpoint{1.710253in}{-3.090542in}}{\pgfqpoint{1.710253in}{-3.101592in}}%
\pgfpathcurveto{\pgfqpoint{1.710253in}{-3.112643in}}{\pgfqpoint{1.714643in}{-3.123242in}}{\pgfqpoint{1.722456in}{-3.131055in}}%
\pgfpathcurveto{\pgfqpoint{1.730270in}{-3.138869in}}{\pgfqpoint{1.740869in}{-3.143259in}}{\pgfqpoint{1.751919in}{-3.143259in}}%
\pgfpathclose%
\pgfusepath{stroke,fill}%
\end{pgfscope}%
\begin{pgfscope}%
\pgfpathrectangle{\pgfqpoint{0.800000in}{0.528000in}}{\pgfqpoint{4.960000in}{3.696000in}} %
\pgfusepath{clip}%
\pgfsetbuttcap%
\pgfsetroundjoin%
\definecolor{currentfill}{rgb}{0.000000,0.500000,0.000000}%
\pgfsetfillcolor{currentfill}%
\pgfsetlinewidth{1.003750pt}%
\definecolor{currentstroke}{rgb}{0.000000,0.500000,0.000000}%
\pgfsetstrokecolor{currentstroke}%
\pgfsetdash{}{0pt}%
\pgfpathmoveto{\pgfqpoint{1.802020in}{0.631821in}}%
\pgfpathcurveto{\pgfqpoint{1.813070in}{0.631821in}}{\pgfqpoint{1.823669in}{0.636211in}}{\pgfqpoint{1.831483in}{0.644025in}}%
\pgfpathcurveto{\pgfqpoint{1.839297in}{0.651838in}}{\pgfqpoint{1.843687in}{0.662437in}}{\pgfqpoint{1.843687in}{0.673487in}}%
\pgfpathcurveto{\pgfqpoint{1.843687in}{0.684537in}}{\pgfqpoint{1.839297in}{0.695136in}}{\pgfqpoint{1.831483in}{0.702950in}}%
\pgfpathcurveto{\pgfqpoint{1.823669in}{0.710764in}}{\pgfqpoint{1.813070in}{0.715154in}}{\pgfqpoint{1.802020in}{0.715154in}}%
\pgfpathcurveto{\pgfqpoint{1.790970in}{0.715154in}}{\pgfqpoint{1.780371in}{0.710764in}}{\pgfqpoint{1.772557in}{0.702950in}}%
\pgfpathcurveto{\pgfqpoint{1.764744in}{0.695136in}}{\pgfqpoint{1.760354in}{0.684537in}}{\pgfqpoint{1.760354in}{0.673487in}}%
\pgfpathcurveto{\pgfqpoint{1.760354in}{0.662437in}}{\pgfqpoint{1.764744in}{0.651838in}}{\pgfqpoint{1.772557in}{0.644025in}}%
\pgfpathcurveto{\pgfqpoint{1.780371in}{0.636211in}}{\pgfqpoint{1.790970in}{0.631821in}}{\pgfqpoint{1.802020in}{0.631821in}}%
\pgfpathclose%
\pgfusepath{stroke,fill}%
\end{pgfscope}%
\begin{pgfscope}%
\pgfpathrectangle{\pgfqpoint{0.800000in}{0.528000in}}{\pgfqpoint{4.960000in}{3.696000in}} %
\pgfusepath{clip}%
\pgfsetbuttcap%
\pgfsetroundjoin%
\definecolor{currentfill}{rgb}{0.000000,0.500000,0.000000}%
\pgfsetfillcolor{currentfill}%
\pgfsetlinewidth{1.003750pt}%
\definecolor{currentstroke}{rgb}{0.000000,0.500000,0.000000}%
\pgfsetstrokecolor{currentstroke}%
\pgfsetdash{}{0pt}%
\pgfpathmoveto{\pgfqpoint{1.852121in}{-1.499330in}}%
\pgfpathcurveto{\pgfqpoint{1.863171in}{-1.499330in}}{\pgfqpoint{1.873770in}{-1.494940in}}{\pgfqpoint{1.881584in}{-1.487127in}}%
\pgfpathcurveto{\pgfqpoint{1.889398in}{-1.479313in}}{\pgfqpoint{1.893788in}{-1.468714in}}{\pgfqpoint{1.893788in}{-1.457664in}}%
\pgfpathcurveto{\pgfqpoint{1.893788in}{-1.446614in}}{\pgfqpoint{1.889398in}{-1.436015in}}{\pgfqpoint{1.881584in}{-1.428201in}}%
\pgfpathcurveto{\pgfqpoint{1.873770in}{-1.420387in}}{\pgfqpoint{1.863171in}{-1.415997in}}{\pgfqpoint{1.852121in}{-1.415997in}}%
\pgfpathcurveto{\pgfqpoint{1.841071in}{-1.415997in}}{\pgfqpoint{1.830472in}{-1.420387in}}{\pgfqpoint{1.822658in}{-1.428201in}}%
\pgfpathcurveto{\pgfqpoint{1.814845in}{-1.436015in}}{\pgfqpoint{1.810455in}{-1.446614in}}{\pgfqpoint{1.810455in}{-1.457664in}}%
\pgfpathcurveto{\pgfqpoint{1.810455in}{-1.468714in}}{\pgfqpoint{1.814845in}{-1.479313in}}{\pgfqpoint{1.822658in}{-1.487127in}}%
\pgfpathcurveto{\pgfqpoint{1.830472in}{-1.494940in}}{\pgfqpoint{1.841071in}{-1.499330in}}{\pgfqpoint{1.852121in}{-1.499330in}}%
\pgfpathclose%
\pgfusepath{stroke,fill}%
\end{pgfscope}%
\begin{pgfscope}%
\pgfpathrectangle{\pgfqpoint{0.800000in}{0.528000in}}{\pgfqpoint{4.960000in}{3.696000in}} %
\pgfusepath{clip}%
\pgfsetbuttcap%
\pgfsetroundjoin%
\definecolor{currentfill}{rgb}{0.000000,0.500000,0.000000}%
\pgfsetfillcolor{currentfill}%
\pgfsetlinewidth{1.003750pt}%
\definecolor{currentstroke}{rgb}{0.000000,0.500000,0.000000}%
\pgfsetstrokecolor{currentstroke}%
\pgfsetdash{}{0pt}%
\pgfpathmoveto{\pgfqpoint{1.902222in}{-2.161634in}}%
\pgfpathcurveto{\pgfqpoint{1.913272in}{-2.161634in}}{\pgfqpoint{1.923871in}{-2.157244in}}{\pgfqpoint{1.931685in}{-2.149430in}}%
\pgfpathcurveto{\pgfqpoint{1.939499in}{-2.141617in}}{\pgfqpoint{1.943889in}{-2.131018in}}{\pgfqpoint{1.943889in}{-2.119968in}}%
\pgfpathcurveto{\pgfqpoint{1.943889in}{-2.108918in}}{\pgfqpoint{1.939499in}{-2.098319in}}{\pgfqpoint{1.931685in}{-2.090505in}}%
\pgfpathcurveto{\pgfqpoint{1.923871in}{-2.082691in}}{\pgfqpoint{1.913272in}{-2.078301in}}{\pgfqpoint{1.902222in}{-2.078301in}}%
\pgfpathcurveto{\pgfqpoint{1.891172in}{-2.078301in}}{\pgfqpoint{1.880573in}{-2.082691in}}{\pgfqpoint{1.872759in}{-2.090505in}}%
\pgfpathcurveto{\pgfqpoint{1.864946in}{-2.098319in}}{\pgfqpoint{1.860556in}{-2.108918in}}{\pgfqpoint{1.860556in}{-2.119968in}}%
\pgfpathcurveto{\pgfqpoint{1.860556in}{-2.131018in}}{\pgfqpoint{1.864946in}{-2.141617in}}{\pgfqpoint{1.872759in}{-2.149430in}}%
\pgfpathcurveto{\pgfqpoint{1.880573in}{-2.157244in}}{\pgfqpoint{1.891172in}{-2.161634in}}{\pgfqpoint{1.902222in}{-2.161634in}}%
\pgfpathclose%
\pgfusepath{stroke,fill}%
\end{pgfscope}%
\begin{pgfscope}%
\pgfpathrectangle{\pgfqpoint{0.800000in}{0.528000in}}{\pgfqpoint{4.960000in}{3.696000in}} %
\pgfusepath{clip}%
\pgfsetbuttcap%
\pgfsetroundjoin%
\definecolor{currentfill}{rgb}{0.000000,0.500000,0.000000}%
\pgfsetfillcolor{currentfill}%
\pgfsetlinewidth{1.003750pt}%
\definecolor{currentstroke}{rgb}{0.000000,0.500000,0.000000}%
\pgfsetstrokecolor{currentstroke}%
\pgfsetdash{}{0pt}%
\pgfpathmoveto{\pgfqpoint{1.952323in}{-2.331141in}}%
\pgfpathcurveto{\pgfqpoint{1.963373in}{-2.331141in}}{\pgfqpoint{1.973972in}{-2.326751in}}{\pgfqpoint{1.981786in}{-2.318937in}}%
\pgfpathcurveto{\pgfqpoint{1.989600in}{-2.311124in}}{\pgfqpoint{1.993990in}{-2.300525in}}{\pgfqpoint{1.993990in}{-2.289475in}}%
\pgfpathcurveto{\pgfqpoint{1.993990in}{-2.278425in}}{\pgfqpoint{1.989600in}{-2.267825in}}{\pgfqpoint{1.981786in}{-2.260012in}}%
\pgfpathcurveto{\pgfqpoint{1.973972in}{-2.252198in}}{\pgfqpoint{1.963373in}{-2.247808in}}{\pgfqpoint{1.952323in}{-2.247808in}}%
\pgfpathcurveto{\pgfqpoint{1.941273in}{-2.247808in}}{\pgfqpoint{1.930674in}{-2.252198in}}{\pgfqpoint{1.922860in}{-2.260012in}}%
\pgfpathcurveto{\pgfqpoint{1.915047in}{-2.267825in}}{\pgfqpoint{1.910657in}{-2.278425in}}{\pgfqpoint{1.910657in}{-2.289475in}}%
\pgfpathcurveto{\pgfqpoint{1.910657in}{-2.300525in}}{\pgfqpoint{1.915047in}{-2.311124in}}{\pgfqpoint{1.922860in}{-2.318937in}}%
\pgfpathcurveto{\pgfqpoint{1.930674in}{-2.326751in}}{\pgfqpoint{1.941273in}{-2.331141in}}{\pgfqpoint{1.952323in}{-2.331141in}}%
\pgfpathclose%
\pgfusepath{stroke,fill}%
\end{pgfscope}%
\begin{pgfscope}%
\pgfpathrectangle{\pgfqpoint{0.800000in}{0.528000in}}{\pgfqpoint{4.960000in}{3.696000in}} %
\pgfusepath{clip}%
\pgfsetbuttcap%
\pgfsetroundjoin%
\definecolor{currentfill}{rgb}{0.000000,0.500000,0.000000}%
\pgfsetfillcolor{currentfill}%
\pgfsetlinewidth{1.003750pt}%
\definecolor{currentstroke}{rgb}{0.000000,0.500000,0.000000}%
\pgfsetstrokecolor{currentstroke}%
\pgfsetdash{}{0pt}%
\pgfpathmoveto{\pgfqpoint{2.002424in}{-1.311050in}}%
\pgfpathcurveto{\pgfqpoint{2.013474in}{-1.311050in}}{\pgfqpoint{2.024073in}{-1.306660in}}{\pgfqpoint{2.031887in}{-1.298846in}}%
\pgfpathcurveto{\pgfqpoint{2.039701in}{-1.291033in}}{\pgfqpoint{2.044091in}{-1.280433in}}{\pgfqpoint{2.044091in}{-1.269383in}}%
\pgfpathcurveto{\pgfqpoint{2.044091in}{-1.258333in}}{\pgfqpoint{2.039701in}{-1.247734in}}{\pgfqpoint{2.031887in}{-1.239921in}}%
\pgfpathcurveto{\pgfqpoint{2.024073in}{-1.232107in}}{\pgfqpoint{2.013474in}{-1.227717in}}{\pgfqpoint{2.002424in}{-1.227717in}}%
\pgfpathcurveto{\pgfqpoint{1.991374in}{-1.227717in}}{\pgfqpoint{1.980775in}{-1.232107in}}{\pgfqpoint{1.972961in}{-1.239921in}}%
\pgfpathcurveto{\pgfqpoint{1.965148in}{-1.247734in}}{\pgfqpoint{1.960758in}{-1.258333in}}{\pgfqpoint{1.960758in}{-1.269383in}}%
\pgfpathcurveto{\pgfqpoint{1.960758in}{-1.280433in}}{\pgfqpoint{1.965148in}{-1.291033in}}{\pgfqpoint{1.972961in}{-1.298846in}}%
\pgfpathcurveto{\pgfqpoint{1.980775in}{-1.306660in}}{\pgfqpoint{1.991374in}{-1.311050in}}{\pgfqpoint{2.002424in}{-1.311050in}}%
\pgfpathclose%
\pgfusepath{stroke,fill}%
\end{pgfscope}%
\begin{pgfscope}%
\pgfpathrectangle{\pgfqpoint{0.800000in}{0.528000in}}{\pgfqpoint{4.960000in}{3.696000in}} %
\pgfusepath{clip}%
\pgfsetbuttcap%
\pgfsetroundjoin%
\definecolor{currentfill}{rgb}{0.000000,0.500000,0.000000}%
\pgfsetfillcolor{currentfill}%
\pgfsetlinewidth{1.003750pt}%
\definecolor{currentstroke}{rgb}{0.000000,0.500000,0.000000}%
\pgfsetstrokecolor{currentstroke}%
\pgfsetdash{}{0pt}%
\pgfpathmoveto{\pgfqpoint{2.052525in}{-0.684279in}}%
\pgfpathcurveto{\pgfqpoint{2.063575in}{-0.684279in}}{\pgfqpoint{2.074174in}{-0.679889in}}{\pgfqpoint{2.081988in}{-0.672076in}}%
\pgfpathcurveto{\pgfqpoint{2.089802in}{-0.664262in}}{\pgfqpoint{2.094192in}{-0.653663in}}{\pgfqpoint{2.094192in}{-0.642613in}}%
\pgfpathcurveto{\pgfqpoint{2.094192in}{-0.631563in}}{\pgfqpoint{2.089802in}{-0.620964in}}{\pgfqpoint{2.081988in}{-0.613150in}}%
\pgfpathcurveto{\pgfqpoint{2.074174in}{-0.605336in}}{\pgfqpoint{2.063575in}{-0.600946in}}{\pgfqpoint{2.052525in}{-0.600946in}}%
\pgfpathcurveto{\pgfqpoint{2.041475in}{-0.600946in}}{\pgfqpoint{2.030876in}{-0.605336in}}{\pgfqpoint{2.023062in}{-0.613150in}}%
\pgfpathcurveto{\pgfqpoint{2.015249in}{-0.620964in}}{\pgfqpoint{2.010859in}{-0.631563in}}{\pgfqpoint{2.010859in}{-0.642613in}}%
\pgfpathcurveto{\pgfqpoint{2.010859in}{-0.653663in}}{\pgfqpoint{2.015249in}{-0.664262in}}{\pgfqpoint{2.023062in}{-0.672076in}}%
\pgfpathcurveto{\pgfqpoint{2.030876in}{-0.679889in}}{\pgfqpoint{2.041475in}{-0.684279in}}{\pgfqpoint{2.052525in}{-0.684279in}}%
\pgfpathclose%
\pgfusepath{stroke,fill}%
\end{pgfscope}%
\begin{pgfscope}%
\pgfpathrectangle{\pgfqpoint{0.800000in}{0.528000in}}{\pgfqpoint{4.960000in}{3.696000in}} %
\pgfusepath{clip}%
\pgfsetbuttcap%
\pgfsetroundjoin%
\definecolor{currentfill}{rgb}{0.000000,0.500000,0.000000}%
\pgfsetfillcolor{currentfill}%
\pgfsetlinewidth{1.003750pt}%
\definecolor{currentstroke}{rgb}{0.000000,0.500000,0.000000}%
\pgfsetstrokecolor{currentstroke}%
\pgfsetdash{}{0pt}%
\pgfpathmoveto{\pgfqpoint{2.102626in}{0.814808in}}%
\pgfpathcurveto{\pgfqpoint{2.113676in}{0.814808in}}{\pgfqpoint{2.124275in}{0.819199in}}{\pgfqpoint{2.132089in}{0.827012in}}%
\pgfpathcurveto{\pgfqpoint{2.139903in}{0.834826in}}{\pgfqpoint{2.144293in}{0.845425in}}{\pgfqpoint{2.144293in}{0.856475in}}%
\pgfpathcurveto{\pgfqpoint{2.144293in}{0.867525in}}{\pgfqpoint{2.139903in}{0.878124in}}{\pgfqpoint{2.132089in}{0.885938in}}%
\pgfpathcurveto{\pgfqpoint{2.124275in}{0.893751in}}{\pgfqpoint{2.113676in}{0.898142in}}{\pgfqpoint{2.102626in}{0.898142in}}%
\pgfpathcurveto{\pgfqpoint{2.091576in}{0.898142in}}{\pgfqpoint{2.080977in}{0.893751in}}{\pgfqpoint{2.073163in}{0.885938in}}%
\pgfpathcurveto{\pgfqpoint{2.065350in}{0.878124in}}{\pgfqpoint{2.060960in}{0.867525in}}{\pgfqpoint{2.060960in}{0.856475in}}%
\pgfpathcurveto{\pgfqpoint{2.060960in}{0.845425in}}{\pgfqpoint{2.065350in}{0.834826in}}{\pgfqpoint{2.073163in}{0.827012in}}%
\pgfpathcurveto{\pgfqpoint{2.080977in}{0.819199in}}{\pgfqpoint{2.091576in}{0.814808in}}{\pgfqpoint{2.102626in}{0.814808in}}%
\pgfpathclose%
\pgfusepath{stroke,fill}%
\end{pgfscope}%
\begin{pgfscope}%
\pgfpathrectangle{\pgfqpoint{0.800000in}{0.528000in}}{\pgfqpoint{4.960000in}{3.696000in}} %
\pgfusepath{clip}%
\pgfsetbuttcap%
\pgfsetroundjoin%
\definecolor{currentfill}{rgb}{0.000000,0.500000,0.000000}%
\pgfsetfillcolor{currentfill}%
\pgfsetlinewidth{1.003750pt}%
\definecolor{currentstroke}{rgb}{0.000000,0.500000,0.000000}%
\pgfsetstrokecolor{currentstroke}%
\pgfsetdash{}{0pt}%
\pgfpathmoveto{\pgfqpoint{2.152727in}{-1.218463in}}%
\pgfpathcurveto{\pgfqpoint{2.163777in}{-1.218463in}}{\pgfqpoint{2.174376in}{-1.214073in}}{\pgfqpoint{2.182190in}{-1.206259in}}%
\pgfpathcurveto{\pgfqpoint{2.190004in}{-1.198445in}}{\pgfqpoint{2.194394in}{-1.187846in}}{\pgfqpoint{2.194394in}{-1.176796in}}%
\pgfpathcurveto{\pgfqpoint{2.194394in}{-1.165746in}}{\pgfqpoint{2.190004in}{-1.155147in}}{\pgfqpoint{2.182190in}{-1.147333in}}%
\pgfpathcurveto{\pgfqpoint{2.174376in}{-1.139520in}}{\pgfqpoint{2.163777in}{-1.135130in}}{\pgfqpoint{2.152727in}{-1.135130in}}%
\pgfpathcurveto{\pgfqpoint{2.141677in}{-1.135130in}}{\pgfqpoint{2.131078in}{-1.139520in}}{\pgfqpoint{2.123264in}{-1.147333in}}%
\pgfpathcurveto{\pgfqpoint{2.115451in}{-1.155147in}}{\pgfqpoint{2.111061in}{-1.165746in}}{\pgfqpoint{2.111061in}{-1.176796in}}%
\pgfpathcurveto{\pgfqpoint{2.111061in}{-1.187846in}}{\pgfqpoint{2.115451in}{-1.198445in}}{\pgfqpoint{2.123264in}{-1.206259in}}%
\pgfpathcurveto{\pgfqpoint{2.131078in}{-1.214073in}}{\pgfqpoint{2.141677in}{-1.218463in}}{\pgfqpoint{2.152727in}{-1.218463in}}%
\pgfpathclose%
\pgfusepath{stroke,fill}%
\end{pgfscope}%
\begin{pgfscope}%
\pgfpathrectangle{\pgfqpoint{0.800000in}{0.528000in}}{\pgfqpoint{4.960000in}{3.696000in}} %
\pgfusepath{clip}%
\pgfsetbuttcap%
\pgfsetroundjoin%
\definecolor{currentfill}{rgb}{0.000000,0.500000,0.000000}%
\pgfsetfillcolor{currentfill}%
\pgfsetlinewidth{1.003750pt}%
\definecolor{currentstroke}{rgb}{0.000000,0.500000,0.000000}%
\pgfsetstrokecolor{currentstroke}%
\pgfsetdash{}{0pt}%
\pgfpathmoveto{\pgfqpoint{2.202828in}{0.536603in}}%
\pgfpathcurveto{\pgfqpoint{2.213878in}{0.536603in}}{\pgfqpoint{2.224477in}{0.540993in}}{\pgfqpoint{2.232291in}{0.548807in}}%
\pgfpathcurveto{\pgfqpoint{2.240105in}{0.556620in}}{\pgfqpoint{2.244495in}{0.567219in}}{\pgfqpoint{2.244495in}{0.578269in}}%
\pgfpathcurveto{\pgfqpoint{2.244495in}{0.589320in}}{\pgfqpoint{2.240105in}{0.599919in}}{\pgfqpoint{2.232291in}{0.607732in}}%
\pgfpathcurveto{\pgfqpoint{2.224477in}{0.615546in}}{\pgfqpoint{2.213878in}{0.619936in}}{\pgfqpoint{2.202828in}{0.619936in}}%
\pgfpathcurveto{\pgfqpoint{2.191778in}{0.619936in}}{\pgfqpoint{2.181179in}{0.615546in}}{\pgfqpoint{2.173366in}{0.607732in}}%
\pgfpathcurveto{\pgfqpoint{2.165552in}{0.599919in}}{\pgfqpoint{2.161162in}{0.589320in}}{\pgfqpoint{2.161162in}{0.578269in}}%
\pgfpathcurveto{\pgfqpoint{2.161162in}{0.567219in}}{\pgfqpoint{2.165552in}{0.556620in}}{\pgfqpoint{2.173366in}{0.548807in}}%
\pgfpathcurveto{\pgfqpoint{2.181179in}{0.540993in}}{\pgfqpoint{2.191778in}{0.536603in}}{\pgfqpoint{2.202828in}{0.536603in}}%
\pgfpathclose%
\pgfusepath{stroke,fill}%
\end{pgfscope}%
\begin{pgfscope}%
\pgfpathrectangle{\pgfqpoint{0.800000in}{0.528000in}}{\pgfqpoint{4.960000in}{3.696000in}} %
\pgfusepath{clip}%
\pgfsetbuttcap%
\pgfsetroundjoin%
\definecolor{currentfill}{rgb}{0.000000,0.500000,0.000000}%
\pgfsetfillcolor{currentfill}%
\pgfsetlinewidth{1.003750pt}%
\definecolor{currentstroke}{rgb}{0.000000,0.500000,0.000000}%
\pgfsetstrokecolor{currentstroke}%
\pgfsetdash{}{0pt}%
\pgfpathmoveto{\pgfqpoint{2.252929in}{-1.034681in}}%
\pgfpathcurveto{\pgfqpoint{2.263979in}{-1.034681in}}{\pgfqpoint{2.274578in}{-1.030291in}}{\pgfqpoint{2.282392in}{-1.022477in}}%
\pgfpathcurveto{\pgfqpoint{2.290206in}{-1.014663in}}{\pgfqpoint{2.294596in}{-1.004064in}}{\pgfqpoint{2.294596in}{-0.993014in}}%
\pgfpathcurveto{\pgfqpoint{2.294596in}{-0.981964in}}{\pgfqpoint{2.290206in}{-0.971365in}}{\pgfqpoint{2.282392in}{-0.963551in}}%
\pgfpathcurveto{\pgfqpoint{2.274578in}{-0.955738in}}{\pgfqpoint{2.263979in}{-0.951348in}}{\pgfqpoint{2.252929in}{-0.951348in}}%
\pgfpathcurveto{\pgfqpoint{2.241879in}{-0.951348in}}{\pgfqpoint{2.231280in}{-0.955738in}}{\pgfqpoint{2.223467in}{-0.963551in}}%
\pgfpathcurveto{\pgfqpoint{2.215653in}{-0.971365in}}{\pgfqpoint{2.211263in}{-0.981964in}}{\pgfqpoint{2.211263in}{-0.993014in}}%
\pgfpathcurveto{\pgfqpoint{2.211263in}{-1.004064in}}{\pgfqpoint{2.215653in}{-1.014663in}}{\pgfqpoint{2.223467in}{-1.022477in}}%
\pgfpathcurveto{\pgfqpoint{2.231280in}{-1.030291in}}{\pgfqpoint{2.241879in}{-1.034681in}}{\pgfqpoint{2.252929in}{-1.034681in}}%
\pgfpathclose%
\pgfusepath{stroke,fill}%
\end{pgfscope}%
\begin{pgfscope}%
\pgfpathrectangle{\pgfqpoint{0.800000in}{0.528000in}}{\pgfqpoint{4.960000in}{3.696000in}} %
\pgfusepath{clip}%
\pgfsetbuttcap%
\pgfsetroundjoin%
\definecolor{currentfill}{rgb}{0.000000,0.500000,0.000000}%
\pgfsetfillcolor{currentfill}%
\pgfsetlinewidth{1.003750pt}%
\definecolor{currentstroke}{rgb}{0.000000,0.500000,0.000000}%
\pgfsetstrokecolor{currentstroke}%
\pgfsetdash{}{0pt}%
\pgfpathmoveto{\pgfqpoint{2.303030in}{0.535905in}}%
\pgfpathcurveto{\pgfqpoint{2.314080in}{0.535905in}}{\pgfqpoint{2.324679in}{0.540295in}}{\pgfqpoint{2.332493in}{0.548109in}}%
\pgfpathcurveto{\pgfqpoint{2.340307in}{0.555922in}}{\pgfqpoint{2.344697in}{0.566521in}}{\pgfqpoint{2.344697in}{0.577571in}}%
\pgfpathcurveto{\pgfqpoint{2.344697in}{0.588621in}}{\pgfqpoint{2.340307in}{0.599220in}}{\pgfqpoint{2.332493in}{0.607034in}}%
\pgfpathcurveto{\pgfqpoint{2.324679in}{0.614848in}}{\pgfqpoint{2.314080in}{0.619238in}}{\pgfqpoint{2.303030in}{0.619238in}}%
\pgfpathcurveto{\pgfqpoint{2.291980in}{0.619238in}}{\pgfqpoint{2.281381in}{0.614848in}}{\pgfqpoint{2.273568in}{0.607034in}}%
\pgfpathcurveto{\pgfqpoint{2.265754in}{0.599220in}}{\pgfqpoint{2.261364in}{0.588621in}}{\pgfqpoint{2.261364in}{0.577571in}}%
\pgfpathcurveto{\pgfqpoint{2.261364in}{0.566521in}}{\pgfqpoint{2.265754in}{0.555922in}}{\pgfqpoint{2.273568in}{0.548109in}}%
\pgfpathcurveto{\pgfqpoint{2.281381in}{0.540295in}}{\pgfqpoint{2.291980in}{0.535905in}}{\pgfqpoint{2.303030in}{0.535905in}}%
\pgfpathclose%
\pgfusepath{stroke,fill}%
\end{pgfscope}%
\begin{pgfscope}%
\pgfpathrectangle{\pgfqpoint{0.800000in}{0.528000in}}{\pgfqpoint{4.960000in}{3.696000in}} %
\pgfusepath{clip}%
\pgfsetbuttcap%
\pgfsetroundjoin%
\definecolor{currentfill}{rgb}{0.000000,0.500000,0.000000}%
\pgfsetfillcolor{currentfill}%
\pgfsetlinewidth{1.003750pt}%
\definecolor{currentstroke}{rgb}{0.000000,0.500000,0.000000}%
\pgfsetstrokecolor{currentstroke}%
\pgfsetdash{}{0pt}%
\pgfpathmoveto{\pgfqpoint{2.353131in}{-2.529487in}}%
\pgfpathcurveto{\pgfqpoint{2.364181in}{-2.529487in}}{\pgfqpoint{2.374780in}{-2.525097in}}{\pgfqpoint{2.382594in}{-2.517283in}}%
\pgfpathcurveto{\pgfqpoint{2.390408in}{-2.509470in}}{\pgfqpoint{2.394798in}{-2.498871in}}{\pgfqpoint{2.394798in}{-2.487821in}}%
\pgfpathcurveto{\pgfqpoint{2.394798in}{-2.476771in}}{\pgfqpoint{2.390408in}{-2.466171in}}{\pgfqpoint{2.382594in}{-2.458358in}}%
\pgfpathcurveto{\pgfqpoint{2.374780in}{-2.450544in}}{\pgfqpoint{2.364181in}{-2.446154in}}{\pgfqpoint{2.353131in}{-2.446154in}}%
\pgfpathcurveto{\pgfqpoint{2.342081in}{-2.446154in}}{\pgfqpoint{2.331482in}{-2.450544in}}{\pgfqpoint{2.323669in}{-2.458358in}}%
\pgfpathcurveto{\pgfqpoint{2.315855in}{-2.466171in}}{\pgfqpoint{2.311465in}{-2.476771in}}{\pgfqpoint{2.311465in}{-2.487821in}}%
\pgfpathcurveto{\pgfqpoint{2.311465in}{-2.498871in}}{\pgfqpoint{2.315855in}{-2.509470in}}{\pgfqpoint{2.323669in}{-2.517283in}}%
\pgfpathcurveto{\pgfqpoint{2.331482in}{-2.525097in}}{\pgfqpoint{2.342081in}{-2.529487in}}{\pgfqpoint{2.353131in}{-2.529487in}}%
\pgfpathclose%
\pgfusepath{stroke,fill}%
\end{pgfscope}%
\begin{pgfscope}%
\pgfpathrectangle{\pgfqpoint{0.800000in}{0.528000in}}{\pgfqpoint{4.960000in}{3.696000in}} %
\pgfusepath{clip}%
\pgfsetbuttcap%
\pgfsetroundjoin%
\definecolor{currentfill}{rgb}{0.000000,0.500000,0.000000}%
\pgfsetfillcolor{currentfill}%
\pgfsetlinewidth{1.003750pt}%
\definecolor{currentstroke}{rgb}{0.000000,0.500000,0.000000}%
\pgfsetstrokecolor{currentstroke}%
\pgfsetdash{}{0pt}%
\pgfpathmoveto{\pgfqpoint{2.403232in}{-0.708868in}}%
\pgfpathcurveto{\pgfqpoint{2.414282in}{-0.708868in}}{\pgfqpoint{2.424881in}{-0.704478in}}{\pgfqpoint{2.432695in}{-0.696665in}}%
\pgfpathcurveto{\pgfqpoint{2.440509in}{-0.688851in}}{\pgfqpoint{2.444899in}{-0.678252in}}{\pgfqpoint{2.444899in}{-0.667202in}}%
\pgfpathcurveto{\pgfqpoint{2.444899in}{-0.656152in}}{\pgfqpoint{2.440509in}{-0.645553in}}{\pgfqpoint{2.432695in}{-0.637739in}}%
\pgfpathcurveto{\pgfqpoint{2.424881in}{-0.629925in}}{\pgfqpoint{2.414282in}{-0.625535in}}{\pgfqpoint{2.403232in}{-0.625535in}}%
\pgfpathcurveto{\pgfqpoint{2.392182in}{-0.625535in}}{\pgfqpoint{2.381583in}{-0.629925in}}{\pgfqpoint{2.373770in}{-0.637739in}}%
\pgfpathcurveto{\pgfqpoint{2.365956in}{-0.645553in}}{\pgfqpoint{2.361566in}{-0.656152in}}{\pgfqpoint{2.361566in}{-0.667202in}}%
\pgfpathcurveto{\pgfqpoint{2.361566in}{-0.678252in}}{\pgfqpoint{2.365956in}{-0.688851in}}{\pgfqpoint{2.373770in}{-0.696665in}}%
\pgfpathcurveto{\pgfqpoint{2.381583in}{-0.704478in}}{\pgfqpoint{2.392182in}{-0.708868in}}{\pgfqpoint{2.403232in}{-0.708868in}}%
\pgfpathclose%
\pgfusepath{stroke,fill}%
\end{pgfscope}%
\begin{pgfscope}%
\pgfpathrectangle{\pgfqpoint{0.800000in}{0.528000in}}{\pgfqpoint{4.960000in}{3.696000in}} %
\pgfusepath{clip}%
\pgfsetbuttcap%
\pgfsetroundjoin%
\definecolor{currentfill}{rgb}{0.000000,0.500000,0.000000}%
\pgfsetfillcolor{currentfill}%
\pgfsetlinewidth{1.003750pt}%
\definecolor{currentstroke}{rgb}{0.000000,0.500000,0.000000}%
\pgfsetstrokecolor{currentstroke}%
\pgfsetdash{}{0pt}%
\pgfpathmoveto{\pgfqpoint{2.453333in}{-0.228484in}}%
\pgfpathcurveto{\pgfqpoint{2.464383in}{-0.228484in}}{\pgfqpoint{2.474982in}{-0.224094in}}{\pgfqpoint{2.482796in}{-0.216280in}}%
\pgfpathcurveto{\pgfqpoint{2.490610in}{-0.208466in}}{\pgfqpoint{2.495000in}{-0.197867in}}{\pgfqpoint{2.495000in}{-0.186817in}}%
\pgfpathcurveto{\pgfqpoint{2.495000in}{-0.175767in}}{\pgfqpoint{2.490610in}{-0.165168in}}{\pgfqpoint{2.482796in}{-0.157354in}}%
\pgfpathcurveto{\pgfqpoint{2.474982in}{-0.149541in}}{\pgfqpoint{2.464383in}{-0.145151in}}{\pgfqpoint{2.453333in}{-0.145151in}}%
\pgfpathcurveto{\pgfqpoint{2.442283in}{-0.145151in}}{\pgfqpoint{2.431684in}{-0.149541in}}{\pgfqpoint{2.423871in}{-0.157354in}}%
\pgfpathcurveto{\pgfqpoint{2.416057in}{-0.165168in}}{\pgfqpoint{2.411667in}{-0.175767in}}{\pgfqpoint{2.411667in}{-0.186817in}}%
\pgfpathcurveto{\pgfqpoint{2.411667in}{-0.197867in}}{\pgfqpoint{2.416057in}{-0.208466in}}{\pgfqpoint{2.423871in}{-0.216280in}}%
\pgfpathcurveto{\pgfqpoint{2.431684in}{-0.224094in}}{\pgfqpoint{2.442283in}{-0.228484in}}{\pgfqpoint{2.453333in}{-0.228484in}}%
\pgfpathclose%
\pgfusepath{stroke,fill}%
\end{pgfscope}%
\begin{pgfscope}%
\pgfpathrectangle{\pgfqpoint{0.800000in}{0.528000in}}{\pgfqpoint{4.960000in}{3.696000in}} %
\pgfusepath{clip}%
\pgfsetbuttcap%
\pgfsetroundjoin%
\definecolor{currentfill}{rgb}{0.000000,0.500000,0.000000}%
\pgfsetfillcolor{currentfill}%
\pgfsetlinewidth{1.003750pt}%
\definecolor{currentstroke}{rgb}{0.000000,0.500000,0.000000}%
\pgfsetstrokecolor{currentstroke}%
\pgfsetdash{}{0pt}%
\pgfpathmoveto{\pgfqpoint{2.503434in}{-0.829136in}}%
\pgfpathcurveto{\pgfqpoint{2.514484in}{-0.829136in}}{\pgfqpoint{2.525084in}{-0.824746in}}{\pgfqpoint{2.532897in}{-0.816932in}}%
\pgfpathcurveto{\pgfqpoint{2.540711in}{-0.809119in}}{\pgfqpoint{2.545101in}{-0.798520in}}{\pgfqpoint{2.545101in}{-0.787469in}}%
\pgfpathcurveto{\pgfqpoint{2.545101in}{-0.776419in}}{\pgfqpoint{2.540711in}{-0.765820in}}{\pgfqpoint{2.532897in}{-0.758007in}}%
\pgfpathcurveto{\pgfqpoint{2.525084in}{-0.750193in}}{\pgfqpoint{2.514484in}{-0.745803in}}{\pgfqpoint{2.503434in}{-0.745803in}}%
\pgfpathcurveto{\pgfqpoint{2.492384in}{-0.745803in}}{\pgfqpoint{2.481785in}{-0.750193in}}{\pgfqpoint{2.473972in}{-0.758007in}}%
\pgfpathcurveto{\pgfqpoint{2.466158in}{-0.765820in}}{\pgfqpoint{2.461768in}{-0.776419in}}{\pgfqpoint{2.461768in}{-0.787469in}}%
\pgfpathcurveto{\pgfqpoint{2.461768in}{-0.798520in}}{\pgfqpoint{2.466158in}{-0.809119in}}{\pgfqpoint{2.473972in}{-0.816932in}}%
\pgfpathcurveto{\pgfqpoint{2.481785in}{-0.824746in}}{\pgfqpoint{2.492384in}{-0.829136in}}{\pgfqpoint{2.503434in}{-0.829136in}}%
\pgfpathclose%
\pgfusepath{stroke,fill}%
\end{pgfscope}%
\begin{pgfscope}%
\pgfpathrectangle{\pgfqpoint{0.800000in}{0.528000in}}{\pgfqpoint{4.960000in}{3.696000in}} %
\pgfusepath{clip}%
\pgfsetbuttcap%
\pgfsetroundjoin%
\definecolor{currentfill}{rgb}{0.000000,0.500000,0.000000}%
\pgfsetfillcolor{currentfill}%
\pgfsetlinewidth{1.003750pt}%
\definecolor{currentstroke}{rgb}{0.000000,0.500000,0.000000}%
\pgfsetstrokecolor{currentstroke}%
\pgfsetdash{}{0pt}%
\pgfpathmoveto{\pgfqpoint{2.553535in}{-0.687790in}}%
\pgfpathcurveto{\pgfqpoint{2.564585in}{-0.687790in}}{\pgfqpoint{2.575185in}{-0.683400in}}{\pgfqpoint{2.582998in}{-0.675586in}}%
\pgfpathcurveto{\pgfqpoint{2.590812in}{-0.667772in}}{\pgfqpoint{2.595202in}{-0.657173in}}{\pgfqpoint{2.595202in}{-0.646123in}}%
\pgfpathcurveto{\pgfqpoint{2.595202in}{-0.635073in}}{\pgfqpoint{2.590812in}{-0.624474in}}{\pgfqpoint{2.582998in}{-0.616660in}}%
\pgfpathcurveto{\pgfqpoint{2.575185in}{-0.608847in}}{\pgfqpoint{2.564585in}{-0.604457in}}{\pgfqpoint{2.553535in}{-0.604457in}}%
\pgfpathcurveto{\pgfqpoint{2.542485in}{-0.604457in}}{\pgfqpoint{2.531886in}{-0.608847in}}{\pgfqpoint{2.524073in}{-0.616660in}}%
\pgfpathcurveto{\pgfqpoint{2.516259in}{-0.624474in}}{\pgfqpoint{2.511869in}{-0.635073in}}{\pgfqpoint{2.511869in}{-0.646123in}}%
\pgfpathcurveto{\pgfqpoint{2.511869in}{-0.657173in}}{\pgfqpoint{2.516259in}{-0.667772in}}{\pgfqpoint{2.524073in}{-0.675586in}}%
\pgfpathcurveto{\pgfqpoint{2.531886in}{-0.683400in}}{\pgfqpoint{2.542485in}{-0.687790in}}{\pgfqpoint{2.553535in}{-0.687790in}}%
\pgfpathclose%
\pgfusepath{stroke,fill}%
\end{pgfscope}%
\begin{pgfscope}%
\pgfpathrectangle{\pgfqpoint{0.800000in}{0.528000in}}{\pgfqpoint{4.960000in}{3.696000in}} %
\pgfusepath{clip}%
\pgfsetbuttcap%
\pgfsetroundjoin%
\definecolor{currentfill}{rgb}{0.000000,0.500000,0.000000}%
\pgfsetfillcolor{currentfill}%
\pgfsetlinewidth{1.003750pt}%
\definecolor{currentstroke}{rgb}{0.000000,0.500000,0.000000}%
\pgfsetstrokecolor{currentstroke}%
\pgfsetdash{}{0pt}%
\pgfpathmoveto{\pgfqpoint{2.603636in}{1.149831in}}%
\pgfpathcurveto{\pgfqpoint{2.614686in}{1.149831in}}{\pgfqpoint{2.625286in}{1.154221in}}{\pgfqpoint{2.633099in}{1.162034in}}%
\pgfpathcurveto{\pgfqpoint{2.640913in}{1.169848in}}{\pgfqpoint{2.645303in}{1.180447in}}{\pgfqpoint{2.645303in}{1.191497in}}%
\pgfpathcurveto{\pgfqpoint{2.645303in}{1.202547in}}{\pgfqpoint{2.640913in}{1.213146in}}{\pgfqpoint{2.633099in}{1.220960in}}%
\pgfpathcurveto{\pgfqpoint{2.625286in}{1.228774in}}{\pgfqpoint{2.614686in}{1.233164in}}{\pgfqpoint{2.603636in}{1.233164in}}%
\pgfpathcurveto{\pgfqpoint{2.592586in}{1.233164in}}{\pgfqpoint{2.581987in}{1.228774in}}{\pgfqpoint{2.574174in}{1.220960in}}%
\pgfpathcurveto{\pgfqpoint{2.566360in}{1.213146in}}{\pgfqpoint{2.561970in}{1.202547in}}{\pgfqpoint{2.561970in}{1.191497in}}%
\pgfpathcurveto{\pgfqpoint{2.561970in}{1.180447in}}{\pgfqpoint{2.566360in}{1.169848in}}{\pgfqpoint{2.574174in}{1.162034in}}%
\pgfpathcurveto{\pgfqpoint{2.581987in}{1.154221in}}{\pgfqpoint{2.592586in}{1.149831in}}{\pgfqpoint{2.603636in}{1.149831in}}%
\pgfpathclose%
\pgfusepath{stroke,fill}%
\end{pgfscope}%
\begin{pgfscope}%
\pgfpathrectangle{\pgfqpoint{0.800000in}{0.528000in}}{\pgfqpoint{4.960000in}{3.696000in}} %
\pgfusepath{clip}%
\pgfsetbuttcap%
\pgfsetroundjoin%
\definecolor{currentfill}{rgb}{0.000000,0.500000,0.000000}%
\pgfsetfillcolor{currentfill}%
\pgfsetlinewidth{1.003750pt}%
\definecolor{currentstroke}{rgb}{0.000000,0.500000,0.000000}%
\pgfsetstrokecolor{currentstroke}%
\pgfsetdash{}{0pt}%
\pgfpathmoveto{\pgfqpoint{2.653737in}{-1.444467in}}%
\pgfpathcurveto{\pgfqpoint{2.664788in}{-1.444467in}}{\pgfqpoint{2.675387in}{-1.440077in}}{\pgfqpoint{2.683200in}{-1.432263in}}%
\pgfpathcurveto{\pgfqpoint{2.691014in}{-1.424450in}}{\pgfqpoint{2.695404in}{-1.413851in}}{\pgfqpoint{2.695404in}{-1.402801in}}%
\pgfpathcurveto{\pgfqpoint{2.695404in}{-1.391751in}}{\pgfqpoint{2.691014in}{-1.381152in}}{\pgfqpoint{2.683200in}{-1.373338in}}%
\pgfpathcurveto{\pgfqpoint{2.675387in}{-1.365524in}}{\pgfqpoint{2.664788in}{-1.361134in}}{\pgfqpoint{2.653737in}{-1.361134in}}%
\pgfpathcurveto{\pgfqpoint{2.642687in}{-1.361134in}}{\pgfqpoint{2.632088in}{-1.365524in}}{\pgfqpoint{2.624275in}{-1.373338in}}%
\pgfpathcurveto{\pgfqpoint{2.616461in}{-1.381152in}}{\pgfqpoint{2.612071in}{-1.391751in}}{\pgfqpoint{2.612071in}{-1.402801in}}%
\pgfpathcurveto{\pgfqpoint{2.612071in}{-1.413851in}}{\pgfqpoint{2.616461in}{-1.424450in}}{\pgfqpoint{2.624275in}{-1.432263in}}%
\pgfpathcurveto{\pgfqpoint{2.632088in}{-1.440077in}}{\pgfqpoint{2.642687in}{-1.444467in}}{\pgfqpoint{2.653737in}{-1.444467in}}%
\pgfpathclose%
\pgfusepath{stroke,fill}%
\end{pgfscope}%
\begin{pgfscope}%
\pgfpathrectangle{\pgfqpoint{0.800000in}{0.528000in}}{\pgfqpoint{4.960000in}{3.696000in}} %
\pgfusepath{clip}%
\pgfsetbuttcap%
\pgfsetroundjoin%
\definecolor{currentfill}{rgb}{0.000000,0.500000,0.000000}%
\pgfsetfillcolor{currentfill}%
\pgfsetlinewidth{1.003750pt}%
\definecolor{currentstroke}{rgb}{0.000000,0.500000,0.000000}%
\pgfsetstrokecolor{currentstroke}%
\pgfsetdash{}{0pt}%
\pgfpathmoveto{\pgfqpoint{2.703838in}{-2.981878in}}%
\pgfpathcurveto{\pgfqpoint{2.714889in}{-2.981878in}}{\pgfqpoint{2.725488in}{-2.977487in}}{\pgfqpoint{2.733301in}{-2.969674in}}%
\pgfpathcurveto{\pgfqpoint{2.741115in}{-2.961860in}}{\pgfqpoint{2.745505in}{-2.951261in}}{\pgfqpoint{2.745505in}{-2.940211in}}%
\pgfpathcurveto{\pgfqpoint{2.745505in}{-2.929161in}}{\pgfqpoint{2.741115in}{-2.918562in}}{\pgfqpoint{2.733301in}{-2.910748in}}%
\pgfpathcurveto{\pgfqpoint{2.725488in}{-2.902935in}}{\pgfqpoint{2.714889in}{-2.898544in}}{\pgfqpoint{2.703838in}{-2.898544in}}%
\pgfpathcurveto{\pgfqpoint{2.692788in}{-2.898544in}}{\pgfqpoint{2.682189in}{-2.902935in}}{\pgfqpoint{2.674376in}{-2.910748in}}%
\pgfpathcurveto{\pgfqpoint{2.666562in}{-2.918562in}}{\pgfqpoint{2.662172in}{-2.929161in}}{\pgfqpoint{2.662172in}{-2.940211in}}%
\pgfpathcurveto{\pgfqpoint{2.662172in}{-2.951261in}}{\pgfqpoint{2.666562in}{-2.961860in}}{\pgfqpoint{2.674376in}{-2.969674in}}%
\pgfpathcurveto{\pgfqpoint{2.682189in}{-2.977487in}}{\pgfqpoint{2.692788in}{-2.981878in}}{\pgfqpoint{2.703838in}{-2.981878in}}%
\pgfpathclose%
\pgfusepath{stroke,fill}%
\end{pgfscope}%
\begin{pgfscope}%
\pgfpathrectangle{\pgfqpoint{0.800000in}{0.528000in}}{\pgfqpoint{4.960000in}{3.696000in}} %
\pgfusepath{clip}%
\pgfsetbuttcap%
\pgfsetroundjoin%
\definecolor{currentfill}{rgb}{0.000000,0.500000,0.000000}%
\pgfsetfillcolor{currentfill}%
\pgfsetlinewidth{1.003750pt}%
\definecolor{currentstroke}{rgb}{0.000000,0.500000,0.000000}%
\pgfsetstrokecolor{currentstroke}%
\pgfsetdash{}{0pt}%
\pgfpathmoveto{\pgfqpoint{2.753939in}{-0.686957in}}%
\pgfpathcurveto{\pgfqpoint{2.764990in}{-0.686957in}}{\pgfqpoint{2.775589in}{-0.682567in}}{\pgfqpoint{2.783402in}{-0.674753in}}%
\pgfpathcurveto{\pgfqpoint{2.791216in}{-0.666939in}}{\pgfqpoint{2.795606in}{-0.656340in}}{\pgfqpoint{2.795606in}{-0.645290in}}%
\pgfpathcurveto{\pgfqpoint{2.795606in}{-0.634240in}}{\pgfqpoint{2.791216in}{-0.623641in}}{\pgfqpoint{2.783402in}{-0.615827in}}%
\pgfpathcurveto{\pgfqpoint{2.775589in}{-0.608014in}}{\pgfqpoint{2.764990in}{-0.603624in}}{\pgfqpoint{2.753939in}{-0.603624in}}%
\pgfpathcurveto{\pgfqpoint{2.742889in}{-0.603624in}}{\pgfqpoint{2.732290in}{-0.608014in}}{\pgfqpoint{2.724477in}{-0.615827in}}%
\pgfpathcurveto{\pgfqpoint{2.716663in}{-0.623641in}}{\pgfqpoint{2.712273in}{-0.634240in}}{\pgfqpoint{2.712273in}{-0.645290in}}%
\pgfpathcurveto{\pgfqpoint{2.712273in}{-0.656340in}}{\pgfqpoint{2.716663in}{-0.666939in}}{\pgfqpoint{2.724477in}{-0.674753in}}%
\pgfpathcurveto{\pgfqpoint{2.732290in}{-0.682567in}}{\pgfqpoint{2.742889in}{-0.686957in}}{\pgfqpoint{2.753939in}{-0.686957in}}%
\pgfpathclose%
\pgfusepath{stroke,fill}%
\end{pgfscope}%
\begin{pgfscope}%
\pgfpathrectangle{\pgfqpoint{0.800000in}{0.528000in}}{\pgfqpoint{4.960000in}{3.696000in}} %
\pgfusepath{clip}%
\pgfsetbuttcap%
\pgfsetroundjoin%
\definecolor{currentfill}{rgb}{0.000000,0.500000,0.000000}%
\pgfsetfillcolor{currentfill}%
\pgfsetlinewidth{1.003750pt}%
\definecolor{currentstroke}{rgb}{0.000000,0.500000,0.000000}%
\pgfsetstrokecolor{currentstroke}%
\pgfsetdash{}{0pt}%
\pgfpathmoveto{\pgfqpoint{2.804040in}{1.378506in}}%
\pgfpathcurveto{\pgfqpoint{2.815091in}{1.378506in}}{\pgfqpoint{2.825690in}{1.382896in}}{\pgfqpoint{2.833503in}{1.390710in}}%
\pgfpathcurveto{\pgfqpoint{2.841317in}{1.398523in}}{\pgfqpoint{2.845707in}{1.409122in}}{\pgfqpoint{2.845707in}{1.420172in}}%
\pgfpathcurveto{\pgfqpoint{2.845707in}{1.431223in}}{\pgfqpoint{2.841317in}{1.441822in}}{\pgfqpoint{2.833503in}{1.449635in}}%
\pgfpathcurveto{\pgfqpoint{2.825690in}{1.457449in}}{\pgfqpoint{2.815091in}{1.461839in}}{\pgfqpoint{2.804040in}{1.461839in}}%
\pgfpathcurveto{\pgfqpoint{2.792990in}{1.461839in}}{\pgfqpoint{2.782391in}{1.457449in}}{\pgfqpoint{2.774578in}{1.449635in}}%
\pgfpathcurveto{\pgfqpoint{2.766764in}{1.441822in}}{\pgfqpoint{2.762374in}{1.431223in}}{\pgfqpoint{2.762374in}{1.420172in}}%
\pgfpathcurveto{\pgfqpoint{2.762374in}{1.409122in}}{\pgfqpoint{2.766764in}{1.398523in}}{\pgfqpoint{2.774578in}{1.390710in}}%
\pgfpathcurveto{\pgfqpoint{2.782391in}{1.382896in}}{\pgfqpoint{2.792990in}{1.378506in}}{\pgfqpoint{2.804040in}{1.378506in}}%
\pgfpathclose%
\pgfusepath{stroke,fill}%
\end{pgfscope}%
\begin{pgfscope}%
\pgfpathrectangle{\pgfqpoint{0.800000in}{0.528000in}}{\pgfqpoint{4.960000in}{3.696000in}} %
\pgfusepath{clip}%
\pgfsetbuttcap%
\pgfsetroundjoin%
\definecolor{currentfill}{rgb}{0.000000,0.500000,0.000000}%
\pgfsetfillcolor{currentfill}%
\pgfsetlinewidth{1.003750pt}%
\definecolor{currentstroke}{rgb}{0.000000,0.500000,0.000000}%
\pgfsetstrokecolor{currentstroke}%
\pgfsetdash{}{0pt}%
\pgfpathmoveto{\pgfqpoint{2.854141in}{-0.991782in}}%
\pgfpathcurveto{\pgfqpoint{2.865192in}{-0.991782in}}{\pgfqpoint{2.875791in}{-0.987392in}}{\pgfqpoint{2.883604in}{-0.979578in}}%
\pgfpathcurveto{\pgfqpoint{2.891418in}{-0.971765in}}{\pgfqpoint{2.895808in}{-0.961166in}}{\pgfqpoint{2.895808in}{-0.950115in}}%
\pgfpathcurveto{\pgfqpoint{2.895808in}{-0.939065in}}{\pgfqpoint{2.891418in}{-0.928466in}}{\pgfqpoint{2.883604in}{-0.920653in}}%
\pgfpathcurveto{\pgfqpoint{2.875791in}{-0.912839in}}{\pgfqpoint{2.865192in}{-0.908449in}}{\pgfqpoint{2.854141in}{-0.908449in}}%
\pgfpathcurveto{\pgfqpoint{2.843091in}{-0.908449in}}{\pgfqpoint{2.832492in}{-0.912839in}}{\pgfqpoint{2.824679in}{-0.920653in}}%
\pgfpathcurveto{\pgfqpoint{2.816865in}{-0.928466in}}{\pgfqpoint{2.812475in}{-0.939065in}}{\pgfqpoint{2.812475in}{-0.950115in}}%
\pgfpathcurveto{\pgfqpoint{2.812475in}{-0.961166in}}{\pgfqpoint{2.816865in}{-0.971765in}}{\pgfqpoint{2.824679in}{-0.979578in}}%
\pgfpathcurveto{\pgfqpoint{2.832492in}{-0.987392in}}{\pgfqpoint{2.843091in}{-0.991782in}}{\pgfqpoint{2.854141in}{-0.991782in}}%
\pgfpathclose%
\pgfusepath{stroke,fill}%
\end{pgfscope}%
\begin{pgfscope}%
\pgfpathrectangle{\pgfqpoint{0.800000in}{0.528000in}}{\pgfqpoint{4.960000in}{3.696000in}} %
\pgfusepath{clip}%
\pgfsetbuttcap%
\pgfsetroundjoin%
\definecolor{currentfill}{rgb}{0.000000,0.500000,0.000000}%
\pgfsetfillcolor{currentfill}%
\pgfsetlinewidth{1.003750pt}%
\definecolor{currentstroke}{rgb}{0.000000,0.500000,0.000000}%
\pgfsetstrokecolor{currentstroke}%
\pgfsetdash{}{0pt}%
\pgfpathmoveto{\pgfqpoint{2.904242in}{-1.786507in}}%
\pgfpathcurveto{\pgfqpoint{2.915293in}{-1.786507in}}{\pgfqpoint{2.925892in}{-1.782117in}}{\pgfqpoint{2.933705in}{-1.774303in}}%
\pgfpathcurveto{\pgfqpoint{2.941519in}{-1.766489in}}{\pgfqpoint{2.945909in}{-1.755890in}}{\pgfqpoint{2.945909in}{-1.744840in}}%
\pgfpathcurveto{\pgfqpoint{2.945909in}{-1.733790in}}{\pgfqpoint{2.941519in}{-1.723191in}}{\pgfqpoint{2.933705in}{-1.715377in}}%
\pgfpathcurveto{\pgfqpoint{2.925892in}{-1.707564in}}{\pgfqpoint{2.915293in}{-1.703174in}}{\pgfqpoint{2.904242in}{-1.703174in}}%
\pgfpathcurveto{\pgfqpoint{2.893192in}{-1.703174in}}{\pgfqpoint{2.882593in}{-1.707564in}}{\pgfqpoint{2.874780in}{-1.715377in}}%
\pgfpathcurveto{\pgfqpoint{2.866966in}{-1.723191in}}{\pgfqpoint{2.862576in}{-1.733790in}}{\pgfqpoint{2.862576in}{-1.744840in}}%
\pgfpathcurveto{\pgfqpoint{2.862576in}{-1.755890in}}{\pgfqpoint{2.866966in}{-1.766489in}}{\pgfqpoint{2.874780in}{-1.774303in}}%
\pgfpathcurveto{\pgfqpoint{2.882593in}{-1.782117in}}{\pgfqpoint{2.893192in}{-1.786507in}}{\pgfqpoint{2.904242in}{-1.786507in}}%
\pgfpathclose%
\pgfusepath{stroke,fill}%
\end{pgfscope}%
\begin{pgfscope}%
\pgfpathrectangle{\pgfqpoint{0.800000in}{0.528000in}}{\pgfqpoint{4.960000in}{3.696000in}} %
\pgfusepath{clip}%
\pgfsetbuttcap%
\pgfsetroundjoin%
\definecolor{currentfill}{rgb}{0.000000,0.500000,0.000000}%
\pgfsetfillcolor{currentfill}%
\pgfsetlinewidth{1.003750pt}%
\definecolor{currentstroke}{rgb}{0.000000,0.500000,0.000000}%
\pgfsetstrokecolor{currentstroke}%
\pgfsetdash{}{0pt}%
\pgfpathmoveto{\pgfqpoint{2.954343in}{1.508471in}}%
\pgfpathcurveto{\pgfqpoint{2.965394in}{1.508471in}}{\pgfqpoint{2.975993in}{1.512862in}}{\pgfqpoint{2.983806in}{1.520675in}}%
\pgfpathcurveto{\pgfqpoint{2.991620in}{1.528489in}}{\pgfqpoint{2.996010in}{1.539088in}}{\pgfqpoint{2.996010in}{1.550138in}}%
\pgfpathcurveto{\pgfqpoint{2.996010in}{1.561188in}}{\pgfqpoint{2.991620in}{1.571787in}}{\pgfqpoint{2.983806in}{1.579601in}}%
\pgfpathcurveto{\pgfqpoint{2.975993in}{1.587414in}}{\pgfqpoint{2.965394in}{1.591805in}}{\pgfqpoint{2.954343in}{1.591805in}}%
\pgfpathcurveto{\pgfqpoint{2.943293in}{1.591805in}}{\pgfqpoint{2.932694in}{1.587414in}}{\pgfqpoint{2.924881in}{1.579601in}}%
\pgfpathcurveto{\pgfqpoint{2.917067in}{1.571787in}}{\pgfqpoint{2.912677in}{1.561188in}}{\pgfqpoint{2.912677in}{1.550138in}}%
\pgfpathcurveto{\pgfqpoint{2.912677in}{1.539088in}}{\pgfqpoint{2.917067in}{1.528489in}}{\pgfqpoint{2.924881in}{1.520675in}}%
\pgfpathcurveto{\pgfqpoint{2.932694in}{1.512862in}}{\pgfqpoint{2.943293in}{1.508471in}}{\pgfqpoint{2.954343in}{1.508471in}}%
\pgfpathclose%
\pgfusepath{stroke,fill}%
\end{pgfscope}%
\begin{pgfscope}%
\pgfpathrectangle{\pgfqpoint{0.800000in}{0.528000in}}{\pgfqpoint{4.960000in}{3.696000in}} %
\pgfusepath{clip}%
\pgfsetbuttcap%
\pgfsetroundjoin%
\definecolor{currentfill}{rgb}{0.000000,0.500000,0.000000}%
\pgfsetfillcolor{currentfill}%
\pgfsetlinewidth{1.003750pt}%
\definecolor{currentstroke}{rgb}{0.000000,0.500000,0.000000}%
\pgfsetstrokecolor{currentstroke}%
\pgfsetdash{}{0pt}%
\pgfpathmoveto{\pgfqpoint{3.004444in}{1.486842in}}%
\pgfpathcurveto{\pgfqpoint{3.015495in}{1.486842in}}{\pgfqpoint{3.026094in}{1.491232in}}{\pgfqpoint{3.033907in}{1.499046in}}%
\pgfpathcurveto{\pgfqpoint{3.041721in}{1.506860in}}{\pgfqpoint{3.046111in}{1.517459in}}{\pgfqpoint{3.046111in}{1.528509in}}%
\pgfpathcurveto{\pgfqpoint{3.046111in}{1.539559in}}{\pgfqpoint{3.041721in}{1.550158in}}{\pgfqpoint{3.033907in}{1.557972in}}%
\pgfpathcurveto{\pgfqpoint{3.026094in}{1.565785in}}{\pgfqpoint{3.015495in}{1.570175in}}{\pgfqpoint{3.004444in}{1.570175in}}%
\pgfpathcurveto{\pgfqpoint{2.993394in}{1.570175in}}{\pgfqpoint{2.982795in}{1.565785in}}{\pgfqpoint{2.974982in}{1.557972in}}%
\pgfpathcurveto{\pgfqpoint{2.967168in}{1.550158in}}{\pgfqpoint{2.962778in}{1.539559in}}{\pgfqpoint{2.962778in}{1.528509in}}%
\pgfpathcurveto{\pgfqpoint{2.962778in}{1.517459in}}{\pgfqpoint{2.967168in}{1.506860in}}{\pgfqpoint{2.974982in}{1.499046in}}%
\pgfpathcurveto{\pgfqpoint{2.982795in}{1.491232in}}{\pgfqpoint{2.993394in}{1.486842in}}{\pgfqpoint{3.004444in}{1.486842in}}%
\pgfpathclose%
\pgfusepath{stroke,fill}%
\end{pgfscope}%
\begin{pgfscope}%
\pgfpathrectangle{\pgfqpoint{0.800000in}{0.528000in}}{\pgfqpoint{4.960000in}{3.696000in}} %
\pgfusepath{clip}%
\pgfsetbuttcap%
\pgfsetroundjoin%
\definecolor{currentfill}{rgb}{0.000000,0.500000,0.000000}%
\pgfsetfillcolor{currentfill}%
\pgfsetlinewidth{1.003750pt}%
\definecolor{currentstroke}{rgb}{0.000000,0.500000,0.000000}%
\pgfsetstrokecolor{currentstroke}%
\pgfsetdash{}{0pt}%
\pgfpathmoveto{\pgfqpoint{3.054545in}{-2.846312in}}%
\pgfpathcurveto{\pgfqpoint{3.065596in}{-2.846312in}}{\pgfqpoint{3.076195in}{-2.841922in}}{\pgfqpoint{3.084008in}{-2.834109in}}%
\pgfpathcurveto{\pgfqpoint{3.091822in}{-2.826295in}}{\pgfqpoint{3.096212in}{-2.815696in}}{\pgfqpoint{3.096212in}{-2.804646in}}%
\pgfpathcurveto{\pgfqpoint{3.096212in}{-2.793596in}}{\pgfqpoint{3.091822in}{-2.782997in}}{\pgfqpoint{3.084008in}{-2.775183in}}%
\pgfpathcurveto{\pgfqpoint{3.076195in}{-2.767369in}}{\pgfqpoint{3.065596in}{-2.762979in}}{\pgfqpoint{3.054545in}{-2.762979in}}%
\pgfpathcurveto{\pgfqpoint{3.043495in}{-2.762979in}}{\pgfqpoint{3.032896in}{-2.767369in}}{\pgfqpoint{3.025083in}{-2.775183in}}%
\pgfpathcurveto{\pgfqpoint{3.017269in}{-2.782997in}}{\pgfqpoint{3.012879in}{-2.793596in}}{\pgfqpoint{3.012879in}{-2.804646in}}%
\pgfpathcurveto{\pgfqpoint{3.012879in}{-2.815696in}}{\pgfqpoint{3.017269in}{-2.826295in}}{\pgfqpoint{3.025083in}{-2.834109in}}%
\pgfpathcurveto{\pgfqpoint{3.032896in}{-2.841922in}}{\pgfqpoint{3.043495in}{-2.846312in}}{\pgfqpoint{3.054545in}{-2.846312in}}%
\pgfpathclose%
\pgfusepath{stroke,fill}%
\end{pgfscope}%
\begin{pgfscope}%
\pgfpathrectangle{\pgfqpoint{0.800000in}{0.528000in}}{\pgfqpoint{4.960000in}{3.696000in}} %
\pgfusepath{clip}%
\pgfsetbuttcap%
\pgfsetroundjoin%
\definecolor{currentfill}{rgb}{0.000000,0.500000,0.000000}%
\pgfsetfillcolor{currentfill}%
\pgfsetlinewidth{1.003750pt}%
\definecolor{currentstroke}{rgb}{0.000000,0.500000,0.000000}%
\pgfsetstrokecolor{currentstroke}%
\pgfsetdash{}{0pt}%
\pgfpathmoveto{\pgfqpoint{3.104646in}{1.595799in}}%
\pgfpathcurveto{\pgfqpoint{3.115697in}{1.595799in}}{\pgfqpoint{3.126296in}{1.600189in}}{\pgfqpoint{3.134109in}{1.608002in}}%
\pgfpathcurveto{\pgfqpoint{3.141923in}{1.615816in}}{\pgfqpoint{3.146313in}{1.626415in}}{\pgfqpoint{3.146313in}{1.637465in}}%
\pgfpathcurveto{\pgfqpoint{3.146313in}{1.648515in}}{\pgfqpoint{3.141923in}{1.659114in}}{\pgfqpoint{3.134109in}{1.666928in}}%
\pgfpathcurveto{\pgfqpoint{3.126296in}{1.674742in}}{\pgfqpoint{3.115697in}{1.679132in}}{\pgfqpoint{3.104646in}{1.679132in}}%
\pgfpathcurveto{\pgfqpoint{3.093596in}{1.679132in}}{\pgfqpoint{3.082997in}{1.674742in}}{\pgfqpoint{3.075184in}{1.666928in}}%
\pgfpathcurveto{\pgfqpoint{3.067370in}{1.659114in}}{\pgfqpoint{3.062980in}{1.648515in}}{\pgfqpoint{3.062980in}{1.637465in}}%
\pgfpathcurveto{\pgfqpoint{3.062980in}{1.626415in}}{\pgfqpoint{3.067370in}{1.615816in}}{\pgfqpoint{3.075184in}{1.608002in}}%
\pgfpathcurveto{\pgfqpoint{3.082997in}{1.600189in}}{\pgfqpoint{3.093596in}{1.595799in}}{\pgfqpoint{3.104646in}{1.595799in}}%
\pgfpathclose%
\pgfusepath{stroke,fill}%
\end{pgfscope}%
\begin{pgfscope}%
\pgfpathrectangle{\pgfqpoint{0.800000in}{0.528000in}}{\pgfqpoint{4.960000in}{3.696000in}} %
\pgfusepath{clip}%
\pgfsetbuttcap%
\pgfsetroundjoin%
\definecolor{currentfill}{rgb}{0.000000,0.500000,0.000000}%
\pgfsetfillcolor{currentfill}%
\pgfsetlinewidth{1.003750pt}%
\definecolor{currentstroke}{rgb}{0.000000,0.500000,0.000000}%
\pgfsetstrokecolor{currentstroke}%
\pgfsetdash{}{0pt}%
\pgfpathmoveto{\pgfqpoint{3.154747in}{-0.943956in}}%
\pgfpathcurveto{\pgfqpoint{3.165798in}{-0.943956in}}{\pgfqpoint{3.176397in}{-0.939565in}}{\pgfqpoint{3.184210in}{-0.931752in}}%
\pgfpathcurveto{\pgfqpoint{3.192024in}{-0.923938in}}{\pgfqpoint{3.196414in}{-0.913339in}}{\pgfqpoint{3.196414in}{-0.902289in}}%
\pgfpathcurveto{\pgfqpoint{3.196414in}{-0.891239in}}{\pgfqpoint{3.192024in}{-0.880640in}}{\pgfqpoint{3.184210in}{-0.872826in}}%
\pgfpathcurveto{\pgfqpoint{3.176397in}{-0.865012in}}{\pgfqpoint{3.165798in}{-0.860622in}}{\pgfqpoint{3.154747in}{-0.860622in}}%
\pgfpathcurveto{\pgfqpoint{3.143697in}{-0.860622in}}{\pgfqpoint{3.133098in}{-0.865012in}}{\pgfqpoint{3.125285in}{-0.872826in}}%
\pgfpathcurveto{\pgfqpoint{3.117471in}{-0.880640in}}{\pgfqpoint{3.113081in}{-0.891239in}}{\pgfqpoint{3.113081in}{-0.902289in}}%
\pgfpathcurveto{\pgfqpoint{3.113081in}{-0.913339in}}{\pgfqpoint{3.117471in}{-0.923938in}}{\pgfqpoint{3.125285in}{-0.931752in}}%
\pgfpathcurveto{\pgfqpoint{3.133098in}{-0.939565in}}{\pgfqpoint{3.143697in}{-0.943956in}}{\pgfqpoint{3.154747in}{-0.943956in}}%
\pgfpathclose%
\pgfusepath{stroke,fill}%
\end{pgfscope}%
\begin{pgfscope}%
\pgfpathrectangle{\pgfqpoint{0.800000in}{0.528000in}}{\pgfqpoint{4.960000in}{3.696000in}} %
\pgfusepath{clip}%
\pgfsetbuttcap%
\pgfsetroundjoin%
\definecolor{currentfill}{rgb}{0.000000,0.500000,0.000000}%
\pgfsetfillcolor{currentfill}%
\pgfsetlinewidth{1.003750pt}%
\definecolor{currentstroke}{rgb}{0.000000,0.500000,0.000000}%
\pgfsetstrokecolor{currentstroke}%
\pgfsetdash{}{0pt}%
\pgfpathmoveto{\pgfqpoint{3.204848in}{-1.671124in}}%
\pgfpathcurveto{\pgfqpoint{3.215899in}{-1.671124in}}{\pgfqpoint{3.226498in}{-1.666734in}}{\pgfqpoint{3.234311in}{-1.658920in}}%
\pgfpathcurveto{\pgfqpoint{3.242125in}{-1.651107in}}{\pgfqpoint{3.246515in}{-1.640508in}}{\pgfqpoint{3.246515in}{-1.629458in}}%
\pgfpathcurveto{\pgfqpoint{3.246515in}{-1.618408in}}{\pgfqpoint{3.242125in}{-1.607809in}}{\pgfqpoint{3.234311in}{-1.599995in}}%
\pgfpathcurveto{\pgfqpoint{3.226498in}{-1.592181in}}{\pgfqpoint{3.215899in}{-1.587791in}}{\pgfqpoint{3.204848in}{-1.587791in}}%
\pgfpathcurveto{\pgfqpoint{3.193798in}{-1.587791in}}{\pgfqpoint{3.183199in}{-1.592181in}}{\pgfqpoint{3.175386in}{-1.599995in}}%
\pgfpathcurveto{\pgfqpoint{3.167572in}{-1.607809in}}{\pgfqpoint{3.163182in}{-1.618408in}}{\pgfqpoint{3.163182in}{-1.629458in}}%
\pgfpathcurveto{\pgfqpoint{3.163182in}{-1.640508in}}{\pgfqpoint{3.167572in}{-1.651107in}}{\pgfqpoint{3.175386in}{-1.658920in}}%
\pgfpathcurveto{\pgfqpoint{3.183199in}{-1.666734in}}{\pgfqpoint{3.193798in}{-1.671124in}}{\pgfqpoint{3.204848in}{-1.671124in}}%
\pgfpathclose%
\pgfusepath{stroke,fill}%
\end{pgfscope}%
\begin{pgfscope}%
\pgfpathrectangle{\pgfqpoint{0.800000in}{0.528000in}}{\pgfqpoint{4.960000in}{3.696000in}} %
\pgfusepath{clip}%
\pgfsetbuttcap%
\pgfsetroundjoin%
\definecolor{currentfill}{rgb}{0.000000,0.500000,0.000000}%
\pgfsetfillcolor{currentfill}%
\pgfsetlinewidth{1.003750pt}%
\definecolor{currentstroke}{rgb}{0.000000,0.500000,0.000000}%
\pgfsetstrokecolor{currentstroke}%
\pgfsetdash{}{0pt}%
\pgfpathmoveto{\pgfqpoint{3.254949in}{1.103529in}}%
\pgfpathcurveto{\pgfqpoint{3.266000in}{1.103529in}}{\pgfqpoint{3.276599in}{1.107920in}}{\pgfqpoint{3.284412in}{1.115733in}}%
\pgfpathcurveto{\pgfqpoint{3.292226in}{1.123547in}}{\pgfqpoint{3.296616in}{1.134146in}}{\pgfqpoint{3.296616in}{1.145196in}}%
\pgfpathcurveto{\pgfqpoint{3.296616in}{1.156246in}}{\pgfqpoint{3.292226in}{1.166845in}}{\pgfqpoint{3.284412in}{1.174659in}}%
\pgfpathcurveto{\pgfqpoint{3.276599in}{1.182472in}}{\pgfqpoint{3.266000in}{1.186863in}}{\pgfqpoint{3.254949in}{1.186863in}}%
\pgfpathcurveto{\pgfqpoint{3.243899in}{1.186863in}}{\pgfqpoint{3.233300in}{1.182472in}}{\pgfqpoint{3.225487in}{1.174659in}}%
\pgfpathcurveto{\pgfqpoint{3.217673in}{1.166845in}}{\pgfqpoint{3.213283in}{1.156246in}}{\pgfqpoint{3.213283in}{1.145196in}}%
\pgfpathcurveto{\pgfqpoint{3.213283in}{1.134146in}}{\pgfqpoint{3.217673in}{1.123547in}}{\pgfqpoint{3.225487in}{1.115733in}}%
\pgfpathcurveto{\pgfqpoint{3.233300in}{1.107920in}}{\pgfqpoint{3.243899in}{1.103529in}}{\pgfqpoint{3.254949in}{1.103529in}}%
\pgfpathclose%
\pgfusepath{stroke,fill}%
\end{pgfscope}%
\begin{pgfscope}%
\pgfpathrectangle{\pgfqpoint{0.800000in}{0.528000in}}{\pgfqpoint{4.960000in}{3.696000in}} %
\pgfusepath{clip}%
\pgfsetbuttcap%
\pgfsetroundjoin%
\definecolor{currentfill}{rgb}{0.000000,0.500000,0.000000}%
\pgfsetfillcolor{currentfill}%
\pgfsetlinewidth{1.003750pt}%
\definecolor{currentstroke}{rgb}{0.000000,0.500000,0.000000}%
\pgfsetstrokecolor{currentstroke}%
\pgfsetdash{}{0pt}%
\pgfpathmoveto{\pgfqpoint{3.305051in}{-0.956379in}}%
\pgfpathcurveto{\pgfqpoint{3.316101in}{-0.956379in}}{\pgfqpoint{3.326700in}{-0.951989in}}{\pgfqpoint{3.334513in}{-0.944175in}}%
\pgfpathcurveto{\pgfqpoint{3.342327in}{-0.936361in}}{\pgfqpoint{3.346717in}{-0.925762in}}{\pgfqpoint{3.346717in}{-0.914712in}}%
\pgfpathcurveto{\pgfqpoint{3.346717in}{-0.903662in}}{\pgfqpoint{3.342327in}{-0.893063in}}{\pgfqpoint{3.334513in}{-0.885249in}}%
\pgfpathcurveto{\pgfqpoint{3.326700in}{-0.877436in}}{\pgfqpoint{3.316101in}{-0.873046in}}{\pgfqpoint{3.305051in}{-0.873046in}}%
\pgfpathcurveto{\pgfqpoint{3.294000in}{-0.873046in}}{\pgfqpoint{3.283401in}{-0.877436in}}{\pgfqpoint{3.275588in}{-0.885249in}}%
\pgfpathcurveto{\pgfqpoint{3.267774in}{-0.893063in}}{\pgfqpoint{3.263384in}{-0.903662in}}{\pgfqpoint{3.263384in}{-0.914712in}}%
\pgfpathcurveto{\pgfqpoint{3.263384in}{-0.925762in}}{\pgfqpoint{3.267774in}{-0.936361in}}{\pgfqpoint{3.275588in}{-0.944175in}}%
\pgfpathcurveto{\pgfqpoint{3.283401in}{-0.951989in}}{\pgfqpoint{3.294000in}{-0.956379in}}{\pgfqpoint{3.305051in}{-0.956379in}}%
\pgfpathclose%
\pgfusepath{stroke,fill}%
\end{pgfscope}%
\begin{pgfscope}%
\pgfpathrectangle{\pgfqpoint{0.800000in}{0.528000in}}{\pgfqpoint{4.960000in}{3.696000in}} %
\pgfusepath{clip}%
\pgfsetbuttcap%
\pgfsetroundjoin%
\definecolor{currentfill}{rgb}{0.000000,0.500000,0.000000}%
\pgfsetfillcolor{currentfill}%
\pgfsetlinewidth{1.003750pt}%
\definecolor{currentstroke}{rgb}{0.000000,0.500000,0.000000}%
\pgfsetstrokecolor{currentstroke}%
\pgfsetdash{}{0pt}%
\pgfpathmoveto{\pgfqpoint{3.355152in}{0.766928in}}%
\pgfpathcurveto{\pgfqpoint{3.366202in}{0.766928in}}{\pgfqpoint{3.376801in}{0.771319in}}{\pgfqpoint{3.384614in}{0.779132in}}%
\pgfpathcurveto{\pgfqpoint{3.392428in}{0.786946in}}{\pgfqpoint{3.396818in}{0.797545in}}{\pgfqpoint{3.396818in}{0.808595in}}%
\pgfpathcurveto{\pgfqpoint{3.396818in}{0.819645in}}{\pgfqpoint{3.392428in}{0.830244in}}{\pgfqpoint{3.384614in}{0.838058in}}%
\pgfpathcurveto{\pgfqpoint{3.376801in}{0.845872in}}{\pgfqpoint{3.366202in}{0.850262in}}{\pgfqpoint{3.355152in}{0.850262in}}%
\pgfpathcurveto{\pgfqpoint{3.344101in}{0.850262in}}{\pgfqpoint{3.333502in}{0.845872in}}{\pgfqpoint{3.325689in}{0.838058in}}%
\pgfpathcurveto{\pgfqpoint{3.317875in}{0.830244in}}{\pgfqpoint{3.313485in}{0.819645in}}{\pgfqpoint{3.313485in}{0.808595in}}%
\pgfpathcurveto{\pgfqpoint{3.313485in}{0.797545in}}{\pgfqpoint{3.317875in}{0.786946in}}{\pgfqpoint{3.325689in}{0.779132in}}%
\pgfpathcurveto{\pgfqpoint{3.333502in}{0.771319in}}{\pgfqpoint{3.344101in}{0.766928in}}{\pgfqpoint{3.355152in}{0.766928in}}%
\pgfpathclose%
\pgfusepath{stroke,fill}%
\end{pgfscope}%
\begin{pgfscope}%
\pgfpathrectangle{\pgfqpoint{0.800000in}{0.528000in}}{\pgfqpoint{4.960000in}{3.696000in}} %
\pgfusepath{clip}%
\pgfsetbuttcap%
\pgfsetroundjoin%
\definecolor{currentfill}{rgb}{0.000000,0.500000,0.000000}%
\pgfsetfillcolor{currentfill}%
\pgfsetlinewidth{1.003750pt}%
\definecolor{currentstroke}{rgb}{0.000000,0.500000,0.000000}%
\pgfsetstrokecolor{currentstroke}%
\pgfsetdash{}{0pt}%
\pgfpathmoveto{\pgfqpoint{3.405253in}{1.733081in}}%
\pgfpathcurveto{\pgfqpoint{3.416303in}{1.733081in}}{\pgfqpoint{3.426902in}{1.737472in}}{\pgfqpoint{3.434715in}{1.745285in}}%
\pgfpathcurveto{\pgfqpoint{3.442529in}{1.753099in}}{\pgfqpoint{3.446919in}{1.763698in}}{\pgfqpoint{3.446919in}{1.774748in}}%
\pgfpathcurveto{\pgfqpoint{3.446919in}{1.785798in}}{\pgfqpoint{3.442529in}{1.796397in}}{\pgfqpoint{3.434715in}{1.804211in}}%
\pgfpathcurveto{\pgfqpoint{3.426902in}{1.812024in}}{\pgfqpoint{3.416303in}{1.816415in}}{\pgfqpoint{3.405253in}{1.816415in}}%
\pgfpathcurveto{\pgfqpoint{3.394202in}{1.816415in}}{\pgfqpoint{3.383603in}{1.812024in}}{\pgfqpoint{3.375790in}{1.804211in}}%
\pgfpathcurveto{\pgfqpoint{3.367976in}{1.796397in}}{\pgfqpoint{3.363586in}{1.785798in}}{\pgfqpoint{3.363586in}{1.774748in}}%
\pgfpathcurveto{\pgfqpoint{3.363586in}{1.763698in}}{\pgfqpoint{3.367976in}{1.753099in}}{\pgfqpoint{3.375790in}{1.745285in}}%
\pgfpathcurveto{\pgfqpoint{3.383603in}{1.737472in}}{\pgfqpoint{3.394202in}{1.733081in}}{\pgfqpoint{3.405253in}{1.733081in}}%
\pgfpathclose%
\pgfusepath{stroke,fill}%
\end{pgfscope}%
\begin{pgfscope}%
\pgfpathrectangle{\pgfqpoint{0.800000in}{0.528000in}}{\pgfqpoint{4.960000in}{3.696000in}} %
\pgfusepath{clip}%
\pgfsetbuttcap%
\pgfsetroundjoin%
\definecolor{currentfill}{rgb}{0.000000,0.500000,0.000000}%
\pgfsetfillcolor{currentfill}%
\pgfsetlinewidth{1.003750pt}%
\definecolor{currentstroke}{rgb}{0.000000,0.500000,0.000000}%
\pgfsetstrokecolor{currentstroke}%
\pgfsetdash{}{0pt}%
\pgfpathmoveto{\pgfqpoint{3.455354in}{-0.976987in}}%
\pgfpathcurveto{\pgfqpoint{3.466404in}{-0.976987in}}{\pgfqpoint{3.477003in}{-0.972597in}}{\pgfqpoint{3.484816in}{-0.964783in}}%
\pgfpathcurveto{\pgfqpoint{3.492630in}{-0.956969in}}{\pgfqpoint{3.497020in}{-0.946370in}}{\pgfqpoint{3.497020in}{-0.935320in}}%
\pgfpathcurveto{\pgfqpoint{3.497020in}{-0.924270in}}{\pgfqpoint{3.492630in}{-0.913671in}}{\pgfqpoint{3.484816in}{-0.905858in}}%
\pgfpathcurveto{\pgfqpoint{3.477003in}{-0.898044in}}{\pgfqpoint{3.466404in}{-0.893654in}}{\pgfqpoint{3.455354in}{-0.893654in}}%
\pgfpathcurveto{\pgfqpoint{3.444303in}{-0.893654in}}{\pgfqpoint{3.433704in}{-0.898044in}}{\pgfqpoint{3.425891in}{-0.905858in}}%
\pgfpathcurveto{\pgfqpoint{3.418077in}{-0.913671in}}{\pgfqpoint{3.413687in}{-0.924270in}}{\pgfqpoint{3.413687in}{-0.935320in}}%
\pgfpathcurveto{\pgfqpoint{3.413687in}{-0.946370in}}{\pgfqpoint{3.418077in}{-0.956969in}}{\pgfqpoint{3.425891in}{-0.964783in}}%
\pgfpathcurveto{\pgfqpoint{3.433704in}{-0.972597in}}{\pgfqpoint{3.444303in}{-0.976987in}}{\pgfqpoint{3.455354in}{-0.976987in}}%
\pgfpathclose%
\pgfusepath{stroke,fill}%
\end{pgfscope}%
\begin{pgfscope}%
\pgfpathrectangle{\pgfqpoint{0.800000in}{0.528000in}}{\pgfqpoint{4.960000in}{3.696000in}} %
\pgfusepath{clip}%
\pgfsetbuttcap%
\pgfsetroundjoin%
\definecolor{currentfill}{rgb}{0.000000,0.500000,0.000000}%
\pgfsetfillcolor{currentfill}%
\pgfsetlinewidth{1.003750pt}%
\definecolor{currentstroke}{rgb}{0.000000,0.500000,0.000000}%
\pgfsetstrokecolor{currentstroke}%
\pgfsetdash{}{0pt}%
\pgfpathmoveto{\pgfqpoint{3.505455in}{1.249754in}}%
\pgfpathcurveto{\pgfqpoint{3.516505in}{1.249754in}}{\pgfqpoint{3.527104in}{1.254144in}}{\pgfqpoint{3.534917in}{1.261958in}}%
\pgfpathcurveto{\pgfqpoint{3.542731in}{1.269771in}}{\pgfqpoint{3.547121in}{1.280370in}}{\pgfqpoint{3.547121in}{1.291420in}}%
\pgfpathcurveto{\pgfqpoint{3.547121in}{1.302471in}}{\pgfqpoint{3.542731in}{1.313070in}}{\pgfqpoint{3.534917in}{1.320883in}}%
\pgfpathcurveto{\pgfqpoint{3.527104in}{1.328697in}}{\pgfqpoint{3.516505in}{1.333087in}}{\pgfqpoint{3.505455in}{1.333087in}}%
\pgfpathcurveto{\pgfqpoint{3.494404in}{1.333087in}}{\pgfqpoint{3.483805in}{1.328697in}}{\pgfqpoint{3.475992in}{1.320883in}}%
\pgfpathcurveto{\pgfqpoint{3.468178in}{1.313070in}}{\pgfqpoint{3.463788in}{1.302471in}}{\pgfqpoint{3.463788in}{1.291420in}}%
\pgfpathcurveto{\pgfqpoint{3.463788in}{1.280370in}}{\pgfqpoint{3.468178in}{1.269771in}}{\pgfqpoint{3.475992in}{1.261958in}}%
\pgfpathcurveto{\pgfqpoint{3.483805in}{1.254144in}}{\pgfqpoint{3.494404in}{1.249754in}}{\pgfqpoint{3.505455in}{1.249754in}}%
\pgfpathclose%
\pgfusepath{stroke,fill}%
\end{pgfscope}%
\begin{pgfscope}%
\pgfpathrectangle{\pgfqpoint{0.800000in}{0.528000in}}{\pgfqpoint{4.960000in}{3.696000in}} %
\pgfusepath{clip}%
\pgfsetbuttcap%
\pgfsetroundjoin%
\definecolor{currentfill}{rgb}{0.000000,0.500000,0.000000}%
\pgfsetfillcolor{currentfill}%
\pgfsetlinewidth{1.003750pt}%
\definecolor{currentstroke}{rgb}{0.000000,0.500000,0.000000}%
\pgfsetstrokecolor{currentstroke}%
\pgfsetdash{}{0pt}%
\pgfpathmoveto{\pgfqpoint{3.555556in}{2.326571in}}%
\pgfpathcurveto{\pgfqpoint{3.566606in}{2.326571in}}{\pgfqpoint{3.577205in}{2.330962in}}{\pgfqpoint{3.585018in}{2.338775in}}%
\pgfpathcurveto{\pgfqpoint{3.592832in}{2.346589in}}{\pgfqpoint{3.597222in}{2.357188in}}{\pgfqpoint{3.597222in}{2.368238in}}%
\pgfpathcurveto{\pgfqpoint{3.597222in}{2.379288in}}{\pgfqpoint{3.592832in}{2.389887in}}{\pgfqpoint{3.585018in}{2.397701in}}%
\pgfpathcurveto{\pgfqpoint{3.577205in}{2.405514in}}{\pgfqpoint{3.566606in}{2.409905in}}{\pgfqpoint{3.555556in}{2.409905in}}%
\pgfpathcurveto{\pgfqpoint{3.544505in}{2.409905in}}{\pgfqpoint{3.533906in}{2.405514in}}{\pgfqpoint{3.526093in}{2.397701in}}%
\pgfpathcurveto{\pgfqpoint{3.518279in}{2.389887in}}{\pgfqpoint{3.513889in}{2.379288in}}{\pgfqpoint{3.513889in}{2.368238in}}%
\pgfpathcurveto{\pgfqpoint{3.513889in}{2.357188in}}{\pgfqpoint{3.518279in}{2.346589in}}{\pgfqpoint{3.526093in}{2.338775in}}%
\pgfpathcurveto{\pgfqpoint{3.533906in}{2.330962in}}{\pgfqpoint{3.544505in}{2.326571in}}{\pgfqpoint{3.555556in}{2.326571in}}%
\pgfpathclose%
\pgfusepath{stroke,fill}%
\end{pgfscope}%
\begin{pgfscope}%
\pgfpathrectangle{\pgfqpoint{0.800000in}{0.528000in}}{\pgfqpoint{4.960000in}{3.696000in}} %
\pgfusepath{clip}%
\pgfsetbuttcap%
\pgfsetroundjoin%
\definecolor{currentfill}{rgb}{0.000000,0.500000,0.000000}%
\pgfsetfillcolor{currentfill}%
\pgfsetlinewidth{1.003750pt}%
\definecolor{currentstroke}{rgb}{0.000000,0.500000,0.000000}%
\pgfsetstrokecolor{currentstroke}%
\pgfsetdash{}{0pt}%
\pgfpathmoveto{\pgfqpoint{3.605657in}{1.189003in}}%
\pgfpathcurveto{\pgfqpoint{3.616707in}{1.189003in}}{\pgfqpoint{3.627306in}{1.193393in}}{\pgfqpoint{3.635119in}{1.201207in}}%
\pgfpathcurveto{\pgfqpoint{3.642933in}{1.209021in}}{\pgfqpoint{3.647323in}{1.219620in}}{\pgfqpoint{3.647323in}{1.230670in}}%
\pgfpathcurveto{\pgfqpoint{3.647323in}{1.241720in}}{\pgfqpoint{3.642933in}{1.252319in}}{\pgfqpoint{3.635119in}{1.260133in}}%
\pgfpathcurveto{\pgfqpoint{3.627306in}{1.267946in}}{\pgfqpoint{3.616707in}{1.272336in}}{\pgfqpoint{3.605657in}{1.272336in}}%
\pgfpathcurveto{\pgfqpoint{3.594606in}{1.272336in}}{\pgfqpoint{3.584007in}{1.267946in}}{\pgfqpoint{3.576194in}{1.260133in}}%
\pgfpathcurveto{\pgfqpoint{3.568380in}{1.252319in}}{\pgfqpoint{3.563990in}{1.241720in}}{\pgfqpoint{3.563990in}{1.230670in}}%
\pgfpathcurveto{\pgfqpoint{3.563990in}{1.219620in}}{\pgfqpoint{3.568380in}{1.209021in}}{\pgfqpoint{3.576194in}{1.201207in}}%
\pgfpathcurveto{\pgfqpoint{3.584007in}{1.193393in}}{\pgfqpoint{3.594606in}{1.189003in}}{\pgfqpoint{3.605657in}{1.189003in}}%
\pgfpathclose%
\pgfusepath{stroke,fill}%
\end{pgfscope}%
\begin{pgfscope}%
\pgfpathrectangle{\pgfqpoint{0.800000in}{0.528000in}}{\pgfqpoint{4.960000in}{3.696000in}} %
\pgfusepath{clip}%
\pgfsetbuttcap%
\pgfsetroundjoin%
\definecolor{currentfill}{rgb}{0.000000,0.500000,0.000000}%
\pgfsetfillcolor{currentfill}%
\pgfsetlinewidth{1.003750pt}%
\definecolor{currentstroke}{rgb}{0.000000,0.500000,0.000000}%
\pgfsetstrokecolor{currentstroke}%
\pgfsetdash{}{0pt}%
\pgfpathmoveto{\pgfqpoint{3.655758in}{0.537667in}}%
\pgfpathcurveto{\pgfqpoint{3.666808in}{0.537667in}}{\pgfqpoint{3.677407in}{0.542057in}}{\pgfqpoint{3.685220in}{0.549871in}}%
\pgfpathcurveto{\pgfqpoint{3.693034in}{0.557684in}}{\pgfqpoint{3.697424in}{0.568283in}}{\pgfqpoint{3.697424in}{0.579333in}}%
\pgfpathcurveto{\pgfqpoint{3.697424in}{0.590384in}}{\pgfqpoint{3.693034in}{0.600983in}}{\pgfqpoint{3.685220in}{0.608796in}}%
\pgfpathcurveto{\pgfqpoint{3.677407in}{0.616610in}}{\pgfqpoint{3.666808in}{0.621000in}}{\pgfqpoint{3.655758in}{0.621000in}}%
\pgfpathcurveto{\pgfqpoint{3.644707in}{0.621000in}}{\pgfqpoint{3.634108in}{0.616610in}}{\pgfqpoint{3.626295in}{0.608796in}}%
\pgfpathcurveto{\pgfqpoint{3.618481in}{0.600983in}}{\pgfqpoint{3.614091in}{0.590384in}}{\pgfqpoint{3.614091in}{0.579333in}}%
\pgfpathcurveto{\pgfqpoint{3.614091in}{0.568283in}}{\pgfqpoint{3.618481in}{0.557684in}}{\pgfqpoint{3.626295in}{0.549871in}}%
\pgfpathcurveto{\pgfqpoint{3.634108in}{0.542057in}}{\pgfqpoint{3.644707in}{0.537667in}}{\pgfqpoint{3.655758in}{0.537667in}}%
\pgfpathclose%
\pgfusepath{stroke,fill}%
\end{pgfscope}%
\begin{pgfscope}%
\pgfpathrectangle{\pgfqpoint{0.800000in}{0.528000in}}{\pgfqpoint{4.960000in}{3.696000in}} %
\pgfusepath{clip}%
\pgfsetbuttcap%
\pgfsetroundjoin%
\definecolor{currentfill}{rgb}{0.000000,0.500000,0.000000}%
\pgfsetfillcolor{currentfill}%
\pgfsetlinewidth{1.003750pt}%
\definecolor{currentstroke}{rgb}{0.000000,0.500000,0.000000}%
\pgfsetstrokecolor{currentstroke}%
\pgfsetdash{}{0pt}%
\pgfpathmoveto{\pgfqpoint{3.705859in}{2.017167in}}%
\pgfpathcurveto{\pgfqpoint{3.716909in}{2.017167in}}{\pgfqpoint{3.727508in}{2.021557in}}{\pgfqpoint{3.735321in}{2.029371in}}%
\pgfpathcurveto{\pgfqpoint{3.743135in}{2.037184in}}{\pgfqpoint{3.747525in}{2.047783in}}{\pgfqpoint{3.747525in}{2.058833in}}%
\pgfpathcurveto{\pgfqpoint{3.747525in}{2.069883in}}{\pgfqpoint{3.743135in}{2.080483in}}{\pgfqpoint{3.735321in}{2.088296in}}%
\pgfpathcurveto{\pgfqpoint{3.727508in}{2.096110in}}{\pgfqpoint{3.716909in}{2.100500in}}{\pgfqpoint{3.705859in}{2.100500in}}%
\pgfpathcurveto{\pgfqpoint{3.694808in}{2.100500in}}{\pgfqpoint{3.684209in}{2.096110in}}{\pgfqpoint{3.676396in}{2.088296in}}%
\pgfpathcurveto{\pgfqpoint{3.668582in}{2.080483in}}{\pgfqpoint{3.664192in}{2.069883in}}{\pgfqpoint{3.664192in}{2.058833in}}%
\pgfpathcurveto{\pgfqpoint{3.664192in}{2.047783in}}{\pgfqpoint{3.668582in}{2.037184in}}{\pgfqpoint{3.676396in}{2.029371in}}%
\pgfpathcurveto{\pgfqpoint{3.684209in}{2.021557in}}{\pgfqpoint{3.694808in}{2.017167in}}{\pgfqpoint{3.705859in}{2.017167in}}%
\pgfpathclose%
\pgfusepath{stroke,fill}%
\end{pgfscope}%
\begin{pgfscope}%
\pgfpathrectangle{\pgfqpoint{0.800000in}{0.528000in}}{\pgfqpoint{4.960000in}{3.696000in}} %
\pgfusepath{clip}%
\pgfsetbuttcap%
\pgfsetroundjoin%
\definecolor{currentfill}{rgb}{0.000000,0.500000,0.000000}%
\pgfsetfillcolor{currentfill}%
\pgfsetlinewidth{1.003750pt}%
\definecolor{currentstroke}{rgb}{0.000000,0.500000,0.000000}%
\pgfsetstrokecolor{currentstroke}%
\pgfsetdash{}{0pt}%
\pgfpathmoveto{\pgfqpoint{3.755960in}{-0.614694in}}%
\pgfpathcurveto{\pgfqpoint{3.767010in}{-0.614694in}}{\pgfqpoint{3.777609in}{-0.610304in}}{\pgfqpoint{3.785422in}{-0.602490in}}%
\pgfpathcurveto{\pgfqpoint{3.793236in}{-0.594677in}}{\pgfqpoint{3.797626in}{-0.584078in}}{\pgfqpoint{3.797626in}{-0.573027in}}%
\pgfpathcurveto{\pgfqpoint{3.797626in}{-0.561977in}}{\pgfqpoint{3.793236in}{-0.551378in}}{\pgfqpoint{3.785422in}{-0.543565in}}%
\pgfpathcurveto{\pgfqpoint{3.777609in}{-0.535751in}}{\pgfqpoint{3.767010in}{-0.531361in}}{\pgfqpoint{3.755960in}{-0.531361in}}%
\pgfpathcurveto{\pgfqpoint{3.744909in}{-0.531361in}}{\pgfqpoint{3.734310in}{-0.535751in}}{\pgfqpoint{3.726497in}{-0.543565in}}%
\pgfpathcurveto{\pgfqpoint{3.718683in}{-0.551378in}}{\pgfqpoint{3.714293in}{-0.561977in}}{\pgfqpoint{3.714293in}{-0.573027in}}%
\pgfpathcurveto{\pgfqpoint{3.714293in}{-0.584078in}}{\pgfqpoint{3.718683in}{-0.594677in}}{\pgfqpoint{3.726497in}{-0.602490in}}%
\pgfpathcurveto{\pgfqpoint{3.734310in}{-0.610304in}}{\pgfqpoint{3.744909in}{-0.614694in}}{\pgfqpoint{3.755960in}{-0.614694in}}%
\pgfpathclose%
\pgfusepath{stroke,fill}%
\end{pgfscope}%
\begin{pgfscope}%
\pgfpathrectangle{\pgfqpoint{0.800000in}{0.528000in}}{\pgfqpoint{4.960000in}{3.696000in}} %
\pgfusepath{clip}%
\pgfsetbuttcap%
\pgfsetroundjoin%
\definecolor{currentfill}{rgb}{0.000000,0.500000,0.000000}%
\pgfsetfillcolor{currentfill}%
\pgfsetlinewidth{1.003750pt}%
\definecolor{currentstroke}{rgb}{0.000000,0.500000,0.000000}%
\pgfsetstrokecolor{currentstroke}%
\pgfsetdash{}{0pt}%
\pgfpathmoveto{\pgfqpoint{3.806061in}{0.616924in}}%
\pgfpathcurveto{\pgfqpoint{3.817111in}{0.616924in}}{\pgfqpoint{3.827710in}{0.621314in}}{\pgfqpoint{3.835523in}{0.629128in}}%
\pgfpathcurveto{\pgfqpoint{3.843337in}{0.636941in}}{\pgfqpoint{3.847727in}{0.647540in}}{\pgfqpoint{3.847727in}{0.658590in}}%
\pgfpathcurveto{\pgfqpoint{3.847727in}{0.669641in}}{\pgfqpoint{3.843337in}{0.680240in}}{\pgfqpoint{3.835523in}{0.688053in}}%
\pgfpathcurveto{\pgfqpoint{3.827710in}{0.695867in}}{\pgfqpoint{3.817111in}{0.700257in}}{\pgfqpoint{3.806061in}{0.700257in}}%
\pgfpathcurveto{\pgfqpoint{3.795010in}{0.700257in}}{\pgfqpoint{3.784411in}{0.695867in}}{\pgfqpoint{3.776598in}{0.688053in}}%
\pgfpathcurveto{\pgfqpoint{3.768784in}{0.680240in}}{\pgfqpoint{3.764394in}{0.669641in}}{\pgfqpoint{3.764394in}{0.658590in}}%
\pgfpathcurveto{\pgfqpoint{3.764394in}{0.647540in}}{\pgfqpoint{3.768784in}{0.636941in}}{\pgfqpoint{3.776598in}{0.629128in}}%
\pgfpathcurveto{\pgfqpoint{3.784411in}{0.621314in}}{\pgfqpoint{3.795010in}{0.616924in}}{\pgfqpoint{3.806061in}{0.616924in}}%
\pgfpathclose%
\pgfusepath{stroke,fill}%
\end{pgfscope}%
\begin{pgfscope}%
\pgfpathrectangle{\pgfqpoint{0.800000in}{0.528000in}}{\pgfqpoint{4.960000in}{3.696000in}} %
\pgfusepath{clip}%
\pgfsetbuttcap%
\pgfsetroundjoin%
\definecolor{currentfill}{rgb}{0.000000,0.500000,0.000000}%
\pgfsetfillcolor{currentfill}%
\pgfsetlinewidth{1.003750pt}%
\definecolor{currentstroke}{rgb}{0.000000,0.500000,0.000000}%
\pgfsetstrokecolor{currentstroke}%
\pgfsetdash{}{0pt}%
\pgfpathmoveto{\pgfqpoint{3.856162in}{-0.479322in}}%
\pgfpathcurveto{\pgfqpoint{3.867212in}{-0.479322in}}{\pgfqpoint{3.877811in}{-0.474932in}}{\pgfqpoint{3.885624in}{-0.467119in}}%
\pgfpathcurveto{\pgfqpoint{3.893438in}{-0.459305in}}{\pgfqpoint{3.897828in}{-0.448706in}}{\pgfqpoint{3.897828in}{-0.437656in}}%
\pgfpathcurveto{\pgfqpoint{3.897828in}{-0.426606in}}{\pgfqpoint{3.893438in}{-0.416007in}}{\pgfqpoint{3.885624in}{-0.408193in}}%
\pgfpathcurveto{\pgfqpoint{3.877811in}{-0.400379in}}{\pgfqpoint{3.867212in}{-0.395989in}}{\pgfqpoint{3.856162in}{-0.395989in}}%
\pgfpathcurveto{\pgfqpoint{3.845111in}{-0.395989in}}{\pgfqpoint{3.834512in}{-0.400379in}}{\pgfqpoint{3.826699in}{-0.408193in}}%
\pgfpathcurveto{\pgfqpoint{3.818885in}{-0.416007in}}{\pgfqpoint{3.814495in}{-0.426606in}}{\pgfqpoint{3.814495in}{-0.437656in}}%
\pgfpathcurveto{\pgfqpoint{3.814495in}{-0.448706in}}{\pgfqpoint{3.818885in}{-0.459305in}}{\pgfqpoint{3.826699in}{-0.467119in}}%
\pgfpathcurveto{\pgfqpoint{3.834512in}{-0.474932in}}{\pgfqpoint{3.845111in}{-0.479322in}}{\pgfqpoint{3.856162in}{-0.479322in}}%
\pgfpathclose%
\pgfusepath{stroke,fill}%
\end{pgfscope}%
\begin{pgfscope}%
\pgfpathrectangle{\pgfqpoint{0.800000in}{0.528000in}}{\pgfqpoint{4.960000in}{3.696000in}} %
\pgfusepath{clip}%
\pgfsetbuttcap%
\pgfsetroundjoin%
\definecolor{currentfill}{rgb}{0.000000,0.500000,0.000000}%
\pgfsetfillcolor{currentfill}%
\pgfsetlinewidth{1.003750pt}%
\definecolor{currentstroke}{rgb}{0.000000,0.500000,0.000000}%
\pgfsetstrokecolor{currentstroke}%
\pgfsetdash{}{0pt}%
\pgfpathmoveto{\pgfqpoint{3.906263in}{1.193995in}}%
\pgfpathcurveto{\pgfqpoint{3.917313in}{1.193995in}}{\pgfqpoint{3.927912in}{1.198385in}}{\pgfqpoint{3.935725in}{1.206199in}}%
\pgfpathcurveto{\pgfqpoint{3.943539in}{1.214013in}}{\pgfqpoint{3.947929in}{1.224612in}}{\pgfqpoint{3.947929in}{1.235662in}}%
\pgfpathcurveto{\pgfqpoint{3.947929in}{1.246712in}}{\pgfqpoint{3.943539in}{1.257311in}}{\pgfqpoint{3.935725in}{1.265125in}}%
\pgfpathcurveto{\pgfqpoint{3.927912in}{1.272938in}}{\pgfqpoint{3.917313in}{1.277329in}}{\pgfqpoint{3.906263in}{1.277329in}}%
\pgfpathcurveto{\pgfqpoint{3.895212in}{1.277329in}}{\pgfqpoint{3.884613in}{1.272938in}}{\pgfqpoint{3.876800in}{1.265125in}}%
\pgfpathcurveto{\pgfqpoint{3.868986in}{1.257311in}}{\pgfqpoint{3.864596in}{1.246712in}}{\pgfqpoint{3.864596in}{1.235662in}}%
\pgfpathcurveto{\pgfqpoint{3.864596in}{1.224612in}}{\pgfqpoint{3.868986in}{1.214013in}}{\pgfqpoint{3.876800in}{1.206199in}}%
\pgfpathcurveto{\pgfqpoint{3.884613in}{1.198385in}}{\pgfqpoint{3.895212in}{1.193995in}}{\pgfqpoint{3.906263in}{1.193995in}}%
\pgfpathclose%
\pgfusepath{stroke,fill}%
\end{pgfscope}%
\begin{pgfscope}%
\pgfpathrectangle{\pgfqpoint{0.800000in}{0.528000in}}{\pgfqpoint{4.960000in}{3.696000in}} %
\pgfusepath{clip}%
\pgfsetbuttcap%
\pgfsetroundjoin%
\definecolor{currentfill}{rgb}{0.000000,0.500000,0.000000}%
\pgfsetfillcolor{currentfill}%
\pgfsetlinewidth{1.003750pt}%
\definecolor{currentstroke}{rgb}{0.000000,0.500000,0.000000}%
\pgfsetstrokecolor{currentstroke}%
\pgfsetdash{}{0pt}%
\pgfpathmoveto{\pgfqpoint{3.956364in}{-1.543579in}}%
\pgfpathcurveto{\pgfqpoint{3.967414in}{-1.543579in}}{\pgfqpoint{3.978013in}{-1.539189in}}{\pgfqpoint{3.985826in}{-1.531375in}}%
\pgfpathcurveto{\pgfqpoint{3.993640in}{-1.523562in}}{\pgfqpoint{3.998030in}{-1.512963in}}{\pgfqpoint{3.998030in}{-1.501912in}}%
\pgfpathcurveto{\pgfqpoint{3.998030in}{-1.490862in}}{\pgfqpoint{3.993640in}{-1.480263in}}{\pgfqpoint{3.985826in}{-1.472450in}}%
\pgfpathcurveto{\pgfqpoint{3.978013in}{-1.464636in}}{\pgfqpoint{3.967414in}{-1.460246in}}{\pgfqpoint{3.956364in}{-1.460246in}}%
\pgfpathcurveto{\pgfqpoint{3.945314in}{-1.460246in}}{\pgfqpoint{3.934714in}{-1.464636in}}{\pgfqpoint{3.926901in}{-1.472450in}}%
\pgfpathcurveto{\pgfqpoint{3.919087in}{-1.480263in}}{\pgfqpoint{3.914697in}{-1.490862in}}{\pgfqpoint{3.914697in}{-1.501912in}}%
\pgfpathcurveto{\pgfqpoint{3.914697in}{-1.512963in}}{\pgfqpoint{3.919087in}{-1.523562in}}{\pgfqpoint{3.926901in}{-1.531375in}}%
\pgfpathcurveto{\pgfqpoint{3.934714in}{-1.539189in}}{\pgfqpoint{3.945314in}{-1.543579in}}{\pgfqpoint{3.956364in}{-1.543579in}}%
\pgfpathclose%
\pgfusepath{stroke,fill}%
\end{pgfscope}%
\begin{pgfscope}%
\pgfpathrectangle{\pgfqpoint{0.800000in}{0.528000in}}{\pgfqpoint{4.960000in}{3.696000in}} %
\pgfusepath{clip}%
\pgfsetbuttcap%
\pgfsetroundjoin%
\definecolor{currentfill}{rgb}{0.000000,0.500000,0.000000}%
\pgfsetfillcolor{currentfill}%
\pgfsetlinewidth{1.003750pt}%
\definecolor{currentstroke}{rgb}{0.000000,0.500000,0.000000}%
\pgfsetstrokecolor{currentstroke}%
\pgfsetdash{}{0pt}%
\pgfpathmoveto{\pgfqpoint{4.006465in}{0.024514in}}%
\pgfpathcurveto{\pgfqpoint{4.017515in}{0.024514in}}{\pgfqpoint{4.028114in}{0.028904in}}{\pgfqpoint{4.035927in}{0.036718in}}%
\pgfpathcurveto{\pgfqpoint{4.043741in}{0.044532in}}{\pgfqpoint{4.048131in}{0.055131in}}{\pgfqpoint{4.048131in}{0.066181in}}%
\pgfpathcurveto{\pgfqpoint{4.048131in}{0.077231in}}{\pgfqpoint{4.043741in}{0.087830in}}{\pgfqpoint{4.035927in}{0.095643in}}%
\pgfpathcurveto{\pgfqpoint{4.028114in}{0.103457in}}{\pgfqpoint{4.017515in}{0.107847in}}{\pgfqpoint{4.006465in}{0.107847in}}%
\pgfpathcurveto{\pgfqpoint{3.995415in}{0.107847in}}{\pgfqpoint{3.984815in}{0.103457in}}{\pgfqpoint{3.977002in}{0.095643in}}%
\pgfpathcurveto{\pgfqpoint{3.969188in}{0.087830in}}{\pgfqpoint{3.964798in}{0.077231in}}{\pgfqpoint{3.964798in}{0.066181in}}%
\pgfpathcurveto{\pgfqpoint{3.964798in}{0.055131in}}{\pgfqpoint{3.969188in}{0.044532in}}{\pgfqpoint{3.977002in}{0.036718in}}%
\pgfpathcurveto{\pgfqpoint{3.984815in}{0.028904in}}{\pgfqpoint{3.995415in}{0.024514in}}{\pgfqpoint{4.006465in}{0.024514in}}%
\pgfpathclose%
\pgfusepath{stroke,fill}%
\end{pgfscope}%
\begin{pgfscope}%
\pgfpathrectangle{\pgfqpoint{0.800000in}{0.528000in}}{\pgfqpoint{4.960000in}{3.696000in}} %
\pgfusepath{clip}%
\pgfsetbuttcap%
\pgfsetroundjoin%
\definecolor{currentfill}{rgb}{0.000000,0.500000,0.000000}%
\pgfsetfillcolor{currentfill}%
\pgfsetlinewidth{1.003750pt}%
\definecolor{currentstroke}{rgb}{0.000000,0.500000,0.000000}%
\pgfsetstrokecolor{currentstroke}%
\pgfsetdash{}{0pt}%
\pgfpathmoveto{\pgfqpoint{4.056566in}{0.434837in}}%
\pgfpathcurveto{\pgfqpoint{4.067616in}{0.434837in}}{\pgfqpoint{4.078215in}{0.439227in}}{\pgfqpoint{4.086028in}{0.447041in}}%
\pgfpathcurveto{\pgfqpoint{4.093842in}{0.454854in}}{\pgfqpoint{4.098232in}{0.465454in}}{\pgfqpoint{4.098232in}{0.476504in}}%
\pgfpathcurveto{\pgfqpoint{4.098232in}{0.487554in}}{\pgfqpoint{4.093842in}{0.498153in}}{\pgfqpoint{4.086028in}{0.505966in}}%
\pgfpathcurveto{\pgfqpoint{4.078215in}{0.513780in}}{\pgfqpoint{4.067616in}{0.518170in}}{\pgfqpoint{4.056566in}{0.518170in}}%
\pgfpathcurveto{\pgfqpoint{4.045516in}{0.518170in}}{\pgfqpoint{4.034916in}{0.513780in}}{\pgfqpoint{4.027103in}{0.505966in}}%
\pgfpathcurveto{\pgfqpoint{4.019289in}{0.498153in}}{\pgfqpoint{4.014899in}{0.487554in}}{\pgfqpoint{4.014899in}{0.476504in}}%
\pgfpathcurveto{\pgfqpoint{4.014899in}{0.465454in}}{\pgfqpoint{4.019289in}{0.454854in}}{\pgfqpoint{4.027103in}{0.447041in}}%
\pgfpathcurveto{\pgfqpoint{4.034916in}{0.439227in}}{\pgfqpoint{4.045516in}{0.434837in}}{\pgfqpoint{4.056566in}{0.434837in}}%
\pgfpathclose%
\pgfusepath{stroke,fill}%
\end{pgfscope}%
\begin{pgfscope}%
\pgfpathrectangle{\pgfqpoint{0.800000in}{0.528000in}}{\pgfqpoint{4.960000in}{3.696000in}} %
\pgfusepath{clip}%
\pgfsetbuttcap%
\pgfsetroundjoin%
\definecolor{currentfill}{rgb}{0.000000,0.500000,0.000000}%
\pgfsetfillcolor{currentfill}%
\pgfsetlinewidth{1.003750pt}%
\definecolor{currentstroke}{rgb}{0.000000,0.500000,0.000000}%
\pgfsetstrokecolor{currentstroke}%
\pgfsetdash{}{0pt}%
\pgfpathmoveto{\pgfqpoint{4.106667in}{2.072927in}}%
\pgfpathcurveto{\pgfqpoint{4.117717in}{2.072927in}}{\pgfqpoint{4.128316in}{2.077318in}}{\pgfqpoint{4.136129in}{2.085131in}}%
\pgfpathcurveto{\pgfqpoint{4.143943in}{2.092945in}}{\pgfqpoint{4.148333in}{2.103544in}}{\pgfqpoint{4.148333in}{2.114594in}}%
\pgfpathcurveto{\pgfqpoint{4.148333in}{2.125644in}}{\pgfqpoint{4.143943in}{2.136243in}}{\pgfqpoint{4.136129in}{2.144057in}}%
\pgfpathcurveto{\pgfqpoint{4.128316in}{2.151870in}}{\pgfqpoint{4.117717in}{2.156261in}}{\pgfqpoint{4.106667in}{2.156261in}}%
\pgfpathcurveto{\pgfqpoint{4.095617in}{2.156261in}}{\pgfqpoint{4.085018in}{2.151870in}}{\pgfqpoint{4.077204in}{2.144057in}}%
\pgfpathcurveto{\pgfqpoint{4.069390in}{2.136243in}}{\pgfqpoint{4.065000in}{2.125644in}}{\pgfqpoint{4.065000in}{2.114594in}}%
\pgfpathcurveto{\pgfqpoint{4.065000in}{2.103544in}}{\pgfqpoint{4.069390in}{2.092945in}}{\pgfqpoint{4.077204in}{2.085131in}}%
\pgfpathcurveto{\pgfqpoint{4.085018in}{2.077318in}}{\pgfqpoint{4.095617in}{2.072927in}}{\pgfqpoint{4.106667in}{2.072927in}}%
\pgfpathclose%
\pgfusepath{stroke,fill}%
\end{pgfscope}%
\begin{pgfscope}%
\pgfpathrectangle{\pgfqpoint{0.800000in}{0.528000in}}{\pgfqpoint{4.960000in}{3.696000in}} %
\pgfusepath{clip}%
\pgfsetbuttcap%
\pgfsetroundjoin%
\definecolor{currentfill}{rgb}{0.000000,0.500000,0.000000}%
\pgfsetfillcolor{currentfill}%
\pgfsetlinewidth{1.003750pt}%
\definecolor{currentstroke}{rgb}{0.000000,0.500000,0.000000}%
\pgfsetstrokecolor{currentstroke}%
\pgfsetdash{}{0pt}%
\pgfpathmoveto{\pgfqpoint{4.156768in}{2.471377in}}%
\pgfpathcurveto{\pgfqpoint{4.167818in}{2.471377in}}{\pgfqpoint{4.178417in}{2.475767in}}{\pgfqpoint{4.186230in}{2.483581in}}%
\pgfpathcurveto{\pgfqpoint{4.194044in}{2.491394in}}{\pgfqpoint{4.198434in}{2.501993in}}{\pgfqpoint{4.198434in}{2.513043in}}%
\pgfpathcurveto{\pgfqpoint{4.198434in}{2.524093in}}{\pgfqpoint{4.194044in}{2.534693in}}{\pgfqpoint{4.186230in}{2.542506in}}%
\pgfpathcurveto{\pgfqpoint{4.178417in}{2.550320in}}{\pgfqpoint{4.167818in}{2.554710in}}{\pgfqpoint{4.156768in}{2.554710in}}%
\pgfpathcurveto{\pgfqpoint{4.145718in}{2.554710in}}{\pgfqpoint{4.135119in}{2.550320in}}{\pgfqpoint{4.127305in}{2.542506in}}%
\pgfpathcurveto{\pgfqpoint{4.119491in}{2.534693in}}{\pgfqpoint{4.115101in}{2.524093in}}{\pgfqpoint{4.115101in}{2.513043in}}%
\pgfpathcurveto{\pgfqpoint{4.115101in}{2.501993in}}{\pgfqpoint{4.119491in}{2.491394in}}{\pgfqpoint{4.127305in}{2.483581in}}%
\pgfpathcurveto{\pgfqpoint{4.135119in}{2.475767in}}{\pgfqpoint{4.145718in}{2.471377in}}{\pgfqpoint{4.156768in}{2.471377in}}%
\pgfpathclose%
\pgfusepath{stroke,fill}%
\end{pgfscope}%
\begin{pgfscope}%
\pgfpathrectangle{\pgfqpoint{0.800000in}{0.528000in}}{\pgfqpoint{4.960000in}{3.696000in}} %
\pgfusepath{clip}%
\pgfsetbuttcap%
\pgfsetroundjoin%
\definecolor{currentfill}{rgb}{0.000000,0.500000,0.000000}%
\pgfsetfillcolor{currentfill}%
\pgfsetlinewidth{1.003750pt}%
\definecolor{currentstroke}{rgb}{0.000000,0.500000,0.000000}%
\pgfsetstrokecolor{currentstroke}%
\pgfsetdash{}{0pt}%
\pgfpathmoveto{\pgfqpoint{4.206869in}{-1.229223in}}%
\pgfpathcurveto{\pgfqpoint{4.217919in}{-1.229223in}}{\pgfqpoint{4.228518in}{-1.224833in}}{\pgfqpoint{4.236331in}{-1.217019in}}%
\pgfpathcurveto{\pgfqpoint{4.244145in}{-1.209206in}}{\pgfqpoint{4.248535in}{-1.198607in}}{\pgfqpoint{4.248535in}{-1.187557in}}%
\pgfpathcurveto{\pgfqpoint{4.248535in}{-1.176507in}}{\pgfqpoint{4.244145in}{-1.165908in}}{\pgfqpoint{4.236331in}{-1.158094in}}%
\pgfpathcurveto{\pgfqpoint{4.228518in}{-1.150280in}}{\pgfqpoint{4.217919in}{-1.145890in}}{\pgfqpoint{4.206869in}{-1.145890in}}%
\pgfpathcurveto{\pgfqpoint{4.195819in}{-1.145890in}}{\pgfqpoint{4.185220in}{-1.150280in}}{\pgfqpoint{4.177406in}{-1.158094in}}%
\pgfpathcurveto{\pgfqpoint{4.169592in}{-1.165908in}}{\pgfqpoint{4.165202in}{-1.176507in}}{\pgfqpoint{4.165202in}{-1.187557in}}%
\pgfpathcurveto{\pgfqpoint{4.165202in}{-1.198607in}}{\pgfqpoint{4.169592in}{-1.209206in}}{\pgfqpoint{4.177406in}{-1.217019in}}%
\pgfpathcurveto{\pgfqpoint{4.185220in}{-1.224833in}}{\pgfqpoint{4.195819in}{-1.229223in}}{\pgfqpoint{4.206869in}{-1.229223in}}%
\pgfpathclose%
\pgfusepath{stroke,fill}%
\end{pgfscope}%
\begin{pgfscope}%
\pgfpathrectangle{\pgfqpoint{0.800000in}{0.528000in}}{\pgfqpoint{4.960000in}{3.696000in}} %
\pgfusepath{clip}%
\pgfsetbuttcap%
\pgfsetroundjoin%
\definecolor{currentfill}{rgb}{0.000000,0.500000,0.000000}%
\pgfsetfillcolor{currentfill}%
\pgfsetlinewidth{1.003750pt}%
\definecolor{currentstroke}{rgb}{0.000000,0.500000,0.000000}%
\pgfsetstrokecolor{currentstroke}%
\pgfsetdash{}{0pt}%
\pgfpathmoveto{\pgfqpoint{4.256970in}{-1.608056in}}%
\pgfpathcurveto{\pgfqpoint{4.268020in}{-1.608056in}}{\pgfqpoint{4.278619in}{-1.603665in}}{\pgfqpoint{4.286432in}{-1.595852in}}%
\pgfpathcurveto{\pgfqpoint{4.294246in}{-1.588038in}}{\pgfqpoint{4.298636in}{-1.577439in}}{\pgfqpoint{4.298636in}{-1.566389in}}%
\pgfpathcurveto{\pgfqpoint{4.298636in}{-1.555339in}}{\pgfqpoint{4.294246in}{-1.544740in}}{\pgfqpoint{4.286432in}{-1.536926in}}%
\pgfpathcurveto{\pgfqpoint{4.278619in}{-1.529113in}}{\pgfqpoint{4.268020in}{-1.524722in}}{\pgfqpoint{4.256970in}{-1.524722in}}%
\pgfpathcurveto{\pgfqpoint{4.245920in}{-1.524722in}}{\pgfqpoint{4.235321in}{-1.529113in}}{\pgfqpoint{4.227507in}{-1.536926in}}%
\pgfpathcurveto{\pgfqpoint{4.219693in}{-1.544740in}}{\pgfqpoint{4.215303in}{-1.555339in}}{\pgfqpoint{4.215303in}{-1.566389in}}%
\pgfpathcurveto{\pgfqpoint{4.215303in}{-1.577439in}}{\pgfqpoint{4.219693in}{-1.588038in}}{\pgfqpoint{4.227507in}{-1.595852in}}%
\pgfpathcurveto{\pgfqpoint{4.235321in}{-1.603665in}}{\pgfqpoint{4.245920in}{-1.608056in}}{\pgfqpoint{4.256970in}{-1.608056in}}%
\pgfpathclose%
\pgfusepath{stroke,fill}%
\end{pgfscope}%
\begin{pgfscope}%
\pgfpathrectangle{\pgfqpoint{0.800000in}{0.528000in}}{\pgfqpoint{4.960000in}{3.696000in}} %
\pgfusepath{clip}%
\pgfsetbuttcap%
\pgfsetroundjoin%
\definecolor{currentfill}{rgb}{0.000000,0.500000,0.000000}%
\pgfsetfillcolor{currentfill}%
\pgfsetlinewidth{1.003750pt}%
\definecolor{currentstroke}{rgb}{0.000000,0.500000,0.000000}%
\pgfsetstrokecolor{currentstroke}%
\pgfsetdash{}{0pt}%
\pgfpathmoveto{\pgfqpoint{4.307071in}{-0.487112in}}%
\pgfpathcurveto{\pgfqpoint{4.318121in}{-0.487112in}}{\pgfqpoint{4.328720in}{-0.482722in}}{\pgfqpoint{4.336533in}{-0.474908in}}%
\pgfpathcurveto{\pgfqpoint{4.344347in}{-0.467095in}}{\pgfqpoint{4.348737in}{-0.456495in}}{\pgfqpoint{4.348737in}{-0.445445in}}%
\pgfpathcurveto{\pgfqpoint{4.348737in}{-0.434395in}}{\pgfqpoint{4.344347in}{-0.423796in}}{\pgfqpoint{4.336533in}{-0.415983in}}%
\pgfpathcurveto{\pgfqpoint{4.328720in}{-0.408169in}}{\pgfqpoint{4.318121in}{-0.403779in}}{\pgfqpoint{4.307071in}{-0.403779in}}%
\pgfpathcurveto{\pgfqpoint{4.296021in}{-0.403779in}}{\pgfqpoint{4.285422in}{-0.408169in}}{\pgfqpoint{4.277608in}{-0.415983in}}%
\pgfpathcurveto{\pgfqpoint{4.269794in}{-0.423796in}}{\pgfqpoint{4.265404in}{-0.434395in}}{\pgfqpoint{4.265404in}{-0.445445in}}%
\pgfpathcurveto{\pgfqpoint{4.265404in}{-0.456495in}}{\pgfqpoint{4.269794in}{-0.467095in}}{\pgfqpoint{4.277608in}{-0.474908in}}%
\pgfpathcurveto{\pgfqpoint{4.285422in}{-0.482722in}}{\pgfqpoint{4.296021in}{-0.487112in}}{\pgfqpoint{4.307071in}{-0.487112in}}%
\pgfpathclose%
\pgfusepath{stroke,fill}%
\end{pgfscope}%
\begin{pgfscope}%
\pgfpathrectangle{\pgfqpoint{0.800000in}{0.528000in}}{\pgfqpoint{4.960000in}{3.696000in}} %
\pgfusepath{clip}%
\pgfsetbuttcap%
\pgfsetroundjoin%
\definecolor{currentfill}{rgb}{0.000000,0.500000,0.000000}%
\pgfsetfillcolor{currentfill}%
\pgfsetlinewidth{1.003750pt}%
\definecolor{currentstroke}{rgb}{0.000000,0.500000,0.000000}%
\pgfsetstrokecolor{currentstroke}%
\pgfsetdash{}{0pt}%
\pgfpathmoveto{\pgfqpoint{4.357172in}{1.878283in}}%
\pgfpathcurveto{\pgfqpoint{4.368222in}{1.878283in}}{\pgfqpoint{4.378821in}{1.882673in}}{\pgfqpoint{4.386634in}{1.890487in}}%
\pgfpathcurveto{\pgfqpoint{4.394448in}{1.898301in}}{\pgfqpoint{4.398838in}{1.908900in}}{\pgfqpoint{4.398838in}{1.919950in}}%
\pgfpathcurveto{\pgfqpoint{4.398838in}{1.931000in}}{\pgfqpoint{4.394448in}{1.941599in}}{\pgfqpoint{4.386634in}{1.949413in}}%
\pgfpathcurveto{\pgfqpoint{4.378821in}{1.957226in}}{\pgfqpoint{4.368222in}{1.961616in}}{\pgfqpoint{4.357172in}{1.961616in}}%
\pgfpathcurveto{\pgfqpoint{4.346122in}{1.961616in}}{\pgfqpoint{4.335523in}{1.957226in}}{\pgfqpoint{4.327709in}{1.949413in}}%
\pgfpathcurveto{\pgfqpoint{4.319895in}{1.941599in}}{\pgfqpoint{4.315505in}{1.931000in}}{\pgfqpoint{4.315505in}{1.919950in}}%
\pgfpathcurveto{\pgfqpoint{4.315505in}{1.908900in}}{\pgfqpoint{4.319895in}{1.898301in}}{\pgfqpoint{4.327709in}{1.890487in}}%
\pgfpathcurveto{\pgfqpoint{4.335523in}{1.882673in}}{\pgfqpoint{4.346122in}{1.878283in}}{\pgfqpoint{4.357172in}{1.878283in}}%
\pgfpathclose%
\pgfusepath{stroke,fill}%
\end{pgfscope}%
\begin{pgfscope}%
\pgfpathrectangle{\pgfqpoint{0.800000in}{0.528000in}}{\pgfqpoint{4.960000in}{3.696000in}} %
\pgfusepath{clip}%
\pgfsetbuttcap%
\pgfsetroundjoin%
\definecolor{currentfill}{rgb}{0.000000,0.500000,0.000000}%
\pgfsetfillcolor{currentfill}%
\pgfsetlinewidth{1.003750pt}%
\definecolor{currentstroke}{rgb}{0.000000,0.500000,0.000000}%
\pgfsetstrokecolor{currentstroke}%
\pgfsetdash{}{0pt}%
\pgfpathmoveto{\pgfqpoint{4.407273in}{-1.599351in}}%
\pgfpathcurveto{\pgfqpoint{4.418323in}{-1.599351in}}{\pgfqpoint{4.428922in}{-1.594961in}}{\pgfqpoint{4.436736in}{-1.587147in}}%
\pgfpathcurveto{\pgfqpoint{4.444549in}{-1.579333in}}{\pgfqpoint{4.448939in}{-1.568734in}}{\pgfqpoint{4.448939in}{-1.557684in}}%
\pgfpathcurveto{\pgfqpoint{4.448939in}{-1.546634in}}{\pgfqpoint{4.444549in}{-1.536035in}}{\pgfqpoint{4.436736in}{-1.528222in}}%
\pgfpathcurveto{\pgfqpoint{4.428922in}{-1.520408in}}{\pgfqpoint{4.418323in}{-1.516018in}}{\pgfqpoint{4.407273in}{-1.516018in}}%
\pgfpathcurveto{\pgfqpoint{4.396223in}{-1.516018in}}{\pgfqpoint{4.385624in}{-1.520408in}}{\pgfqpoint{4.377810in}{-1.528222in}}%
\pgfpathcurveto{\pgfqpoint{4.369996in}{-1.536035in}}{\pgfqpoint{4.365606in}{-1.546634in}}{\pgfqpoint{4.365606in}{-1.557684in}}%
\pgfpathcurveto{\pgfqpoint{4.365606in}{-1.568734in}}{\pgfqpoint{4.369996in}{-1.579333in}}{\pgfqpoint{4.377810in}{-1.587147in}}%
\pgfpathcurveto{\pgfqpoint{4.385624in}{-1.594961in}}{\pgfqpoint{4.396223in}{-1.599351in}}{\pgfqpoint{4.407273in}{-1.599351in}}%
\pgfpathclose%
\pgfusepath{stroke,fill}%
\end{pgfscope}%
\begin{pgfscope}%
\pgfpathrectangle{\pgfqpoint{0.800000in}{0.528000in}}{\pgfqpoint{4.960000in}{3.696000in}} %
\pgfusepath{clip}%
\pgfsetbuttcap%
\pgfsetroundjoin%
\definecolor{currentfill}{rgb}{0.000000,0.500000,0.000000}%
\pgfsetfillcolor{currentfill}%
\pgfsetlinewidth{1.003750pt}%
\definecolor{currentstroke}{rgb}{0.000000,0.500000,0.000000}%
\pgfsetstrokecolor{currentstroke}%
\pgfsetdash{}{0pt}%
\pgfpathmoveto{\pgfqpoint{4.457374in}{-0.033806in}}%
\pgfpathcurveto{\pgfqpoint{4.468424in}{-0.033806in}}{\pgfqpoint{4.479023in}{-0.029416in}}{\pgfqpoint{4.486837in}{-0.021602in}}%
\pgfpathcurveto{\pgfqpoint{4.494650in}{-0.013789in}}{\pgfqpoint{4.499040in}{-0.003190in}}{\pgfqpoint{4.499040in}{0.007861in}}%
\pgfpathcurveto{\pgfqpoint{4.499040in}{0.018911in}}{\pgfqpoint{4.494650in}{0.029510in}}{\pgfqpoint{4.486837in}{0.037323in}}%
\pgfpathcurveto{\pgfqpoint{4.479023in}{0.045137in}}{\pgfqpoint{4.468424in}{0.049527in}}{\pgfqpoint{4.457374in}{0.049527in}}%
\pgfpathcurveto{\pgfqpoint{4.446324in}{0.049527in}}{\pgfqpoint{4.435725in}{0.045137in}}{\pgfqpoint{4.427911in}{0.037323in}}%
\pgfpathcurveto{\pgfqpoint{4.420097in}{0.029510in}}{\pgfqpoint{4.415707in}{0.018911in}}{\pgfqpoint{4.415707in}{0.007861in}}%
\pgfpathcurveto{\pgfqpoint{4.415707in}{-0.003190in}}{\pgfqpoint{4.420097in}{-0.013789in}}{\pgfqpoint{4.427911in}{-0.021602in}}%
\pgfpathcurveto{\pgfqpoint{4.435725in}{-0.029416in}}{\pgfqpoint{4.446324in}{-0.033806in}}{\pgfqpoint{4.457374in}{-0.033806in}}%
\pgfpathclose%
\pgfusepath{stroke,fill}%
\end{pgfscope}%
\begin{pgfscope}%
\pgfpathrectangle{\pgfqpoint{0.800000in}{0.528000in}}{\pgfqpoint{4.960000in}{3.696000in}} %
\pgfusepath{clip}%
\pgfsetbuttcap%
\pgfsetroundjoin%
\definecolor{currentfill}{rgb}{0.000000,0.500000,0.000000}%
\pgfsetfillcolor{currentfill}%
\pgfsetlinewidth{1.003750pt}%
\definecolor{currentstroke}{rgb}{0.000000,0.500000,0.000000}%
\pgfsetstrokecolor{currentstroke}%
\pgfsetdash{}{0pt}%
\pgfpathmoveto{\pgfqpoint{4.507475in}{1.635490in}}%
\pgfpathcurveto{\pgfqpoint{4.518525in}{1.635490in}}{\pgfqpoint{4.529124in}{1.639880in}}{\pgfqpoint{4.536938in}{1.647694in}}%
\pgfpathcurveto{\pgfqpoint{4.544751in}{1.655508in}}{\pgfqpoint{4.549141in}{1.666107in}}{\pgfqpoint{4.549141in}{1.677157in}}%
\pgfpathcurveto{\pgfqpoint{4.549141in}{1.688207in}}{\pgfqpoint{4.544751in}{1.698806in}}{\pgfqpoint{4.536938in}{1.706620in}}%
\pgfpathcurveto{\pgfqpoint{4.529124in}{1.714433in}}{\pgfqpoint{4.518525in}{1.718823in}}{\pgfqpoint{4.507475in}{1.718823in}}%
\pgfpathcurveto{\pgfqpoint{4.496425in}{1.718823in}}{\pgfqpoint{4.485826in}{1.714433in}}{\pgfqpoint{4.478012in}{1.706620in}}%
\pgfpathcurveto{\pgfqpoint{4.470198in}{1.698806in}}{\pgfqpoint{4.465808in}{1.688207in}}{\pgfqpoint{4.465808in}{1.677157in}}%
\pgfpathcurveto{\pgfqpoint{4.465808in}{1.666107in}}{\pgfqpoint{4.470198in}{1.655508in}}{\pgfqpoint{4.478012in}{1.647694in}}%
\pgfpathcurveto{\pgfqpoint{4.485826in}{1.639880in}}{\pgfqpoint{4.496425in}{1.635490in}}{\pgfqpoint{4.507475in}{1.635490in}}%
\pgfpathclose%
\pgfusepath{stroke,fill}%
\end{pgfscope}%
\begin{pgfscope}%
\pgfpathrectangle{\pgfqpoint{0.800000in}{0.528000in}}{\pgfqpoint{4.960000in}{3.696000in}} %
\pgfusepath{clip}%
\pgfsetbuttcap%
\pgfsetroundjoin%
\definecolor{currentfill}{rgb}{0.000000,0.500000,0.000000}%
\pgfsetfillcolor{currentfill}%
\pgfsetlinewidth{1.003750pt}%
\definecolor{currentstroke}{rgb}{0.000000,0.500000,0.000000}%
\pgfsetstrokecolor{currentstroke}%
\pgfsetdash{}{0pt}%
\pgfpathmoveto{\pgfqpoint{4.557576in}{-0.499582in}}%
\pgfpathcurveto{\pgfqpoint{4.568626in}{-0.499582in}}{\pgfqpoint{4.579225in}{-0.495192in}}{\pgfqpoint{4.587039in}{-0.487378in}}%
\pgfpathcurveto{\pgfqpoint{4.594852in}{-0.479564in}}{\pgfqpoint{4.599242in}{-0.468965in}}{\pgfqpoint{4.599242in}{-0.457915in}}%
\pgfpathcurveto{\pgfqpoint{4.599242in}{-0.446865in}}{\pgfqpoint{4.594852in}{-0.436266in}}{\pgfqpoint{4.587039in}{-0.428452in}}%
\pgfpathcurveto{\pgfqpoint{4.579225in}{-0.420639in}}{\pgfqpoint{4.568626in}{-0.416249in}}{\pgfqpoint{4.557576in}{-0.416249in}}%
\pgfpathcurveto{\pgfqpoint{4.546526in}{-0.416249in}}{\pgfqpoint{4.535927in}{-0.420639in}}{\pgfqpoint{4.528113in}{-0.428452in}}%
\pgfpathcurveto{\pgfqpoint{4.520299in}{-0.436266in}}{\pgfqpoint{4.515909in}{-0.446865in}}{\pgfqpoint{4.515909in}{-0.457915in}}%
\pgfpathcurveto{\pgfqpoint{4.515909in}{-0.468965in}}{\pgfqpoint{4.520299in}{-0.479564in}}{\pgfqpoint{4.528113in}{-0.487378in}}%
\pgfpathcurveto{\pgfqpoint{4.535927in}{-0.495192in}}{\pgfqpoint{4.546526in}{-0.499582in}}{\pgfqpoint{4.557576in}{-0.499582in}}%
\pgfpathclose%
\pgfusepath{stroke,fill}%
\end{pgfscope}%
\begin{pgfscope}%
\pgfpathrectangle{\pgfqpoint{0.800000in}{0.528000in}}{\pgfqpoint{4.960000in}{3.696000in}} %
\pgfusepath{clip}%
\pgfsetbuttcap%
\pgfsetroundjoin%
\definecolor{currentfill}{rgb}{0.000000,0.500000,0.000000}%
\pgfsetfillcolor{currentfill}%
\pgfsetlinewidth{1.003750pt}%
\definecolor{currentstroke}{rgb}{0.000000,0.500000,0.000000}%
\pgfsetstrokecolor{currentstroke}%
\pgfsetdash{}{0pt}%
\pgfpathmoveto{\pgfqpoint{4.607677in}{2.092573in}}%
\pgfpathcurveto{\pgfqpoint{4.618727in}{2.092573in}}{\pgfqpoint{4.629326in}{2.096963in}}{\pgfqpoint{4.637140in}{2.104777in}}%
\pgfpathcurveto{\pgfqpoint{4.644953in}{2.112590in}}{\pgfqpoint{4.649343in}{2.123189in}}{\pgfqpoint{4.649343in}{2.134239in}}%
\pgfpathcurveto{\pgfqpoint{4.649343in}{2.145290in}}{\pgfqpoint{4.644953in}{2.155889in}}{\pgfqpoint{4.637140in}{2.163702in}}%
\pgfpathcurveto{\pgfqpoint{4.629326in}{2.171516in}}{\pgfqpoint{4.618727in}{2.175906in}}{\pgfqpoint{4.607677in}{2.175906in}}%
\pgfpathcurveto{\pgfqpoint{4.596627in}{2.175906in}}{\pgfqpoint{4.586028in}{2.171516in}}{\pgfqpoint{4.578214in}{2.163702in}}%
\pgfpathcurveto{\pgfqpoint{4.570400in}{2.155889in}}{\pgfqpoint{4.566010in}{2.145290in}}{\pgfqpoint{4.566010in}{2.134239in}}%
\pgfpathcurveto{\pgfqpoint{4.566010in}{2.123189in}}{\pgfqpoint{4.570400in}{2.112590in}}{\pgfqpoint{4.578214in}{2.104777in}}%
\pgfpathcurveto{\pgfqpoint{4.586028in}{2.096963in}}{\pgfqpoint{4.596627in}{2.092573in}}{\pgfqpoint{4.607677in}{2.092573in}}%
\pgfpathclose%
\pgfusepath{stroke,fill}%
\end{pgfscope}%
\begin{pgfscope}%
\pgfpathrectangle{\pgfqpoint{0.800000in}{0.528000in}}{\pgfqpoint{4.960000in}{3.696000in}} %
\pgfusepath{clip}%
\pgfsetbuttcap%
\pgfsetroundjoin%
\definecolor{currentfill}{rgb}{0.000000,0.500000,0.000000}%
\pgfsetfillcolor{currentfill}%
\pgfsetlinewidth{1.003750pt}%
\definecolor{currentstroke}{rgb}{0.000000,0.500000,0.000000}%
\pgfsetstrokecolor{currentstroke}%
\pgfsetdash{}{0pt}%
\pgfpathmoveto{\pgfqpoint{4.657778in}{-0.139076in}}%
\pgfpathcurveto{\pgfqpoint{4.668828in}{-0.139076in}}{\pgfqpoint{4.679427in}{-0.134686in}}{\pgfqpoint{4.687241in}{-0.126873in}}%
\pgfpathcurveto{\pgfqpoint{4.695054in}{-0.119059in}}{\pgfqpoint{4.699444in}{-0.108460in}}{\pgfqpoint{4.699444in}{-0.097410in}}%
\pgfpathcurveto{\pgfqpoint{4.699444in}{-0.086360in}}{\pgfqpoint{4.695054in}{-0.075761in}}{\pgfqpoint{4.687241in}{-0.067947in}}%
\pgfpathcurveto{\pgfqpoint{4.679427in}{-0.060133in}}{\pgfqpoint{4.668828in}{-0.055743in}}{\pgfqpoint{4.657778in}{-0.055743in}}%
\pgfpathcurveto{\pgfqpoint{4.646728in}{-0.055743in}}{\pgfqpoint{4.636129in}{-0.060133in}}{\pgfqpoint{4.628315in}{-0.067947in}}%
\pgfpathcurveto{\pgfqpoint{4.620501in}{-0.075761in}}{\pgfqpoint{4.616111in}{-0.086360in}}{\pgfqpoint{4.616111in}{-0.097410in}}%
\pgfpathcurveto{\pgfqpoint{4.616111in}{-0.108460in}}{\pgfqpoint{4.620501in}{-0.119059in}}{\pgfqpoint{4.628315in}{-0.126873in}}%
\pgfpathcurveto{\pgfqpoint{4.636129in}{-0.134686in}}{\pgfqpoint{4.646728in}{-0.139076in}}{\pgfqpoint{4.657778in}{-0.139076in}}%
\pgfpathclose%
\pgfusepath{stroke,fill}%
\end{pgfscope}%
\begin{pgfscope}%
\pgfpathrectangle{\pgfqpoint{0.800000in}{0.528000in}}{\pgfqpoint{4.960000in}{3.696000in}} %
\pgfusepath{clip}%
\pgfsetbuttcap%
\pgfsetroundjoin%
\definecolor{currentfill}{rgb}{0.000000,0.500000,0.000000}%
\pgfsetfillcolor{currentfill}%
\pgfsetlinewidth{1.003750pt}%
\definecolor{currentstroke}{rgb}{0.000000,0.500000,0.000000}%
\pgfsetstrokecolor{currentstroke}%
\pgfsetdash{}{0pt}%
\pgfpathmoveto{\pgfqpoint{4.707879in}{-1.167236in}}%
\pgfpathcurveto{\pgfqpoint{4.718929in}{-1.167236in}}{\pgfqpoint{4.729528in}{-1.162845in}}{\pgfqpoint{4.737342in}{-1.155032in}}%
\pgfpathcurveto{\pgfqpoint{4.745155in}{-1.147218in}}{\pgfqpoint{4.749545in}{-1.136619in}}{\pgfqpoint{4.749545in}{-1.125569in}}%
\pgfpathcurveto{\pgfqpoint{4.749545in}{-1.114519in}}{\pgfqpoint{4.745155in}{-1.103920in}}{\pgfqpoint{4.737342in}{-1.096106in}}%
\pgfpathcurveto{\pgfqpoint{4.729528in}{-1.088293in}}{\pgfqpoint{4.718929in}{-1.083902in}}{\pgfqpoint{4.707879in}{-1.083902in}}%
\pgfpathcurveto{\pgfqpoint{4.696829in}{-1.083902in}}{\pgfqpoint{4.686230in}{-1.088293in}}{\pgfqpoint{4.678416in}{-1.096106in}}%
\pgfpathcurveto{\pgfqpoint{4.670602in}{-1.103920in}}{\pgfqpoint{4.666212in}{-1.114519in}}{\pgfqpoint{4.666212in}{-1.125569in}}%
\pgfpathcurveto{\pgfqpoint{4.666212in}{-1.136619in}}{\pgfqpoint{4.670602in}{-1.147218in}}{\pgfqpoint{4.678416in}{-1.155032in}}%
\pgfpathcurveto{\pgfqpoint{4.686230in}{-1.162845in}}{\pgfqpoint{4.696829in}{-1.167236in}}{\pgfqpoint{4.707879in}{-1.167236in}}%
\pgfpathclose%
\pgfusepath{stroke,fill}%
\end{pgfscope}%
\begin{pgfscope}%
\pgfpathrectangle{\pgfqpoint{0.800000in}{0.528000in}}{\pgfqpoint{4.960000in}{3.696000in}} %
\pgfusepath{clip}%
\pgfsetbuttcap%
\pgfsetroundjoin%
\definecolor{currentfill}{rgb}{0.000000,0.500000,0.000000}%
\pgfsetfillcolor{currentfill}%
\pgfsetlinewidth{1.003750pt}%
\definecolor{currentstroke}{rgb}{0.000000,0.500000,0.000000}%
\pgfsetstrokecolor{currentstroke}%
\pgfsetdash{}{0pt}%
\pgfpathmoveto{\pgfqpoint{4.757980in}{-0.816509in}}%
\pgfpathcurveto{\pgfqpoint{4.769030in}{-0.816509in}}{\pgfqpoint{4.779629in}{-0.812119in}}{\pgfqpoint{4.787443in}{-0.804305in}}%
\pgfpathcurveto{\pgfqpoint{4.795256in}{-0.796491in}}{\pgfqpoint{4.799646in}{-0.785892in}}{\pgfqpoint{4.799646in}{-0.774842in}}%
\pgfpathcurveto{\pgfqpoint{4.799646in}{-0.763792in}}{\pgfqpoint{4.795256in}{-0.753193in}}{\pgfqpoint{4.787443in}{-0.745379in}}%
\pgfpathcurveto{\pgfqpoint{4.779629in}{-0.737566in}}{\pgfqpoint{4.769030in}{-0.733176in}}{\pgfqpoint{4.757980in}{-0.733176in}}%
\pgfpathcurveto{\pgfqpoint{4.746930in}{-0.733176in}}{\pgfqpoint{4.736331in}{-0.737566in}}{\pgfqpoint{4.728517in}{-0.745379in}}%
\pgfpathcurveto{\pgfqpoint{4.720703in}{-0.753193in}}{\pgfqpoint{4.716313in}{-0.763792in}}{\pgfqpoint{4.716313in}{-0.774842in}}%
\pgfpathcurveto{\pgfqpoint{4.716313in}{-0.785892in}}{\pgfqpoint{4.720703in}{-0.796491in}}{\pgfqpoint{4.728517in}{-0.804305in}}%
\pgfpathcurveto{\pgfqpoint{4.736331in}{-0.812119in}}{\pgfqpoint{4.746930in}{-0.816509in}}{\pgfqpoint{4.757980in}{-0.816509in}}%
\pgfpathclose%
\pgfusepath{stroke,fill}%
\end{pgfscope}%
\begin{pgfscope}%
\pgfpathrectangle{\pgfqpoint{0.800000in}{0.528000in}}{\pgfqpoint{4.960000in}{3.696000in}} %
\pgfusepath{clip}%
\pgfsetbuttcap%
\pgfsetroundjoin%
\definecolor{currentfill}{rgb}{0.000000,0.500000,0.000000}%
\pgfsetfillcolor{currentfill}%
\pgfsetlinewidth{1.003750pt}%
\definecolor{currentstroke}{rgb}{0.000000,0.500000,0.000000}%
\pgfsetstrokecolor{currentstroke}%
\pgfsetdash{}{0pt}%
\pgfpathmoveto{\pgfqpoint{4.808081in}{2.887262in}}%
\pgfpathcurveto{\pgfqpoint{4.819131in}{2.887262in}}{\pgfqpoint{4.829730in}{2.891652in}}{\pgfqpoint{4.837544in}{2.899466in}}%
\pgfpathcurveto{\pgfqpoint{4.845357in}{2.907279in}}{\pgfqpoint{4.849747in}{2.917878in}}{\pgfqpoint{4.849747in}{2.928928in}}%
\pgfpathcurveto{\pgfqpoint{4.849747in}{2.939979in}}{\pgfqpoint{4.845357in}{2.950578in}}{\pgfqpoint{4.837544in}{2.958391in}}%
\pgfpathcurveto{\pgfqpoint{4.829730in}{2.966205in}}{\pgfqpoint{4.819131in}{2.970595in}}{\pgfqpoint{4.808081in}{2.970595in}}%
\pgfpathcurveto{\pgfqpoint{4.797031in}{2.970595in}}{\pgfqpoint{4.786432in}{2.966205in}}{\pgfqpoint{4.778618in}{2.958391in}}%
\pgfpathcurveto{\pgfqpoint{4.770804in}{2.950578in}}{\pgfqpoint{4.766414in}{2.939979in}}{\pgfqpoint{4.766414in}{2.928928in}}%
\pgfpathcurveto{\pgfqpoint{4.766414in}{2.917878in}}{\pgfqpoint{4.770804in}{2.907279in}}{\pgfqpoint{4.778618in}{2.899466in}}%
\pgfpathcurveto{\pgfqpoint{4.786432in}{2.891652in}}{\pgfqpoint{4.797031in}{2.887262in}}{\pgfqpoint{4.808081in}{2.887262in}}%
\pgfpathclose%
\pgfusepath{stroke,fill}%
\end{pgfscope}%
\begin{pgfscope}%
\pgfpathrectangle{\pgfqpoint{0.800000in}{0.528000in}}{\pgfqpoint{4.960000in}{3.696000in}} %
\pgfusepath{clip}%
\pgfsetbuttcap%
\pgfsetroundjoin%
\definecolor{currentfill}{rgb}{0.000000,0.500000,0.000000}%
\pgfsetfillcolor{currentfill}%
\pgfsetlinewidth{1.003750pt}%
\definecolor{currentstroke}{rgb}{0.000000,0.500000,0.000000}%
\pgfsetstrokecolor{currentstroke}%
\pgfsetdash{}{0pt}%
\pgfpathmoveto{\pgfqpoint{4.858182in}{1.526323in}}%
\pgfpathcurveto{\pgfqpoint{4.869232in}{1.526323in}}{\pgfqpoint{4.879831in}{1.530713in}}{\pgfqpoint{4.887645in}{1.538527in}}%
\pgfpathcurveto{\pgfqpoint{4.895458in}{1.546341in}}{\pgfqpoint{4.899848in}{1.556940in}}{\pgfqpoint{4.899848in}{1.567990in}}%
\pgfpathcurveto{\pgfqpoint{4.899848in}{1.579040in}}{\pgfqpoint{4.895458in}{1.589639in}}{\pgfqpoint{4.887645in}{1.597452in}}%
\pgfpathcurveto{\pgfqpoint{4.879831in}{1.605266in}}{\pgfqpoint{4.869232in}{1.609656in}}{\pgfqpoint{4.858182in}{1.609656in}}%
\pgfpathcurveto{\pgfqpoint{4.847132in}{1.609656in}}{\pgfqpoint{4.836533in}{1.605266in}}{\pgfqpoint{4.828719in}{1.597452in}}%
\pgfpathcurveto{\pgfqpoint{4.820905in}{1.589639in}}{\pgfqpoint{4.816515in}{1.579040in}}{\pgfqpoint{4.816515in}{1.567990in}}%
\pgfpathcurveto{\pgfqpoint{4.816515in}{1.556940in}}{\pgfqpoint{4.820905in}{1.546341in}}{\pgfqpoint{4.828719in}{1.538527in}}%
\pgfpathcurveto{\pgfqpoint{4.836533in}{1.530713in}}{\pgfqpoint{4.847132in}{1.526323in}}{\pgfqpoint{4.858182in}{1.526323in}}%
\pgfpathclose%
\pgfusepath{stroke,fill}%
\end{pgfscope}%
\begin{pgfscope}%
\pgfpathrectangle{\pgfqpoint{0.800000in}{0.528000in}}{\pgfqpoint{4.960000in}{3.696000in}} %
\pgfusepath{clip}%
\pgfsetbuttcap%
\pgfsetroundjoin%
\definecolor{currentfill}{rgb}{0.000000,0.500000,0.000000}%
\pgfsetfillcolor{currentfill}%
\pgfsetlinewidth{1.003750pt}%
\definecolor{currentstroke}{rgb}{0.000000,0.500000,0.000000}%
\pgfsetstrokecolor{currentstroke}%
\pgfsetdash{}{0pt}%
\pgfpathmoveto{\pgfqpoint{4.908283in}{-1.316729in}}%
\pgfpathcurveto{\pgfqpoint{4.919333in}{-1.316729in}}{\pgfqpoint{4.929932in}{-1.312339in}}{\pgfqpoint{4.937746in}{-1.304525in}}%
\pgfpathcurveto{\pgfqpoint{4.945559in}{-1.296711in}}{\pgfqpoint{4.949949in}{-1.286112in}}{\pgfqpoint{4.949949in}{-1.275062in}}%
\pgfpathcurveto{\pgfqpoint{4.949949in}{-1.264012in}}{\pgfqpoint{4.945559in}{-1.253413in}}{\pgfqpoint{4.937746in}{-1.245599in}}%
\pgfpathcurveto{\pgfqpoint{4.929932in}{-1.237786in}}{\pgfqpoint{4.919333in}{-1.233396in}}{\pgfqpoint{4.908283in}{-1.233396in}}%
\pgfpathcurveto{\pgfqpoint{4.897233in}{-1.233396in}}{\pgfqpoint{4.886634in}{-1.237786in}}{\pgfqpoint{4.878820in}{-1.245599in}}%
\pgfpathcurveto{\pgfqpoint{4.871006in}{-1.253413in}}{\pgfqpoint{4.866616in}{-1.264012in}}{\pgfqpoint{4.866616in}{-1.275062in}}%
\pgfpathcurveto{\pgfqpoint{4.866616in}{-1.286112in}}{\pgfqpoint{4.871006in}{-1.296711in}}{\pgfqpoint{4.878820in}{-1.304525in}}%
\pgfpathcurveto{\pgfqpoint{4.886634in}{-1.312339in}}{\pgfqpoint{4.897233in}{-1.316729in}}{\pgfqpoint{4.908283in}{-1.316729in}}%
\pgfpathclose%
\pgfusepath{stroke,fill}%
\end{pgfscope}%
\begin{pgfscope}%
\pgfpathrectangle{\pgfqpoint{0.800000in}{0.528000in}}{\pgfqpoint{4.960000in}{3.696000in}} %
\pgfusepath{clip}%
\pgfsetbuttcap%
\pgfsetroundjoin%
\definecolor{currentfill}{rgb}{0.000000,0.500000,0.000000}%
\pgfsetfillcolor{currentfill}%
\pgfsetlinewidth{1.003750pt}%
\definecolor{currentstroke}{rgb}{0.000000,0.500000,0.000000}%
\pgfsetstrokecolor{currentstroke}%
\pgfsetdash{}{0pt}%
\pgfpathmoveto{\pgfqpoint{4.958384in}{1.926007in}}%
\pgfpathcurveto{\pgfqpoint{4.969434in}{1.926007in}}{\pgfqpoint{4.980033in}{1.930397in}}{\pgfqpoint{4.987847in}{1.938211in}}%
\pgfpathcurveto{\pgfqpoint{4.995660in}{1.946024in}}{\pgfqpoint{5.000051in}{1.956623in}}{\pgfqpoint{5.000051in}{1.967674in}}%
\pgfpathcurveto{\pgfqpoint{5.000051in}{1.978724in}}{\pgfqpoint{4.995660in}{1.989323in}}{\pgfqpoint{4.987847in}{1.997136in}}%
\pgfpathcurveto{\pgfqpoint{4.980033in}{2.004950in}}{\pgfqpoint{4.969434in}{2.009340in}}{\pgfqpoint{4.958384in}{2.009340in}}%
\pgfpathcurveto{\pgfqpoint{4.947334in}{2.009340in}}{\pgfqpoint{4.936735in}{2.004950in}}{\pgfqpoint{4.928921in}{1.997136in}}%
\pgfpathcurveto{\pgfqpoint{4.921107in}{1.989323in}}{\pgfqpoint{4.916717in}{1.978724in}}{\pgfqpoint{4.916717in}{1.967674in}}%
\pgfpathcurveto{\pgfqpoint{4.916717in}{1.956623in}}{\pgfqpoint{4.921107in}{1.946024in}}{\pgfqpoint{4.928921in}{1.938211in}}%
\pgfpathcurveto{\pgfqpoint{4.936735in}{1.930397in}}{\pgfqpoint{4.947334in}{1.926007in}}{\pgfqpoint{4.958384in}{1.926007in}}%
\pgfpathclose%
\pgfusepath{stroke,fill}%
\end{pgfscope}%
\begin{pgfscope}%
\pgfpathrectangle{\pgfqpoint{0.800000in}{0.528000in}}{\pgfqpoint{4.960000in}{3.696000in}} %
\pgfusepath{clip}%
\pgfsetbuttcap%
\pgfsetroundjoin%
\definecolor{currentfill}{rgb}{0.000000,0.500000,0.000000}%
\pgfsetfillcolor{currentfill}%
\pgfsetlinewidth{1.003750pt}%
\definecolor{currentstroke}{rgb}{0.000000,0.500000,0.000000}%
\pgfsetstrokecolor{currentstroke}%
\pgfsetdash{}{0pt}%
\pgfpathmoveto{\pgfqpoint{5.008485in}{2.410626in}}%
\pgfpathcurveto{\pgfqpoint{5.019535in}{2.410626in}}{\pgfqpoint{5.030134in}{2.415016in}}{\pgfqpoint{5.037948in}{2.422830in}}%
\pgfpathcurveto{\pgfqpoint{5.045761in}{2.430643in}}{\pgfqpoint{5.050152in}{2.441242in}}{\pgfqpoint{5.050152in}{2.452293in}}%
\pgfpathcurveto{\pgfqpoint{5.050152in}{2.463343in}}{\pgfqpoint{5.045761in}{2.473942in}}{\pgfqpoint{5.037948in}{2.481755in}}%
\pgfpathcurveto{\pgfqpoint{5.030134in}{2.489569in}}{\pgfqpoint{5.019535in}{2.493959in}}{\pgfqpoint{5.008485in}{2.493959in}}%
\pgfpathcurveto{\pgfqpoint{4.997435in}{2.493959in}}{\pgfqpoint{4.986836in}{2.489569in}}{\pgfqpoint{4.979022in}{2.481755in}}%
\pgfpathcurveto{\pgfqpoint{4.971208in}{2.473942in}}{\pgfqpoint{4.966818in}{2.463343in}}{\pgfqpoint{4.966818in}{2.452293in}}%
\pgfpathcurveto{\pgfqpoint{4.966818in}{2.441242in}}{\pgfqpoint{4.971208in}{2.430643in}}{\pgfqpoint{4.979022in}{2.422830in}}%
\pgfpathcurveto{\pgfqpoint{4.986836in}{2.415016in}}{\pgfqpoint{4.997435in}{2.410626in}}{\pgfqpoint{5.008485in}{2.410626in}}%
\pgfpathclose%
\pgfusepath{stroke,fill}%
\end{pgfscope}%
\begin{pgfscope}%
\pgfpathrectangle{\pgfqpoint{0.800000in}{0.528000in}}{\pgfqpoint{4.960000in}{3.696000in}} %
\pgfusepath{clip}%
\pgfsetbuttcap%
\pgfsetroundjoin%
\definecolor{currentfill}{rgb}{0.000000,0.500000,0.000000}%
\pgfsetfillcolor{currentfill}%
\pgfsetlinewidth{1.003750pt}%
\definecolor{currentstroke}{rgb}{0.000000,0.500000,0.000000}%
\pgfsetstrokecolor{currentstroke}%
\pgfsetdash{}{0pt}%
\pgfpathmoveto{\pgfqpoint{5.058586in}{3.393339in}}%
\pgfpathcurveto{\pgfqpoint{5.069636in}{3.393339in}}{\pgfqpoint{5.080235in}{3.397729in}}{\pgfqpoint{5.088049in}{3.405543in}}%
\pgfpathcurveto{\pgfqpoint{5.095862in}{3.413357in}}{\pgfqpoint{5.100253in}{3.423956in}}{\pgfqpoint{5.100253in}{3.435006in}}%
\pgfpathcurveto{\pgfqpoint{5.100253in}{3.446056in}}{\pgfqpoint{5.095862in}{3.456655in}}{\pgfqpoint{5.088049in}{3.464468in}}%
\pgfpathcurveto{\pgfqpoint{5.080235in}{3.472282in}}{\pgfqpoint{5.069636in}{3.476672in}}{\pgfqpoint{5.058586in}{3.476672in}}%
\pgfpathcurveto{\pgfqpoint{5.047536in}{3.476672in}}{\pgfqpoint{5.036937in}{3.472282in}}{\pgfqpoint{5.029123in}{3.464468in}}%
\pgfpathcurveto{\pgfqpoint{5.021309in}{3.456655in}}{\pgfqpoint{5.016919in}{3.446056in}}{\pgfqpoint{5.016919in}{3.435006in}}%
\pgfpathcurveto{\pgfqpoint{5.016919in}{3.423956in}}{\pgfqpoint{5.021309in}{3.413357in}}{\pgfqpoint{5.029123in}{3.405543in}}%
\pgfpathcurveto{\pgfqpoint{5.036937in}{3.397729in}}{\pgfqpoint{5.047536in}{3.393339in}}{\pgfqpoint{5.058586in}{3.393339in}}%
\pgfpathclose%
\pgfusepath{stroke,fill}%
\end{pgfscope}%
\begin{pgfscope}%
\pgfpathrectangle{\pgfqpoint{0.800000in}{0.528000in}}{\pgfqpoint{4.960000in}{3.696000in}} %
\pgfusepath{clip}%
\pgfsetbuttcap%
\pgfsetroundjoin%
\definecolor{currentfill}{rgb}{0.000000,0.500000,0.000000}%
\pgfsetfillcolor{currentfill}%
\pgfsetlinewidth{1.003750pt}%
\definecolor{currentstroke}{rgb}{0.000000,0.500000,0.000000}%
\pgfsetstrokecolor{currentstroke}%
\pgfsetdash{}{0pt}%
\pgfpathmoveto{\pgfqpoint{5.108687in}{-0.259895in}}%
\pgfpathcurveto{\pgfqpoint{5.119737in}{-0.259895in}}{\pgfqpoint{5.130336in}{-0.255504in}}{\pgfqpoint{5.138150in}{-0.247691in}}%
\pgfpathcurveto{\pgfqpoint{5.145963in}{-0.239877in}}{\pgfqpoint{5.150354in}{-0.229278in}}{\pgfqpoint{5.150354in}{-0.218228in}}%
\pgfpathcurveto{\pgfqpoint{5.150354in}{-0.207178in}}{\pgfqpoint{5.145963in}{-0.196579in}}{\pgfqpoint{5.138150in}{-0.188765in}}%
\pgfpathcurveto{\pgfqpoint{5.130336in}{-0.180952in}}{\pgfqpoint{5.119737in}{-0.176561in}}{\pgfqpoint{5.108687in}{-0.176561in}}%
\pgfpathcurveto{\pgfqpoint{5.097637in}{-0.176561in}}{\pgfqpoint{5.087038in}{-0.180952in}}{\pgfqpoint{5.079224in}{-0.188765in}}%
\pgfpathcurveto{\pgfqpoint{5.071410in}{-0.196579in}}{\pgfqpoint{5.067020in}{-0.207178in}}{\pgfqpoint{5.067020in}{-0.218228in}}%
\pgfpathcurveto{\pgfqpoint{5.067020in}{-0.229278in}}{\pgfqpoint{5.071410in}{-0.239877in}}{\pgfqpoint{5.079224in}{-0.247691in}}%
\pgfpathcurveto{\pgfqpoint{5.087038in}{-0.255504in}}{\pgfqpoint{5.097637in}{-0.259895in}}{\pgfqpoint{5.108687in}{-0.259895in}}%
\pgfpathclose%
\pgfusepath{stroke,fill}%
\end{pgfscope}%
\begin{pgfscope}%
\pgfpathrectangle{\pgfqpoint{0.800000in}{0.528000in}}{\pgfqpoint{4.960000in}{3.696000in}} %
\pgfusepath{clip}%
\pgfsetbuttcap%
\pgfsetroundjoin%
\definecolor{currentfill}{rgb}{0.000000,0.500000,0.000000}%
\pgfsetfillcolor{currentfill}%
\pgfsetlinewidth{1.003750pt}%
\definecolor{currentstroke}{rgb}{0.000000,0.500000,0.000000}%
\pgfsetstrokecolor{currentstroke}%
\pgfsetdash{}{0pt}%
\pgfpathmoveto{\pgfqpoint{5.158788in}{-0.308708in}}%
\pgfpathcurveto{\pgfqpoint{5.169838in}{-0.308708in}}{\pgfqpoint{5.180437in}{-0.304318in}}{\pgfqpoint{5.188251in}{-0.296504in}}%
\pgfpathcurveto{\pgfqpoint{5.196064in}{-0.288690in}}{\pgfqpoint{5.200455in}{-0.278091in}}{\pgfqpoint{5.200455in}{-0.267041in}}%
\pgfpathcurveto{\pgfqpoint{5.200455in}{-0.255991in}}{\pgfqpoint{5.196064in}{-0.245392in}}{\pgfqpoint{5.188251in}{-0.237578in}}%
\pgfpathcurveto{\pgfqpoint{5.180437in}{-0.229765in}}{\pgfqpoint{5.169838in}{-0.225375in}}{\pgfqpoint{5.158788in}{-0.225375in}}%
\pgfpathcurveto{\pgfqpoint{5.147738in}{-0.225375in}}{\pgfqpoint{5.137139in}{-0.229765in}}{\pgfqpoint{5.129325in}{-0.237578in}}%
\pgfpathcurveto{\pgfqpoint{5.121511in}{-0.245392in}}{\pgfqpoint{5.117121in}{-0.255991in}}{\pgfqpoint{5.117121in}{-0.267041in}}%
\pgfpathcurveto{\pgfqpoint{5.117121in}{-0.278091in}}{\pgfqpoint{5.121511in}{-0.288690in}}{\pgfqpoint{5.129325in}{-0.296504in}}%
\pgfpathcurveto{\pgfqpoint{5.137139in}{-0.304318in}}{\pgfqpoint{5.147738in}{-0.308708in}}{\pgfqpoint{5.158788in}{-0.308708in}}%
\pgfpathclose%
\pgfusepath{stroke,fill}%
\end{pgfscope}%
\begin{pgfscope}%
\pgfpathrectangle{\pgfqpoint{0.800000in}{0.528000in}}{\pgfqpoint{4.960000in}{3.696000in}} %
\pgfusepath{clip}%
\pgfsetbuttcap%
\pgfsetroundjoin%
\definecolor{currentfill}{rgb}{0.000000,0.500000,0.000000}%
\pgfsetfillcolor{currentfill}%
\pgfsetlinewidth{1.003750pt}%
\definecolor{currentstroke}{rgb}{0.000000,0.500000,0.000000}%
\pgfsetstrokecolor{currentstroke}%
\pgfsetdash{}{0pt}%
\pgfpathmoveto{\pgfqpoint{5.208889in}{2.282065in}}%
\pgfpathcurveto{\pgfqpoint{5.219939in}{2.282065in}}{\pgfqpoint{5.230538in}{2.286455in}}{\pgfqpoint{5.238352in}{2.294269in}}%
\pgfpathcurveto{\pgfqpoint{5.246165in}{2.302082in}}{\pgfqpoint{5.250556in}{2.312681in}}{\pgfqpoint{5.250556in}{2.323731in}}%
\pgfpathcurveto{\pgfqpoint{5.250556in}{2.334781in}}{\pgfqpoint{5.246165in}{2.345381in}}{\pgfqpoint{5.238352in}{2.353194in}}%
\pgfpathcurveto{\pgfqpoint{5.230538in}{2.361008in}}{\pgfqpoint{5.219939in}{2.365398in}}{\pgfqpoint{5.208889in}{2.365398in}}%
\pgfpathcurveto{\pgfqpoint{5.197839in}{2.365398in}}{\pgfqpoint{5.187240in}{2.361008in}}{\pgfqpoint{5.179426in}{2.353194in}}%
\pgfpathcurveto{\pgfqpoint{5.171612in}{2.345381in}}{\pgfqpoint{5.167222in}{2.334781in}}{\pgfqpoint{5.167222in}{2.323731in}}%
\pgfpathcurveto{\pgfqpoint{5.167222in}{2.312681in}}{\pgfqpoint{5.171612in}{2.302082in}}{\pgfqpoint{5.179426in}{2.294269in}}%
\pgfpathcurveto{\pgfqpoint{5.187240in}{2.286455in}}{\pgfqpoint{5.197839in}{2.282065in}}{\pgfqpoint{5.208889in}{2.282065in}}%
\pgfpathclose%
\pgfusepath{stroke,fill}%
\end{pgfscope}%
\begin{pgfscope}%
\pgfpathrectangle{\pgfqpoint{0.800000in}{0.528000in}}{\pgfqpoint{4.960000in}{3.696000in}} %
\pgfusepath{clip}%
\pgfsetbuttcap%
\pgfsetroundjoin%
\definecolor{currentfill}{rgb}{0.000000,0.500000,0.000000}%
\pgfsetfillcolor{currentfill}%
\pgfsetlinewidth{1.003750pt}%
\definecolor{currentstroke}{rgb}{0.000000,0.500000,0.000000}%
\pgfsetstrokecolor{currentstroke}%
\pgfsetdash{}{0pt}%
\pgfpathmoveto{\pgfqpoint{5.258990in}{2.944829in}}%
\pgfpathcurveto{\pgfqpoint{5.270040in}{2.944829in}}{\pgfqpoint{5.280639in}{2.949219in}}{\pgfqpoint{5.288453in}{2.957033in}}%
\pgfpathcurveto{\pgfqpoint{5.296266in}{2.964847in}}{\pgfqpoint{5.300657in}{2.975446in}}{\pgfqpoint{5.300657in}{2.986496in}}%
\pgfpathcurveto{\pgfqpoint{5.300657in}{2.997546in}}{\pgfqpoint{5.296266in}{3.008145in}}{\pgfqpoint{5.288453in}{3.015959in}}%
\pgfpathcurveto{\pgfqpoint{5.280639in}{3.023772in}}{\pgfqpoint{5.270040in}{3.028162in}}{\pgfqpoint{5.258990in}{3.028162in}}%
\pgfpathcurveto{\pgfqpoint{5.247940in}{3.028162in}}{\pgfqpoint{5.237341in}{3.023772in}}{\pgfqpoint{5.229527in}{3.015959in}}%
\pgfpathcurveto{\pgfqpoint{5.221713in}{3.008145in}}{\pgfqpoint{5.217323in}{2.997546in}}{\pgfqpoint{5.217323in}{2.986496in}}%
\pgfpathcurveto{\pgfqpoint{5.217323in}{2.975446in}}{\pgfqpoint{5.221713in}{2.964847in}}{\pgfqpoint{5.229527in}{2.957033in}}%
\pgfpathcurveto{\pgfqpoint{5.237341in}{2.949219in}}{\pgfqpoint{5.247940in}{2.944829in}}{\pgfqpoint{5.258990in}{2.944829in}}%
\pgfpathclose%
\pgfusepath{stroke,fill}%
\end{pgfscope}%
\begin{pgfscope}%
\pgfpathrectangle{\pgfqpoint{0.800000in}{0.528000in}}{\pgfqpoint{4.960000in}{3.696000in}} %
\pgfusepath{clip}%
\pgfsetbuttcap%
\pgfsetroundjoin%
\definecolor{currentfill}{rgb}{0.000000,0.500000,0.000000}%
\pgfsetfillcolor{currentfill}%
\pgfsetlinewidth{1.003750pt}%
\definecolor{currentstroke}{rgb}{0.000000,0.500000,0.000000}%
\pgfsetstrokecolor{currentstroke}%
\pgfsetdash{}{0pt}%
\pgfpathmoveto{\pgfqpoint{5.309091in}{-0.657110in}}%
\pgfpathcurveto{\pgfqpoint{5.320141in}{-0.657110in}}{\pgfqpoint{5.330740in}{-0.652720in}}{\pgfqpoint{5.338554in}{-0.644906in}}%
\pgfpathcurveto{\pgfqpoint{5.346367in}{-0.637092in}}{\pgfqpoint{5.350758in}{-0.626493in}}{\pgfqpoint{5.350758in}{-0.615443in}}%
\pgfpathcurveto{\pgfqpoint{5.350758in}{-0.604393in}}{\pgfqpoint{5.346367in}{-0.593794in}}{\pgfqpoint{5.338554in}{-0.585980in}}%
\pgfpathcurveto{\pgfqpoint{5.330740in}{-0.578167in}}{\pgfqpoint{5.320141in}{-0.573777in}}{\pgfqpoint{5.309091in}{-0.573777in}}%
\pgfpathcurveto{\pgfqpoint{5.298041in}{-0.573777in}}{\pgfqpoint{5.287442in}{-0.578167in}}{\pgfqpoint{5.279628in}{-0.585980in}}%
\pgfpathcurveto{\pgfqpoint{5.271815in}{-0.593794in}}{\pgfqpoint{5.267424in}{-0.604393in}}{\pgfqpoint{5.267424in}{-0.615443in}}%
\pgfpathcurveto{\pgfqpoint{5.267424in}{-0.626493in}}{\pgfqpoint{5.271815in}{-0.637092in}}{\pgfqpoint{5.279628in}{-0.644906in}}%
\pgfpathcurveto{\pgfqpoint{5.287442in}{-0.652720in}}{\pgfqpoint{5.298041in}{-0.657110in}}{\pgfqpoint{5.309091in}{-0.657110in}}%
\pgfpathclose%
\pgfusepath{stroke,fill}%
\end{pgfscope}%
\begin{pgfscope}%
\pgfpathrectangle{\pgfqpoint{0.800000in}{0.528000in}}{\pgfqpoint{4.960000in}{3.696000in}} %
\pgfusepath{clip}%
\pgfsetbuttcap%
\pgfsetroundjoin%
\definecolor{currentfill}{rgb}{0.000000,0.500000,0.000000}%
\pgfsetfillcolor{currentfill}%
\pgfsetlinewidth{1.003750pt}%
\definecolor{currentstroke}{rgb}{0.000000,0.500000,0.000000}%
\pgfsetstrokecolor{currentstroke}%
\pgfsetdash{}{0pt}%
\pgfpathmoveto{\pgfqpoint{5.359192in}{2.088842in}}%
\pgfpathcurveto{\pgfqpoint{5.370242in}{2.088842in}}{\pgfqpoint{5.380841in}{2.093233in}}{\pgfqpoint{5.388655in}{2.101046in}}%
\pgfpathcurveto{\pgfqpoint{5.396468in}{2.108860in}}{\pgfqpoint{5.400859in}{2.119459in}}{\pgfqpoint{5.400859in}{2.130509in}}%
\pgfpathcurveto{\pgfqpoint{5.400859in}{2.141559in}}{\pgfqpoint{5.396468in}{2.152158in}}{\pgfqpoint{5.388655in}{2.159972in}}%
\pgfpathcurveto{\pgfqpoint{5.380841in}{2.167786in}}{\pgfqpoint{5.370242in}{2.172176in}}{\pgfqpoint{5.359192in}{2.172176in}}%
\pgfpathcurveto{\pgfqpoint{5.348142in}{2.172176in}}{\pgfqpoint{5.337543in}{2.167786in}}{\pgfqpoint{5.329729in}{2.159972in}}%
\pgfpathcurveto{\pgfqpoint{5.321916in}{2.152158in}}{\pgfqpoint{5.317525in}{2.141559in}}{\pgfqpoint{5.317525in}{2.130509in}}%
\pgfpathcurveto{\pgfqpoint{5.317525in}{2.119459in}}{\pgfqpoint{5.321916in}{2.108860in}}{\pgfqpoint{5.329729in}{2.101046in}}%
\pgfpathcurveto{\pgfqpoint{5.337543in}{2.093233in}}{\pgfqpoint{5.348142in}{2.088842in}}{\pgfqpoint{5.359192in}{2.088842in}}%
\pgfpathclose%
\pgfusepath{stroke,fill}%
\end{pgfscope}%
\begin{pgfscope}%
\pgfpathrectangle{\pgfqpoint{0.800000in}{0.528000in}}{\pgfqpoint{4.960000in}{3.696000in}} %
\pgfusepath{clip}%
\pgfsetbuttcap%
\pgfsetroundjoin%
\definecolor{currentfill}{rgb}{0.000000,0.500000,0.000000}%
\pgfsetfillcolor{currentfill}%
\pgfsetlinewidth{1.003750pt}%
\definecolor{currentstroke}{rgb}{0.000000,0.500000,0.000000}%
\pgfsetstrokecolor{currentstroke}%
\pgfsetdash{}{0pt}%
\pgfpathmoveto{\pgfqpoint{5.409293in}{1.291735in}}%
\pgfpathcurveto{\pgfqpoint{5.420343in}{1.291735in}}{\pgfqpoint{5.430942in}{1.296126in}}{\pgfqpoint{5.438756in}{1.303939in}}%
\pgfpathcurveto{\pgfqpoint{5.446569in}{1.311753in}}{\pgfqpoint{5.450960in}{1.322352in}}{\pgfqpoint{5.450960in}{1.333402in}}%
\pgfpathcurveto{\pgfqpoint{5.450960in}{1.344452in}}{\pgfqpoint{5.446569in}{1.355051in}}{\pgfqpoint{5.438756in}{1.362865in}}%
\pgfpathcurveto{\pgfqpoint{5.430942in}{1.370679in}}{\pgfqpoint{5.420343in}{1.375069in}}{\pgfqpoint{5.409293in}{1.375069in}}%
\pgfpathcurveto{\pgfqpoint{5.398243in}{1.375069in}}{\pgfqpoint{5.387644in}{1.370679in}}{\pgfqpoint{5.379830in}{1.362865in}}%
\pgfpathcurveto{\pgfqpoint{5.372017in}{1.355051in}}{\pgfqpoint{5.367626in}{1.344452in}}{\pgfqpoint{5.367626in}{1.333402in}}%
\pgfpathcurveto{\pgfqpoint{5.367626in}{1.322352in}}{\pgfqpoint{5.372017in}{1.311753in}}{\pgfqpoint{5.379830in}{1.303939in}}%
\pgfpathcurveto{\pgfqpoint{5.387644in}{1.296126in}}{\pgfqpoint{5.398243in}{1.291735in}}{\pgfqpoint{5.409293in}{1.291735in}}%
\pgfpathclose%
\pgfusepath{stroke,fill}%
\end{pgfscope}%
\begin{pgfscope}%
\pgfpathrectangle{\pgfqpoint{0.800000in}{0.528000in}}{\pgfqpoint{4.960000in}{3.696000in}} %
\pgfusepath{clip}%
\pgfsetbuttcap%
\pgfsetroundjoin%
\definecolor{currentfill}{rgb}{0.000000,0.500000,0.000000}%
\pgfsetfillcolor{currentfill}%
\pgfsetlinewidth{1.003750pt}%
\definecolor{currentstroke}{rgb}{0.000000,0.500000,0.000000}%
\pgfsetstrokecolor{currentstroke}%
\pgfsetdash{}{0pt}%
\pgfpathmoveto{\pgfqpoint{5.459394in}{2.160995in}}%
\pgfpathcurveto{\pgfqpoint{5.470444in}{2.160995in}}{\pgfqpoint{5.481043in}{2.165385in}}{\pgfqpoint{5.488857in}{2.173199in}}%
\pgfpathcurveto{\pgfqpoint{5.496670in}{2.181012in}}{\pgfqpoint{5.501061in}{2.191611in}}{\pgfqpoint{5.501061in}{2.202662in}}%
\pgfpathcurveto{\pgfqpoint{5.501061in}{2.213712in}}{\pgfqpoint{5.496670in}{2.224311in}}{\pgfqpoint{5.488857in}{2.232124in}}%
\pgfpathcurveto{\pgfqpoint{5.481043in}{2.239938in}}{\pgfqpoint{5.470444in}{2.244328in}}{\pgfqpoint{5.459394in}{2.244328in}}%
\pgfpathcurveto{\pgfqpoint{5.448344in}{2.244328in}}{\pgfqpoint{5.437745in}{2.239938in}}{\pgfqpoint{5.429931in}{2.232124in}}%
\pgfpathcurveto{\pgfqpoint{5.422118in}{2.224311in}}{\pgfqpoint{5.417727in}{2.213712in}}{\pgfqpoint{5.417727in}{2.202662in}}%
\pgfpathcurveto{\pgfqpoint{5.417727in}{2.191611in}}{\pgfqpoint{5.422118in}{2.181012in}}{\pgfqpoint{5.429931in}{2.173199in}}%
\pgfpathcurveto{\pgfqpoint{5.437745in}{2.165385in}}{\pgfqpoint{5.448344in}{2.160995in}}{\pgfqpoint{5.459394in}{2.160995in}}%
\pgfpathclose%
\pgfusepath{stroke,fill}%
\end{pgfscope}%
\begin{pgfscope}%
\pgfpathrectangle{\pgfqpoint{0.800000in}{0.528000in}}{\pgfqpoint{4.960000in}{3.696000in}} %
\pgfusepath{clip}%
\pgfsetbuttcap%
\pgfsetroundjoin%
\definecolor{currentfill}{rgb}{0.000000,0.500000,0.000000}%
\pgfsetfillcolor{currentfill}%
\pgfsetlinewidth{1.003750pt}%
\definecolor{currentstroke}{rgb}{0.000000,0.500000,0.000000}%
\pgfsetstrokecolor{currentstroke}%
\pgfsetdash{}{0pt}%
\pgfpathmoveto{\pgfqpoint{5.509495in}{2.546164in}}%
\pgfpathcurveto{\pgfqpoint{5.520545in}{2.546164in}}{\pgfqpoint{5.531144in}{2.550554in}}{\pgfqpoint{5.538958in}{2.558368in}}%
\pgfpathcurveto{\pgfqpoint{5.546771in}{2.566181in}}{\pgfqpoint{5.551162in}{2.576780in}}{\pgfqpoint{5.551162in}{2.587830in}}%
\pgfpathcurveto{\pgfqpoint{5.551162in}{2.598881in}}{\pgfqpoint{5.546771in}{2.609480in}}{\pgfqpoint{5.538958in}{2.617293in}}%
\pgfpathcurveto{\pgfqpoint{5.531144in}{2.625107in}}{\pgfqpoint{5.520545in}{2.629497in}}{\pgfqpoint{5.509495in}{2.629497in}}%
\pgfpathcurveto{\pgfqpoint{5.498445in}{2.629497in}}{\pgfqpoint{5.487846in}{2.625107in}}{\pgfqpoint{5.480032in}{2.617293in}}%
\pgfpathcurveto{\pgfqpoint{5.472219in}{2.609480in}}{\pgfqpoint{5.467828in}{2.598881in}}{\pgfqpoint{5.467828in}{2.587830in}}%
\pgfpathcurveto{\pgfqpoint{5.467828in}{2.576780in}}{\pgfqpoint{5.472219in}{2.566181in}}{\pgfqpoint{5.480032in}{2.558368in}}%
\pgfpathcurveto{\pgfqpoint{5.487846in}{2.550554in}}{\pgfqpoint{5.498445in}{2.546164in}}{\pgfqpoint{5.509495in}{2.546164in}}%
\pgfpathclose%
\pgfusepath{stroke,fill}%
\end{pgfscope}%
\begin{pgfscope}%
\pgfpathrectangle{\pgfqpoint{0.800000in}{0.528000in}}{\pgfqpoint{4.960000in}{3.696000in}} %
\pgfusepath{clip}%
\pgfsetbuttcap%
\pgfsetroundjoin%
\definecolor{currentfill}{rgb}{0.000000,0.500000,0.000000}%
\pgfsetfillcolor{currentfill}%
\pgfsetlinewidth{1.003750pt}%
\definecolor{currentstroke}{rgb}{0.000000,0.500000,0.000000}%
\pgfsetstrokecolor{currentstroke}%
\pgfsetdash{}{0pt}%
\pgfpathmoveto{\pgfqpoint{5.559596in}{2.201759in}}%
\pgfpathcurveto{\pgfqpoint{5.570646in}{2.201759in}}{\pgfqpoint{5.581245in}{2.206149in}}{\pgfqpoint{5.589059in}{2.213963in}}%
\pgfpathcurveto{\pgfqpoint{5.596872in}{2.221777in}}{\pgfqpoint{5.601263in}{2.232376in}}{\pgfqpoint{5.601263in}{2.243426in}}%
\pgfpathcurveto{\pgfqpoint{5.601263in}{2.254476in}}{\pgfqpoint{5.596872in}{2.265075in}}{\pgfqpoint{5.589059in}{2.272888in}}%
\pgfpathcurveto{\pgfqpoint{5.581245in}{2.280702in}}{\pgfqpoint{5.570646in}{2.285092in}}{\pgfqpoint{5.559596in}{2.285092in}}%
\pgfpathcurveto{\pgfqpoint{5.548546in}{2.285092in}}{\pgfqpoint{5.537947in}{2.280702in}}{\pgfqpoint{5.530133in}{2.272888in}}%
\pgfpathcurveto{\pgfqpoint{5.522320in}{2.265075in}}{\pgfqpoint{5.517929in}{2.254476in}}{\pgfqpoint{5.517929in}{2.243426in}}%
\pgfpathcurveto{\pgfqpoint{5.517929in}{2.232376in}}{\pgfqpoint{5.522320in}{2.221777in}}{\pgfqpoint{5.530133in}{2.213963in}}%
\pgfpathcurveto{\pgfqpoint{5.537947in}{2.206149in}}{\pgfqpoint{5.548546in}{2.201759in}}{\pgfqpoint{5.559596in}{2.201759in}}%
\pgfpathclose%
\pgfusepath{stroke,fill}%
\end{pgfscope}%
\begin{pgfscope}%
\pgfpathrectangle{\pgfqpoint{0.800000in}{0.528000in}}{\pgfqpoint{4.960000in}{3.696000in}} %
\pgfusepath{clip}%
\pgfsetbuttcap%
\pgfsetroundjoin%
\definecolor{currentfill}{rgb}{0.000000,0.500000,0.000000}%
\pgfsetfillcolor{currentfill}%
\pgfsetlinewidth{1.003750pt}%
\definecolor{currentstroke}{rgb}{0.000000,0.500000,0.000000}%
\pgfsetstrokecolor{currentstroke}%
\pgfsetdash{}{0pt}%
\pgfpathmoveto{\pgfqpoint{5.609697in}{0.757946in}}%
\pgfpathcurveto{\pgfqpoint{5.620747in}{0.757946in}}{\pgfqpoint{5.631346in}{0.762336in}}{\pgfqpoint{5.639160in}{0.770149in}}%
\pgfpathcurveto{\pgfqpoint{5.646973in}{0.777963in}}{\pgfqpoint{5.651364in}{0.788562in}}{\pgfqpoint{5.651364in}{0.799612in}}%
\pgfpathcurveto{\pgfqpoint{5.651364in}{0.810662in}}{\pgfqpoint{5.646973in}{0.821261in}}{\pgfqpoint{5.639160in}{0.829075in}}%
\pgfpathcurveto{\pgfqpoint{5.631346in}{0.836889in}}{\pgfqpoint{5.620747in}{0.841279in}}{\pgfqpoint{5.609697in}{0.841279in}}%
\pgfpathcurveto{\pgfqpoint{5.598647in}{0.841279in}}{\pgfqpoint{5.588048in}{0.836889in}}{\pgfqpoint{5.580234in}{0.829075in}}%
\pgfpathcurveto{\pgfqpoint{5.572421in}{0.821261in}}{\pgfqpoint{5.568030in}{0.810662in}}{\pgfqpoint{5.568030in}{0.799612in}}%
\pgfpathcurveto{\pgfqpoint{5.568030in}{0.788562in}}{\pgfqpoint{5.572421in}{0.777963in}}{\pgfqpoint{5.580234in}{0.770149in}}%
\pgfpathcurveto{\pgfqpoint{5.588048in}{0.762336in}}{\pgfqpoint{5.598647in}{0.757946in}}{\pgfqpoint{5.609697in}{0.757946in}}%
\pgfpathclose%
\pgfusepath{stroke,fill}%
\end{pgfscope}%
\begin{pgfscope}%
\pgfpathrectangle{\pgfqpoint{0.800000in}{0.528000in}}{\pgfqpoint{4.960000in}{3.696000in}} %
\pgfusepath{clip}%
\pgfsetbuttcap%
\pgfsetroundjoin%
\definecolor{currentfill}{rgb}{0.000000,0.500000,0.000000}%
\pgfsetfillcolor{currentfill}%
\pgfsetlinewidth{1.003750pt}%
\definecolor{currentstroke}{rgb}{0.000000,0.500000,0.000000}%
\pgfsetstrokecolor{currentstroke}%
\pgfsetdash{}{0pt}%
\pgfpathmoveto{\pgfqpoint{5.659798in}{0.124574in}}%
\pgfpathcurveto{\pgfqpoint{5.670848in}{0.124574in}}{\pgfqpoint{5.681447in}{0.128965in}}{\pgfqpoint{5.689261in}{0.136778in}}%
\pgfpathcurveto{\pgfqpoint{5.697074in}{0.144592in}}{\pgfqpoint{5.701465in}{0.155191in}}{\pgfqpoint{5.701465in}{0.166241in}}%
\pgfpathcurveto{\pgfqpoint{5.701465in}{0.177291in}}{\pgfqpoint{5.697074in}{0.187890in}}{\pgfqpoint{5.689261in}{0.195704in}}%
\pgfpathcurveto{\pgfqpoint{5.681447in}{0.203518in}}{\pgfqpoint{5.670848in}{0.207908in}}{\pgfqpoint{5.659798in}{0.207908in}}%
\pgfpathcurveto{\pgfqpoint{5.648748in}{0.207908in}}{\pgfqpoint{5.638149in}{0.203518in}}{\pgfqpoint{5.630335in}{0.195704in}}%
\pgfpathcurveto{\pgfqpoint{5.622522in}{0.187890in}}{\pgfqpoint{5.618131in}{0.177291in}}{\pgfqpoint{5.618131in}{0.166241in}}%
\pgfpathcurveto{\pgfqpoint{5.618131in}{0.155191in}}{\pgfqpoint{5.622522in}{0.144592in}}{\pgfqpoint{5.630335in}{0.136778in}}%
\pgfpathcurveto{\pgfqpoint{5.638149in}{0.128965in}}{\pgfqpoint{5.648748in}{0.124574in}}{\pgfqpoint{5.659798in}{0.124574in}}%
\pgfpathclose%
\pgfusepath{stroke,fill}%
\end{pgfscope}%
\begin{pgfscope}%
\pgfpathrectangle{\pgfqpoint{0.800000in}{0.528000in}}{\pgfqpoint{4.960000in}{3.696000in}} %
\pgfusepath{clip}%
\pgfsetbuttcap%
\pgfsetroundjoin%
\definecolor{currentfill}{rgb}{0.000000,0.500000,0.000000}%
\pgfsetfillcolor{currentfill}%
\pgfsetlinewidth{1.003750pt}%
\definecolor{currentstroke}{rgb}{0.000000,0.500000,0.000000}%
\pgfsetstrokecolor{currentstroke}%
\pgfsetdash{}{0pt}%
\pgfpathmoveto{\pgfqpoint{5.709899in}{0.257535in}}%
\pgfpathcurveto{\pgfqpoint{5.720949in}{0.257535in}}{\pgfqpoint{5.731548in}{0.261925in}}{\pgfqpoint{5.739362in}{0.269739in}}%
\pgfpathcurveto{\pgfqpoint{5.747175in}{0.277552in}}{\pgfqpoint{5.751566in}{0.288151in}}{\pgfqpoint{5.751566in}{0.299201in}}%
\pgfpathcurveto{\pgfqpoint{5.751566in}{0.310251in}}{\pgfqpoint{5.747175in}{0.320851in}}{\pgfqpoint{5.739362in}{0.328664in}}%
\pgfpathcurveto{\pgfqpoint{5.731548in}{0.336478in}}{\pgfqpoint{5.720949in}{0.340868in}}{\pgfqpoint{5.709899in}{0.340868in}}%
\pgfpathcurveto{\pgfqpoint{5.698849in}{0.340868in}}{\pgfqpoint{5.688250in}{0.336478in}}{\pgfqpoint{5.680436in}{0.328664in}}%
\pgfpathcurveto{\pgfqpoint{5.672623in}{0.320851in}}{\pgfqpoint{5.668232in}{0.310251in}}{\pgfqpoint{5.668232in}{0.299201in}}%
\pgfpathcurveto{\pgfqpoint{5.668232in}{0.288151in}}{\pgfqpoint{5.672623in}{0.277552in}}{\pgfqpoint{5.680436in}{0.269739in}}%
\pgfpathcurveto{\pgfqpoint{5.688250in}{0.261925in}}{\pgfqpoint{5.698849in}{0.257535in}}{\pgfqpoint{5.709899in}{0.257535in}}%
\pgfpathclose%
\pgfusepath{stroke,fill}%
\end{pgfscope}%
\begin{pgfscope}%
\pgfpathrectangle{\pgfqpoint{0.800000in}{0.528000in}}{\pgfqpoint{4.960000in}{3.696000in}} %
\pgfusepath{clip}%
\pgfsetbuttcap%
\pgfsetroundjoin%
\definecolor{currentfill}{rgb}{0.000000,0.500000,0.000000}%
\pgfsetfillcolor{currentfill}%
\pgfsetlinewidth{1.003750pt}%
\definecolor{currentstroke}{rgb}{0.000000,0.500000,0.000000}%
\pgfsetstrokecolor{currentstroke}%
\pgfsetdash{}{0pt}%
\pgfpathmoveto{\pgfqpoint{5.760000in}{-0.075474in}}%
\pgfpathcurveto{\pgfqpoint{5.771050in}{-0.075474in}}{\pgfqpoint{5.781649in}{-0.071084in}}{\pgfqpoint{5.789463in}{-0.063270in}}%
\pgfpathcurveto{\pgfqpoint{5.797276in}{-0.055456in}}{\pgfqpoint{5.801667in}{-0.044857in}}{\pgfqpoint{5.801667in}{-0.033807in}}%
\pgfpathcurveto{\pgfqpoint{5.801667in}{-0.022757in}}{\pgfqpoint{5.797276in}{-0.012158in}}{\pgfqpoint{5.789463in}{-0.004344in}}%
\pgfpathcurveto{\pgfqpoint{5.781649in}{0.003469in}}{\pgfqpoint{5.771050in}{0.007859in}}{\pgfqpoint{5.760000in}{0.007859in}}%
\pgfpathcurveto{\pgfqpoint{5.748950in}{0.007859in}}{\pgfqpoint{5.738351in}{0.003469in}}{\pgfqpoint{5.730537in}{-0.004344in}}%
\pgfpathcurveto{\pgfqpoint{5.722724in}{-0.012158in}}{\pgfqpoint{5.718333in}{-0.022757in}}{\pgfqpoint{5.718333in}{-0.033807in}}%
\pgfpathcurveto{\pgfqpoint{5.718333in}{-0.044857in}}{\pgfqpoint{5.722724in}{-0.055456in}}{\pgfqpoint{5.730537in}{-0.063270in}}%
\pgfpathcurveto{\pgfqpoint{5.738351in}{-0.071084in}}{\pgfqpoint{5.748950in}{-0.075474in}}{\pgfqpoint{5.760000in}{-0.075474in}}%
\pgfpathclose%
\pgfusepath{stroke,fill}%
\end{pgfscope}%
\begin{pgfscope}%
\pgfsetbuttcap%
\pgfsetroundjoin%
\definecolor{currentfill}{rgb}{0.000000,0.000000,0.000000}%
\pgfsetfillcolor{currentfill}%
\pgfsetlinewidth{0.803000pt}%
\definecolor{currentstroke}{rgb}{0.000000,0.000000,0.000000}%
\pgfsetstrokecolor{currentstroke}%
\pgfsetdash{}{0pt}%
\pgfsys@defobject{currentmarker}{\pgfqpoint{0.000000in}{-0.048611in}}{\pgfqpoint{0.000000in}{0.000000in}}{%
\pgfpathmoveto{\pgfqpoint{0.000000in}{0.000000in}}%
\pgfpathlineto{\pgfqpoint{0.000000in}{-0.048611in}}%
\pgfusepath{stroke,fill}%
}%
\begin{pgfscope}%
\pgfsys@transformshift{0.800000in}{0.528000in}%
\pgfsys@useobject{currentmarker}{}%
\end{pgfscope}%
\end{pgfscope}%
\begin{pgfscope}%
\pgftext[x=0.800000in,y=0.430778in,,top]{\sffamily\fontsize{10.000000}{12.000000}\selectfont 1.0}%
\end{pgfscope}%
\begin{pgfscope}%
\pgfsetbuttcap%
\pgfsetroundjoin%
\definecolor{currentfill}{rgb}{0.000000,0.000000,0.000000}%
\pgfsetfillcolor{currentfill}%
\pgfsetlinewidth{0.803000pt}%
\definecolor{currentstroke}{rgb}{0.000000,0.000000,0.000000}%
\pgfsetstrokecolor{currentstroke}%
\pgfsetdash{}{0pt}%
\pgfsys@defobject{currentmarker}{\pgfqpoint{0.000000in}{-0.048611in}}{\pgfqpoint{0.000000in}{0.000000in}}{%
\pgfpathmoveto{\pgfqpoint{0.000000in}{0.000000in}}%
\pgfpathlineto{\pgfqpoint{0.000000in}{-0.048611in}}%
\pgfusepath{stroke,fill}%
}%
\begin{pgfscope}%
\pgfsys@transformshift{1.420000in}{0.528000in}%
\pgfsys@useobject{currentmarker}{}%
\end{pgfscope}%
\end{pgfscope}%
\begin{pgfscope}%
\pgftext[x=1.420000in,y=0.430778in,,top]{\sffamily\fontsize{10.000000}{12.000000}\selectfont 1.5}%
\end{pgfscope}%
\begin{pgfscope}%
\pgfsetbuttcap%
\pgfsetroundjoin%
\definecolor{currentfill}{rgb}{0.000000,0.000000,0.000000}%
\pgfsetfillcolor{currentfill}%
\pgfsetlinewidth{0.803000pt}%
\definecolor{currentstroke}{rgb}{0.000000,0.000000,0.000000}%
\pgfsetstrokecolor{currentstroke}%
\pgfsetdash{}{0pt}%
\pgfsys@defobject{currentmarker}{\pgfqpoint{0.000000in}{-0.048611in}}{\pgfqpoint{0.000000in}{0.000000in}}{%
\pgfpathmoveto{\pgfqpoint{0.000000in}{0.000000in}}%
\pgfpathlineto{\pgfqpoint{0.000000in}{-0.048611in}}%
\pgfusepath{stroke,fill}%
}%
\begin{pgfscope}%
\pgfsys@transformshift{2.040000in}{0.528000in}%
\pgfsys@useobject{currentmarker}{}%
\end{pgfscope}%
\end{pgfscope}%
\begin{pgfscope}%
\pgftext[x=2.040000in,y=0.430778in,,top]{\sffamily\fontsize{10.000000}{12.000000}\selectfont 2.0}%
\end{pgfscope}%
\begin{pgfscope}%
\pgfsetbuttcap%
\pgfsetroundjoin%
\definecolor{currentfill}{rgb}{0.000000,0.000000,0.000000}%
\pgfsetfillcolor{currentfill}%
\pgfsetlinewidth{0.803000pt}%
\definecolor{currentstroke}{rgb}{0.000000,0.000000,0.000000}%
\pgfsetstrokecolor{currentstroke}%
\pgfsetdash{}{0pt}%
\pgfsys@defobject{currentmarker}{\pgfqpoint{0.000000in}{-0.048611in}}{\pgfqpoint{0.000000in}{0.000000in}}{%
\pgfpathmoveto{\pgfqpoint{0.000000in}{0.000000in}}%
\pgfpathlineto{\pgfqpoint{0.000000in}{-0.048611in}}%
\pgfusepath{stroke,fill}%
}%
\begin{pgfscope}%
\pgfsys@transformshift{2.660000in}{0.528000in}%
\pgfsys@useobject{currentmarker}{}%
\end{pgfscope}%
\end{pgfscope}%
\begin{pgfscope}%
\pgftext[x=2.660000in,y=0.430778in,,top]{\sffamily\fontsize{10.000000}{12.000000}\selectfont 2.5}%
\end{pgfscope}%
\begin{pgfscope}%
\pgfsetbuttcap%
\pgfsetroundjoin%
\definecolor{currentfill}{rgb}{0.000000,0.000000,0.000000}%
\pgfsetfillcolor{currentfill}%
\pgfsetlinewidth{0.803000pt}%
\definecolor{currentstroke}{rgb}{0.000000,0.000000,0.000000}%
\pgfsetstrokecolor{currentstroke}%
\pgfsetdash{}{0pt}%
\pgfsys@defobject{currentmarker}{\pgfqpoint{0.000000in}{-0.048611in}}{\pgfqpoint{0.000000in}{0.000000in}}{%
\pgfpathmoveto{\pgfqpoint{0.000000in}{0.000000in}}%
\pgfpathlineto{\pgfqpoint{0.000000in}{-0.048611in}}%
\pgfusepath{stroke,fill}%
}%
\begin{pgfscope}%
\pgfsys@transformshift{3.280000in}{0.528000in}%
\pgfsys@useobject{currentmarker}{}%
\end{pgfscope}%
\end{pgfscope}%
\begin{pgfscope}%
\pgftext[x=3.280000in,y=0.430778in,,top]{\sffamily\fontsize{10.000000}{12.000000}\selectfont 3.0}%
\end{pgfscope}%
\begin{pgfscope}%
\pgfsetbuttcap%
\pgfsetroundjoin%
\definecolor{currentfill}{rgb}{0.000000,0.000000,0.000000}%
\pgfsetfillcolor{currentfill}%
\pgfsetlinewidth{0.803000pt}%
\definecolor{currentstroke}{rgb}{0.000000,0.000000,0.000000}%
\pgfsetstrokecolor{currentstroke}%
\pgfsetdash{}{0pt}%
\pgfsys@defobject{currentmarker}{\pgfqpoint{0.000000in}{-0.048611in}}{\pgfqpoint{0.000000in}{0.000000in}}{%
\pgfpathmoveto{\pgfqpoint{0.000000in}{0.000000in}}%
\pgfpathlineto{\pgfqpoint{0.000000in}{-0.048611in}}%
\pgfusepath{stroke,fill}%
}%
\begin{pgfscope}%
\pgfsys@transformshift{3.900000in}{0.528000in}%
\pgfsys@useobject{currentmarker}{}%
\end{pgfscope}%
\end{pgfscope}%
\begin{pgfscope}%
\pgftext[x=3.900000in,y=0.430778in,,top]{\sffamily\fontsize{10.000000}{12.000000}\selectfont 3.5}%
\end{pgfscope}%
\begin{pgfscope}%
\pgfsetbuttcap%
\pgfsetroundjoin%
\definecolor{currentfill}{rgb}{0.000000,0.000000,0.000000}%
\pgfsetfillcolor{currentfill}%
\pgfsetlinewidth{0.803000pt}%
\definecolor{currentstroke}{rgb}{0.000000,0.000000,0.000000}%
\pgfsetstrokecolor{currentstroke}%
\pgfsetdash{}{0pt}%
\pgfsys@defobject{currentmarker}{\pgfqpoint{0.000000in}{-0.048611in}}{\pgfqpoint{0.000000in}{0.000000in}}{%
\pgfpathmoveto{\pgfqpoint{0.000000in}{0.000000in}}%
\pgfpathlineto{\pgfqpoint{0.000000in}{-0.048611in}}%
\pgfusepath{stroke,fill}%
}%
\begin{pgfscope}%
\pgfsys@transformshift{4.520000in}{0.528000in}%
\pgfsys@useobject{currentmarker}{}%
\end{pgfscope}%
\end{pgfscope}%
\begin{pgfscope}%
\pgftext[x=4.520000in,y=0.430778in,,top]{\sffamily\fontsize{10.000000}{12.000000}\selectfont 4.0}%
\end{pgfscope}%
\begin{pgfscope}%
\pgfsetbuttcap%
\pgfsetroundjoin%
\definecolor{currentfill}{rgb}{0.000000,0.000000,0.000000}%
\pgfsetfillcolor{currentfill}%
\pgfsetlinewidth{0.803000pt}%
\definecolor{currentstroke}{rgb}{0.000000,0.000000,0.000000}%
\pgfsetstrokecolor{currentstroke}%
\pgfsetdash{}{0pt}%
\pgfsys@defobject{currentmarker}{\pgfqpoint{0.000000in}{-0.048611in}}{\pgfqpoint{0.000000in}{0.000000in}}{%
\pgfpathmoveto{\pgfqpoint{0.000000in}{0.000000in}}%
\pgfpathlineto{\pgfqpoint{0.000000in}{-0.048611in}}%
\pgfusepath{stroke,fill}%
}%
\begin{pgfscope}%
\pgfsys@transformshift{5.140000in}{0.528000in}%
\pgfsys@useobject{currentmarker}{}%
\end{pgfscope}%
\end{pgfscope}%
\begin{pgfscope}%
\pgftext[x=5.140000in,y=0.430778in,,top]{\sffamily\fontsize{10.000000}{12.000000}\selectfont 4.5}%
\end{pgfscope}%
\begin{pgfscope}%
\pgfsetbuttcap%
\pgfsetroundjoin%
\definecolor{currentfill}{rgb}{0.000000,0.000000,0.000000}%
\pgfsetfillcolor{currentfill}%
\pgfsetlinewidth{0.803000pt}%
\definecolor{currentstroke}{rgb}{0.000000,0.000000,0.000000}%
\pgfsetstrokecolor{currentstroke}%
\pgfsetdash{}{0pt}%
\pgfsys@defobject{currentmarker}{\pgfqpoint{0.000000in}{-0.048611in}}{\pgfqpoint{0.000000in}{0.000000in}}{%
\pgfpathmoveto{\pgfqpoint{0.000000in}{0.000000in}}%
\pgfpathlineto{\pgfqpoint{0.000000in}{-0.048611in}}%
\pgfusepath{stroke,fill}%
}%
\begin{pgfscope}%
\pgfsys@transformshift{5.760000in}{0.528000in}%
\pgfsys@useobject{currentmarker}{}%
\end{pgfscope}%
\end{pgfscope}%
\begin{pgfscope}%
\pgftext[x=5.760000in,y=0.430778in,,top]{\sffamily\fontsize{10.000000}{12.000000}\selectfont 5.0}%
\end{pgfscope}%
\begin{pgfscope}%
\pgfsetbuttcap%
\pgfsetroundjoin%
\definecolor{currentfill}{rgb}{0.000000,0.000000,0.000000}%
\pgfsetfillcolor{currentfill}%
\pgfsetlinewidth{0.803000pt}%
\definecolor{currentstroke}{rgb}{0.000000,0.000000,0.000000}%
\pgfsetstrokecolor{currentstroke}%
\pgfsetdash{}{0pt}%
\pgfsys@defobject{currentmarker}{\pgfqpoint{-0.048611in}{0.000000in}}{\pgfqpoint{0.000000in}{0.000000in}}{%
\pgfpathmoveto{\pgfqpoint{0.000000in}{0.000000in}}%
\pgfpathlineto{\pgfqpoint{-0.048611in}{0.000000in}}%
\pgfusepath{stroke,fill}%
}%
\begin{pgfscope}%
\pgfsys@transformshift{0.800000in}{0.528000in}%
\pgfsys@useobject{currentmarker}{}%
\end{pgfscope}%
\end{pgfscope}%
\begin{pgfscope}%
\pgftext[x=0.614413in,y=0.475238in,left,base]{\sffamily\fontsize{10.000000}{12.000000}\selectfont 3}%
\end{pgfscope}%
\begin{pgfscope}%
\pgfsetbuttcap%
\pgfsetroundjoin%
\definecolor{currentfill}{rgb}{0.000000,0.000000,0.000000}%
\pgfsetfillcolor{currentfill}%
\pgfsetlinewidth{0.803000pt}%
\definecolor{currentstroke}{rgb}{0.000000,0.000000,0.000000}%
\pgfsetstrokecolor{currentstroke}%
\pgfsetdash{}{0pt}%
\pgfsys@defobject{currentmarker}{\pgfqpoint{-0.048611in}{0.000000in}}{\pgfqpoint{0.000000in}{0.000000in}}{%
\pgfpathmoveto{\pgfqpoint{0.000000in}{0.000000in}}%
\pgfpathlineto{\pgfqpoint{-0.048611in}{0.000000in}}%
\pgfusepath{stroke,fill}%
}%
\begin{pgfscope}%
\pgfsys@transformshift{0.800000in}{0.990000in}%
\pgfsys@useobject{currentmarker}{}%
\end{pgfscope}%
\end{pgfscope}%
\begin{pgfscope}%
\pgftext[x=0.614413in,y=0.937238in,left,base]{\sffamily\fontsize{10.000000}{12.000000}\selectfont 4}%
\end{pgfscope}%
\begin{pgfscope}%
\pgfsetbuttcap%
\pgfsetroundjoin%
\definecolor{currentfill}{rgb}{0.000000,0.000000,0.000000}%
\pgfsetfillcolor{currentfill}%
\pgfsetlinewidth{0.803000pt}%
\definecolor{currentstroke}{rgb}{0.000000,0.000000,0.000000}%
\pgfsetstrokecolor{currentstroke}%
\pgfsetdash{}{0pt}%
\pgfsys@defobject{currentmarker}{\pgfqpoint{-0.048611in}{0.000000in}}{\pgfqpoint{0.000000in}{0.000000in}}{%
\pgfpathmoveto{\pgfqpoint{0.000000in}{0.000000in}}%
\pgfpathlineto{\pgfqpoint{-0.048611in}{0.000000in}}%
\pgfusepath{stroke,fill}%
}%
\begin{pgfscope}%
\pgfsys@transformshift{0.800000in}{1.452000in}%
\pgfsys@useobject{currentmarker}{}%
\end{pgfscope}%
\end{pgfscope}%
\begin{pgfscope}%
\pgftext[x=0.614413in,y=1.399238in,left,base]{\sffamily\fontsize{10.000000}{12.000000}\selectfont 5}%
\end{pgfscope}%
\begin{pgfscope}%
\pgfsetbuttcap%
\pgfsetroundjoin%
\definecolor{currentfill}{rgb}{0.000000,0.000000,0.000000}%
\pgfsetfillcolor{currentfill}%
\pgfsetlinewidth{0.803000pt}%
\definecolor{currentstroke}{rgb}{0.000000,0.000000,0.000000}%
\pgfsetstrokecolor{currentstroke}%
\pgfsetdash{}{0pt}%
\pgfsys@defobject{currentmarker}{\pgfqpoint{-0.048611in}{0.000000in}}{\pgfqpoint{0.000000in}{0.000000in}}{%
\pgfpathmoveto{\pgfqpoint{0.000000in}{0.000000in}}%
\pgfpathlineto{\pgfqpoint{-0.048611in}{0.000000in}}%
\pgfusepath{stroke,fill}%
}%
\begin{pgfscope}%
\pgfsys@transformshift{0.800000in}{1.914000in}%
\pgfsys@useobject{currentmarker}{}%
\end{pgfscope}%
\end{pgfscope}%
\begin{pgfscope}%
\pgftext[x=0.614413in,y=1.861238in,left,base]{\sffamily\fontsize{10.000000}{12.000000}\selectfont 6}%
\end{pgfscope}%
\begin{pgfscope}%
\pgfsetbuttcap%
\pgfsetroundjoin%
\definecolor{currentfill}{rgb}{0.000000,0.000000,0.000000}%
\pgfsetfillcolor{currentfill}%
\pgfsetlinewidth{0.803000pt}%
\definecolor{currentstroke}{rgb}{0.000000,0.000000,0.000000}%
\pgfsetstrokecolor{currentstroke}%
\pgfsetdash{}{0pt}%
\pgfsys@defobject{currentmarker}{\pgfqpoint{-0.048611in}{0.000000in}}{\pgfqpoint{0.000000in}{0.000000in}}{%
\pgfpathmoveto{\pgfqpoint{0.000000in}{0.000000in}}%
\pgfpathlineto{\pgfqpoint{-0.048611in}{0.000000in}}%
\pgfusepath{stroke,fill}%
}%
\begin{pgfscope}%
\pgfsys@transformshift{0.800000in}{2.376000in}%
\pgfsys@useobject{currentmarker}{}%
\end{pgfscope}%
\end{pgfscope}%
\begin{pgfscope}%
\pgftext[x=0.614413in,y=2.323238in,left,base]{\sffamily\fontsize{10.000000}{12.000000}\selectfont 7}%
\end{pgfscope}%
\begin{pgfscope}%
\pgfsetbuttcap%
\pgfsetroundjoin%
\definecolor{currentfill}{rgb}{0.000000,0.000000,0.000000}%
\pgfsetfillcolor{currentfill}%
\pgfsetlinewidth{0.803000pt}%
\definecolor{currentstroke}{rgb}{0.000000,0.000000,0.000000}%
\pgfsetstrokecolor{currentstroke}%
\pgfsetdash{}{0pt}%
\pgfsys@defobject{currentmarker}{\pgfqpoint{-0.048611in}{0.000000in}}{\pgfqpoint{0.000000in}{0.000000in}}{%
\pgfpathmoveto{\pgfqpoint{0.000000in}{0.000000in}}%
\pgfpathlineto{\pgfqpoint{-0.048611in}{0.000000in}}%
\pgfusepath{stroke,fill}%
}%
\begin{pgfscope}%
\pgfsys@transformshift{0.800000in}{2.838000in}%
\pgfsys@useobject{currentmarker}{}%
\end{pgfscope}%
\end{pgfscope}%
\begin{pgfscope}%
\pgftext[x=0.614413in,y=2.785238in,left,base]{\sffamily\fontsize{10.000000}{12.000000}\selectfont 8}%
\end{pgfscope}%
\begin{pgfscope}%
\pgfsetbuttcap%
\pgfsetroundjoin%
\definecolor{currentfill}{rgb}{0.000000,0.000000,0.000000}%
\pgfsetfillcolor{currentfill}%
\pgfsetlinewidth{0.803000pt}%
\definecolor{currentstroke}{rgb}{0.000000,0.000000,0.000000}%
\pgfsetstrokecolor{currentstroke}%
\pgfsetdash{}{0pt}%
\pgfsys@defobject{currentmarker}{\pgfqpoint{-0.048611in}{0.000000in}}{\pgfqpoint{0.000000in}{0.000000in}}{%
\pgfpathmoveto{\pgfqpoint{0.000000in}{0.000000in}}%
\pgfpathlineto{\pgfqpoint{-0.048611in}{0.000000in}}%
\pgfusepath{stroke,fill}%
}%
\begin{pgfscope}%
\pgfsys@transformshift{0.800000in}{3.300000in}%
\pgfsys@useobject{currentmarker}{}%
\end{pgfscope}%
\end{pgfscope}%
\begin{pgfscope}%
\pgftext[x=0.614413in,y=3.247238in,left,base]{\sffamily\fontsize{10.000000}{12.000000}\selectfont 9}%
\end{pgfscope}%
\begin{pgfscope}%
\pgfsetbuttcap%
\pgfsetroundjoin%
\definecolor{currentfill}{rgb}{0.000000,0.000000,0.000000}%
\pgfsetfillcolor{currentfill}%
\pgfsetlinewidth{0.803000pt}%
\definecolor{currentstroke}{rgb}{0.000000,0.000000,0.000000}%
\pgfsetstrokecolor{currentstroke}%
\pgfsetdash{}{0pt}%
\pgfsys@defobject{currentmarker}{\pgfqpoint{-0.048611in}{0.000000in}}{\pgfqpoint{0.000000in}{0.000000in}}{%
\pgfpathmoveto{\pgfqpoint{0.000000in}{0.000000in}}%
\pgfpathlineto{\pgfqpoint{-0.048611in}{0.000000in}}%
\pgfusepath{stroke,fill}%
}%
\begin{pgfscope}%
\pgfsys@transformshift{0.800000in}{3.762000in}%
\pgfsys@useobject{currentmarker}{}%
\end{pgfscope}%
\end{pgfscope}%
\begin{pgfscope}%
\pgftext[x=0.526047in,y=3.709238in,left,base]{\sffamily\fontsize{10.000000}{12.000000}\selectfont 10}%
\end{pgfscope}%
\begin{pgfscope}%
\pgfsetbuttcap%
\pgfsetroundjoin%
\definecolor{currentfill}{rgb}{0.000000,0.000000,0.000000}%
\pgfsetfillcolor{currentfill}%
\pgfsetlinewidth{0.803000pt}%
\definecolor{currentstroke}{rgb}{0.000000,0.000000,0.000000}%
\pgfsetstrokecolor{currentstroke}%
\pgfsetdash{}{0pt}%
\pgfsys@defobject{currentmarker}{\pgfqpoint{-0.048611in}{0.000000in}}{\pgfqpoint{0.000000in}{0.000000in}}{%
\pgfpathmoveto{\pgfqpoint{0.000000in}{0.000000in}}%
\pgfpathlineto{\pgfqpoint{-0.048611in}{0.000000in}}%
\pgfusepath{stroke,fill}%
}%
\begin{pgfscope}%
\pgfsys@transformshift{0.800000in}{4.224000in}%
\pgfsys@useobject{currentmarker}{}%
\end{pgfscope}%
\end{pgfscope}%
\begin{pgfscope}%
\pgftext[x=0.526047in,y=4.171238in,left,base]{\sffamily\fontsize{10.000000}{12.000000}\selectfont 11}%
\end{pgfscope}%
\begin{pgfscope}%
\pgfpathrectangle{\pgfqpoint{0.800000in}{0.528000in}}{\pgfqpoint{4.960000in}{3.696000in}} %
\pgfusepath{clip}%
\pgfsetrectcap%
\pgfsetroundjoin%
\pgfsetlinewidth{2.007500pt}%
\definecolor{currentstroke}{rgb}{0.121569,0.466667,0.705882}%
\pgfsetstrokecolor{currentstroke}%
\pgfsetdash{}{0pt}%
\pgfpathmoveto{\pgfqpoint{0.800000in}{0.528000in}}%
\pgfpathlineto{\pgfqpoint{0.850101in}{0.565333in}}%
\pgfpathlineto{\pgfqpoint{0.900202in}{0.602667in}}%
\pgfpathlineto{\pgfqpoint{0.950303in}{0.640000in}}%
\pgfpathlineto{\pgfqpoint{1.000404in}{0.677333in}}%
\pgfpathlineto{\pgfqpoint{1.050505in}{0.714667in}}%
\pgfpathlineto{\pgfqpoint{1.100606in}{0.752000in}}%
\pgfpathlineto{\pgfqpoint{1.150707in}{0.789333in}}%
\pgfpathlineto{\pgfqpoint{1.200808in}{0.826667in}}%
\pgfpathlineto{\pgfqpoint{1.250909in}{0.864000in}}%
\pgfpathlineto{\pgfqpoint{1.301010in}{0.901333in}}%
\pgfpathlineto{\pgfqpoint{1.351111in}{0.938667in}}%
\pgfpathlineto{\pgfqpoint{1.401212in}{0.976000in}}%
\pgfpathlineto{\pgfqpoint{1.451313in}{1.013333in}}%
\pgfpathlineto{\pgfqpoint{1.501414in}{1.050667in}}%
\pgfpathlineto{\pgfqpoint{1.551515in}{1.088000in}}%
\pgfpathlineto{\pgfqpoint{1.601616in}{1.125333in}}%
\pgfpathlineto{\pgfqpoint{1.651717in}{1.162667in}}%
\pgfpathlineto{\pgfqpoint{1.701818in}{1.200000in}}%
\pgfpathlineto{\pgfqpoint{1.751919in}{1.237333in}}%
\pgfpathlineto{\pgfqpoint{1.802020in}{1.274667in}}%
\pgfpathlineto{\pgfqpoint{1.852121in}{1.312000in}}%
\pgfpathlineto{\pgfqpoint{1.902222in}{1.349333in}}%
\pgfpathlineto{\pgfqpoint{1.952323in}{1.386667in}}%
\pgfpathlineto{\pgfqpoint{2.002424in}{1.424000in}}%
\pgfpathlineto{\pgfqpoint{2.052525in}{1.461333in}}%
\pgfpathlineto{\pgfqpoint{2.102626in}{1.498667in}}%
\pgfpathlineto{\pgfqpoint{2.152727in}{1.536000in}}%
\pgfpathlineto{\pgfqpoint{2.202828in}{1.573333in}}%
\pgfpathlineto{\pgfqpoint{2.252929in}{1.610667in}}%
\pgfpathlineto{\pgfqpoint{2.303030in}{1.648000in}}%
\pgfpathlineto{\pgfqpoint{2.353131in}{1.685333in}}%
\pgfpathlineto{\pgfqpoint{2.403232in}{1.722667in}}%
\pgfpathlineto{\pgfqpoint{2.453333in}{1.760000in}}%
\pgfpathlineto{\pgfqpoint{2.503434in}{1.797333in}}%
\pgfpathlineto{\pgfqpoint{2.553535in}{1.834667in}}%
\pgfpathlineto{\pgfqpoint{2.603636in}{1.872000in}}%
\pgfpathlineto{\pgfqpoint{2.653737in}{1.909333in}}%
\pgfpathlineto{\pgfqpoint{2.703838in}{1.946667in}}%
\pgfpathlineto{\pgfqpoint{2.753939in}{1.984000in}}%
\pgfpathlineto{\pgfqpoint{2.804040in}{2.021333in}}%
\pgfpathlineto{\pgfqpoint{2.854141in}{2.058667in}}%
\pgfpathlineto{\pgfqpoint{2.904242in}{2.096000in}}%
\pgfpathlineto{\pgfqpoint{2.954343in}{2.133333in}}%
\pgfpathlineto{\pgfqpoint{3.004444in}{2.170667in}}%
\pgfpathlineto{\pgfqpoint{3.054545in}{2.208000in}}%
\pgfpathlineto{\pgfqpoint{3.104646in}{2.245333in}}%
\pgfpathlineto{\pgfqpoint{3.154747in}{2.282667in}}%
\pgfpathlineto{\pgfqpoint{3.204848in}{2.320000in}}%
\pgfpathlineto{\pgfqpoint{3.254949in}{2.357333in}}%
\pgfpathlineto{\pgfqpoint{3.305051in}{2.394667in}}%
\pgfpathlineto{\pgfqpoint{3.355152in}{2.432000in}}%
\pgfpathlineto{\pgfqpoint{3.405253in}{2.469333in}}%
\pgfpathlineto{\pgfqpoint{3.455354in}{2.506667in}}%
\pgfpathlineto{\pgfqpoint{3.505455in}{2.544000in}}%
\pgfpathlineto{\pgfqpoint{3.555556in}{2.581333in}}%
\pgfpathlineto{\pgfqpoint{3.605657in}{2.618667in}}%
\pgfpathlineto{\pgfqpoint{3.655758in}{2.656000in}}%
\pgfpathlineto{\pgfqpoint{3.705859in}{2.693333in}}%
\pgfpathlineto{\pgfqpoint{3.755960in}{2.730667in}}%
\pgfpathlineto{\pgfqpoint{3.806061in}{2.768000in}}%
\pgfpathlineto{\pgfqpoint{3.856162in}{2.805333in}}%
\pgfpathlineto{\pgfqpoint{3.906263in}{2.842667in}}%
\pgfpathlineto{\pgfqpoint{3.956364in}{2.880000in}}%
\pgfpathlineto{\pgfqpoint{4.006465in}{2.917333in}}%
\pgfpathlineto{\pgfqpoint{4.056566in}{2.954667in}}%
\pgfpathlineto{\pgfqpoint{4.106667in}{2.992000in}}%
\pgfpathlineto{\pgfqpoint{4.156768in}{3.029333in}}%
\pgfpathlineto{\pgfqpoint{4.206869in}{3.066667in}}%
\pgfpathlineto{\pgfqpoint{4.256970in}{3.104000in}}%
\pgfpathlineto{\pgfqpoint{4.307071in}{3.141333in}}%
\pgfpathlineto{\pgfqpoint{4.357172in}{3.178667in}}%
\pgfpathlineto{\pgfqpoint{4.407273in}{3.216000in}}%
\pgfpathlineto{\pgfqpoint{4.457374in}{3.253333in}}%
\pgfpathlineto{\pgfqpoint{4.507475in}{3.290667in}}%
\pgfpathlineto{\pgfqpoint{4.557576in}{3.328000in}}%
\pgfpathlineto{\pgfqpoint{4.607677in}{3.365333in}}%
\pgfpathlineto{\pgfqpoint{4.657778in}{3.402667in}}%
\pgfpathlineto{\pgfqpoint{4.707879in}{3.440000in}}%
\pgfpathlineto{\pgfqpoint{4.757980in}{3.477333in}}%
\pgfpathlineto{\pgfqpoint{4.808081in}{3.514667in}}%
\pgfpathlineto{\pgfqpoint{4.858182in}{3.552000in}}%
\pgfpathlineto{\pgfqpoint{4.908283in}{3.589333in}}%
\pgfpathlineto{\pgfqpoint{4.958384in}{3.626667in}}%
\pgfpathlineto{\pgfqpoint{5.008485in}{3.664000in}}%
\pgfpathlineto{\pgfqpoint{5.058586in}{3.701333in}}%
\pgfpathlineto{\pgfqpoint{5.108687in}{3.738667in}}%
\pgfpathlineto{\pgfqpoint{5.158788in}{3.776000in}}%
\pgfpathlineto{\pgfqpoint{5.208889in}{3.813333in}}%
\pgfpathlineto{\pgfqpoint{5.258990in}{3.850667in}}%
\pgfpathlineto{\pgfqpoint{5.309091in}{3.888000in}}%
\pgfpathlineto{\pgfqpoint{5.359192in}{3.925333in}}%
\pgfpathlineto{\pgfqpoint{5.409293in}{3.962667in}}%
\pgfpathlineto{\pgfqpoint{5.459394in}{4.000000in}}%
\pgfpathlineto{\pgfqpoint{5.509495in}{4.037333in}}%
\pgfpathlineto{\pgfqpoint{5.559596in}{4.074667in}}%
\pgfpathlineto{\pgfqpoint{5.609697in}{4.112000in}}%
\pgfpathlineto{\pgfqpoint{5.659798in}{4.149333in}}%
\pgfpathlineto{\pgfqpoint{5.709899in}{4.186667in}}%
\pgfpathlineto{\pgfqpoint{5.760000in}{4.224000in}}%
\pgfusepath{stroke}%
\end{pgfscope}%
\begin{pgfscope}%
\pgfsetrectcap%
\pgfsetmiterjoin%
\pgfsetlinewidth{0.803000pt}%
\definecolor{currentstroke}{rgb}{0.000000,0.000000,0.000000}%
\pgfsetstrokecolor{currentstroke}%
\pgfsetdash{}{0pt}%
\pgfpathmoveto{\pgfqpoint{0.800000in}{0.528000in}}%
\pgfpathlineto{\pgfqpoint{0.800000in}{4.224000in}}%
\pgfusepath{stroke}%
\end{pgfscope}%
\begin{pgfscope}%
\pgfsetrectcap%
\pgfsetmiterjoin%
\pgfsetlinewidth{0.803000pt}%
\definecolor{currentstroke}{rgb}{0.000000,0.000000,0.000000}%
\pgfsetstrokecolor{currentstroke}%
\pgfsetdash{}{0pt}%
\pgfpathmoveto{\pgfqpoint{5.760000in}{0.528000in}}%
\pgfpathlineto{\pgfqpoint{5.760000in}{4.224000in}}%
\pgfusepath{stroke}%
\end{pgfscope}%
\begin{pgfscope}%
\pgfsetrectcap%
\pgfsetmiterjoin%
\pgfsetlinewidth{0.803000pt}%
\definecolor{currentstroke}{rgb}{0.000000,0.000000,0.000000}%
\pgfsetstrokecolor{currentstroke}%
\pgfsetdash{}{0pt}%
\pgfpathmoveto{\pgfqpoint{0.800000in}{0.528000in}}%
\pgfpathlineto{\pgfqpoint{5.760000in}{0.528000in}}%
\pgfusepath{stroke}%
\end{pgfscope}%
\begin{pgfscope}%
\pgfsetrectcap%
\pgfsetmiterjoin%
\pgfsetlinewidth{0.803000pt}%
\definecolor{currentstroke}{rgb}{0.000000,0.000000,0.000000}%
\pgfsetstrokecolor{currentstroke}%
\pgfsetdash{}{0pt}%
\pgfpathmoveto{\pgfqpoint{0.800000in}{4.224000in}}%
\pgfpathlineto{\pgfqpoint{5.760000in}{4.224000in}}%
\pgfusepath{stroke}%
\end{pgfscope}%
\end{pgfpicture}%
\makeatother%
\endgroup%
}
	\caption{Binarna klasifikacija tačaka u skladu sa položajem u odnosu na pravu $2x+1$}
	\label{fig:bin_klas}
\end{figure}

\par
Regresija se odnosi na skup problema (i rešenja) u kojima je ciljna promenljiva neprekidna. Na primer, cene nekretnina mogu se predvideti na osnovu površine, lokacije, populacije koja živi u komšiluku, itd. Često korišćena vrsta regresije jeste linearna regresija. U slučaju linearne regresije, podrazumeva se da je funkcija $f_w(x)$ linearna u odnosu na parametar $w$. Iako se ovo na prvi pogled čini kao prilično jako ograničenje, to nije slučaj; kako za atribute ne postoji zahtev za linearnosti, oni pre pravljenja linearne kombinacije mogu biti proizvoljno transformisani. Primer linearne regresije jeste aproksimacija polinomom:
\begin{center}
	$f_w(x) = w_0 + \sum_{i=1}^{N}w_ix^i$
\end{center}

\subsection{Nenadgledano mašinsko učenje}

Nenadgledano učenje obuhvata skup problema (i njihovih rešenja) u kojima sistem prihvata ulazne podatke bez izlaznih. Ovo znači da sistem sam mora da zaključi kakve zakonitosti važe u podacima.  Kako nije moguće odrediti preciznost sistema pa je cilj naći najbolji model u odnosu na neki kriterijum koji je unapred zadat.
Jedan primer nenadgledanog mašinskog učenja je klasterovanje: sistem grupiše neoznačene podatke u odnosu na  neki kriterijum koji nije unapred poznat. Svaka grupa (klaster) sastoji se iz podataka koji su međusobno slični i različiti od elemenata preostalih grupa u odnosu na taj kriterijum. Jednostavan primer klasterovanja po numeričkim atributima x i y može se videti na slici \ref{fig:klaster}.
	
\begin{figure}
	\centering
	\resizebox{.8\linewidth}{!}{%% Creator: Matplotlib, PGF backend
%%
%% To include the figure in your LaTeX document, write
%%   \input{<filename>.pgf}
%%
%% Make sure the required packages are loaded in your preamble
%%   \usepackage{pgf}
%%
%% Figures using additional raster images can only be included by \input if
%% they are in the same directory as the main LaTeX file. For loading figures
%% from other directories you can use the `import` package
%%   \usepackage{import}
%% and then include the figures with
%%   \import{<path to file>}{<filename>.pgf}
%%
%% Matplotlib used the following preamble
%%   \usepackage{fontspec}
%%   \setmainfont{DejaVu Serif}
%%   \setsansfont{DejaVu Sans}
%%   \setmonofont{DejaVu Sans Mono}
%%
\begingroup%
\makeatletter%
\begin{pgfpicture}%
\pgfpathrectangle{\pgfpointorigin}{\pgfqpoint{6.400000in}{4.800000in}}%
\pgfusepath{use as bounding box, clip}%
\begin{pgfscope}%
\pgfsetbuttcap%
\pgfsetmiterjoin%
\definecolor{currentfill}{rgb}{1.000000,1.000000,1.000000}%
\pgfsetfillcolor{currentfill}%
\pgfsetlinewidth{0.000000pt}%
\definecolor{currentstroke}{rgb}{1.000000,1.000000,1.000000}%
\pgfsetstrokecolor{currentstroke}%
\pgfsetdash{}{0pt}%
\pgfpathmoveto{\pgfqpoint{0.000000in}{0.000000in}}%
\pgfpathlineto{\pgfqpoint{6.400000in}{0.000000in}}%
\pgfpathlineto{\pgfqpoint{6.400000in}{4.800000in}}%
\pgfpathlineto{\pgfqpoint{0.000000in}{4.800000in}}%
\pgfpathclose%
\pgfusepath{fill}%
\end{pgfscope}%
\begin{pgfscope}%
\pgfsetbuttcap%
\pgfsetmiterjoin%
\definecolor{currentfill}{rgb}{1.000000,1.000000,1.000000}%
\pgfsetfillcolor{currentfill}%
\pgfsetlinewidth{0.000000pt}%
\definecolor{currentstroke}{rgb}{0.000000,0.000000,0.000000}%
\pgfsetstrokecolor{currentstroke}%
\pgfsetstrokeopacity{0.000000}%
\pgfsetdash{}{0pt}%
\pgfpathmoveto{\pgfqpoint{1.115200in}{0.528000in}}%
\pgfpathlineto{\pgfqpoint{5.444800in}{0.528000in}}%
\pgfpathlineto{\pgfqpoint{5.444800in}{4.224000in}}%
\pgfpathlineto{\pgfqpoint{1.115200in}{4.224000in}}%
\pgfpathclose%
\pgfusepath{fill}%
\end{pgfscope}%
\begin{pgfscope}%
\pgfpathrectangle{\pgfqpoint{1.115200in}{0.528000in}}{\pgfqpoint{4.329600in}{3.696000in}} %
\pgfusepath{clip}%
\pgfsetbuttcap%
\pgfsetroundjoin%
\definecolor{currentfill}{rgb}{0.000000,0.501961,0.000000}%
\pgfsetfillcolor{currentfill}%
\pgfsetlinewidth{1.003750pt}%
\definecolor{currentstroke}{rgb}{0.000000,0.501961,0.000000}%
\pgfsetstrokecolor{currentstroke}%
\pgfsetdash{}{0pt}%
\pgfpathmoveto{\pgfqpoint{3.794951in}{2.912157in}}%
\pgfpathcurveto{\pgfqpoint{3.806001in}{2.912157in}}{\pgfqpoint{3.816600in}{2.916547in}}{\pgfqpoint{3.824414in}{2.924361in}}%
\pgfpathcurveto{\pgfqpoint{3.832228in}{2.932174in}}{\pgfqpoint{3.836618in}{2.942773in}}{\pgfqpoint{3.836618in}{2.953824in}}%
\pgfpathcurveto{\pgfqpoint{3.836618in}{2.964874in}}{\pgfqpoint{3.832228in}{2.975473in}}{\pgfqpoint{3.824414in}{2.983286in}}%
\pgfpathcurveto{\pgfqpoint{3.816600in}{2.991100in}}{\pgfqpoint{3.806001in}{2.995490in}}{\pgfqpoint{3.794951in}{2.995490in}}%
\pgfpathcurveto{\pgfqpoint{3.783901in}{2.995490in}}{\pgfqpoint{3.773302in}{2.991100in}}{\pgfqpoint{3.765489in}{2.983286in}}%
\pgfpathcurveto{\pgfqpoint{3.757675in}{2.975473in}}{\pgfqpoint{3.753285in}{2.964874in}}{\pgfqpoint{3.753285in}{2.953824in}}%
\pgfpathcurveto{\pgfqpoint{3.753285in}{2.942773in}}{\pgfqpoint{3.757675in}{2.932174in}}{\pgfqpoint{3.765489in}{2.924361in}}%
\pgfpathcurveto{\pgfqpoint{3.773302in}{2.916547in}}{\pgfqpoint{3.783901in}{2.912157in}}{\pgfqpoint{3.794951in}{2.912157in}}%
\pgfpathclose%
\pgfusepath{stroke,fill}%
\end{pgfscope}%
\begin{pgfscope}%
\pgfpathrectangle{\pgfqpoint{1.115200in}{0.528000in}}{\pgfqpoint{4.329600in}{3.696000in}} %
\pgfusepath{clip}%
\pgfsetbuttcap%
\pgfsetroundjoin%
\definecolor{currentfill}{rgb}{0.000000,0.501961,0.000000}%
\pgfsetfillcolor{currentfill}%
\pgfsetlinewidth{1.003750pt}%
\definecolor{currentstroke}{rgb}{0.000000,0.501961,0.000000}%
\pgfsetstrokecolor{currentstroke}%
\pgfsetdash{}{0pt}%
\pgfpathmoveto{\pgfqpoint{4.014766in}{3.578152in}}%
\pgfpathcurveto{\pgfqpoint{4.025816in}{3.578152in}}{\pgfqpoint{4.036415in}{3.582542in}}{\pgfqpoint{4.044228in}{3.590355in}}%
\pgfpathcurveto{\pgfqpoint{4.052042in}{3.598169in}}{\pgfqpoint{4.056432in}{3.608768in}}{\pgfqpoint{4.056432in}{3.619818in}}%
\pgfpathcurveto{\pgfqpoint{4.056432in}{3.630868in}}{\pgfqpoint{4.052042in}{3.641467in}}{\pgfqpoint{4.044228in}{3.649281in}}%
\pgfpathcurveto{\pgfqpoint{4.036415in}{3.657095in}}{\pgfqpoint{4.025816in}{3.661485in}}{\pgfqpoint{4.014766in}{3.661485in}}%
\pgfpathcurveto{\pgfqpoint{4.003715in}{3.661485in}}{\pgfqpoint{3.993116in}{3.657095in}}{\pgfqpoint{3.985303in}{3.649281in}}%
\pgfpathcurveto{\pgfqpoint{3.977489in}{3.641467in}}{\pgfqpoint{3.973099in}{3.630868in}}{\pgfqpoint{3.973099in}{3.619818in}}%
\pgfpathcurveto{\pgfqpoint{3.973099in}{3.608768in}}{\pgfqpoint{3.977489in}{3.598169in}}{\pgfqpoint{3.985303in}{3.590355in}}%
\pgfpathcurveto{\pgfqpoint{3.993116in}{3.582542in}}{\pgfqpoint{4.003715in}{3.578152in}}{\pgfqpoint{4.014766in}{3.578152in}}%
\pgfpathclose%
\pgfusepath{stroke,fill}%
\end{pgfscope}%
\begin{pgfscope}%
\pgfpathrectangle{\pgfqpoint{1.115200in}{0.528000in}}{\pgfqpoint{4.329600in}{3.696000in}} %
\pgfusepath{clip}%
\pgfsetbuttcap%
\pgfsetroundjoin%
\definecolor{currentfill}{rgb}{0.000000,0.501961,0.000000}%
\pgfsetfillcolor{currentfill}%
\pgfsetlinewidth{1.003750pt}%
\definecolor{currentstroke}{rgb}{0.000000,0.501961,0.000000}%
\pgfsetstrokecolor{currentstroke}%
\pgfsetdash{}{0pt}%
\pgfpathmoveto{\pgfqpoint{3.906804in}{3.141140in}}%
\pgfpathcurveto{\pgfqpoint{3.917854in}{3.141140in}}{\pgfqpoint{3.928453in}{3.145530in}}{\pgfqpoint{3.936267in}{3.153344in}}%
\pgfpathcurveto{\pgfqpoint{3.944080in}{3.161157in}}{\pgfqpoint{3.948470in}{3.171756in}}{\pgfqpoint{3.948470in}{3.182806in}}%
\pgfpathcurveto{\pgfqpoint{3.948470in}{3.193857in}}{\pgfqpoint{3.944080in}{3.204456in}}{\pgfqpoint{3.936267in}{3.212269in}}%
\pgfpathcurveto{\pgfqpoint{3.928453in}{3.220083in}}{\pgfqpoint{3.917854in}{3.224473in}}{\pgfqpoint{3.906804in}{3.224473in}}%
\pgfpathcurveto{\pgfqpoint{3.895754in}{3.224473in}}{\pgfqpoint{3.885155in}{3.220083in}}{\pgfqpoint{3.877341in}{3.212269in}}%
\pgfpathcurveto{\pgfqpoint{3.869527in}{3.204456in}}{\pgfqpoint{3.865137in}{3.193857in}}{\pgfqpoint{3.865137in}{3.182806in}}%
\pgfpathcurveto{\pgfqpoint{3.865137in}{3.171756in}}{\pgfqpoint{3.869527in}{3.161157in}}{\pgfqpoint{3.877341in}{3.153344in}}%
\pgfpathcurveto{\pgfqpoint{3.885155in}{3.145530in}}{\pgfqpoint{3.895754in}{3.141140in}}{\pgfqpoint{3.906804in}{3.141140in}}%
\pgfpathclose%
\pgfusepath{stroke,fill}%
\end{pgfscope}%
\begin{pgfscope}%
\pgfpathrectangle{\pgfqpoint{1.115200in}{0.528000in}}{\pgfqpoint{4.329600in}{3.696000in}} %
\pgfusepath{clip}%
\pgfsetbuttcap%
\pgfsetroundjoin%
\definecolor{currentfill}{rgb}{0.000000,0.501961,0.000000}%
\pgfsetfillcolor{currentfill}%
\pgfsetlinewidth{1.003750pt}%
\definecolor{currentstroke}{rgb}{0.000000,0.501961,0.000000}%
\pgfsetstrokecolor{currentstroke}%
\pgfsetdash{}{0pt}%
\pgfpathmoveto{\pgfqpoint{3.670912in}{3.787660in}}%
\pgfpathcurveto{\pgfqpoint{3.681962in}{3.787660in}}{\pgfqpoint{3.692561in}{3.792050in}}{\pgfqpoint{3.700375in}{3.799864in}}%
\pgfpathcurveto{\pgfqpoint{3.708188in}{3.807677in}}{\pgfqpoint{3.712578in}{3.818276in}}{\pgfqpoint{3.712578in}{3.829326in}}%
\pgfpathcurveto{\pgfqpoint{3.712578in}{3.840377in}}{\pgfqpoint{3.708188in}{3.850976in}}{\pgfqpoint{3.700375in}{3.858789in}}%
\pgfpathcurveto{\pgfqpoint{3.692561in}{3.866603in}}{\pgfqpoint{3.681962in}{3.870993in}}{\pgfqpoint{3.670912in}{3.870993in}}%
\pgfpathcurveto{\pgfqpoint{3.659862in}{3.870993in}}{\pgfqpoint{3.649263in}{3.866603in}}{\pgfqpoint{3.641449in}{3.858789in}}%
\pgfpathcurveto{\pgfqpoint{3.633635in}{3.850976in}}{\pgfqpoint{3.629245in}{3.840377in}}{\pgfqpoint{3.629245in}{3.829326in}}%
\pgfpathcurveto{\pgfqpoint{3.629245in}{3.818276in}}{\pgfqpoint{3.633635in}{3.807677in}}{\pgfqpoint{3.641449in}{3.799864in}}%
\pgfpathcurveto{\pgfqpoint{3.649263in}{3.792050in}}{\pgfqpoint{3.659862in}{3.787660in}}{\pgfqpoint{3.670912in}{3.787660in}}%
\pgfpathclose%
\pgfusepath{stroke,fill}%
\end{pgfscope}%
\begin{pgfscope}%
\pgfpathrectangle{\pgfqpoint{1.115200in}{0.528000in}}{\pgfqpoint{4.329600in}{3.696000in}} %
\pgfusepath{clip}%
\pgfsetbuttcap%
\pgfsetroundjoin%
\definecolor{currentfill}{rgb}{0.000000,0.501961,0.000000}%
\pgfsetfillcolor{currentfill}%
\pgfsetlinewidth{1.003750pt}%
\definecolor{currentstroke}{rgb}{0.000000,0.501961,0.000000}%
\pgfsetstrokecolor{currentstroke}%
\pgfsetdash{}{0pt}%
\pgfpathmoveto{\pgfqpoint{3.335592in}{3.359038in}}%
\pgfpathcurveto{\pgfqpoint{3.346642in}{3.359038in}}{\pgfqpoint{3.357241in}{3.363428in}}{\pgfqpoint{3.365055in}{3.371242in}}%
\pgfpathcurveto{\pgfqpoint{3.372868in}{3.379056in}}{\pgfqpoint{3.377259in}{3.389655in}}{\pgfqpoint{3.377259in}{3.400705in}}%
\pgfpathcurveto{\pgfqpoint{3.377259in}{3.411755in}}{\pgfqpoint{3.372868in}{3.422354in}}{\pgfqpoint{3.365055in}{3.430168in}}%
\pgfpathcurveto{\pgfqpoint{3.357241in}{3.437981in}}{\pgfqpoint{3.346642in}{3.442372in}}{\pgfqpoint{3.335592in}{3.442372in}}%
\pgfpathcurveto{\pgfqpoint{3.324542in}{3.442372in}}{\pgfqpoint{3.313943in}{3.437981in}}{\pgfqpoint{3.306129in}{3.430168in}}%
\pgfpathcurveto{\pgfqpoint{3.298316in}{3.422354in}}{\pgfqpoint{3.293925in}{3.411755in}}{\pgfqpoint{3.293925in}{3.400705in}}%
\pgfpathcurveto{\pgfqpoint{3.293925in}{3.389655in}}{\pgfqpoint{3.298316in}{3.379056in}}{\pgfqpoint{3.306129in}{3.371242in}}%
\pgfpathcurveto{\pgfqpoint{3.313943in}{3.363428in}}{\pgfqpoint{3.324542in}{3.359038in}}{\pgfqpoint{3.335592in}{3.359038in}}%
\pgfpathclose%
\pgfusepath{stroke,fill}%
\end{pgfscope}%
\begin{pgfscope}%
\pgfpathrectangle{\pgfqpoint{1.115200in}{0.528000in}}{\pgfqpoint{4.329600in}{3.696000in}} %
\pgfusepath{clip}%
\pgfsetbuttcap%
\pgfsetroundjoin%
\definecolor{currentfill}{rgb}{0.000000,0.501961,0.000000}%
\pgfsetfillcolor{currentfill}%
\pgfsetlinewidth{1.003750pt}%
\definecolor{currentstroke}{rgb}{0.000000,0.501961,0.000000}%
\pgfsetstrokecolor{currentstroke}%
\pgfsetdash{}{0pt}%
\pgfpathmoveto{\pgfqpoint{4.053679in}{3.160995in}}%
\pgfpathcurveto{\pgfqpoint{4.064729in}{3.160995in}}{\pgfqpoint{4.075328in}{3.165385in}}{\pgfqpoint{4.083142in}{3.173199in}}%
\pgfpathcurveto{\pgfqpoint{4.090955in}{3.181013in}}{\pgfqpoint{4.095345in}{3.191612in}}{\pgfqpoint{4.095345in}{3.202662in}}%
\pgfpathcurveto{\pgfqpoint{4.095345in}{3.213712in}}{\pgfqpoint{4.090955in}{3.224311in}}{\pgfqpoint{4.083142in}{3.232125in}}%
\pgfpathcurveto{\pgfqpoint{4.075328in}{3.239938in}}{\pgfqpoint{4.064729in}{3.244328in}}{\pgfqpoint{4.053679in}{3.244328in}}%
\pgfpathcurveto{\pgfqpoint{4.042629in}{3.244328in}}{\pgfqpoint{4.032030in}{3.239938in}}{\pgfqpoint{4.024216in}{3.232125in}}%
\pgfpathcurveto{\pgfqpoint{4.016402in}{3.224311in}}{\pgfqpoint{4.012012in}{3.213712in}}{\pgfqpoint{4.012012in}{3.202662in}}%
\pgfpathcurveto{\pgfqpoint{4.012012in}{3.191612in}}{\pgfqpoint{4.016402in}{3.181013in}}{\pgfqpoint{4.024216in}{3.173199in}}%
\pgfpathcurveto{\pgfqpoint{4.032030in}{3.165385in}}{\pgfqpoint{4.042629in}{3.160995in}}{\pgfqpoint{4.053679in}{3.160995in}}%
\pgfpathclose%
\pgfusepath{stroke,fill}%
\end{pgfscope}%
\begin{pgfscope}%
\pgfpathrectangle{\pgfqpoint{1.115200in}{0.528000in}}{\pgfqpoint{4.329600in}{3.696000in}} %
\pgfusepath{clip}%
\pgfsetbuttcap%
\pgfsetroundjoin%
\definecolor{currentfill}{rgb}{0.000000,0.501961,0.000000}%
\pgfsetfillcolor{currentfill}%
\pgfsetlinewidth{1.003750pt}%
\definecolor{currentstroke}{rgb}{0.000000,0.501961,0.000000}%
\pgfsetstrokecolor{currentstroke}%
\pgfsetdash{}{0pt}%
\pgfpathmoveto{\pgfqpoint{4.098836in}{3.493344in}}%
\pgfpathcurveto{\pgfqpoint{4.109886in}{3.493344in}}{\pgfqpoint{4.120485in}{3.497734in}}{\pgfqpoint{4.128299in}{3.505547in}}%
\pgfpathcurveto{\pgfqpoint{4.136112in}{3.513361in}}{\pgfqpoint{4.140503in}{3.523960in}}{\pgfqpoint{4.140503in}{3.535010in}}%
\pgfpathcurveto{\pgfqpoint{4.140503in}{3.546060in}}{\pgfqpoint{4.136112in}{3.556659in}}{\pgfqpoint{4.128299in}{3.564473in}}%
\pgfpathcurveto{\pgfqpoint{4.120485in}{3.572287in}}{\pgfqpoint{4.109886in}{3.576677in}}{\pgfqpoint{4.098836in}{3.576677in}}%
\pgfpathcurveto{\pgfqpoint{4.087786in}{3.576677in}}{\pgfqpoint{4.077187in}{3.572287in}}{\pgfqpoint{4.069373in}{3.564473in}}%
\pgfpathcurveto{\pgfqpoint{4.061559in}{3.556659in}}{\pgfqpoint{4.057169in}{3.546060in}}{\pgfqpoint{4.057169in}{3.535010in}}%
\pgfpathcurveto{\pgfqpoint{4.057169in}{3.523960in}}{\pgfqpoint{4.061559in}{3.513361in}}{\pgfqpoint{4.069373in}{3.505547in}}%
\pgfpathcurveto{\pgfqpoint{4.077187in}{3.497734in}}{\pgfqpoint{4.087786in}{3.493344in}}{\pgfqpoint{4.098836in}{3.493344in}}%
\pgfpathclose%
\pgfusepath{stroke,fill}%
\end{pgfscope}%
\begin{pgfscope}%
\pgfpathrectangle{\pgfqpoint{1.115200in}{0.528000in}}{\pgfqpoint{4.329600in}{3.696000in}} %
\pgfusepath{clip}%
\pgfsetbuttcap%
\pgfsetroundjoin%
\definecolor{currentfill}{rgb}{1.000000,0.000000,0.000000}%
\pgfsetfillcolor{currentfill}%
\pgfsetlinewidth{1.003750pt}%
\definecolor{currentstroke}{rgb}{1.000000,0.000000,0.000000}%
\pgfsetstrokecolor{currentstroke}%
\pgfsetdash{}{0pt}%
\pgfpathmoveto{\pgfqpoint{4.635205in}{1.486440in}}%
\pgfpathcurveto{\pgfqpoint{4.646256in}{1.486440in}}{\pgfqpoint{4.656855in}{1.490830in}}{\pgfqpoint{4.664668in}{1.498644in}}%
\pgfpathcurveto{\pgfqpoint{4.672482in}{1.506457in}}{\pgfqpoint{4.676872in}{1.517056in}}{\pgfqpoint{4.676872in}{1.528107in}}%
\pgfpathcurveto{\pgfqpoint{4.676872in}{1.539157in}}{\pgfqpoint{4.672482in}{1.549756in}}{\pgfqpoint{4.664668in}{1.557569in}}%
\pgfpathcurveto{\pgfqpoint{4.656855in}{1.565383in}}{\pgfqpoint{4.646256in}{1.569773in}}{\pgfqpoint{4.635205in}{1.569773in}}%
\pgfpathcurveto{\pgfqpoint{4.624155in}{1.569773in}}{\pgfqpoint{4.613556in}{1.565383in}}{\pgfqpoint{4.605743in}{1.557569in}}%
\pgfpathcurveto{\pgfqpoint{4.597929in}{1.549756in}}{\pgfqpoint{4.593539in}{1.539157in}}{\pgfqpoint{4.593539in}{1.528107in}}%
\pgfpathcurveto{\pgfqpoint{4.593539in}{1.517056in}}{\pgfqpoint{4.597929in}{1.506457in}}{\pgfqpoint{4.605743in}{1.498644in}}%
\pgfpathcurveto{\pgfqpoint{4.613556in}{1.490830in}}{\pgfqpoint{4.624155in}{1.486440in}}{\pgfqpoint{4.635205in}{1.486440in}}%
\pgfpathclose%
\pgfusepath{stroke,fill}%
\end{pgfscope}%
\begin{pgfscope}%
\pgfpathrectangle{\pgfqpoint{1.115200in}{0.528000in}}{\pgfqpoint{4.329600in}{3.696000in}} %
\pgfusepath{clip}%
\pgfsetbuttcap%
\pgfsetroundjoin%
\definecolor{currentfill}{rgb}{1.000000,0.000000,0.000000}%
\pgfsetfillcolor{currentfill}%
\pgfsetlinewidth{1.003750pt}%
\definecolor{currentstroke}{rgb}{1.000000,0.000000,0.000000}%
\pgfsetstrokecolor{currentstroke}%
\pgfsetdash{}{0pt}%
\pgfpathmoveto{\pgfqpoint{4.640915in}{2.354764in}}%
\pgfpathcurveto{\pgfqpoint{4.651965in}{2.354764in}}{\pgfqpoint{4.662564in}{2.359154in}}{\pgfqpoint{4.670378in}{2.366968in}}%
\pgfpathcurveto{\pgfqpoint{4.678192in}{2.374781in}}{\pgfqpoint{4.682582in}{2.385380in}}{\pgfqpoint{4.682582in}{2.396430in}}%
\pgfpathcurveto{\pgfqpoint{4.682582in}{2.407480in}}{\pgfqpoint{4.678192in}{2.418079in}}{\pgfqpoint{4.670378in}{2.425893in}}%
\pgfpathcurveto{\pgfqpoint{4.662564in}{2.433707in}}{\pgfqpoint{4.651965in}{2.438097in}}{\pgfqpoint{4.640915in}{2.438097in}}%
\pgfpathcurveto{\pgfqpoint{4.629865in}{2.438097in}}{\pgfqpoint{4.619266in}{2.433707in}}{\pgfqpoint{4.611452in}{2.425893in}}%
\pgfpathcurveto{\pgfqpoint{4.603639in}{2.418079in}}{\pgfqpoint{4.599248in}{2.407480in}}{\pgfqpoint{4.599248in}{2.396430in}}%
\pgfpathcurveto{\pgfqpoint{4.599248in}{2.385380in}}{\pgfqpoint{4.603639in}{2.374781in}}{\pgfqpoint{4.611452in}{2.366968in}}%
\pgfpathcurveto{\pgfqpoint{4.619266in}{2.359154in}}{\pgfqpoint{4.629865in}{2.354764in}}{\pgfqpoint{4.640915in}{2.354764in}}%
\pgfpathclose%
\pgfusepath{stroke,fill}%
\end{pgfscope}%
\begin{pgfscope}%
\pgfpathrectangle{\pgfqpoint{1.115200in}{0.528000in}}{\pgfqpoint{4.329600in}{3.696000in}} %
\pgfusepath{clip}%
\pgfsetbuttcap%
\pgfsetroundjoin%
\definecolor{currentfill}{rgb}{1.000000,0.000000,0.000000}%
\pgfsetfillcolor{currentfill}%
\pgfsetlinewidth{1.003750pt}%
\definecolor{currentstroke}{rgb}{1.000000,0.000000,0.000000}%
\pgfsetstrokecolor{currentstroke}%
\pgfsetdash{}{0pt}%
\pgfpathmoveto{\pgfqpoint{4.624117in}{2.084918in}}%
\pgfpathcurveto{\pgfqpoint{4.635167in}{2.084918in}}{\pgfqpoint{4.645766in}{2.089309in}}{\pgfqpoint{4.653580in}{2.097122in}}%
\pgfpathcurveto{\pgfqpoint{4.661394in}{2.104936in}}{\pgfqpoint{4.665784in}{2.115535in}}{\pgfqpoint{4.665784in}{2.126585in}}%
\pgfpathcurveto{\pgfqpoint{4.665784in}{2.137635in}}{\pgfqpoint{4.661394in}{2.148234in}}{\pgfqpoint{4.653580in}{2.156048in}}%
\pgfpathcurveto{\pgfqpoint{4.645766in}{2.163861in}}{\pgfqpoint{4.635167in}{2.168252in}}{\pgfqpoint{4.624117in}{2.168252in}}%
\pgfpathcurveto{\pgfqpoint{4.613067in}{2.168252in}}{\pgfqpoint{4.602468in}{2.163861in}}{\pgfqpoint{4.594654in}{2.156048in}}%
\pgfpathcurveto{\pgfqpoint{4.586841in}{2.148234in}}{\pgfqpoint{4.582451in}{2.137635in}}{\pgfqpoint{4.582451in}{2.126585in}}%
\pgfpathcurveto{\pgfqpoint{4.582451in}{2.115535in}}{\pgfqpoint{4.586841in}{2.104936in}}{\pgfqpoint{4.594654in}{2.097122in}}%
\pgfpathcurveto{\pgfqpoint{4.602468in}{2.089309in}}{\pgfqpoint{4.613067in}{2.084918in}}{\pgfqpoint{4.624117in}{2.084918in}}%
\pgfpathclose%
\pgfusepath{stroke,fill}%
\end{pgfscope}%
\begin{pgfscope}%
\pgfpathrectangle{\pgfqpoint{1.115200in}{0.528000in}}{\pgfqpoint{4.329600in}{3.696000in}} %
\pgfusepath{clip}%
\pgfsetbuttcap%
\pgfsetroundjoin%
\definecolor{currentfill}{rgb}{1.000000,0.000000,0.000000}%
\pgfsetfillcolor{currentfill}%
\pgfsetlinewidth{1.003750pt}%
\definecolor{currentstroke}{rgb}{1.000000,0.000000,0.000000}%
\pgfsetstrokecolor{currentstroke}%
\pgfsetdash{}{0pt}%
\pgfpathmoveto{\pgfqpoint{4.550855in}{1.801790in}}%
\pgfpathcurveto{\pgfqpoint{4.561906in}{1.801790in}}{\pgfqpoint{4.572505in}{1.806181in}}{\pgfqpoint{4.580318in}{1.813994in}}%
\pgfpathcurveto{\pgfqpoint{4.588132in}{1.821808in}}{\pgfqpoint{4.592522in}{1.832407in}}{\pgfqpoint{4.592522in}{1.843457in}}%
\pgfpathcurveto{\pgfqpoint{4.592522in}{1.854507in}}{\pgfqpoint{4.588132in}{1.865106in}}{\pgfqpoint{4.580318in}{1.872920in}}%
\pgfpathcurveto{\pgfqpoint{4.572505in}{1.880734in}}{\pgfqpoint{4.561906in}{1.885124in}}{\pgfqpoint{4.550855in}{1.885124in}}%
\pgfpathcurveto{\pgfqpoint{4.539805in}{1.885124in}}{\pgfqpoint{4.529206in}{1.880734in}}{\pgfqpoint{4.521393in}{1.872920in}}%
\pgfpathcurveto{\pgfqpoint{4.513579in}{1.865106in}}{\pgfqpoint{4.509189in}{1.854507in}}{\pgfqpoint{4.509189in}{1.843457in}}%
\pgfpathcurveto{\pgfqpoint{4.509189in}{1.832407in}}{\pgfqpoint{4.513579in}{1.821808in}}{\pgfqpoint{4.521393in}{1.813994in}}%
\pgfpathcurveto{\pgfqpoint{4.529206in}{1.806181in}}{\pgfqpoint{4.539805in}{1.801790in}}{\pgfqpoint{4.550855in}{1.801790in}}%
\pgfpathclose%
\pgfusepath{stroke,fill}%
\end{pgfscope}%
\begin{pgfscope}%
\pgfpathrectangle{\pgfqpoint{1.115200in}{0.528000in}}{\pgfqpoint{4.329600in}{3.696000in}} %
\pgfusepath{clip}%
\pgfsetbuttcap%
\pgfsetroundjoin%
\definecolor{currentfill}{rgb}{1.000000,0.000000,0.000000}%
\pgfsetfillcolor{currentfill}%
\pgfsetlinewidth{1.003750pt}%
\definecolor{currentstroke}{rgb}{1.000000,0.000000,0.000000}%
\pgfsetstrokecolor{currentstroke}%
\pgfsetdash{}{0pt}%
\pgfpathmoveto{\pgfqpoint{4.800923in}{2.028368in}}%
\pgfpathcurveto{\pgfqpoint{4.811973in}{2.028368in}}{\pgfqpoint{4.822572in}{2.032758in}}{\pgfqpoint{4.830385in}{2.040572in}}%
\pgfpathcurveto{\pgfqpoint{4.838199in}{2.048385in}}{\pgfqpoint{4.842589in}{2.058984in}}{\pgfqpoint{4.842589in}{2.070034in}}%
\pgfpathcurveto{\pgfqpoint{4.842589in}{2.081085in}}{\pgfqpoint{4.838199in}{2.091684in}}{\pgfqpoint{4.830385in}{2.099497in}}%
\pgfpathcurveto{\pgfqpoint{4.822572in}{2.107311in}}{\pgfqpoint{4.811973in}{2.111701in}}{\pgfqpoint{4.800923in}{2.111701in}}%
\pgfpathcurveto{\pgfqpoint{4.789872in}{2.111701in}}{\pgfqpoint{4.779273in}{2.107311in}}{\pgfqpoint{4.771460in}{2.099497in}}%
\pgfpathcurveto{\pgfqpoint{4.763646in}{2.091684in}}{\pgfqpoint{4.759256in}{2.081085in}}{\pgfqpoint{4.759256in}{2.070034in}}%
\pgfpathcurveto{\pgfqpoint{4.759256in}{2.058984in}}{\pgfqpoint{4.763646in}{2.048385in}}{\pgfqpoint{4.771460in}{2.040572in}}%
\pgfpathcurveto{\pgfqpoint{4.779273in}{2.032758in}}{\pgfqpoint{4.789872in}{2.028368in}}{\pgfqpoint{4.800923in}{2.028368in}}%
\pgfpathclose%
\pgfusepath{stroke,fill}%
\end{pgfscope}%
\begin{pgfscope}%
\pgfpathrectangle{\pgfqpoint{1.115200in}{0.528000in}}{\pgfqpoint{4.329600in}{3.696000in}} %
\pgfusepath{clip}%
\pgfsetbuttcap%
\pgfsetroundjoin%
\definecolor{currentfill}{rgb}{1.000000,0.000000,0.000000}%
\pgfsetfillcolor{currentfill}%
\pgfsetlinewidth{1.003750pt}%
\definecolor{currentstroke}{rgb}{1.000000,0.000000,0.000000}%
\pgfsetstrokecolor{currentstroke}%
\pgfsetdash{}{0pt}%
\pgfpathmoveto{\pgfqpoint{4.856816in}{1.673294in}}%
\pgfpathcurveto{\pgfqpoint{4.867866in}{1.673294in}}{\pgfqpoint{4.878465in}{1.677685in}}{\pgfqpoint{4.886278in}{1.685498in}}%
\pgfpathcurveto{\pgfqpoint{4.894092in}{1.693312in}}{\pgfqpoint{4.898482in}{1.703911in}}{\pgfqpoint{4.898482in}{1.714961in}}%
\pgfpathcurveto{\pgfqpoint{4.898482in}{1.726011in}}{\pgfqpoint{4.894092in}{1.736610in}}{\pgfqpoint{4.886278in}{1.744424in}}%
\pgfpathcurveto{\pgfqpoint{4.878465in}{1.752237in}}{\pgfqpoint{4.867866in}{1.756628in}}{\pgfqpoint{4.856816in}{1.756628in}}%
\pgfpathcurveto{\pgfqpoint{4.845765in}{1.756628in}}{\pgfqpoint{4.835166in}{1.752237in}}{\pgfqpoint{4.827353in}{1.744424in}}%
\pgfpathcurveto{\pgfqpoint{4.819539in}{1.736610in}}{\pgfqpoint{4.815149in}{1.726011in}}{\pgfqpoint{4.815149in}{1.714961in}}%
\pgfpathcurveto{\pgfqpoint{4.815149in}{1.703911in}}{\pgfqpoint{4.819539in}{1.693312in}}{\pgfqpoint{4.827353in}{1.685498in}}%
\pgfpathcurveto{\pgfqpoint{4.835166in}{1.677685in}}{\pgfqpoint{4.845765in}{1.673294in}}{\pgfqpoint{4.856816in}{1.673294in}}%
\pgfpathclose%
\pgfusepath{stroke,fill}%
\end{pgfscope}%
\begin{pgfscope}%
\pgfpathrectangle{\pgfqpoint{1.115200in}{0.528000in}}{\pgfqpoint{4.329600in}{3.696000in}} %
\pgfusepath{clip}%
\pgfsetbuttcap%
\pgfsetroundjoin%
\definecolor{currentfill}{rgb}{1.000000,0.000000,0.000000}%
\pgfsetfillcolor{currentfill}%
\pgfsetlinewidth{1.003750pt}%
\definecolor{currentstroke}{rgb}{1.000000,0.000000,0.000000}%
\pgfsetstrokecolor{currentstroke}%
\pgfsetdash{}{0pt}%
\pgfpathmoveto{\pgfqpoint{4.712403in}{1.389682in}}%
\pgfpathcurveto{\pgfqpoint{4.723453in}{1.389682in}}{\pgfqpoint{4.734053in}{1.394072in}}{\pgfqpoint{4.741866in}{1.401886in}}%
\pgfpathcurveto{\pgfqpoint{4.749680in}{1.409700in}}{\pgfqpoint{4.754070in}{1.420299in}}{\pgfqpoint{4.754070in}{1.431349in}}%
\pgfpathcurveto{\pgfqpoint{4.754070in}{1.442399in}}{\pgfqpoint{4.749680in}{1.452998in}}{\pgfqpoint{4.741866in}{1.460812in}}%
\pgfpathcurveto{\pgfqpoint{4.734053in}{1.468625in}}{\pgfqpoint{4.723453in}{1.473016in}}{\pgfqpoint{4.712403in}{1.473016in}}%
\pgfpathcurveto{\pgfqpoint{4.701353in}{1.473016in}}{\pgfqpoint{4.690754in}{1.468625in}}{\pgfqpoint{4.682941in}{1.460812in}}%
\pgfpathcurveto{\pgfqpoint{4.675127in}{1.452998in}}{\pgfqpoint{4.670737in}{1.442399in}}{\pgfqpoint{4.670737in}{1.431349in}}%
\pgfpathcurveto{\pgfqpoint{4.670737in}{1.420299in}}{\pgfqpoint{4.675127in}{1.409700in}}{\pgfqpoint{4.682941in}{1.401886in}}%
\pgfpathcurveto{\pgfqpoint{4.690754in}{1.394072in}}{\pgfqpoint{4.701353in}{1.389682in}}{\pgfqpoint{4.712403in}{1.389682in}}%
\pgfpathclose%
\pgfusepath{stroke,fill}%
\end{pgfscope}%
\begin{pgfscope}%
\pgfpathrectangle{\pgfqpoint{1.115200in}{0.528000in}}{\pgfqpoint{4.329600in}{3.696000in}} %
\pgfusepath{clip}%
\pgfsetbuttcap%
\pgfsetroundjoin%
\definecolor{currentfill}{rgb}{1.000000,0.000000,0.000000}%
\pgfsetfillcolor{currentfill}%
\pgfsetlinewidth{1.003750pt}%
\definecolor{currentstroke}{rgb}{1.000000,0.000000,0.000000}%
\pgfsetstrokecolor{currentstroke}%
\pgfsetdash{}{0pt}%
\pgfpathmoveto{\pgfqpoint{5.174229in}{2.130709in}}%
\pgfpathcurveto{\pgfqpoint{5.185279in}{2.130709in}}{\pgfqpoint{5.195878in}{2.135100in}}{\pgfqpoint{5.203691in}{2.142913in}}%
\pgfpathcurveto{\pgfqpoint{5.211505in}{2.150727in}}{\pgfqpoint{5.215895in}{2.161326in}}{\pgfqpoint{5.215895in}{2.172376in}}%
\pgfpathcurveto{\pgfqpoint{5.215895in}{2.183426in}}{\pgfqpoint{5.211505in}{2.194025in}}{\pgfqpoint{5.203691in}{2.201839in}}%
\pgfpathcurveto{\pgfqpoint{5.195878in}{2.209652in}}{\pgfqpoint{5.185279in}{2.214043in}}{\pgfqpoint{5.174229in}{2.214043in}}%
\pgfpathcurveto{\pgfqpoint{5.163178in}{2.214043in}}{\pgfqpoint{5.152579in}{2.209652in}}{\pgfqpoint{5.144766in}{2.201839in}}%
\pgfpathcurveto{\pgfqpoint{5.136952in}{2.194025in}}{\pgfqpoint{5.132562in}{2.183426in}}{\pgfqpoint{5.132562in}{2.172376in}}%
\pgfpathcurveto{\pgfqpoint{5.132562in}{2.161326in}}{\pgfqpoint{5.136952in}{2.150727in}}{\pgfqpoint{5.144766in}{2.142913in}}%
\pgfpathcurveto{\pgfqpoint{5.152579in}{2.135100in}}{\pgfqpoint{5.163178in}{2.130709in}}{\pgfqpoint{5.174229in}{2.130709in}}%
\pgfpathclose%
\pgfusepath{stroke,fill}%
\end{pgfscope}%
\begin{pgfscope}%
\pgfpathrectangle{\pgfqpoint{1.115200in}{0.528000in}}{\pgfqpoint{4.329600in}{3.696000in}} %
\pgfusepath{clip}%
\pgfsetbuttcap%
\pgfsetroundjoin%
\definecolor{currentfill}{rgb}{1.000000,0.000000,0.000000}%
\pgfsetfillcolor{currentfill}%
\pgfsetlinewidth{1.003750pt}%
\definecolor{currentstroke}{rgb}{1.000000,0.000000,0.000000}%
\pgfsetstrokecolor{currentstroke}%
\pgfsetdash{}{0pt}%
\pgfpathmoveto{\pgfqpoint{4.170268in}{2.157907in}}%
\pgfpathcurveto{\pgfqpoint{4.181319in}{2.157907in}}{\pgfqpoint{4.191918in}{2.162297in}}{\pgfqpoint{4.199731in}{2.170111in}}%
\pgfpathcurveto{\pgfqpoint{4.207545in}{2.177925in}}{\pgfqpoint{4.211935in}{2.188524in}}{\pgfqpoint{4.211935in}{2.199574in}}%
\pgfpathcurveto{\pgfqpoint{4.211935in}{2.210624in}}{\pgfqpoint{4.207545in}{2.221223in}}{\pgfqpoint{4.199731in}{2.229037in}}%
\pgfpathcurveto{\pgfqpoint{4.191918in}{2.236850in}}{\pgfqpoint{4.181319in}{2.241240in}}{\pgfqpoint{4.170268in}{2.241240in}}%
\pgfpathcurveto{\pgfqpoint{4.159218in}{2.241240in}}{\pgfqpoint{4.148619in}{2.236850in}}{\pgfqpoint{4.140806in}{2.229037in}}%
\pgfpathcurveto{\pgfqpoint{4.132992in}{2.221223in}}{\pgfqpoint{4.128602in}{2.210624in}}{\pgfqpoint{4.128602in}{2.199574in}}%
\pgfpathcurveto{\pgfqpoint{4.128602in}{2.188524in}}{\pgfqpoint{4.132992in}{2.177925in}}{\pgfqpoint{4.140806in}{2.170111in}}%
\pgfpathcurveto{\pgfqpoint{4.148619in}{2.162297in}}{\pgfqpoint{4.159218in}{2.157907in}}{\pgfqpoint{4.170268in}{2.157907in}}%
\pgfpathclose%
\pgfusepath{stroke,fill}%
\end{pgfscope}%
\begin{pgfscope}%
\pgfpathrectangle{\pgfqpoint{1.115200in}{0.528000in}}{\pgfqpoint{4.329600in}{3.696000in}} %
\pgfusepath{clip}%
\pgfsetbuttcap%
\pgfsetroundjoin%
\definecolor{currentfill}{rgb}{1.000000,0.000000,0.000000}%
\pgfsetfillcolor{currentfill}%
\pgfsetlinewidth{1.003750pt}%
\definecolor{currentstroke}{rgb}{1.000000,0.000000,0.000000}%
\pgfsetstrokecolor{currentstroke}%
\pgfsetdash{}{0pt}%
\pgfpathmoveto{\pgfqpoint{4.453121in}{1.895388in}}%
\pgfpathcurveto{\pgfqpoint{4.464172in}{1.895388in}}{\pgfqpoint{4.474771in}{1.899779in}}{\pgfqpoint{4.482584in}{1.907592in}}%
\pgfpathcurveto{\pgfqpoint{4.490398in}{1.915406in}}{\pgfqpoint{4.494788in}{1.926005in}}{\pgfqpoint{4.494788in}{1.937055in}}%
\pgfpathcurveto{\pgfqpoint{4.494788in}{1.948105in}}{\pgfqpoint{4.490398in}{1.958704in}}{\pgfqpoint{4.482584in}{1.966518in}}%
\pgfpathcurveto{\pgfqpoint{4.474771in}{1.974331in}}{\pgfqpoint{4.464172in}{1.978722in}}{\pgfqpoint{4.453121in}{1.978722in}}%
\pgfpathcurveto{\pgfqpoint{4.442071in}{1.978722in}}{\pgfqpoint{4.431472in}{1.974331in}}{\pgfqpoint{4.423659in}{1.966518in}}%
\pgfpathcurveto{\pgfqpoint{4.415845in}{1.958704in}}{\pgfqpoint{4.411455in}{1.948105in}}{\pgfqpoint{4.411455in}{1.937055in}}%
\pgfpathcurveto{\pgfqpoint{4.411455in}{1.926005in}}{\pgfqpoint{4.415845in}{1.915406in}}{\pgfqpoint{4.423659in}{1.907592in}}%
\pgfpathcurveto{\pgfqpoint{4.431472in}{1.899779in}}{\pgfqpoint{4.442071in}{1.895388in}}{\pgfqpoint{4.453121in}{1.895388in}}%
\pgfpathclose%
\pgfusepath{stroke,fill}%
\end{pgfscope}%
\begin{pgfscope}%
\pgfpathrectangle{\pgfqpoint{1.115200in}{0.528000in}}{\pgfqpoint{4.329600in}{3.696000in}} %
\pgfusepath{clip}%
\pgfsetbuttcap%
\pgfsetroundjoin%
\definecolor{currentfill}{rgb}{1.000000,0.000000,0.000000}%
\pgfsetfillcolor{currentfill}%
\pgfsetlinewidth{1.003750pt}%
\definecolor{currentstroke}{rgb}{1.000000,0.000000,0.000000}%
\pgfsetstrokecolor{currentstroke}%
\pgfsetdash{}{0pt}%
\pgfpathmoveto{\pgfqpoint{4.861250in}{1.852177in}}%
\pgfpathcurveto{\pgfqpoint{4.872300in}{1.852177in}}{\pgfqpoint{4.882899in}{1.856568in}}{\pgfqpoint{4.890713in}{1.864381in}}%
\pgfpathcurveto{\pgfqpoint{4.898527in}{1.872195in}}{\pgfqpoint{4.902917in}{1.882794in}}{\pgfqpoint{4.902917in}{1.893844in}}%
\pgfpathcurveto{\pgfqpoint{4.902917in}{1.904894in}}{\pgfqpoint{4.898527in}{1.915493in}}{\pgfqpoint{4.890713in}{1.923307in}}%
\pgfpathcurveto{\pgfqpoint{4.882899in}{1.931121in}}{\pgfqpoint{4.872300in}{1.935511in}}{\pgfqpoint{4.861250in}{1.935511in}}%
\pgfpathcurveto{\pgfqpoint{4.850200in}{1.935511in}}{\pgfqpoint{4.839601in}{1.931121in}}{\pgfqpoint{4.831788in}{1.923307in}}%
\pgfpathcurveto{\pgfqpoint{4.823974in}{1.915493in}}{\pgfqpoint{4.819584in}{1.904894in}}{\pgfqpoint{4.819584in}{1.893844in}}%
\pgfpathcurveto{\pgfqpoint{4.819584in}{1.882794in}}{\pgfqpoint{4.823974in}{1.872195in}}{\pgfqpoint{4.831788in}{1.864381in}}%
\pgfpathcurveto{\pgfqpoint{4.839601in}{1.856568in}}{\pgfqpoint{4.850200in}{1.852177in}}{\pgfqpoint{4.861250in}{1.852177in}}%
\pgfpathclose%
\pgfusepath{stroke,fill}%
\end{pgfscope}%
\begin{pgfscope}%
\pgfpathrectangle{\pgfqpoint{1.115200in}{0.528000in}}{\pgfqpoint{4.329600in}{3.696000in}} %
\pgfusepath{clip}%
\pgfsetbuttcap%
\pgfsetroundjoin%
\definecolor{currentfill}{rgb}{0.000000,0.000000,1.000000}%
\pgfsetfillcolor{currentfill}%
\pgfsetlinewidth{1.003750pt}%
\definecolor{currentstroke}{rgb}{0.000000,0.000000,1.000000}%
\pgfsetstrokecolor{currentstroke}%
\pgfsetdash{}{0pt}%
\pgfpathmoveto{\pgfqpoint{1.834260in}{1.780375in}}%
\pgfpathcurveto{\pgfqpoint{1.845310in}{1.780375in}}{\pgfqpoint{1.855909in}{1.784765in}}{\pgfqpoint{1.863723in}{1.792578in}}%
\pgfpathcurveto{\pgfqpoint{1.871537in}{1.800392in}}{\pgfqpoint{1.875927in}{1.810991in}}{\pgfqpoint{1.875927in}{1.822041in}}%
\pgfpathcurveto{\pgfqpoint{1.875927in}{1.833091in}}{\pgfqpoint{1.871537in}{1.843690in}}{\pgfqpoint{1.863723in}{1.851504in}}%
\pgfpathcurveto{\pgfqpoint{1.855909in}{1.859318in}}{\pgfqpoint{1.845310in}{1.863708in}}{\pgfqpoint{1.834260in}{1.863708in}}%
\pgfpathcurveto{\pgfqpoint{1.823210in}{1.863708in}}{\pgfqpoint{1.812611in}{1.859318in}}{\pgfqpoint{1.804798in}{1.851504in}}%
\pgfpathcurveto{\pgfqpoint{1.796984in}{1.843690in}}{\pgfqpoint{1.792594in}{1.833091in}}{\pgfqpoint{1.792594in}{1.822041in}}%
\pgfpathcurveto{\pgfqpoint{1.792594in}{1.810991in}}{\pgfqpoint{1.796984in}{1.800392in}}{\pgfqpoint{1.804798in}{1.792578in}}%
\pgfpathcurveto{\pgfqpoint{1.812611in}{1.784765in}}{\pgfqpoint{1.823210in}{1.780375in}}{\pgfqpoint{1.834260in}{1.780375in}}%
\pgfpathclose%
\pgfusepath{stroke,fill}%
\end{pgfscope}%
\begin{pgfscope}%
\pgfpathrectangle{\pgfqpoint{1.115200in}{0.528000in}}{\pgfqpoint{4.329600in}{3.696000in}} %
\pgfusepath{clip}%
\pgfsetbuttcap%
\pgfsetroundjoin%
\definecolor{currentfill}{rgb}{0.000000,0.000000,1.000000}%
\pgfsetfillcolor{currentfill}%
\pgfsetlinewidth{1.003750pt}%
\definecolor{currentstroke}{rgb}{0.000000,0.000000,1.000000}%
\pgfsetstrokecolor{currentstroke}%
\pgfsetdash{}{0pt}%
\pgfpathmoveto{\pgfqpoint{2.039030in}{1.448013in}}%
\pgfpathcurveto{\pgfqpoint{2.050080in}{1.448013in}}{\pgfqpoint{2.060679in}{1.452403in}}{\pgfqpoint{2.068493in}{1.460216in}}%
\pgfpathcurveto{\pgfqpoint{2.076306in}{1.468030in}}{\pgfqpoint{2.080697in}{1.478629in}}{\pgfqpoint{2.080697in}{1.489679in}}%
\pgfpathcurveto{\pgfqpoint{2.080697in}{1.500729in}}{\pgfqpoint{2.076306in}{1.511328in}}{\pgfqpoint{2.068493in}{1.519142in}}%
\pgfpathcurveto{\pgfqpoint{2.060679in}{1.526956in}}{\pgfqpoint{2.050080in}{1.531346in}}{\pgfqpoint{2.039030in}{1.531346in}}%
\pgfpathcurveto{\pgfqpoint{2.027980in}{1.531346in}}{\pgfqpoint{2.017381in}{1.526956in}}{\pgfqpoint{2.009567in}{1.519142in}}%
\pgfpathcurveto{\pgfqpoint{2.001753in}{1.511328in}}{\pgfqpoint{1.997363in}{1.500729in}}{\pgfqpoint{1.997363in}{1.489679in}}%
\pgfpathcurveto{\pgfqpoint{1.997363in}{1.478629in}}{\pgfqpoint{2.001753in}{1.468030in}}{\pgfqpoint{2.009567in}{1.460216in}}%
\pgfpathcurveto{\pgfqpoint{2.017381in}{1.452403in}}{\pgfqpoint{2.027980in}{1.448013in}}{\pgfqpoint{2.039030in}{1.448013in}}%
\pgfpathclose%
\pgfusepath{stroke,fill}%
\end{pgfscope}%
\begin{pgfscope}%
\pgfpathrectangle{\pgfqpoint{1.115200in}{0.528000in}}{\pgfqpoint{4.329600in}{3.696000in}} %
\pgfusepath{clip}%
\pgfsetbuttcap%
\pgfsetroundjoin%
\definecolor{currentfill}{rgb}{0.000000,0.000000,1.000000}%
\pgfsetfillcolor{currentfill}%
\pgfsetlinewidth{1.003750pt}%
\definecolor{currentstroke}{rgb}{0.000000,0.000000,1.000000}%
\pgfsetstrokecolor{currentstroke}%
\pgfsetdash{}{0pt}%
\pgfpathmoveto{\pgfqpoint{2.097463in}{1.626023in}}%
\pgfpathcurveto{\pgfqpoint{2.108514in}{1.626023in}}{\pgfqpoint{2.119113in}{1.630413in}}{\pgfqpoint{2.126926in}{1.638226in}}%
\pgfpathcurveto{\pgfqpoint{2.134740in}{1.646040in}}{\pgfqpoint{2.139130in}{1.656639in}}{\pgfqpoint{2.139130in}{1.667689in}}%
\pgfpathcurveto{\pgfqpoint{2.139130in}{1.678739in}}{\pgfqpoint{2.134740in}{1.689338in}}{\pgfqpoint{2.126926in}{1.697152in}}%
\pgfpathcurveto{\pgfqpoint{2.119113in}{1.704966in}}{\pgfqpoint{2.108514in}{1.709356in}}{\pgfqpoint{2.097463in}{1.709356in}}%
\pgfpathcurveto{\pgfqpoint{2.086413in}{1.709356in}}{\pgfqpoint{2.075814in}{1.704966in}}{\pgfqpoint{2.068001in}{1.697152in}}%
\pgfpathcurveto{\pgfqpoint{2.060187in}{1.689338in}}{\pgfqpoint{2.055797in}{1.678739in}}{\pgfqpoint{2.055797in}{1.667689in}}%
\pgfpathcurveto{\pgfqpoint{2.055797in}{1.656639in}}{\pgfqpoint{2.060187in}{1.646040in}}{\pgfqpoint{2.068001in}{1.638226in}}%
\pgfpathcurveto{\pgfqpoint{2.075814in}{1.630413in}}{\pgfqpoint{2.086413in}{1.626023in}}{\pgfqpoint{2.097463in}{1.626023in}}%
\pgfpathclose%
\pgfusepath{stroke,fill}%
\end{pgfscope}%
\begin{pgfscope}%
\pgfpathrectangle{\pgfqpoint{1.115200in}{0.528000in}}{\pgfqpoint{4.329600in}{3.696000in}} %
\pgfusepath{clip}%
\pgfsetbuttcap%
\pgfsetroundjoin%
\definecolor{currentfill}{rgb}{0.000000,0.000000,1.000000}%
\pgfsetfillcolor{currentfill}%
\pgfsetlinewidth{1.003750pt}%
\definecolor{currentstroke}{rgb}{0.000000,0.000000,1.000000}%
\pgfsetstrokecolor{currentstroke}%
\pgfsetdash{}{0pt}%
\pgfpathmoveto{\pgfqpoint{2.208220in}{1.691202in}}%
\pgfpathcurveto{\pgfqpoint{2.219270in}{1.691202in}}{\pgfqpoint{2.229869in}{1.695592in}}{\pgfqpoint{2.237683in}{1.703405in}}%
\pgfpathcurveto{\pgfqpoint{2.245496in}{1.711219in}}{\pgfqpoint{2.249886in}{1.721818in}}{\pgfqpoint{2.249886in}{1.732868in}}%
\pgfpathcurveto{\pgfqpoint{2.249886in}{1.743918in}}{\pgfqpoint{2.245496in}{1.754517in}}{\pgfqpoint{2.237683in}{1.762331in}}%
\pgfpathcurveto{\pgfqpoint{2.229869in}{1.770145in}}{\pgfqpoint{2.219270in}{1.774535in}}{\pgfqpoint{2.208220in}{1.774535in}}%
\pgfpathcurveto{\pgfqpoint{2.197170in}{1.774535in}}{\pgfqpoint{2.186571in}{1.770145in}}{\pgfqpoint{2.178757in}{1.762331in}}%
\pgfpathcurveto{\pgfqpoint{2.170943in}{1.754517in}}{\pgfqpoint{2.166553in}{1.743918in}}{\pgfqpoint{2.166553in}{1.732868in}}%
\pgfpathcurveto{\pgfqpoint{2.166553in}{1.721818in}}{\pgfqpoint{2.170943in}{1.711219in}}{\pgfqpoint{2.178757in}{1.703405in}}%
\pgfpathcurveto{\pgfqpoint{2.186571in}{1.695592in}}{\pgfqpoint{2.197170in}{1.691202in}}{\pgfqpoint{2.208220in}{1.691202in}}%
\pgfpathclose%
\pgfusepath{stroke,fill}%
\end{pgfscope}%
\begin{pgfscope}%
\pgfpathrectangle{\pgfqpoint{1.115200in}{0.528000in}}{\pgfqpoint{4.329600in}{3.696000in}} %
\pgfusepath{clip}%
\pgfsetbuttcap%
\pgfsetroundjoin%
\definecolor{currentfill}{rgb}{0.000000,0.000000,1.000000}%
\pgfsetfillcolor{currentfill}%
\pgfsetlinewidth{1.003750pt}%
\definecolor{currentstroke}{rgb}{0.000000,0.000000,1.000000}%
\pgfsetstrokecolor{currentstroke}%
\pgfsetdash{}{0pt}%
\pgfpathmoveto{\pgfqpoint{2.390590in}{1.609857in}}%
\pgfpathcurveto{\pgfqpoint{2.401640in}{1.609857in}}{\pgfqpoint{2.412239in}{1.614247in}}{\pgfqpoint{2.420052in}{1.622061in}}%
\pgfpathcurveto{\pgfqpoint{2.427866in}{1.629874in}}{\pgfqpoint{2.432256in}{1.640473in}}{\pgfqpoint{2.432256in}{1.651524in}}%
\pgfpathcurveto{\pgfqpoint{2.432256in}{1.662574in}}{\pgfqpoint{2.427866in}{1.673173in}}{\pgfqpoint{2.420052in}{1.680986in}}%
\pgfpathcurveto{\pgfqpoint{2.412239in}{1.688800in}}{\pgfqpoint{2.401640in}{1.693190in}}{\pgfqpoint{2.390590in}{1.693190in}}%
\pgfpathcurveto{\pgfqpoint{2.379539in}{1.693190in}}{\pgfqpoint{2.368940in}{1.688800in}}{\pgfqpoint{2.361127in}{1.680986in}}%
\pgfpathcurveto{\pgfqpoint{2.353313in}{1.673173in}}{\pgfqpoint{2.348923in}{1.662574in}}{\pgfqpoint{2.348923in}{1.651524in}}%
\pgfpathcurveto{\pgfqpoint{2.348923in}{1.640473in}}{\pgfqpoint{2.353313in}{1.629874in}}{\pgfqpoint{2.361127in}{1.622061in}}%
\pgfpathcurveto{\pgfqpoint{2.368940in}{1.614247in}}{\pgfqpoint{2.379539in}{1.609857in}}{\pgfqpoint{2.390590in}{1.609857in}}%
\pgfpathclose%
\pgfusepath{stroke,fill}%
\end{pgfscope}%
\begin{pgfscope}%
\pgfpathrectangle{\pgfqpoint{1.115200in}{0.528000in}}{\pgfqpoint{4.329600in}{3.696000in}} %
\pgfusepath{clip}%
\pgfsetbuttcap%
\pgfsetroundjoin%
\definecolor{currentfill}{rgb}{0.000000,0.000000,1.000000}%
\pgfsetfillcolor{currentfill}%
\pgfsetlinewidth{1.003750pt}%
\definecolor{currentstroke}{rgb}{0.000000,0.000000,1.000000}%
\pgfsetstrokecolor{currentstroke}%
\pgfsetdash{}{0pt}%
\pgfpathmoveto{\pgfqpoint{2.088891in}{1.422241in}}%
\pgfpathcurveto{\pgfqpoint{2.099941in}{1.422241in}}{\pgfqpoint{2.110540in}{1.426632in}}{\pgfqpoint{2.118354in}{1.434445in}}%
\pgfpathcurveto{\pgfqpoint{2.126167in}{1.442259in}}{\pgfqpoint{2.130558in}{1.452858in}}{\pgfqpoint{2.130558in}{1.463908in}}%
\pgfpathcurveto{\pgfqpoint{2.130558in}{1.474958in}}{\pgfqpoint{2.126167in}{1.485557in}}{\pgfqpoint{2.118354in}{1.493371in}}%
\pgfpathcurveto{\pgfqpoint{2.110540in}{1.501184in}}{\pgfqpoint{2.099941in}{1.505575in}}{\pgfqpoint{2.088891in}{1.505575in}}%
\pgfpathcurveto{\pgfqpoint{2.077841in}{1.505575in}}{\pgfqpoint{2.067242in}{1.501184in}}{\pgfqpoint{2.059428in}{1.493371in}}%
\pgfpathcurveto{\pgfqpoint{2.051615in}{1.485557in}}{\pgfqpoint{2.047224in}{1.474958in}}{\pgfqpoint{2.047224in}{1.463908in}}%
\pgfpathcurveto{\pgfqpoint{2.047224in}{1.452858in}}{\pgfqpoint{2.051615in}{1.442259in}}{\pgfqpoint{2.059428in}{1.434445in}}%
\pgfpathcurveto{\pgfqpoint{2.067242in}{1.426632in}}{\pgfqpoint{2.077841in}{1.422241in}}{\pgfqpoint{2.088891in}{1.422241in}}%
\pgfpathclose%
\pgfusepath{stroke,fill}%
\end{pgfscope}%
\begin{pgfscope}%
\pgfpathrectangle{\pgfqpoint{1.115200in}{0.528000in}}{\pgfqpoint{4.329600in}{3.696000in}} %
\pgfusepath{clip}%
\pgfsetbuttcap%
\pgfsetroundjoin%
\definecolor{currentfill}{rgb}{0.000000,0.000000,1.000000}%
\pgfsetfillcolor{currentfill}%
\pgfsetlinewidth{1.003750pt}%
\definecolor{currentstroke}{rgb}{0.000000,0.000000,1.000000}%
\pgfsetstrokecolor{currentstroke}%
\pgfsetdash{}{0pt}%
\pgfpathmoveto{\pgfqpoint{2.047624in}{1.516614in}}%
\pgfpathcurveto{\pgfqpoint{2.058674in}{1.516614in}}{\pgfqpoint{2.069273in}{1.521005in}}{\pgfqpoint{2.077087in}{1.528818in}}%
\pgfpathcurveto{\pgfqpoint{2.084900in}{1.536632in}}{\pgfqpoint{2.089291in}{1.547231in}}{\pgfqpoint{2.089291in}{1.558281in}}%
\pgfpathcurveto{\pgfqpoint{2.089291in}{1.569331in}}{\pgfqpoint{2.084900in}{1.579930in}}{\pgfqpoint{2.077087in}{1.587744in}}%
\pgfpathcurveto{\pgfqpoint{2.069273in}{1.595557in}}{\pgfqpoint{2.058674in}{1.599948in}}{\pgfqpoint{2.047624in}{1.599948in}}%
\pgfpathcurveto{\pgfqpoint{2.036574in}{1.599948in}}{\pgfqpoint{2.025975in}{1.595557in}}{\pgfqpoint{2.018161in}{1.587744in}}%
\pgfpathcurveto{\pgfqpoint{2.010347in}{1.579930in}}{\pgfqpoint{2.005957in}{1.569331in}}{\pgfqpoint{2.005957in}{1.558281in}}%
\pgfpathcurveto{\pgfqpoint{2.005957in}{1.547231in}}{\pgfqpoint{2.010347in}{1.536632in}}{\pgfqpoint{2.018161in}{1.528818in}}%
\pgfpathcurveto{\pgfqpoint{2.025975in}{1.521005in}}{\pgfqpoint{2.036574in}{1.516614in}}{\pgfqpoint{2.047624in}{1.516614in}}%
\pgfpathclose%
\pgfusepath{stroke,fill}%
\end{pgfscope}%
\begin{pgfscope}%
\pgfpathrectangle{\pgfqpoint{1.115200in}{0.528000in}}{\pgfqpoint{4.329600in}{3.696000in}} %
\pgfusepath{clip}%
\pgfsetbuttcap%
\pgfsetroundjoin%
\definecolor{currentfill}{rgb}{0.000000,0.000000,1.000000}%
\pgfsetfillcolor{currentfill}%
\pgfsetlinewidth{1.003750pt}%
\definecolor{currentstroke}{rgb}{0.000000,0.000000,1.000000}%
\pgfsetstrokecolor{currentstroke}%
\pgfsetdash{}{0pt}%
\pgfpathmoveto{\pgfqpoint{2.322569in}{1.657460in}}%
\pgfpathcurveto{\pgfqpoint{2.333619in}{1.657460in}}{\pgfqpoint{2.344218in}{1.661850in}}{\pgfqpoint{2.352031in}{1.669664in}}%
\pgfpathcurveto{\pgfqpoint{2.359845in}{1.677478in}}{\pgfqpoint{2.364235in}{1.688077in}}{\pgfqpoint{2.364235in}{1.699127in}}%
\pgfpathcurveto{\pgfqpoint{2.364235in}{1.710177in}}{\pgfqpoint{2.359845in}{1.720776in}}{\pgfqpoint{2.352031in}{1.728589in}}%
\pgfpathcurveto{\pgfqpoint{2.344218in}{1.736403in}}{\pgfqpoint{2.333619in}{1.740793in}}{\pgfqpoint{2.322569in}{1.740793in}}%
\pgfpathcurveto{\pgfqpoint{2.311518in}{1.740793in}}{\pgfqpoint{2.300919in}{1.736403in}}{\pgfqpoint{2.293106in}{1.728589in}}%
\pgfpathcurveto{\pgfqpoint{2.285292in}{1.720776in}}{\pgfqpoint{2.280902in}{1.710177in}}{\pgfqpoint{2.280902in}{1.699127in}}%
\pgfpathcurveto{\pgfqpoint{2.280902in}{1.688077in}}{\pgfqpoint{2.285292in}{1.677478in}}{\pgfqpoint{2.293106in}{1.669664in}}%
\pgfpathcurveto{\pgfqpoint{2.300919in}{1.661850in}}{\pgfqpoint{2.311518in}{1.657460in}}{\pgfqpoint{2.322569in}{1.657460in}}%
\pgfpathclose%
\pgfusepath{stroke,fill}%
\end{pgfscope}%
\begin{pgfscope}%
\pgfpathrectangle{\pgfqpoint{1.115200in}{0.528000in}}{\pgfqpoint{4.329600in}{3.696000in}} %
\pgfusepath{clip}%
\pgfsetbuttcap%
\pgfsetroundjoin%
\definecolor{currentfill}{rgb}{0.000000,0.000000,1.000000}%
\pgfsetfillcolor{currentfill}%
\pgfsetlinewidth{1.003750pt}%
\definecolor{currentstroke}{rgb}{0.000000,0.000000,1.000000}%
\pgfsetstrokecolor{currentstroke}%
\pgfsetdash{}{0pt}%
\pgfpathmoveto{\pgfqpoint{2.151125in}{1.612676in}}%
\pgfpathcurveto{\pgfqpoint{2.162175in}{1.612676in}}{\pgfqpoint{2.172774in}{1.617066in}}{\pgfqpoint{2.180588in}{1.624880in}}%
\pgfpathcurveto{\pgfqpoint{2.188401in}{1.632694in}}{\pgfqpoint{2.192792in}{1.643293in}}{\pgfqpoint{2.192792in}{1.654343in}}%
\pgfpathcurveto{\pgfqpoint{2.192792in}{1.665393in}}{\pgfqpoint{2.188401in}{1.675992in}}{\pgfqpoint{2.180588in}{1.683806in}}%
\pgfpathcurveto{\pgfqpoint{2.172774in}{1.691619in}}{\pgfqpoint{2.162175in}{1.696009in}}{\pgfqpoint{2.151125in}{1.696009in}}%
\pgfpathcurveto{\pgfqpoint{2.140075in}{1.696009in}}{\pgfqpoint{2.129476in}{1.691619in}}{\pgfqpoint{2.121662in}{1.683806in}}%
\pgfpathcurveto{\pgfqpoint{2.113849in}{1.675992in}}{\pgfqpoint{2.109458in}{1.665393in}}{\pgfqpoint{2.109458in}{1.654343in}}%
\pgfpathcurveto{\pgfqpoint{2.109458in}{1.643293in}}{\pgfqpoint{2.113849in}{1.632694in}}{\pgfqpoint{2.121662in}{1.624880in}}%
\pgfpathcurveto{\pgfqpoint{2.129476in}{1.617066in}}{\pgfqpoint{2.140075in}{1.612676in}}{\pgfqpoint{2.151125in}{1.612676in}}%
\pgfpathclose%
\pgfusepath{stroke,fill}%
\end{pgfscope}%
\begin{pgfscope}%
\pgfpathrectangle{\pgfqpoint{1.115200in}{0.528000in}}{\pgfqpoint{4.329600in}{3.696000in}} %
\pgfusepath{clip}%
\pgfsetbuttcap%
\pgfsetroundjoin%
\definecolor{currentfill}{rgb}{0.000000,0.000000,1.000000}%
\pgfsetfillcolor{currentfill}%
\pgfsetlinewidth{1.003750pt}%
\definecolor{currentstroke}{rgb}{0.000000,0.000000,1.000000}%
\pgfsetstrokecolor{currentstroke}%
\pgfsetdash{}{0pt}%
\pgfpathmoveto{\pgfqpoint{1.954412in}{1.406613in}}%
\pgfpathcurveto{\pgfqpoint{1.965462in}{1.406613in}}{\pgfqpoint{1.976061in}{1.411003in}}{\pgfqpoint{1.983875in}{1.418817in}}%
\pgfpathcurveto{\pgfqpoint{1.991688in}{1.426630in}}{\pgfqpoint{1.996078in}{1.437229in}}{\pgfqpoint{1.996078in}{1.448279in}}%
\pgfpathcurveto{\pgfqpoint{1.996078in}{1.459329in}}{\pgfqpoint{1.991688in}{1.469928in}}{\pgfqpoint{1.983875in}{1.477742in}}%
\pgfpathcurveto{\pgfqpoint{1.976061in}{1.485556in}}{\pgfqpoint{1.965462in}{1.489946in}}{\pgfqpoint{1.954412in}{1.489946in}}%
\pgfpathcurveto{\pgfqpoint{1.943362in}{1.489946in}}{\pgfqpoint{1.932763in}{1.485556in}}{\pgfqpoint{1.924949in}{1.477742in}}%
\pgfpathcurveto{\pgfqpoint{1.917135in}{1.469928in}}{\pgfqpoint{1.912745in}{1.459329in}}{\pgfqpoint{1.912745in}{1.448279in}}%
\pgfpathcurveto{\pgfqpoint{1.912745in}{1.437229in}}{\pgfqpoint{1.917135in}{1.426630in}}{\pgfqpoint{1.924949in}{1.418817in}}%
\pgfpathcurveto{\pgfqpoint{1.932763in}{1.411003in}}{\pgfqpoint{1.943362in}{1.406613in}}{\pgfqpoint{1.954412in}{1.406613in}}%
\pgfpathclose%
\pgfusepath{stroke,fill}%
\end{pgfscope}%
\begin{pgfscope}%
\pgfpathrectangle{\pgfqpoint{1.115200in}{0.528000in}}{\pgfqpoint{4.329600in}{3.696000in}} %
\pgfusepath{clip}%
\pgfsetbuttcap%
\pgfsetroundjoin%
\definecolor{currentfill}{rgb}{0.000000,0.000000,1.000000}%
\pgfsetfillcolor{currentfill}%
\pgfsetlinewidth{1.003750pt}%
\definecolor{currentstroke}{rgb}{0.000000,0.000000,1.000000}%
\pgfsetstrokecolor{currentstroke}%
\pgfsetdash{}{0pt}%
\pgfpathmoveto{\pgfqpoint{2.364207in}{1.613127in}}%
\pgfpathcurveto{\pgfqpoint{2.375257in}{1.613127in}}{\pgfqpoint{2.385856in}{1.617517in}}{\pgfqpoint{2.393670in}{1.625331in}}%
\pgfpathcurveto{\pgfqpoint{2.401484in}{1.633144in}}{\pgfqpoint{2.405874in}{1.643743in}}{\pgfqpoint{2.405874in}{1.654793in}}%
\pgfpathcurveto{\pgfqpoint{2.405874in}{1.665844in}}{\pgfqpoint{2.401484in}{1.676443in}}{\pgfqpoint{2.393670in}{1.684256in}}%
\pgfpathcurveto{\pgfqpoint{2.385856in}{1.692070in}}{\pgfqpoint{2.375257in}{1.696460in}}{\pgfqpoint{2.364207in}{1.696460in}}%
\pgfpathcurveto{\pgfqpoint{2.353157in}{1.696460in}}{\pgfqpoint{2.342558in}{1.692070in}}{\pgfqpoint{2.334745in}{1.684256in}}%
\pgfpathcurveto{\pgfqpoint{2.326931in}{1.676443in}}{\pgfqpoint{2.322541in}{1.665844in}}{\pgfqpoint{2.322541in}{1.654793in}}%
\pgfpathcurveto{\pgfqpoint{2.322541in}{1.643743in}}{\pgfqpoint{2.326931in}{1.633144in}}{\pgfqpoint{2.334745in}{1.625331in}}%
\pgfpathcurveto{\pgfqpoint{2.342558in}{1.617517in}}{\pgfqpoint{2.353157in}{1.613127in}}{\pgfqpoint{2.364207in}{1.613127in}}%
\pgfpathclose%
\pgfusepath{stroke,fill}%
\end{pgfscope}%
\begin{pgfscope}%
\pgfpathrectangle{\pgfqpoint{1.115200in}{0.528000in}}{\pgfqpoint{4.329600in}{3.696000in}} %
\pgfusepath{clip}%
\pgfsetbuttcap%
\pgfsetroundjoin%
\definecolor{currentfill}{rgb}{0.000000,0.000000,1.000000}%
\pgfsetfillcolor{currentfill}%
\pgfsetlinewidth{1.003750pt}%
\definecolor{currentstroke}{rgb}{0.000000,0.000000,1.000000}%
\pgfsetstrokecolor{currentstroke}%
\pgfsetdash{}{0pt}%
\pgfpathmoveto{\pgfqpoint{2.035372in}{1.381613in}}%
\pgfpathcurveto{\pgfqpoint{2.046422in}{1.381613in}}{\pgfqpoint{2.057021in}{1.386003in}}{\pgfqpoint{2.064835in}{1.393817in}}%
\pgfpathcurveto{\pgfqpoint{2.072648in}{1.401631in}}{\pgfqpoint{2.077038in}{1.412230in}}{\pgfqpoint{2.077038in}{1.423280in}}%
\pgfpathcurveto{\pgfqpoint{2.077038in}{1.434330in}}{\pgfqpoint{2.072648in}{1.444929in}}{\pgfqpoint{2.064835in}{1.452743in}}%
\pgfpathcurveto{\pgfqpoint{2.057021in}{1.460556in}}{\pgfqpoint{2.046422in}{1.464946in}}{\pgfqpoint{2.035372in}{1.464946in}}%
\pgfpathcurveto{\pgfqpoint{2.024322in}{1.464946in}}{\pgfqpoint{2.013723in}{1.460556in}}{\pgfqpoint{2.005909in}{1.452743in}}%
\pgfpathcurveto{\pgfqpoint{1.998095in}{1.444929in}}{\pgfqpoint{1.993705in}{1.434330in}}{\pgfqpoint{1.993705in}{1.423280in}}%
\pgfpathcurveto{\pgfqpoint{1.993705in}{1.412230in}}{\pgfqpoint{1.998095in}{1.401631in}}{\pgfqpoint{2.005909in}{1.393817in}}%
\pgfpathcurveto{\pgfqpoint{2.013723in}{1.386003in}}{\pgfqpoint{2.024322in}{1.381613in}}{\pgfqpoint{2.035372in}{1.381613in}}%
\pgfpathclose%
\pgfusepath{stroke,fill}%
\end{pgfscope}%
\begin{pgfscope}%
\pgfpathrectangle{\pgfqpoint{1.115200in}{0.528000in}}{\pgfqpoint{4.329600in}{3.696000in}} %
\pgfusepath{clip}%
\pgfsetbuttcap%
\pgfsetroundjoin%
\definecolor{currentfill}{rgb}{0.000000,0.000000,1.000000}%
\pgfsetfillcolor{currentfill}%
\pgfsetlinewidth{1.003750pt}%
\definecolor{currentstroke}{rgb}{0.000000,0.000000,1.000000}%
\pgfsetstrokecolor{currentstroke}%
\pgfsetdash{}{0pt}%
\pgfpathmoveto{\pgfqpoint{2.082263in}{1.204573in}}%
\pgfpathcurveto{\pgfqpoint{2.093313in}{1.204573in}}{\pgfqpoint{2.103912in}{1.208964in}}{\pgfqpoint{2.111726in}{1.216777in}}%
\pgfpathcurveto{\pgfqpoint{2.119539in}{1.224591in}}{\pgfqpoint{2.123930in}{1.235190in}}{\pgfqpoint{2.123930in}{1.246240in}}%
\pgfpathcurveto{\pgfqpoint{2.123930in}{1.257290in}}{\pgfqpoint{2.119539in}{1.267889in}}{\pgfqpoint{2.111726in}{1.275703in}}%
\pgfpathcurveto{\pgfqpoint{2.103912in}{1.283516in}}{\pgfqpoint{2.093313in}{1.287907in}}{\pgfqpoint{2.082263in}{1.287907in}}%
\pgfpathcurveto{\pgfqpoint{2.071213in}{1.287907in}}{\pgfqpoint{2.060614in}{1.283516in}}{\pgfqpoint{2.052800in}{1.275703in}}%
\pgfpathcurveto{\pgfqpoint{2.044987in}{1.267889in}}{\pgfqpoint{2.040596in}{1.257290in}}{\pgfqpoint{2.040596in}{1.246240in}}%
\pgfpathcurveto{\pgfqpoint{2.040596in}{1.235190in}}{\pgfqpoint{2.044987in}{1.224591in}}{\pgfqpoint{2.052800in}{1.216777in}}%
\pgfpathcurveto{\pgfqpoint{2.060614in}{1.208964in}}{\pgfqpoint{2.071213in}{1.204573in}}{\pgfqpoint{2.082263in}{1.204573in}}%
\pgfpathclose%
\pgfusepath{stroke,fill}%
\end{pgfscope}%
\begin{pgfscope}%
\pgfpathrectangle{\pgfqpoint{1.115200in}{0.528000in}}{\pgfqpoint{4.329600in}{3.696000in}} %
\pgfusepath{clip}%
\pgfsetbuttcap%
\pgfsetroundjoin%
\definecolor{currentfill}{rgb}{0.000000,0.000000,1.000000}%
\pgfsetfillcolor{currentfill}%
\pgfsetlinewidth{1.003750pt}%
\definecolor{currentstroke}{rgb}{0.000000,0.000000,1.000000}%
\pgfsetstrokecolor{currentstroke}%
\pgfsetdash{}{0pt}%
\pgfpathmoveto{\pgfqpoint{2.007490in}{1.569873in}}%
\pgfpathcurveto{\pgfqpoint{2.018541in}{1.569873in}}{\pgfqpoint{2.029140in}{1.574264in}}{\pgfqpoint{2.036953in}{1.582077in}}%
\pgfpathcurveto{\pgfqpoint{2.044767in}{1.589891in}}{\pgfqpoint{2.049157in}{1.600490in}}{\pgfqpoint{2.049157in}{1.611540in}}%
\pgfpathcurveto{\pgfqpoint{2.049157in}{1.622590in}}{\pgfqpoint{2.044767in}{1.633189in}}{\pgfqpoint{2.036953in}{1.641003in}}%
\pgfpathcurveto{\pgfqpoint{2.029140in}{1.648816in}}{\pgfqpoint{2.018541in}{1.653207in}}{\pgfqpoint{2.007490in}{1.653207in}}%
\pgfpathcurveto{\pgfqpoint{1.996440in}{1.653207in}}{\pgfqpoint{1.985841in}{1.648816in}}{\pgfqpoint{1.978028in}{1.641003in}}%
\pgfpathcurveto{\pgfqpoint{1.970214in}{1.633189in}}{\pgfqpoint{1.965824in}{1.622590in}}{\pgfqpoint{1.965824in}{1.611540in}}%
\pgfpathcurveto{\pgfqpoint{1.965824in}{1.600490in}}{\pgfqpoint{1.970214in}{1.589891in}}{\pgfqpoint{1.978028in}{1.582077in}}%
\pgfpathcurveto{\pgfqpoint{1.985841in}{1.574264in}}{\pgfqpoint{1.996440in}{1.569873in}}{\pgfqpoint{2.007490in}{1.569873in}}%
\pgfpathclose%
\pgfusepath{stroke,fill}%
\end{pgfscope}%
\begin{pgfscope}%
\pgfpathrectangle{\pgfqpoint{1.115200in}{0.528000in}}{\pgfqpoint{4.329600in}{3.696000in}} %
\pgfusepath{clip}%
\pgfsetbuttcap%
\pgfsetroundjoin%
\definecolor{currentfill}{rgb}{0.000000,0.000000,1.000000}%
\pgfsetfillcolor{currentfill}%
\pgfsetlinewidth{1.003750pt}%
\definecolor{currentstroke}{rgb}{0.000000,0.000000,1.000000}%
\pgfsetstrokecolor{currentstroke}%
\pgfsetdash{}{0pt}%
\pgfpathmoveto{\pgfqpoint{2.228029in}{1.693004in}}%
\pgfpathcurveto{\pgfqpoint{2.239079in}{1.693004in}}{\pgfqpoint{2.249678in}{1.697394in}}{\pgfqpoint{2.257492in}{1.705207in}}%
\pgfpathcurveto{\pgfqpoint{2.265306in}{1.713021in}}{\pgfqpoint{2.269696in}{1.723620in}}{\pgfqpoint{2.269696in}{1.734670in}}%
\pgfpathcurveto{\pgfqpoint{2.269696in}{1.745720in}}{\pgfqpoint{2.265306in}{1.756319in}}{\pgfqpoint{2.257492in}{1.764133in}}%
\pgfpathcurveto{\pgfqpoint{2.249678in}{1.771947in}}{\pgfqpoint{2.239079in}{1.776337in}}{\pgfqpoint{2.228029in}{1.776337in}}%
\pgfpathcurveto{\pgfqpoint{2.216979in}{1.776337in}}{\pgfqpoint{2.206380in}{1.771947in}}{\pgfqpoint{2.198567in}{1.764133in}}%
\pgfpathcurveto{\pgfqpoint{2.190753in}{1.756319in}}{\pgfqpoint{2.186363in}{1.745720in}}{\pgfqpoint{2.186363in}{1.734670in}}%
\pgfpathcurveto{\pgfqpoint{2.186363in}{1.723620in}}{\pgfqpoint{2.190753in}{1.713021in}}{\pgfqpoint{2.198567in}{1.705207in}}%
\pgfpathcurveto{\pgfqpoint{2.206380in}{1.697394in}}{\pgfqpoint{2.216979in}{1.693004in}}{\pgfqpoint{2.228029in}{1.693004in}}%
\pgfpathclose%
\pgfusepath{stroke,fill}%
\end{pgfscope}%
\begin{pgfscope}%
\pgfpathrectangle{\pgfqpoint{1.115200in}{0.528000in}}{\pgfqpoint{4.329600in}{3.696000in}} %
\pgfusepath{clip}%
\pgfsetbuttcap%
\pgfsetroundjoin%
\definecolor{currentfill}{rgb}{0.000000,0.000000,1.000000}%
\pgfsetfillcolor{currentfill}%
\pgfsetlinewidth{1.003750pt}%
\definecolor{currentstroke}{rgb}{0.000000,0.000000,1.000000}%
\pgfsetstrokecolor{currentstroke}%
\pgfsetdash{}{0pt}%
\pgfpathmoveto{\pgfqpoint{2.341271in}{1.612330in}}%
\pgfpathcurveto{\pgfqpoint{2.352321in}{1.612330in}}{\pgfqpoint{2.362920in}{1.616720in}}{\pgfqpoint{2.370734in}{1.624534in}}%
\pgfpathcurveto{\pgfqpoint{2.378548in}{1.632347in}}{\pgfqpoint{2.382938in}{1.642946in}}{\pgfqpoint{2.382938in}{1.653997in}}%
\pgfpathcurveto{\pgfqpoint{2.382938in}{1.665047in}}{\pgfqpoint{2.378548in}{1.675646in}}{\pgfqpoint{2.370734in}{1.683459in}}%
\pgfpathcurveto{\pgfqpoint{2.362920in}{1.691273in}}{\pgfqpoint{2.352321in}{1.695663in}}{\pgfqpoint{2.341271in}{1.695663in}}%
\pgfpathcurveto{\pgfqpoint{2.330221in}{1.695663in}}{\pgfqpoint{2.319622in}{1.691273in}}{\pgfqpoint{2.311808in}{1.683459in}}%
\pgfpathcurveto{\pgfqpoint{2.303995in}{1.675646in}}{\pgfqpoint{2.299604in}{1.665047in}}{\pgfqpoint{2.299604in}{1.653997in}}%
\pgfpathcurveto{\pgfqpoint{2.299604in}{1.642946in}}{\pgfqpoint{2.303995in}{1.632347in}}{\pgfqpoint{2.311808in}{1.624534in}}%
\pgfpathcurveto{\pgfqpoint{2.319622in}{1.616720in}}{\pgfqpoint{2.330221in}{1.612330in}}{\pgfqpoint{2.341271in}{1.612330in}}%
\pgfpathclose%
\pgfusepath{stroke,fill}%
\end{pgfscope}%
\begin{pgfscope}%
\pgfpathrectangle{\pgfqpoint{1.115200in}{0.528000in}}{\pgfqpoint{4.329600in}{3.696000in}} %
\pgfusepath{clip}%
\pgfsetbuttcap%
\pgfsetroundjoin%
\definecolor{currentfill}{rgb}{0.000000,0.000000,1.000000}%
\pgfsetfillcolor{currentfill}%
\pgfsetlinewidth{1.003750pt}%
\definecolor{currentstroke}{rgb}{0.000000,0.000000,1.000000}%
\pgfsetstrokecolor{currentstroke}%
\pgfsetdash{}{0pt}%
\pgfpathmoveto{\pgfqpoint{2.111720in}{1.775026in}}%
\pgfpathcurveto{\pgfqpoint{2.122770in}{1.775026in}}{\pgfqpoint{2.133369in}{1.779416in}}{\pgfqpoint{2.141183in}{1.787230in}}%
\pgfpathcurveto{\pgfqpoint{2.148996in}{1.795043in}}{\pgfqpoint{2.153387in}{1.805642in}}{\pgfqpoint{2.153387in}{1.816692in}}%
\pgfpathcurveto{\pgfqpoint{2.153387in}{1.827742in}}{\pgfqpoint{2.148996in}{1.838342in}}{\pgfqpoint{2.141183in}{1.846155in}}%
\pgfpathcurveto{\pgfqpoint{2.133369in}{1.853969in}}{\pgfqpoint{2.122770in}{1.858359in}}{\pgfqpoint{2.111720in}{1.858359in}}%
\pgfpathcurveto{\pgfqpoint{2.100670in}{1.858359in}}{\pgfqpoint{2.090071in}{1.853969in}}{\pgfqpoint{2.082257in}{1.846155in}}%
\pgfpathcurveto{\pgfqpoint{2.074444in}{1.838342in}}{\pgfqpoint{2.070053in}{1.827742in}}{\pgfqpoint{2.070053in}{1.816692in}}%
\pgfpathcurveto{\pgfqpoint{2.070053in}{1.805642in}}{\pgfqpoint{2.074444in}{1.795043in}}{\pgfqpoint{2.082257in}{1.787230in}}%
\pgfpathcurveto{\pgfqpoint{2.090071in}{1.779416in}}{\pgfqpoint{2.100670in}{1.775026in}}{\pgfqpoint{2.111720in}{1.775026in}}%
\pgfpathclose%
\pgfusepath{stroke,fill}%
\end{pgfscope}%
\begin{pgfscope}%
\pgfpathrectangle{\pgfqpoint{1.115200in}{0.528000in}}{\pgfqpoint{4.329600in}{3.696000in}} %
\pgfusepath{clip}%
\pgfsetbuttcap%
\pgfsetroundjoin%
\definecolor{currentfill}{rgb}{0.000000,0.000000,1.000000}%
\pgfsetfillcolor{currentfill}%
\pgfsetlinewidth{1.003750pt}%
\definecolor{currentstroke}{rgb}{0.000000,0.000000,1.000000}%
\pgfsetstrokecolor{currentstroke}%
\pgfsetdash{}{0pt}%
\pgfpathmoveto{\pgfqpoint{2.233985in}{1.819457in}}%
\pgfpathcurveto{\pgfqpoint{2.245035in}{1.819457in}}{\pgfqpoint{2.255634in}{1.823847in}}{\pgfqpoint{2.263447in}{1.831661in}}%
\pgfpathcurveto{\pgfqpoint{2.271261in}{1.839474in}}{\pgfqpoint{2.275651in}{1.850073in}}{\pgfqpoint{2.275651in}{1.861123in}}%
\pgfpathcurveto{\pgfqpoint{2.275651in}{1.872174in}}{\pgfqpoint{2.271261in}{1.882773in}}{\pgfqpoint{2.263447in}{1.890586in}}%
\pgfpathcurveto{\pgfqpoint{2.255634in}{1.898400in}}{\pgfqpoint{2.245035in}{1.902790in}}{\pgfqpoint{2.233985in}{1.902790in}}%
\pgfpathcurveto{\pgfqpoint{2.222934in}{1.902790in}}{\pgfqpoint{2.212335in}{1.898400in}}{\pgfqpoint{2.204522in}{1.890586in}}%
\pgfpathcurveto{\pgfqpoint{2.196708in}{1.882773in}}{\pgfqpoint{2.192318in}{1.872174in}}{\pgfqpoint{2.192318in}{1.861123in}}%
\pgfpathcurveto{\pgfqpoint{2.192318in}{1.850073in}}{\pgfqpoint{2.196708in}{1.839474in}}{\pgfqpoint{2.204522in}{1.831661in}}%
\pgfpathcurveto{\pgfqpoint{2.212335in}{1.823847in}}{\pgfqpoint{2.222934in}{1.819457in}}{\pgfqpoint{2.233985in}{1.819457in}}%
\pgfpathclose%
\pgfusepath{stroke,fill}%
\end{pgfscope}%
\begin{pgfscope}%
\pgfpathrectangle{\pgfqpoint{1.115200in}{0.528000in}}{\pgfqpoint{4.329600in}{3.696000in}} %
\pgfusepath{clip}%
\pgfsetbuttcap%
\pgfsetroundjoin%
\definecolor{currentfill}{rgb}{0.000000,0.000000,1.000000}%
\pgfsetfillcolor{currentfill}%
\pgfsetlinewidth{1.003750pt}%
\definecolor{currentstroke}{rgb}{0.000000,0.000000,1.000000}%
\pgfsetstrokecolor{currentstroke}%
\pgfsetdash{}{0pt}%
\pgfpathmoveto{\pgfqpoint{2.096523in}{1.598333in}}%
\pgfpathcurveto{\pgfqpoint{2.107573in}{1.598333in}}{\pgfqpoint{2.118172in}{1.602723in}}{\pgfqpoint{2.125986in}{1.610536in}}%
\pgfpathcurveto{\pgfqpoint{2.133799in}{1.618350in}}{\pgfqpoint{2.138190in}{1.628949in}}{\pgfqpoint{2.138190in}{1.639999in}}%
\pgfpathcurveto{\pgfqpoint{2.138190in}{1.651049in}}{\pgfqpoint{2.133799in}{1.661648in}}{\pgfqpoint{2.125986in}{1.669462in}}%
\pgfpathcurveto{\pgfqpoint{2.118172in}{1.677276in}}{\pgfqpoint{2.107573in}{1.681666in}}{\pgfqpoint{2.096523in}{1.681666in}}%
\pgfpathcurveto{\pgfqpoint{2.085473in}{1.681666in}}{\pgfqpoint{2.074874in}{1.677276in}}{\pgfqpoint{2.067060in}{1.669462in}}%
\pgfpathcurveto{\pgfqpoint{2.059247in}{1.661648in}}{\pgfqpoint{2.054856in}{1.651049in}}{\pgfqpoint{2.054856in}{1.639999in}}%
\pgfpathcurveto{\pgfqpoint{2.054856in}{1.628949in}}{\pgfqpoint{2.059247in}{1.618350in}}{\pgfqpoint{2.067060in}{1.610536in}}%
\pgfpathcurveto{\pgfqpoint{2.074874in}{1.602723in}}{\pgfqpoint{2.085473in}{1.598333in}}{\pgfqpoint{2.096523in}{1.598333in}}%
\pgfpathclose%
\pgfusepath{stroke,fill}%
\end{pgfscope}%
\begin{pgfscope}%
\pgfpathrectangle{\pgfqpoint{1.115200in}{0.528000in}}{\pgfqpoint{4.329600in}{3.696000in}} %
\pgfusepath{clip}%
\pgfsetbuttcap%
\pgfsetroundjoin%
\definecolor{currentfill}{rgb}{0.000000,0.000000,1.000000}%
\pgfsetfillcolor{currentfill}%
\pgfsetlinewidth{1.003750pt}%
\definecolor{currentstroke}{rgb}{0.000000,0.000000,1.000000}%
\pgfsetstrokecolor{currentstroke}%
\pgfsetdash{}{0pt}%
\pgfpathmoveto{\pgfqpoint{2.541319in}{1.440429in}}%
\pgfpathcurveto{\pgfqpoint{2.552369in}{1.440429in}}{\pgfqpoint{2.562968in}{1.444819in}}{\pgfqpoint{2.570781in}{1.452633in}}%
\pgfpathcurveto{\pgfqpoint{2.578595in}{1.460446in}}{\pgfqpoint{2.582985in}{1.471045in}}{\pgfqpoint{2.582985in}{1.482095in}}%
\pgfpathcurveto{\pgfqpoint{2.582985in}{1.493146in}}{\pgfqpoint{2.578595in}{1.503745in}}{\pgfqpoint{2.570781in}{1.511558in}}%
\pgfpathcurveto{\pgfqpoint{2.562968in}{1.519372in}}{\pgfqpoint{2.552369in}{1.523762in}}{\pgfqpoint{2.541319in}{1.523762in}}%
\pgfpathcurveto{\pgfqpoint{2.530269in}{1.523762in}}{\pgfqpoint{2.519670in}{1.519372in}}{\pgfqpoint{2.511856in}{1.511558in}}%
\pgfpathcurveto{\pgfqpoint{2.504042in}{1.503745in}}{\pgfqpoint{2.499652in}{1.493146in}}{\pgfqpoint{2.499652in}{1.482095in}}%
\pgfpathcurveto{\pgfqpoint{2.499652in}{1.471045in}}{\pgfqpoint{2.504042in}{1.460446in}}{\pgfqpoint{2.511856in}{1.452633in}}%
\pgfpathcurveto{\pgfqpoint{2.519670in}{1.444819in}}{\pgfqpoint{2.530269in}{1.440429in}}{\pgfqpoint{2.541319in}{1.440429in}}%
\pgfpathclose%
\pgfusepath{stroke,fill}%
\end{pgfscope}%
\begin{pgfscope}%
\pgfpathrectangle{\pgfqpoint{1.115200in}{0.528000in}}{\pgfqpoint{4.329600in}{3.696000in}} %
\pgfusepath{clip}%
\pgfsetbuttcap%
\pgfsetmiterjoin%
\pgfsetlinewidth{1.003750pt}%
\definecolor{currentstroke}{rgb}{0.000000,0.500000,0.000000}%
\pgfsetstrokecolor{currentstroke}%
\pgfsetdash{{3.700000pt}{1.600000pt}}{0.000000pt}%
\pgfpathmoveto{\pgfqpoint{3.839363in}{2.811342in}}%
\pgfpathcurveto{\pgfqpoint{3.992603in}{2.811342in}}{\pgfqpoint{4.139588in}{2.872225in}}{\pgfqpoint{4.247945in}{2.980582in}}%
\pgfpathcurveto{\pgfqpoint{4.356303in}{3.088939in}}{\pgfqpoint{4.417186in}{3.235924in}}{\pgfqpoint{4.417186in}{3.389165in}}%
\pgfpathcurveto{\pgfqpoint{4.417186in}{3.542405in}}{\pgfqpoint{4.356303in}{3.689390in}}{\pgfqpoint{4.247945in}{3.797747in}}%
\pgfpathcurveto{\pgfqpoint{4.139588in}{3.906104in}}{\pgfqpoint{3.992603in}{3.966987in}}{\pgfqpoint{3.839363in}{3.966987in}}%
\pgfpathcurveto{\pgfqpoint{3.686122in}{3.966987in}}{\pgfqpoint{3.539138in}{3.906104in}}{\pgfqpoint{3.430780in}{3.797747in}}%
\pgfpathcurveto{\pgfqpoint{3.322423in}{3.689390in}}{\pgfqpoint{3.261540in}{3.542405in}}{\pgfqpoint{3.261540in}{3.389165in}}%
\pgfpathcurveto{\pgfqpoint{3.261540in}{3.235924in}}{\pgfqpoint{3.322423in}{3.088939in}}{\pgfqpoint{3.430780in}{2.980582in}}%
\pgfpathcurveto{\pgfqpoint{3.539138in}{2.872225in}}{\pgfqpoint{3.686122in}{2.811342in}}{\pgfqpoint{3.839363in}{2.811342in}}%
\pgfpathclose%
\pgfusepath{stroke}%
\end{pgfscope}%
\begin{pgfscope}%
\pgfpathrectangle{\pgfqpoint{1.115200in}{0.528000in}}{\pgfqpoint{4.329600in}{3.696000in}} %
\pgfusepath{clip}%
\pgfsetbuttcap%
\pgfsetmiterjoin%
\pgfsetlinewidth{1.003750pt}%
\definecolor{currentstroke}{rgb}{1.000000,0.000000,0.000000}%
\pgfsetstrokecolor{currentstroke}%
\pgfsetdash{{3.700000pt}{1.600000pt}}{0.000000pt}%
\pgfpathmoveto{\pgfqpoint{4.680009in}{1.290583in}}%
\pgfpathcurveto{\pgfqpoint{4.851604in}{1.290583in}}{\pgfqpoint{5.016194in}{1.358759in}}{\pgfqpoint{5.137530in}{1.480095in}}%
\pgfpathcurveto{\pgfqpoint{5.258866in}{1.601431in}}{\pgfqpoint{5.327042in}{1.766021in}}{\pgfqpoint{5.327042in}{1.937616in}}%
\pgfpathcurveto{\pgfqpoint{5.327042in}{2.109211in}}{\pgfqpoint{5.258866in}{2.273801in}}{\pgfqpoint{5.137530in}{2.395137in}}%
\pgfpathcurveto{\pgfqpoint{5.016194in}{2.516473in}}{\pgfqpoint{4.851604in}{2.584648in}}{\pgfqpoint{4.680009in}{2.584648in}}%
\pgfpathcurveto{\pgfqpoint{4.508414in}{2.584648in}}{\pgfqpoint{4.343824in}{2.516473in}}{\pgfqpoint{4.222488in}{2.395137in}}%
\pgfpathcurveto{\pgfqpoint{4.101152in}{2.273801in}}{\pgfqpoint{4.032977in}{2.109211in}}{\pgfqpoint{4.032977in}{1.937616in}}%
\pgfpathcurveto{\pgfqpoint{4.032977in}{1.766021in}}{\pgfqpoint{4.101152in}{1.601431in}}{\pgfqpoint{4.222488in}{1.480095in}}%
\pgfpathcurveto{\pgfqpoint{4.343824in}{1.358759in}}{\pgfqpoint{4.508414in}{1.290583in}}{\pgfqpoint{4.680009in}{1.290583in}}%
\pgfpathclose%
\pgfusepath{stroke}%
\end{pgfscope}%
\begin{pgfscope}%
\pgfpathrectangle{\pgfqpoint{1.115200in}{0.528000in}}{\pgfqpoint{4.329600in}{3.696000in}} %
\pgfusepath{clip}%
\pgfsetbuttcap%
\pgfsetmiterjoin%
\pgfsetlinewidth{1.003750pt}%
\definecolor{currentstroke}{rgb}{0.000000,0.000000,1.000000}%
\pgfsetstrokecolor{currentstroke}%
\pgfsetdash{{3.700000pt}{1.600000pt}}{0.000000pt}%
\pgfpathmoveto{\pgfqpoint{2.158818in}{1.136556in}}%
\pgfpathcurveto{\pgfqpoint{2.285864in}{1.136556in}}{\pgfqpoint{2.407724in}{1.187032in}}{\pgfqpoint{2.497559in}{1.276867in}}%
\pgfpathcurveto{\pgfqpoint{2.587395in}{1.366702in}}{\pgfqpoint{2.637871in}{1.488562in}}{\pgfqpoint{2.637871in}{1.615608in}}%
\pgfpathcurveto{\pgfqpoint{2.637871in}{1.742655in}}{\pgfqpoint{2.587395in}{1.864515in}}{\pgfqpoint{2.497559in}{1.954350in}}%
\pgfpathcurveto{\pgfqpoint{2.407724in}{2.044185in}}{\pgfqpoint{2.285864in}{2.094661in}}{\pgfqpoint{2.158818in}{2.094661in}}%
\pgfpathcurveto{\pgfqpoint{2.031772in}{2.094661in}}{\pgfqpoint{1.909912in}{2.044185in}}{\pgfqpoint{1.820077in}{1.954350in}}%
\pgfpathcurveto{\pgfqpoint{1.730242in}{1.864515in}}{\pgfqpoint{1.679766in}{1.742655in}}{\pgfqpoint{1.679766in}{1.615608in}}%
\pgfpathcurveto{\pgfqpoint{1.679766in}{1.488562in}}{\pgfqpoint{1.730242in}{1.366702in}}{\pgfqpoint{1.820077in}{1.276867in}}%
\pgfpathcurveto{\pgfqpoint{1.909912in}{1.187032in}}{\pgfqpoint{2.031772in}{1.136556in}}{\pgfqpoint{2.158818in}{1.136556in}}%
\pgfpathclose%
\pgfusepath{stroke}%
\end{pgfscope}%
\begin{pgfscope}%
\pgfsetbuttcap%
\pgfsetroundjoin%
\definecolor{currentfill}{rgb}{0.000000,0.000000,0.000000}%
\pgfsetfillcolor{currentfill}%
\pgfsetlinewidth{0.803000pt}%
\definecolor{currentstroke}{rgb}{0.000000,0.000000,0.000000}%
\pgfsetstrokecolor{currentstroke}%
\pgfsetdash{}{0pt}%
\pgfsys@defobject{currentmarker}{\pgfqpoint{0.000000in}{-0.048611in}}{\pgfqpoint{0.000000in}{0.000000in}}{%
\pgfpathmoveto{\pgfqpoint{0.000000in}{0.000000in}}%
\pgfpathlineto{\pgfqpoint{0.000000in}{-0.048611in}}%
\pgfusepath{stroke,fill}%
}%
\begin{pgfscope}%
\pgfsys@transformshift{1.115200in}{0.528000in}%
\pgfsys@useobject{currentmarker}{}%
\end{pgfscope}%
\end{pgfscope}%
\begin{pgfscope}%
\pgftext[x=1.115200in,y=0.430778in,,top]{\sffamily\fontsize{10.000000}{12.000000}\selectfont −2.0}%
\end{pgfscope}%
\begin{pgfscope}%
\pgfsetbuttcap%
\pgfsetroundjoin%
\definecolor{currentfill}{rgb}{0.000000,0.000000,0.000000}%
\pgfsetfillcolor{currentfill}%
\pgfsetlinewidth{0.803000pt}%
\definecolor{currentstroke}{rgb}{0.000000,0.000000,0.000000}%
\pgfsetstrokecolor{currentstroke}%
\pgfsetdash{}{0pt}%
\pgfsys@defobject{currentmarker}{\pgfqpoint{0.000000in}{-0.048611in}}{\pgfqpoint{0.000000in}{0.000000in}}{%
\pgfpathmoveto{\pgfqpoint{0.000000in}{0.000000in}}%
\pgfpathlineto{\pgfqpoint{0.000000in}{-0.048611in}}%
\pgfusepath{stroke,fill}%
}%
\begin{pgfscope}%
\pgfsys@transformshift{1.643200in}{0.528000in}%
\pgfsys@useobject{currentmarker}{}%
\end{pgfscope}%
\end{pgfscope}%
\begin{pgfscope}%
\pgftext[x=1.643200in,y=0.430778in,,top]{\sffamily\fontsize{10.000000}{12.000000}\selectfont −1.5}%
\end{pgfscope}%
\begin{pgfscope}%
\pgfsetbuttcap%
\pgfsetroundjoin%
\definecolor{currentfill}{rgb}{0.000000,0.000000,0.000000}%
\pgfsetfillcolor{currentfill}%
\pgfsetlinewidth{0.803000pt}%
\definecolor{currentstroke}{rgb}{0.000000,0.000000,0.000000}%
\pgfsetstrokecolor{currentstroke}%
\pgfsetdash{}{0pt}%
\pgfsys@defobject{currentmarker}{\pgfqpoint{0.000000in}{-0.048611in}}{\pgfqpoint{0.000000in}{0.000000in}}{%
\pgfpathmoveto{\pgfqpoint{0.000000in}{0.000000in}}%
\pgfpathlineto{\pgfqpoint{0.000000in}{-0.048611in}}%
\pgfusepath{stroke,fill}%
}%
\begin{pgfscope}%
\pgfsys@transformshift{2.171200in}{0.528000in}%
\pgfsys@useobject{currentmarker}{}%
\end{pgfscope}%
\end{pgfscope}%
\begin{pgfscope}%
\pgftext[x=2.171200in,y=0.430778in,,top]{\sffamily\fontsize{10.000000}{12.000000}\selectfont −1.0}%
\end{pgfscope}%
\begin{pgfscope}%
\pgfsetbuttcap%
\pgfsetroundjoin%
\definecolor{currentfill}{rgb}{0.000000,0.000000,0.000000}%
\pgfsetfillcolor{currentfill}%
\pgfsetlinewidth{0.803000pt}%
\definecolor{currentstroke}{rgb}{0.000000,0.000000,0.000000}%
\pgfsetstrokecolor{currentstroke}%
\pgfsetdash{}{0pt}%
\pgfsys@defobject{currentmarker}{\pgfqpoint{0.000000in}{-0.048611in}}{\pgfqpoint{0.000000in}{0.000000in}}{%
\pgfpathmoveto{\pgfqpoint{0.000000in}{0.000000in}}%
\pgfpathlineto{\pgfqpoint{0.000000in}{-0.048611in}}%
\pgfusepath{stroke,fill}%
}%
\begin{pgfscope}%
\pgfsys@transformshift{2.699200in}{0.528000in}%
\pgfsys@useobject{currentmarker}{}%
\end{pgfscope}%
\end{pgfscope}%
\begin{pgfscope}%
\pgftext[x=2.699200in,y=0.430778in,,top]{\sffamily\fontsize{10.000000}{12.000000}\selectfont −0.5}%
\end{pgfscope}%
\begin{pgfscope}%
\pgfsetbuttcap%
\pgfsetroundjoin%
\definecolor{currentfill}{rgb}{0.000000,0.000000,0.000000}%
\pgfsetfillcolor{currentfill}%
\pgfsetlinewidth{0.803000pt}%
\definecolor{currentstroke}{rgb}{0.000000,0.000000,0.000000}%
\pgfsetstrokecolor{currentstroke}%
\pgfsetdash{}{0pt}%
\pgfsys@defobject{currentmarker}{\pgfqpoint{0.000000in}{-0.048611in}}{\pgfqpoint{0.000000in}{0.000000in}}{%
\pgfpathmoveto{\pgfqpoint{0.000000in}{0.000000in}}%
\pgfpathlineto{\pgfqpoint{0.000000in}{-0.048611in}}%
\pgfusepath{stroke,fill}%
}%
\begin{pgfscope}%
\pgfsys@transformshift{3.227200in}{0.528000in}%
\pgfsys@useobject{currentmarker}{}%
\end{pgfscope}%
\end{pgfscope}%
\begin{pgfscope}%
\pgftext[x=3.227200in,y=0.430778in,,top]{\sffamily\fontsize{10.000000}{12.000000}\selectfont 0.0}%
\end{pgfscope}%
\begin{pgfscope}%
\pgfsetbuttcap%
\pgfsetroundjoin%
\definecolor{currentfill}{rgb}{0.000000,0.000000,0.000000}%
\pgfsetfillcolor{currentfill}%
\pgfsetlinewidth{0.803000pt}%
\definecolor{currentstroke}{rgb}{0.000000,0.000000,0.000000}%
\pgfsetstrokecolor{currentstroke}%
\pgfsetdash{}{0pt}%
\pgfsys@defobject{currentmarker}{\pgfqpoint{0.000000in}{-0.048611in}}{\pgfqpoint{0.000000in}{0.000000in}}{%
\pgfpathmoveto{\pgfqpoint{0.000000in}{0.000000in}}%
\pgfpathlineto{\pgfqpoint{0.000000in}{-0.048611in}}%
\pgfusepath{stroke,fill}%
}%
\begin{pgfscope}%
\pgfsys@transformshift{3.755200in}{0.528000in}%
\pgfsys@useobject{currentmarker}{}%
\end{pgfscope}%
\end{pgfscope}%
\begin{pgfscope}%
\pgftext[x=3.755200in,y=0.430778in,,top]{\sffamily\fontsize{10.000000}{12.000000}\selectfont 0.5}%
\end{pgfscope}%
\begin{pgfscope}%
\pgfsetbuttcap%
\pgfsetroundjoin%
\definecolor{currentfill}{rgb}{0.000000,0.000000,0.000000}%
\pgfsetfillcolor{currentfill}%
\pgfsetlinewidth{0.803000pt}%
\definecolor{currentstroke}{rgb}{0.000000,0.000000,0.000000}%
\pgfsetstrokecolor{currentstroke}%
\pgfsetdash{}{0pt}%
\pgfsys@defobject{currentmarker}{\pgfqpoint{0.000000in}{-0.048611in}}{\pgfqpoint{0.000000in}{0.000000in}}{%
\pgfpathmoveto{\pgfqpoint{0.000000in}{0.000000in}}%
\pgfpathlineto{\pgfqpoint{0.000000in}{-0.048611in}}%
\pgfusepath{stroke,fill}%
}%
\begin{pgfscope}%
\pgfsys@transformshift{4.283200in}{0.528000in}%
\pgfsys@useobject{currentmarker}{}%
\end{pgfscope}%
\end{pgfscope}%
\begin{pgfscope}%
\pgftext[x=4.283200in,y=0.430778in,,top]{\sffamily\fontsize{10.000000}{12.000000}\selectfont 1.0}%
\end{pgfscope}%
\begin{pgfscope}%
\pgfsetbuttcap%
\pgfsetroundjoin%
\definecolor{currentfill}{rgb}{0.000000,0.000000,0.000000}%
\pgfsetfillcolor{currentfill}%
\pgfsetlinewidth{0.803000pt}%
\definecolor{currentstroke}{rgb}{0.000000,0.000000,0.000000}%
\pgfsetstrokecolor{currentstroke}%
\pgfsetdash{}{0pt}%
\pgfsys@defobject{currentmarker}{\pgfqpoint{0.000000in}{-0.048611in}}{\pgfqpoint{0.000000in}{0.000000in}}{%
\pgfpathmoveto{\pgfqpoint{0.000000in}{0.000000in}}%
\pgfpathlineto{\pgfqpoint{0.000000in}{-0.048611in}}%
\pgfusepath{stroke,fill}%
}%
\begin{pgfscope}%
\pgfsys@transformshift{4.811200in}{0.528000in}%
\pgfsys@useobject{currentmarker}{}%
\end{pgfscope}%
\end{pgfscope}%
\begin{pgfscope}%
\pgftext[x=4.811200in,y=0.430778in,,top]{\sffamily\fontsize{10.000000}{12.000000}\selectfont 1.5}%
\end{pgfscope}%
\begin{pgfscope}%
\pgfsetbuttcap%
\pgfsetroundjoin%
\definecolor{currentfill}{rgb}{0.000000,0.000000,0.000000}%
\pgfsetfillcolor{currentfill}%
\pgfsetlinewidth{0.803000pt}%
\definecolor{currentstroke}{rgb}{0.000000,0.000000,0.000000}%
\pgfsetstrokecolor{currentstroke}%
\pgfsetdash{}{0pt}%
\pgfsys@defobject{currentmarker}{\pgfqpoint{0.000000in}{-0.048611in}}{\pgfqpoint{0.000000in}{0.000000in}}{%
\pgfpathmoveto{\pgfqpoint{0.000000in}{0.000000in}}%
\pgfpathlineto{\pgfqpoint{0.000000in}{-0.048611in}}%
\pgfusepath{stroke,fill}%
}%
\begin{pgfscope}%
\pgfsys@transformshift{5.339200in}{0.528000in}%
\pgfsys@useobject{currentmarker}{}%
\end{pgfscope}%
\end{pgfscope}%
\begin{pgfscope}%
\pgftext[x=5.339200in,y=0.430778in,,top]{\sffamily\fontsize{10.000000}{12.000000}\selectfont 2.0}%
\end{pgfscope}%
\begin{pgfscope}%
\pgftext[x=3.280000in,y=0.240809in,,top]{\sffamily\fontsize{10.000000}{12.000000}\selectfont x}%
\end{pgfscope}%
\begin{pgfscope}%
\pgfsetbuttcap%
\pgfsetroundjoin%
\definecolor{currentfill}{rgb}{0.000000,0.000000,0.000000}%
\pgfsetfillcolor{currentfill}%
\pgfsetlinewidth{0.803000pt}%
\definecolor{currentstroke}{rgb}{0.000000,0.000000,0.000000}%
\pgfsetstrokecolor{currentstroke}%
\pgfsetdash{}{0pt}%
\pgfsys@defobject{currentmarker}{\pgfqpoint{-0.048611in}{0.000000in}}{\pgfqpoint{0.000000in}{0.000000in}}{%
\pgfpathmoveto{\pgfqpoint{0.000000in}{0.000000in}}%
\pgfpathlineto{\pgfqpoint{-0.048611in}{0.000000in}}%
\pgfusepath{stroke,fill}%
}%
\begin{pgfscope}%
\pgfsys@transformshift{1.115200in}{0.528000in}%
\pgfsys@useobject{currentmarker}{}%
\end{pgfscope}%
\end{pgfscope}%
\begin{pgfscope}%
\pgftext[x=0.680725in,y=0.475238in,left,base]{\sffamily\fontsize{10.000000}{12.000000}\selectfont −2.0}%
\end{pgfscope}%
\begin{pgfscope}%
\pgfsetbuttcap%
\pgfsetroundjoin%
\definecolor{currentfill}{rgb}{0.000000,0.000000,0.000000}%
\pgfsetfillcolor{currentfill}%
\pgfsetlinewidth{0.803000pt}%
\definecolor{currentstroke}{rgb}{0.000000,0.000000,0.000000}%
\pgfsetstrokecolor{currentstroke}%
\pgfsetdash{}{0pt}%
\pgfsys@defobject{currentmarker}{\pgfqpoint{-0.048611in}{0.000000in}}{\pgfqpoint{0.000000in}{0.000000in}}{%
\pgfpathmoveto{\pgfqpoint{0.000000in}{0.000000in}}%
\pgfpathlineto{\pgfqpoint{-0.048611in}{0.000000in}}%
\pgfusepath{stroke,fill}%
}%
\begin{pgfscope}%
\pgfsys@transformshift{1.115200in}{1.056000in}%
\pgfsys@useobject{currentmarker}{}%
\end{pgfscope}%
\end{pgfscope}%
\begin{pgfscope}%
\pgftext[x=0.680725in,y=1.003238in,left,base]{\sffamily\fontsize{10.000000}{12.000000}\selectfont −1.5}%
\end{pgfscope}%
\begin{pgfscope}%
\pgfsetbuttcap%
\pgfsetroundjoin%
\definecolor{currentfill}{rgb}{0.000000,0.000000,0.000000}%
\pgfsetfillcolor{currentfill}%
\pgfsetlinewidth{0.803000pt}%
\definecolor{currentstroke}{rgb}{0.000000,0.000000,0.000000}%
\pgfsetstrokecolor{currentstroke}%
\pgfsetdash{}{0pt}%
\pgfsys@defobject{currentmarker}{\pgfqpoint{-0.048611in}{0.000000in}}{\pgfqpoint{0.000000in}{0.000000in}}{%
\pgfpathmoveto{\pgfqpoint{0.000000in}{0.000000in}}%
\pgfpathlineto{\pgfqpoint{-0.048611in}{0.000000in}}%
\pgfusepath{stroke,fill}%
}%
\begin{pgfscope}%
\pgfsys@transformshift{1.115200in}{1.584000in}%
\pgfsys@useobject{currentmarker}{}%
\end{pgfscope}%
\end{pgfscope}%
\begin{pgfscope}%
\pgftext[x=0.680725in,y=1.531238in,left,base]{\sffamily\fontsize{10.000000}{12.000000}\selectfont −1.0}%
\end{pgfscope}%
\begin{pgfscope}%
\pgfsetbuttcap%
\pgfsetroundjoin%
\definecolor{currentfill}{rgb}{0.000000,0.000000,0.000000}%
\pgfsetfillcolor{currentfill}%
\pgfsetlinewidth{0.803000pt}%
\definecolor{currentstroke}{rgb}{0.000000,0.000000,0.000000}%
\pgfsetstrokecolor{currentstroke}%
\pgfsetdash{}{0pt}%
\pgfsys@defobject{currentmarker}{\pgfqpoint{-0.048611in}{0.000000in}}{\pgfqpoint{0.000000in}{0.000000in}}{%
\pgfpathmoveto{\pgfqpoint{0.000000in}{0.000000in}}%
\pgfpathlineto{\pgfqpoint{-0.048611in}{0.000000in}}%
\pgfusepath{stroke,fill}%
}%
\begin{pgfscope}%
\pgfsys@transformshift{1.115200in}{2.112000in}%
\pgfsys@useobject{currentmarker}{}%
\end{pgfscope}%
\end{pgfscope}%
\begin{pgfscope}%
\pgftext[x=0.680725in,y=2.059238in,left,base]{\sffamily\fontsize{10.000000}{12.000000}\selectfont −0.5}%
\end{pgfscope}%
\begin{pgfscope}%
\pgfsetbuttcap%
\pgfsetroundjoin%
\definecolor{currentfill}{rgb}{0.000000,0.000000,0.000000}%
\pgfsetfillcolor{currentfill}%
\pgfsetlinewidth{0.803000pt}%
\definecolor{currentstroke}{rgb}{0.000000,0.000000,0.000000}%
\pgfsetstrokecolor{currentstroke}%
\pgfsetdash{}{0pt}%
\pgfsys@defobject{currentmarker}{\pgfqpoint{-0.048611in}{0.000000in}}{\pgfqpoint{0.000000in}{0.000000in}}{%
\pgfpathmoveto{\pgfqpoint{0.000000in}{0.000000in}}%
\pgfpathlineto{\pgfqpoint{-0.048611in}{0.000000in}}%
\pgfusepath{stroke,fill}%
}%
\begin{pgfscope}%
\pgfsys@transformshift{1.115200in}{2.640000in}%
\pgfsys@useobject{currentmarker}{}%
\end{pgfscope}%
\end{pgfscope}%
\begin{pgfscope}%
\pgftext[x=0.797098in,y=2.587238in,left,base]{\sffamily\fontsize{10.000000}{12.000000}\selectfont 0.0}%
\end{pgfscope}%
\begin{pgfscope}%
\pgfsetbuttcap%
\pgfsetroundjoin%
\definecolor{currentfill}{rgb}{0.000000,0.000000,0.000000}%
\pgfsetfillcolor{currentfill}%
\pgfsetlinewidth{0.803000pt}%
\definecolor{currentstroke}{rgb}{0.000000,0.000000,0.000000}%
\pgfsetstrokecolor{currentstroke}%
\pgfsetdash{}{0pt}%
\pgfsys@defobject{currentmarker}{\pgfqpoint{-0.048611in}{0.000000in}}{\pgfqpoint{0.000000in}{0.000000in}}{%
\pgfpathmoveto{\pgfqpoint{0.000000in}{0.000000in}}%
\pgfpathlineto{\pgfqpoint{-0.048611in}{0.000000in}}%
\pgfusepath{stroke,fill}%
}%
\begin{pgfscope}%
\pgfsys@transformshift{1.115200in}{3.168000in}%
\pgfsys@useobject{currentmarker}{}%
\end{pgfscope}%
\end{pgfscope}%
\begin{pgfscope}%
\pgftext[x=0.797098in,y=3.115238in,left,base]{\sffamily\fontsize{10.000000}{12.000000}\selectfont 0.5}%
\end{pgfscope}%
\begin{pgfscope}%
\pgfsetbuttcap%
\pgfsetroundjoin%
\definecolor{currentfill}{rgb}{0.000000,0.000000,0.000000}%
\pgfsetfillcolor{currentfill}%
\pgfsetlinewidth{0.803000pt}%
\definecolor{currentstroke}{rgb}{0.000000,0.000000,0.000000}%
\pgfsetstrokecolor{currentstroke}%
\pgfsetdash{}{0pt}%
\pgfsys@defobject{currentmarker}{\pgfqpoint{-0.048611in}{0.000000in}}{\pgfqpoint{0.000000in}{0.000000in}}{%
\pgfpathmoveto{\pgfqpoint{0.000000in}{0.000000in}}%
\pgfpathlineto{\pgfqpoint{-0.048611in}{0.000000in}}%
\pgfusepath{stroke,fill}%
}%
\begin{pgfscope}%
\pgfsys@transformshift{1.115200in}{3.696000in}%
\pgfsys@useobject{currentmarker}{}%
\end{pgfscope}%
\end{pgfscope}%
\begin{pgfscope}%
\pgftext[x=0.797098in,y=3.643238in,left,base]{\sffamily\fontsize{10.000000}{12.000000}\selectfont 1.0}%
\end{pgfscope}%
\begin{pgfscope}%
\pgfsetbuttcap%
\pgfsetroundjoin%
\definecolor{currentfill}{rgb}{0.000000,0.000000,0.000000}%
\pgfsetfillcolor{currentfill}%
\pgfsetlinewidth{0.803000pt}%
\definecolor{currentstroke}{rgb}{0.000000,0.000000,0.000000}%
\pgfsetstrokecolor{currentstroke}%
\pgfsetdash{}{0pt}%
\pgfsys@defobject{currentmarker}{\pgfqpoint{-0.048611in}{0.000000in}}{\pgfqpoint{0.000000in}{0.000000in}}{%
\pgfpathmoveto{\pgfqpoint{0.000000in}{0.000000in}}%
\pgfpathlineto{\pgfqpoint{-0.048611in}{0.000000in}}%
\pgfusepath{stroke,fill}%
}%
\begin{pgfscope}%
\pgfsys@transformshift{1.115200in}{4.224000in}%
\pgfsys@useobject{currentmarker}{}%
\end{pgfscope}%
\end{pgfscope}%
\begin{pgfscope}%
\pgftext[x=0.797098in,y=4.171238in,left,base]{\sffamily\fontsize{10.000000}{12.000000}\selectfont 1.5}%
\end{pgfscope}%
\begin{pgfscope}%
\pgftext[x=0.625169in,y=2.376000in,,bottom]{\sffamily\fontsize{10.000000}{12.000000}\selectfont y}%
\end{pgfscope}%
\begin{pgfscope}%
\pgfsetrectcap%
\pgfsetmiterjoin%
\pgfsetlinewidth{0.803000pt}%
\definecolor{currentstroke}{rgb}{0.000000,0.000000,0.000000}%
\pgfsetstrokecolor{currentstroke}%
\pgfsetdash{}{0pt}%
\pgfpathmoveto{\pgfqpoint{1.115200in}{0.528000in}}%
\pgfpathlineto{\pgfqpoint{1.115200in}{4.224000in}}%
\pgfusepath{stroke}%
\end{pgfscope}%
\begin{pgfscope}%
\pgfsetrectcap%
\pgfsetmiterjoin%
\pgfsetlinewidth{0.803000pt}%
\definecolor{currentstroke}{rgb}{0.000000,0.000000,0.000000}%
\pgfsetstrokecolor{currentstroke}%
\pgfsetdash{}{0pt}%
\pgfpathmoveto{\pgfqpoint{5.444800in}{0.528000in}}%
\pgfpathlineto{\pgfqpoint{5.444800in}{4.224000in}}%
\pgfusepath{stroke}%
\end{pgfscope}%
\begin{pgfscope}%
\pgfsetrectcap%
\pgfsetmiterjoin%
\pgfsetlinewidth{0.803000pt}%
\definecolor{currentstroke}{rgb}{0.000000,0.000000,0.000000}%
\pgfsetstrokecolor{currentstroke}%
\pgfsetdash{}{0pt}%
\pgfpathmoveto{\pgfqpoint{1.115200in}{0.528000in}}%
\pgfpathlineto{\pgfqpoint{5.444800in}{0.528000in}}%
\pgfusepath{stroke}%
\end{pgfscope}%
\begin{pgfscope}%
\pgfsetrectcap%
\pgfsetmiterjoin%
\pgfsetlinewidth{0.803000pt}%
\definecolor{currentstroke}{rgb}{0.000000,0.000000,0.000000}%
\pgfsetstrokecolor{currentstroke}%
\pgfsetdash{}{0pt}%
\pgfpathmoveto{\pgfqpoint{1.115200in}{4.224000in}}%
\pgfpathlineto{\pgfqpoint{5.444800in}{4.224000in}}%
\pgfusepath{stroke}%
\end{pgfscope}%
\end{pgfpicture}%
\makeatother%
\endgroup%
}
	\caption{Klasterovanje}
	\label{fig:klaster}
\end{figure}

\section{Dizajn sistema za mašinsko učenje}

Okvirno, koraci u rešavanju problema su sledeći:
\begin{itemize}
	\item Prepoznavanje problema mašinskog učenja (nadgledano učenje, nenadgledano učenje, učenje potkrepljivanjem);
	\item Prikupljanje i obrada podataka, zajedno sa odabirom atributa;
	\item Odabir skupa dopustivih modela;
	\item Odabir algoritma učenja; moguće je odabrati postojeći algoritam ili razviti neki novi koji bolje odgovara problemu
	\item Izbor mere kvaliteta učenja;
	\item Obuka, evaluacija i, ukoliko je neophodno, ponavljanje nekog od prethodnih koraka radi unapređenja naučenog modela 
\end{itemize}

Prilikom odabira modela treba imati na umu vrstu problema koja se rešava, količinu podataka, zakonitosti koje važe u podacima, itd.
Slika \ref{fig:odabir} prikazuje razliku između dva modela iz skupa dopustivih modela za linearnu regresiju polinomom nad 10 različitih tačaka. Na levom delu slike prikazan je polinom reda 1 (prava) a na desnom delu prikazan je polinom reda 10. Iako će polinom reda 10 savršeno opisivati 10 tačaka sa slike, vidi se da su one raspoređene blizu prave i, uprkos većem odstupanju modela od podataka za učenje, jasno je da je prava bolji izbor.

\begin{figure}
	\centering
	\resizebox{.45\linewidth}{!}{%% Creator: Matplotlib, PGF backend
%%
%% To include the figure in your LaTeX document, write
%%   \input{<filename>.pgf}
%%
%% Make sure the required packages are loaded in your preamble
%%   \usepackage{pgf}
%%
%% Figures using additional raster images can only be included by \input if
%% they are in the same directory as the main LaTeX file. For loading figures
%% from other directories you can use the `import` package
%%   \usepackage{import}
%% and then include the figures with
%%   \import{<path to file>}{<filename>.pgf}
%%
%% Matplotlib used the following preamble
%%   \usepackage{fontspec}
%%   \setmainfont{Bitstream Vera Serif}
%%   \setsansfont{Bitstream Vera Sans}
%%   \setmonofont{Bitstream Vera Sans Mono}
%%
\begingroup%
\makeatletter%
\begin{pgfpicture}%
\pgfpathrectangle{\pgfpointorigin}{\pgfqpoint{8.000000in}{6.000000in}}%
\pgfusepath{use as bounding box, clip}%
\begin{pgfscope}%
\pgfsetbuttcap%
\pgfsetmiterjoin%
\definecolor{currentfill}{rgb}{1.000000,1.000000,1.000000}%
\pgfsetfillcolor{currentfill}%
\pgfsetlinewidth{0.000000pt}%
\definecolor{currentstroke}{rgb}{1.000000,1.000000,1.000000}%
\pgfsetstrokecolor{currentstroke}%
\pgfsetdash{}{0pt}%
\pgfpathmoveto{\pgfqpoint{0.000000in}{0.000000in}}%
\pgfpathlineto{\pgfqpoint{8.000000in}{0.000000in}}%
\pgfpathlineto{\pgfqpoint{8.000000in}{6.000000in}}%
\pgfpathlineto{\pgfqpoint{0.000000in}{6.000000in}}%
\pgfpathclose%
\pgfusepath{fill}%
\end{pgfscope}%
\begin{pgfscope}%
\pgfsetbuttcap%
\pgfsetmiterjoin%
\definecolor{currentfill}{rgb}{1.000000,1.000000,1.000000}%
\pgfsetfillcolor{currentfill}%
\pgfsetlinewidth{0.000000pt}%
\definecolor{currentstroke}{rgb}{0.000000,0.000000,0.000000}%
\pgfsetstrokecolor{currentstroke}%
\pgfsetstrokeopacity{0.000000}%
\pgfsetdash{}{0pt}%
\pgfpathmoveto{\pgfqpoint{1.000000in}{0.600000in}}%
\pgfpathlineto{\pgfqpoint{7.200000in}{0.600000in}}%
\pgfpathlineto{\pgfqpoint{7.200000in}{5.400000in}}%
\pgfpathlineto{\pgfqpoint{1.000000in}{5.400000in}}%
\pgfpathclose%
\pgfusepath{fill}%
\end{pgfscope}%
\begin{pgfscope}%
\pgfpathrectangle{\pgfqpoint{1.000000in}{0.600000in}}{\pgfqpoint{6.200000in}{4.800000in}} %
\pgfusepath{clip}%
\pgfsetbuttcap%
\pgfsetroundjoin%
\definecolor{currentfill}{rgb}{0.000000,0.500000,0.000000}%
\pgfsetfillcolor{currentfill}%
\pgfsetlinewidth{1.003750pt}%
\definecolor{currentstroke}{rgb}{0.000000,0.500000,0.000000}%
\pgfsetstrokecolor{currentstroke}%
\pgfsetdash{}{0pt}%
\pgfpathmoveto{\pgfqpoint{1.442857in}{0.663119in}}%
\pgfpathcurveto{\pgfqpoint{1.459330in}{0.663119in}}{\pgfqpoint{1.475130in}{0.669664in}}{\pgfqpoint{1.486778in}{0.681312in}}%
\pgfpathcurveto{\pgfqpoint{1.498426in}{0.692960in}}{\pgfqpoint{1.504970in}{0.708760in}}{\pgfqpoint{1.504970in}{0.725232in}}%
\pgfpathcurveto{\pgfqpoint{1.504970in}{0.741705in}}{\pgfqpoint{1.498426in}{0.757505in}}{\pgfqpoint{1.486778in}{0.769153in}}%
\pgfpathcurveto{\pgfqpoint{1.475130in}{0.780801in}}{\pgfqpoint{1.459330in}{0.787345in}}{\pgfqpoint{1.442857in}{0.787345in}}%
\pgfpathcurveto{\pgfqpoint{1.426385in}{0.787345in}}{\pgfqpoint{1.410584in}{0.780801in}}{\pgfqpoint{1.398937in}{0.769153in}}%
\pgfpathcurveto{\pgfqpoint{1.387289in}{0.757505in}}{\pgfqpoint{1.380744in}{0.741705in}}{\pgfqpoint{1.380744in}{0.725232in}}%
\pgfpathcurveto{\pgfqpoint{1.380744in}{0.708760in}}{\pgfqpoint{1.387289in}{0.692960in}}{\pgfqpoint{1.398937in}{0.681312in}}%
\pgfpathcurveto{\pgfqpoint{1.410584in}{0.669664in}}{\pgfqpoint{1.426385in}{0.663119in}}{\pgfqpoint{1.442857in}{0.663119in}}%
\pgfpathclose%
\pgfusepath{stroke,fill}%
\end{pgfscope}%
\begin{pgfscope}%
\pgfpathrectangle{\pgfqpoint{1.000000in}{0.600000in}}{\pgfqpoint{6.200000in}{4.800000in}} %
\pgfusepath{clip}%
\pgfsetbuttcap%
\pgfsetroundjoin%
\definecolor{currentfill}{rgb}{0.000000,0.500000,0.000000}%
\pgfsetfillcolor{currentfill}%
\pgfsetlinewidth{1.003750pt}%
\definecolor{currentstroke}{rgb}{0.000000,0.500000,0.000000}%
\pgfsetstrokecolor{currentstroke}%
\pgfsetdash{}{0pt}%
\pgfpathmoveto{\pgfqpoint{2.057937in}{1.339529in}}%
\pgfpathcurveto{\pgfqpoint{2.074409in}{1.339529in}}{\pgfqpoint{2.090209in}{1.346074in}}{\pgfqpoint{2.101857in}{1.357722in}}%
\pgfpathcurveto{\pgfqpoint{2.113505in}{1.369370in}}{\pgfqpoint{2.120050in}{1.385170in}}{\pgfqpoint{2.120050in}{1.401642in}}%
\pgfpathcurveto{\pgfqpoint{2.120050in}{1.418115in}}{\pgfqpoint{2.113505in}{1.433915in}}{\pgfqpoint{2.101857in}{1.445563in}}%
\pgfpathcurveto{\pgfqpoint{2.090209in}{1.457211in}}{\pgfqpoint{2.074409in}{1.463755in}}{\pgfqpoint{2.057937in}{1.463755in}}%
\pgfpathcurveto{\pgfqpoint{2.041464in}{1.463755in}}{\pgfqpoint{2.025664in}{1.457211in}}{\pgfqpoint{2.014016in}{1.445563in}}%
\pgfpathcurveto{\pgfqpoint{2.002368in}{1.433915in}}{\pgfqpoint{1.995824in}{1.418115in}}{\pgfqpoint{1.995824in}{1.401642in}}%
\pgfpathcurveto{\pgfqpoint{1.995824in}{1.385170in}}{\pgfqpoint{2.002368in}{1.369370in}}{\pgfqpoint{2.014016in}{1.357722in}}%
\pgfpathcurveto{\pgfqpoint{2.025664in}{1.346074in}}{\pgfqpoint{2.041464in}{1.339529in}}{\pgfqpoint{2.057937in}{1.339529in}}%
\pgfpathclose%
\pgfusepath{stroke,fill}%
\end{pgfscope}%
\begin{pgfscope}%
\pgfpathrectangle{\pgfqpoint{1.000000in}{0.600000in}}{\pgfqpoint{6.200000in}{4.800000in}} %
\pgfusepath{clip}%
\pgfsetbuttcap%
\pgfsetroundjoin%
\definecolor{currentfill}{rgb}{0.000000,0.500000,0.000000}%
\pgfsetfillcolor{currentfill}%
\pgfsetlinewidth{1.003750pt}%
\definecolor{currentstroke}{rgb}{0.000000,0.500000,0.000000}%
\pgfsetstrokecolor{currentstroke}%
\pgfsetdash{}{0pt}%
\pgfpathmoveto{\pgfqpoint{2.673016in}{1.632439in}}%
\pgfpathcurveto{\pgfqpoint{2.689488in}{1.632439in}}{\pgfqpoint{2.705289in}{1.638984in}}{\pgfqpoint{2.716936in}{1.650632in}}%
\pgfpathcurveto{\pgfqpoint{2.728584in}{1.662279in}}{\pgfqpoint{2.735129in}{1.678080in}}{\pgfqpoint{2.735129in}{1.694552in}}%
\pgfpathcurveto{\pgfqpoint{2.735129in}{1.711025in}}{\pgfqpoint{2.728584in}{1.726825in}}{\pgfqpoint{2.716936in}{1.738473in}}%
\pgfpathcurveto{\pgfqpoint{2.705289in}{1.750120in}}{\pgfqpoint{2.689488in}{1.756665in}}{\pgfqpoint{2.673016in}{1.756665in}}%
\pgfpathcurveto{\pgfqpoint{2.656543in}{1.756665in}}{\pgfqpoint{2.640743in}{1.750120in}}{\pgfqpoint{2.629095in}{1.738473in}}%
\pgfpathcurveto{\pgfqpoint{2.617447in}{1.726825in}}{\pgfqpoint{2.610903in}{1.711025in}}{\pgfqpoint{2.610903in}{1.694552in}}%
\pgfpathcurveto{\pgfqpoint{2.610903in}{1.678080in}}{\pgfqpoint{2.617447in}{1.662279in}}{\pgfqpoint{2.629095in}{1.650632in}}%
\pgfpathcurveto{\pgfqpoint{2.640743in}{1.638984in}}{\pgfqpoint{2.656543in}{1.632439in}}{\pgfqpoint{2.673016in}{1.632439in}}%
\pgfpathclose%
\pgfusepath{stroke,fill}%
\end{pgfscope}%
\begin{pgfscope}%
\pgfpathrectangle{\pgfqpoint{1.000000in}{0.600000in}}{\pgfqpoint{6.200000in}{4.800000in}} %
\pgfusepath{clip}%
\pgfsetbuttcap%
\pgfsetroundjoin%
\definecolor{currentfill}{rgb}{0.000000,0.500000,0.000000}%
\pgfsetfillcolor{currentfill}%
\pgfsetlinewidth{1.003750pt}%
\definecolor{currentstroke}{rgb}{0.000000,0.500000,0.000000}%
\pgfsetstrokecolor{currentstroke}%
\pgfsetdash{}{0pt}%
\pgfpathmoveto{\pgfqpoint{3.288095in}{2.327298in}}%
\pgfpathcurveto{\pgfqpoint{3.304568in}{2.327298in}}{\pgfqpoint{3.320368in}{2.333843in}}{\pgfqpoint{3.332016in}{2.345491in}}%
\pgfpathcurveto{\pgfqpoint{3.343664in}{2.357138in}}{\pgfqpoint{3.350208in}{2.372939in}}{\pgfqpoint{3.350208in}{2.389411in}}%
\pgfpathcurveto{\pgfqpoint{3.350208in}{2.405884in}}{\pgfqpoint{3.343664in}{2.421684in}}{\pgfqpoint{3.332016in}{2.433332in}}%
\pgfpathcurveto{\pgfqpoint{3.320368in}{2.444979in}}{\pgfqpoint{3.304568in}{2.451524in}}{\pgfqpoint{3.288095in}{2.451524in}}%
\pgfpathcurveto{\pgfqpoint{3.271623in}{2.451524in}}{\pgfqpoint{3.255823in}{2.444979in}}{\pgfqpoint{3.244175in}{2.433332in}}%
\pgfpathcurveto{\pgfqpoint{3.232527in}{2.421684in}}{\pgfqpoint{3.225982in}{2.405884in}}{\pgfqpoint{3.225982in}{2.389411in}}%
\pgfpathcurveto{\pgfqpoint{3.225982in}{2.372939in}}{\pgfqpoint{3.232527in}{2.357138in}}{\pgfqpoint{3.244175in}{2.345491in}}%
\pgfpathcurveto{\pgfqpoint{3.255823in}{2.333843in}}{\pgfqpoint{3.271623in}{2.327298in}}{\pgfqpoint{3.288095in}{2.327298in}}%
\pgfpathclose%
\pgfusepath{stroke,fill}%
\end{pgfscope}%
\begin{pgfscope}%
\pgfpathrectangle{\pgfqpoint{1.000000in}{0.600000in}}{\pgfqpoint{6.200000in}{4.800000in}} %
\pgfusepath{clip}%
\pgfsetbuttcap%
\pgfsetroundjoin%
\definecolor{currentfill}{rgb}{0.000000,0.500000,0.000000}%
\pgfsetfillcolor{currentfill}%
\pgfsetlinewidth{1.003750pt}%
\definecolor{currentstroke}{rgb}{0.000000,0.500000,0.000000}%
\pgfsetstrokecolor{currentstroke}%
\pgfsetdash{}{0pt}%
\pgfpathmoveto{\pgfqpoint{3.903175in}{2.688513in}}%
\pgfpathcurveto{\pgfqpoint{3.919647in}{2.688513in}}{\pgfqpoint{3.935447in}{2.695057in}}{\pgfqpoint{3.947095in}{2.706705in}}%
\pgfpathcurveto{\pgfqpoint{3.958743in}{2.718353in}}{\pgfqpoint{3.965288in}{2.734153in}}{\pgfqpoint{3.965288in}{2.750626in}}%
\pgfpathcurveto{\pgfqpoint{3.965288in}{2.767098in}}{\pgfqpoint{3.958743in}{2.782898in}}{\pgfqpoint{3.947095in}{2.794546in}}%
\pgfpathcurveto{\pgfqpoint{3.935447in}{2.806194in}}{\pgfqpoint{3.919647in}{2.812739in}}{\pgfqpoint{3.903175in}{2.812739in}}%
\pgfpathcurveto{\pgfqpoint{3.886702in}{2.812739in}}{\pgfqpoint{3.870902in}{2.806194in}}{\pgfqpoint{3.859254in}{2.794546in}}%
\pgfpathcurveto{\pgfqpoint{3.847606in}{2.782898in}}{\pgfqpoint{3.841062in}{2.767098in}}{\pgfqpoint{3.841062in}{2.750626in}}%
\pgfpathcurveto{\pgfqpoint{3.841062in}{2.734153in}}{\pgfqpoint{3.847606in}{2.718353in}}{\pgfqpoint{3.859254in}{2.706705in}}%
\pgfpathcurveto{\pgfqpoint{3.870902in}{2.695057in}}{\pgfqpoint{3.886702in}{2.688513in}}{\pgfqpoint{3.903175in}{2.688513in}}%
\pgfpathclose%
\pgfusepath{stroke,fill}%
\end{pgfscope}%
\begin{pgfscope}%
\pgfpathrectangle{\pgfqpoint{1.000000in}{0.600000in}}{\pgfqpoint{6.200000in}{4.800000in}} %
\pgfusepath{clip}%
\pgfsetbuttcap%
\pgfsetroundjoin%
\definecolor{currentfill}{rgb}{0.000000,0.500000,0.000000}%
\pgfsetfillcolor{currentfill}%
\pgfsetlinewidth{1.003750pt}%
\definecolor{currentstroke}{rgb}{0.000000,0.500000,0.000000}%
\pgfsetstrokecolor{currentstroke}%
\pgfsetdash{}{0pt}%
\pgfpathmoveto{\pgfqpoint{4.518254in}{3.211433in}}%
\pgfpathcurveto{\pgfqpoint{4.534727in}{3.211433in}}{\pgfqpoint{4.550527in}{3.217978in}}{\pgfqpoint{4.562174in}{3.229626in}}%
\pgfpathcurveto{\pgfqpoint{4.573822in}{3.241274in}}{\pgfqpoint{4.580367in}{3.257074in}}{\pgfqpoint{4.580367in}{3.273546in}}%
\pgfpathcurveto{\pgfqpoint{4.580367in}{3.290019in}}{\pgfqpoint{4.573822in}{3.305819in}}{\pgfqpoint{4.562174in}{3.317467in}}%
\pgfpathcurveto{\pgfqpoint{4.550527in}{3.329115in}}{\pgfqpoint{4.534727in}{3.335659in}}{\pgfqpoint{4.518254in}{3.335659in}}%
\pgfpathcurveto{\pgfqpoint{4.501781in}{3.335659in}}{\pgfqpoint{4.485981in}{3.329115in}}{\pgfqpoint{4.474333in}{3.317467in}}%
\pgfpathcurveto{\pgfqpoint{4.462686in}{3.305819in}}{\pgfqpoint{4.456141in}{3.290019in}}{\pgfqpoint{4.456141in}{3.273546in}}%
\pgfpathcurveto{\pgfqpoint{4.456141in}{3.257074in}}{\pgfqpoint{4.462686in}{3.241274in}}{\pgfqpoint{4.474333in}{3.229626in}}%
\pgfpathcurveto{\pgfqpoint{4.485981in}{3.217978in}}{\pgfqpoint{4.501781in}{3.211433in}}{\pgfqpoint{4.518254in}{3.211433in}}%
\pgfpathclose%
\pgfusepath{stroke,fill}%
\end{pgfscope}%
\begin{pgfscope}%
\pgfpathrectangle{\pgfqpoint{1.000000in}{0.600000in}}{\pgfqpoint{6.200000in}{4.800000in}} %
\pgfusepath{clip}%
\pgfsetbuttcap%
\pgfsetroundjoin%
\definecolor{currentfill}{rgb}{0.000000,0.500000,0.000000}%
\pgfsetfillcolor{currentfill}%
\pgfsetlinewidth{1.003750pt}%
\definecolor{currentstroke}{rgb}{0.000000,0.500000,0.000000}%
\pgfsetstrokecolor{currentstroke}%
\pgfsetdash{}{0pt}%
\pgfpathmoveto{\pgfqpoint{5.133333in}{3.743361in}}%
\pgfpathcurveto{\pgfqpoint{5.149806in}{3.743361in}}{\pgfqpoint{5.165606in}{3.749906in}}{\pgfqpoint{5.177254in}{3.761553in}}%
\pgfpathcurveto{\pgfqpoint{5.188902in}{3.773201in}}{\pgfqpoint{5.195446in}{3.789001in}}{\pgfqpoint{5.195446in}{3.805474in}}%
\pgfpathcurveto{\pgfqpoint{5.195446in}{3.821946in}}{\pgfqpoint{5.188902in}{3.837747in}}{\pgfqpoint{5.177254in}{3.849394in}}%
\pgfpathcurveto{\pgfqpoint{5.165606in}{3.861042in}}{\pgfqpoint{5.149806in}{3.867587in}}{\pgfqpoint{5.133333in}{3.867587in}}%
\pgfpathcurveto{\pgfqpoint{5.116861in}{3.867587in}}{\pgfqpoint{5.101061in}{3.861042in}}{\pgfqpoint{5.089413in}{3.849394in}}%
\pgfpathcurveto{\pgfqpoint{5.077765in}{3.837747in}}{\pgfqpoint{5.071220in}{3.821946in}}{\pgfqpoint{5.071220in}{3.805474in}}%
\pgfpathcurveto{\pgfqpoint{5.071220in}{3.789001in}}{\pgfqpoint{5.077765in}{3.773201in}}{\pgfqpoint{5.089413in}{3.761553in}}%
\pgfpathcurveto{\pgfqpoint{5.101061in}{3.749906in}}{\pgfqpoint{5.116861in}{3.743361in}}{\pgfqpoint{5.133333in}{3.743361in}}%
\pgfpathclose%
\pgfusepath{stroke,fill}%
\end{pgfscope}%
\begin{pgfscope}%
\pgfpathrectangle{\pgfqpoint{1.000000in}{0.600000in}}{\pgfqpoint{6.200000in}{4.800000in}} %
\pgfusepath{clip}%
\pgfsetbuttcap%
\pgfsetroundjoin%
\definecolor{currentfill}{rgb}{0.000000,0.500000,0.000000}%
\pgfsetfillcolor{currentfill}%
\pgfsetlinewidth{1.003750pt}%
\definecolor{currentstroke}{rgb}{0.000000,0.500000,0.000000}%
\pgfsetstrokecolor{currentstroke}%
\pgfsetdash{}{0pt}%
\pgfpathmoveto{\pgfqpoint{5.748413in}{4.157951in}}%
\pgfpathcurveto{\pgfqpoint{5.764885in}{4.157951in}}{\pgfqpoint{5.780685in}{4.164496in}}{\pgfqpoint{5.792333in}{4.176144in}}%
\pgfpathcurveto{\pgfqpoint{5.803981in}{4.187792in}}{\pgfqpoint{5.810526in}{4.203592in}}{\pgfqpoint{5.810526in}{4.220064in}}%
\pgfpathcurveto{\pgfqpoint{5.810526in}{4.236537in}}{\pgfqpoint{5.803981in}{4.252337in}}{\pgfqpoint{5.792333in}{4.263985in}}%
\pgfpathcurveto{\pgfqpoint{5.780685in}{4.275633in}}{\pgfqpoint{5.764885in}{4.282177in}}{\pgfqpoint{5.748413in}{4.282177in}}%
\pgfpathcurveto{\pgfqpoint{5.731940in}{4.282177in}}{\pgfqpoint{5.716140in}{4.275633in}}{\pgfqpoint{5.704492in}{4.263985in}}%
\pgfpathcurveto{\pgfqpoint{5.692844in}{4.252337in}}{\pgfqpoint{5.686300in}{4.236537in}}{\pgfqpoint{5.686300in}{4.220064in}}%
\pgfpathcurveto{\pgfqpoint{5.686300in}{4.203592in}}{\pgfqpoint{5.692844in}{4.187792in}}{\pgfqpoint{5.704492in}{4.176144in}}%
\pgfpathcurveto{\pgfqpoint{5.716140in}{4.164496in}}{\pgfqpoint{5.731940in}{4.157951in}}{\pgfqpoint{5.748413in}{4.157951in}}%
\pgfpathclose%
\pgfusepath{stroke,fill}%
\end{pgfscope}%
\begin{pgfscope}%
\pgfpathrectangle{\pgfqpoint{1.000000in}{0.600000in}}{\pgfqpoint{6.200000in}{4.800000in}} %
\pgfusepath{clip}%
\pgfsetbuttcap%
\pgfsetroundjoin%
\definecolor{currentfill}{rgb}{0.000000,0.500000,0.000000}%
\pgfsetfillcolor{currentfill}%
\pgfsetlinewidth{1.003750pt}%
\definecolor{currentstroke}{rgb}{0.000000,0.500000,0.000000}%
\pgfsetstrokecolor{currentstroke}%
\pgfsetdash{}{0pt}%
\pgfpathmoveto{\pgfqpoint{6.363492in}{4.580100in}}%
\pgfpathcurveto{\pgfqpoint{6.379965in}{4.580100in}}{\pgfqpoint{6.395765in}{4.586645in}}{\pgfqpoint{6.407413in}{4.598293in}}%
\pgfpathcurveto{\pgfqpoint{6.419060in}{4.609941in}}{\pgfqpoint{6.425605in}{4.625741in}}{\pgfqpoint{6.425605in}{4.642213in}}%
\pgfpathcurveto{\pgfqpoint{6.425605in}{4.658686in}}{\pgfqpoint{6.419060in}{4.674486in}}{\pgfqpoint{6.407413in}{4.686134in}}%
\pgfpathcurveto{\pgfqpoint{6.395765in}{4.697782in}}{\pgfqpoint{6.379965in}{4.704326in}}{\pgfqpoint{6.363492in}{4.704326in}}%
\pgfpathcurveto{\pgfqpoint{6.347020in}{4.704326in}}{\pgfqpoint{6.331219in}{4.697782in}}{\pgfqpoint{6.319572in}{4.686134in}}%
\pgfpathcurveto{\pgfqpoint{6.307924in}{4.674486in}}{\pgfqpoint{6.301379in}{4.658686in}}{\pgfqpoint{6.301379in}{4.642213in}}%
\pgfpathcurveto{\pgfqpoint{6.301379in}{4.625741in}}{\pgfqpoint{6.307924in}{4.609941in}}{\pgfqpoint{6.319572in}{4.598293in}}%
\pgfpathcurveto{\pgfqpoint{6.331219in}{4.586645in}}{\pgfqpoint{6.347020in}{4.580100in}}{\pgfqpoint{6.363492in}{4.580100in}}%
\pgfpathclose%
\pgfusepath{stroke,fill}%
\end{pgfscope}%
\begin{pgfscope}%
\pgfpathrectangle{\pgfqpoint{1.000000in}{0.600000in}}{\pgfqpoint{6.200000in}{4.800000in}} %
\pgfusepath{clip}%
\pgfsetbuttcap%
\pgfsetroundjoin%
\definecolor{currentfill}{rgb}{0.000000,0.500000,0.000000}%
\pgfsetfillcolor{currentfill}%
\pgfsetlinewidth{1.003750pt}%
\definecolor{currentstroke}{rgb}{0.000000,0.500000,0.000000}%
\pgfsetstrokecolor{currentstroke}%
\pgfsetdash{}{0pt}%
\pgfpathmoveto{\pgfqpoint{6.978571in}{5.035125in}}%
\pgfpathcurveto{\pgfqpoint{6.995044in}{5.035125in}}{\pgfqpoint{7.010844in}{5.041670in}}{\pgfqpoint{7.022492in}{5.053318in}}%
\pgfpathcurveto{\pgfqpoint{7.034140in}{5.064966in}}{\pgfqpoint{7.040684in}{5.080766in}}{\pgfqpoint{7.040684in}{5.097238in}}%
\pgfpathcurveto{\pgfqpoint{7.040684in}{5.113711in}}{\pgfqpoint{7.034140in}{5.129511in}}{\pgfqpoint{7.022492in}{5.141159in}}%
\pgfpathcurveto{\pgfqpoint{7.010844in}{5.152807in}}{\pgfqpoint{6.995044in}{5.159351in}}{\pgfqpoint{6.978571in}{5.159351in}}%
\pgfpathcurveto{\pgfqpoint{6.962099in}{5.159351in}}{\pgfqpoint{6.946299in}{5.152807in}}{\pgfqpoint{6.934651in}{5.141159in}}%
\pgfpathcurveto{\pgfqpoint{6.923003in}{5.129511in}}{\pgfqpoint{6.916458in}{5.113711in}}{\pgfqpoint{6.916458in}{5.097238in}}%
\pgfpathcurveto{\pgfqpoint{6.916458in}{5.080766in}}{\pgfqpoint{6.923003in}{5.064966in}}{\pgfqpoint{6.934651in}{5.053318in}}%
\pgfpathcurveto{\pgfqpoint{6.946299in}{5.041670in}}{\pgfqpoint{6.962099in}{5.035125in}}{\pgfqpoint{6.978571in}{5.035125in}}%
\pgfpathclose%
\pgfusepath{stroke,fill}%
\end{pgfscope}%
\begin{pgfscope}%
\pgfpathrectangle{\pgfqpoint{1.000000in}{0.600000in}}{\pgfqpoint{6.200000in}{4.800000in}} %
\pgfusepath{clip}%
\pgfsetrectcap%
\pgfsetroundjoin%
\pgfsetlinewidth{2.509375pt}%
\definecolor{currentstroke}{rgb}{1.000000,0.000000,0.000000}%
\pgfsetstrokecolor{currentstroke}%
\pgfsetdash{}{0pt}%
\pgfpathmoveto{\pgfqpoint{1.442857in}{0.833708in}}%
\pgfpathlineto{\pgfqpoint{1.555831in}{0.922128in}}%
\pgfpathlineto{\pgfqpoint{1.668805in}{1.010548in}}%
\pgfpathlineto{\pgfqpoint{1.781778in}{1.098968in}}%
\pgfpathlineto{\pgfqpoint{1.894752in}{1.187388in}}%
\pgfpathlineto{\pgfqpoint{2.007726in}{1.275808in}}%
\pgfpathlineto{\pgfqpoint{2.120700in}{1.364228in}}%
\pgfpathlineto{\pgfqpoint{2.233673in}{1.452648in}}%
\pgfpathlineto{\pgfqpoint{2.346647in}{1.541068in}}%
\pgfpathlineto{\pgfqpoint{2.459621in}{1.629488in}}%
\pgfpathlineto{\pgfqpoint{2.572595in}{1.717909in}}%
\pgfpathlineto{\pgfqpoint{2.685569in}{1.806329in}}%
\pgfpathlineto{\pgfqpoint{2.798542in}{1.894749in}}%
\pgfpathlineto{\pgfqpoint{2.911516in}{1.983169in}}%
\pgfpathlineto{\pgfqpoint{3.024490in}{2.071589in}}%
\pgfpathlineto{\pgfqpoint{3.137464in}{2.160009in}}%
\pgfpathlineto{\pgfqpoint{3.250437in}{2.248429in}}%
\pgfpathlineto{\pgfqpoint{3.363411in}{2.336849in}}%
\pgfpathlineto{\pgfqpoint{3.476385in}{2.425269in}}%
\pgfpathlineto{\pgfqpoint{3.589359in}{2.513689in}}%
\pgfpathlineto{\pgfqpoint{3.702332in}{2.602110in}}%
\pgfpathlineto{\pgfqpoint{3.815306in}{2.690530in}}%
\pgfpathlineto{\pgfqpoint{3.928280in}{2.778950in}}%
\pgfpathlineto{\pgfqpoint{4.041254in}{2.867370in}}%
\pgfpathlineto{\pgfqpoint{4.154227in}{2.955790in}}%
\pgfpathlineto{\pgfqpoint{4.267201in}{3.044210in}}%
\pgfpathlineto{\pgfqpoint{4.380175in}{3.132630in}}%
\pgfpathlineto{\pgfqpoint{4.493149in}{3.221050in}}%
\pgfpathlineto{\pgfqpoint{4.606122in}{3.309470in}}%
\pgfpathlineto{\pgfqpoint{4.719096in}{3.397890in}}%
\pgfpathlineto{\pgfqpoint{4.832070in}{3.486311in}}%
\pgfpathlineto{\pgfqpoint{4.945044in}{3.574731in}}%
\pgfpathlineto{\pgfqpoint{5.058017in}{3.663151in}}%
\pgfpathlineto{\pgfqpoint{5.170991in}{3.751571in}}%
\pgfpathlineto{\pgfqpoint{5.283965in}{3.839991in}}%
\pgfpathlineto{\pgfqpoint{5.396939in}{3.928411in}}%
\pgfpathlineto{\pgfqpoint{5.509913in}{4.016831in}}%
\pgfpathlineto{\pgfqpoint{5.622886in}{4.105251in}}%
\pgfpathlineto{\pgfqpoint{5.735860in}{4.193671in}}%
\pgfpathlineto{\pgfqpoint{5.848834in}{4.282091in}}%
\pgfpathlineto{\pgfqpoint{5.961808in}{4.370512in}}%
\pgfpathlineto{\pgfqpoint{6.074781in}{4.458932in}}%
\pgfpathlineto{\pgfqpoint{6.187755in}{4.547352in}}%
\pgfpathlineto{\pgfqpoint{6.300729in}{4.635772in}}%
\pgfpathlineto{\pgfqpoint{6.413703in}{4.724192in}}%
\pgfpathlineto{\pgfqpoint{6.526676in}{4.812612in}}%
\pgfpathlineto{\pgfqpoint{6.639650in}{4.901032in}}%
\pgfpathlineto{\pgfqpoint{6.752624in}{4.989452in}}%
\pgfpathlineto{\pgfqpoint{6.865598in}{5.077872in}}%
\pgfpathlineto{\pgfqpoint{6.978571in}{5.166292in}}%
\pgfusepath{stroke}%
\end{pgfscope}%
\begin{pgfscope}%
\pgfsetrectcap%
\pgfsetmiterjoin%
\pgfsetlinewidth{1.003750pt}%
\definecolor{currentstroke}{rgb}{0.000000,0.000000,0.000000}%
\pgfsetstrokecolor{currentstroke}%
\pgfsetdash{}{0pt}%
\pgfpathmoveto{\pgfqpoint{7.200000in}{0.600000in}}%
\pgfpathlineto{\pgfqpoint{7.200000in}{5.400000in}}%
\pgfusepath{stroke}%
\end{pgfscope}%
\begin{pgfscope}%
\pgfsetrectcap%
\pgfsetmiterjoin%
\pgfsetlinewidth{1.003750pt}%
\definecolor{currentstroke}{rgb}{0.000000,0.000000,0.000000}%
\pgfsetstrokecolor{currentstroke}%
\pgfsetdash{}{0pt}%
\pgfpathmoveto{\pgfqpoint{1.000000in}{0.600000in}}%
\pgfpathlineto{\pgfqpoint{7.200000in}{0.600000in}}%
\pgfusepath{stroke}%
\end{pgfscope}%
\begin{pgfscope}%
\pgfsetrectcap%
\pgfsetmiterjoin%
\pgfsetlinewidth{1.003750pt}%
\definecolor{currentstroke}{rgb}{0.000000,0.000000,0.000000}%
\pgfsetstrokecolor{currentstroke}%
\pgfsetdash{}{0pt}%
\pgfpathmoveto{\pgfqpoint{1.000000in}{0.600000in}}%
\pgfpathlineto{\pgfqpoint{1.000000in}{5.400000in}}%
\pgfusepath{stroke}%
\end{pgfscope}%
\begin{pgfscope}%
\pgfsetrectcap%
\pgfsetmiterjoin%
\pgfsetlinewidth{1.003750pt}%
\definecolor{currentstroke}{rgb}{0.000000,0.000000,0.000000}%
\pgfsetstrokecolor{currentstroke}%
\pgfsetdash{}{0pt}%
\pgfpathmoveto{\pgfqpoint{1.000000in}{5.400000in}}%
\pgfpathlineto{\pgfqpoint{7.200000in}{5.400000in}}%
\pgfusepath{stroke}%
\end{pgfscope}%
\begin{pgfscope}%
\pgfsetbuttcap%
\pgfsetroundjoin%
\definecolor{currentfill}{rgb}{0.000000,0.000000,0.000000}%
\pgfsetfillcolor{currentfill}%
\pgfsetlinewidth{0.501875pt}%
\definecolor{currentstroke}{rgb}{0.000000,0.000000,0.000000}%
\pgfsetstrokecolor{currentstroke}%
\pgfsetdash{}{0pt}%
\pgfsys@defobject{currentmarker}{\pgfqpoint{0.000000in}{0.000000in}}{\pgfqpoint{0.000000in}{0.055556in}}{%
\pgfpathmoveto{\pgfqpoint{0.000000in}{0.000000in}}%
\pgfpathlineto{\pgfqpoint{0.000000in}{0.055556in}}%
\pgfusepath{stroke,fill}%
}%
\begin{pgfscope}%
\pgfsys@transformshift{1.442857in}{0.600000in}%
\pgfsys@useobject{currentmarker}{}%
\end{pgfscope}%
\end{pgfscope}%
\begin{pgfscope}%
\pgfsetbuttcap%
\pgfsetroundjoin%
\definecolor{currentfill}{rgb}{0.000000,0.000000,0.000000}%
\pgfsetfillcolor{currentfill}%
\pgfsetlinewidth{0.501875pt}%
\definecolor{currentstroke}{rgb}{0.000000,0.000000,0.000000}%
\pgfsetstrokecolor{currentstroke}%
\pgfsetdash{}{0pt}%
\pgfsys@defobject{currentmarker}{\pgfqpoint{0.000000in}{-0.055556in}}{\pgfqpoint{0.000000in}{0.000000in}}{%
\pgfpathmoveto{\pgfqpoint{0.000000in}{0.000000in}}%
\pgfpathlineto{\pgfqpoint{0.000000in}{-0.055556in}}%
\pgfusepath{stroke,fill}%
}%
\begin{pgfscope}%
\pgfsys@transformshift{1.442857in}{5.400000in}%
\pgfsys@useobject{currentmarker}{}%
\end{pgfscope}%
\end{pgfscope}%
\begin{pgfscope}%
\pgftext[x=1.442857in,y=0.544444in,,top]{\sffamily\fontsize{12.000000}{14.400000}\selectfont 0}%
\end{pgfscope}%
\begin{pgfscope}%
\pgfsetbuttcap%
\pgfsetroundjoin%
\definecolor{currentfill}{rgb}{0.000000,0.000000,0.000000}%
\pgfsetfillcolor{currentfill}%
\pgfsetlinewidth{0.501875pt}%
\definecolor{currentstroke}{rgb}{0.000000,0.000000,0.000000}%
\pgfsetstrokecolor{currentstroke}%
\pgfsetdash{}{0pt}%
\pgfsys@defobject{currentmarker}{\pgfqpoint{0.000000in}{0.000000in}}{\pgfqpoint{0.000000in}{0.055556in}}{%
\pgfpathmoveto{\pgfqpoint{0.000000in}{0.000000in}}%
\pgfpathlineto{\pgfqpoint{0.000000in}{0.055556in}}%
\pgfusepath{stroke,fill}%
}%
\begin{pgfscope}%
\pgfsys@transformshift{2.550000in}{0.600000in}%
\pgfsys@useobject{currentmarker}{}%
\end{pgfscope}%
\end{pgfscope}%
\begin{pgfscope}%
\pgfsetbuttcap%
\pgfsetroundjoin%
\definecolor{currentfill}{rgb}{0.000000,0.000000,0.000000}%
\pgfsetfillcolor{currentfill}%
\pgfsetlinewidth{0.501875pt}%
\definecolor{currentstroke}{rgb}{0.000000,0.000000,0.000000}%
\pgfsetstrokecolor{currentstroke}%
\pgfsetdash{}{0pt}%
\pgfsys@defobject{currentmarker}{\pgfqpoint{0.000000in}{-0.055556in}}{\pgfqpoint{0.000000in}{0.000000in}}{%
\pgfpathmoveto{\pgfqpoint{0.000000in}{0.000000in}}%
\pgfpathlineto{\pgfqpoint{0.000000in}{-0.055556in}}%
\pgfusepath{stroke,fill}%
}%
\begin{pgfscope}%
\pgfsys@transformshift{2.550000in}{5.400000in}%
\pgfsys@useobject{currentmarker}{}%
\end{pgfscope}%
\end{pgfscope}%
\begin{pgfscope}%
\pgftext[x=2.550000in,y=0.544444in,,top]{\sffamily\fontsize{12.000000}{14.400000}\selectfont 1}%
\end{pgfscope}%
\begin{pgfscope}%
\pgfsetbuttcap%
\pgfsetroundjoin%
\definecolor{currentfill}{rgb}{0.000000,0.000000,0.000000}%
\pgfsetfillcolor{currentfill}%
\pgfsetlinewidth{0.501875pt}%
\definecolor{currentstroke}{rgb}{0.000000,0.000000,0.000000}%
\pgfsetstrokecolor{currentstroke}%
\pgfsetdash{}{0pt}%
\pgfsys@defobject{currentmarker}{\pgfqpoint{0.000000in}{0.000000in}}{\pgfqpoint{0.000000in}{0.055556in}}{%
\pgfpathmoveto{\pgfqpoint{0.000000in}{0.000000in}}%
\pgfpathlineto{\pgfqpoint{0.000000in}{0.055556in}}%
\pgfusepath{stroke,fill}%
}%
\begin{pgfscope}%
\pgfsys@transformshift{3.657143in}{0.600000in}%
\pgfsys@useobject{currentmarker}{}%
\end{pgfscope}%
\end{pgfscope}%
\begin{pgfscope}%
\pgfsetbuttcap%
\pgfsetroundjoin%
\definecolor{currentfill}{rgb}{0.000000,0.000000,0.000000}%
\pgfsetfillcolor{currentfill}%
\pgfsetlinewidth{0.501875pt}%
\definecolor{currentstroke}{rgb}{0.000000,0.000000,0.000000}%
\pgfsetstrokecolor{currentstroke}%
\pgfsetdash{}{0pt}%
\pgfsys@defobject{currentmarker}{\pgfqpoint{0.000000in}{-0.055556in}}{\pgfqpoint{0.000000in}{0.000000in}}{%
\pgfpathmoveto{\pgfqpoint{0.000000in}{0.000000in}}%
\pgfpathlineto{\pgfqpoint{0.000000in}{-0.055556in}}%
\pgfusepath{stroke,fill}%
}%
\begin{pgfscope}%
\pgfsys@transformshift{3.657143in}{5.400000in}%
\pgfsys@useobject{currentmarker}{}%
\end{pgfscope}%
\end{pgfscope}%
\begin{pgfscope}%
\pgftext[x=3.657143in,y=0.544444in,,top]{\sffamily\fontsize{12.000000}{14.400000}\selectfont 2}%
\end{pgfscope}%
\begin{pgfscope}%
\pgfsetbuttcap%
\pgfsetroundjoin%
\definecolor{currentfill}{rgb}{0.000000,0.000000,0.000000}%
\pgfsetfillcolor{currentfill}%
\pgfsetlinewidth{0.501875pt}%
\definecolor{currentstroke}{rgb}{0.000000,0.000000,0.000000}%
\pgfsetstrokecolor{currentstroke}%
\pgfsetdash{}{0pt}%
\pgfsys@defobject{currentmarker}{\pgfqpoint{0.000000in}{0.000000in}}{\pgfqpoint{0.000000in}{0.055556in}}{%
\pgfpathmoveto{\pgfqpoint{0.000000in}{0.000000in}}%
\pgfpathlineto{\pgfqpoint{0.000000in}{0.055556in}}%
\pgfusepath{stroke,fill}%
}%
\begin{pgfscope}%
\pgfsys@transformshift{4.764286in}{0.600000in}%
\pgfsys@useobject{currentmarker}{}%
\end{pgfscope}%
\end{pgfscope}%
\begin{pgfscope}%
\pgfsetbuttcap%
\pgfsetroundjoin%
\definecolor{currentfill}{rgb}{0.000000,0.000000,0.000000}%
\pgfsetfillcolor{currentfill}%
\pgfsetlinewidth{0.501875pt}%
\definecolor{currentstroke}{rgb}{0.000000,0.000000,0.000000}%
\pgfsetstrokecolor{currentstroke}%
\pgfsetdash{}{0pt}%
\pgfsys@defobject{currentmarker}{\pgfqpoint{0.000000in}{-0.055556in}}{\pgfqpoint{0.000000in}{0.000000in}}{%
\pgfpathmoveto{\pgfqpoint{0.000000in}{0.000000in}}%
\pgfpathlineto{\pgfqpoint{0.000000in}{-0.055556in}}%
\pgfusepath{stroke,fill}%
}%
\begin{pgfscope}%
\pgfsys@transformshift{4.764286in}{5.400000in}%
\pgfsys@useobject{currentmarker}{}%
\end{pgfscope}%
\end{pgfscope}%
\begin{pgfscope}%
\pgftext[x=4.764286in,y=0.544444in,,top]{\sffamily\fontsize{12.000000}{14.400000}\selectfont 3}%
\end{pgfscope}%
\begin{pgfscope}%
\pgfsetbuttcap%
\pgfsetroundjoin%
\definecolor{currentfill}{rgb}{0.000000,0.000000,0.000000}%
\pgfsetfillcolor{currentfill}%
\pgfsetlinewidth{0.501875pt}%
\definecolor{currentstroke}{rgb}{0.000000,0.000000,0.000000}%
\pgfsetstrokecolor{currentstroke}%
\pgfsetdash{}{0pt}%
\pgfsys@defobject{currentmarker}{\pgfqpoint{0.000000in}{0.000000in}}{\pgfqpoint{0.000000in}{0.055556in}}{%
\pgfpathmoveto{\pgfqpoint{0.000000in}{0.000000in}}%
\pgfpathlineto{\pgfqpoint{0.000000in}{0.055556in}}%
\pgfusepath{stroke,fill}%
}%
\begin{pgfscope}%
\pgfsys@transformshift{5.871429in}{0.600000in}%
\pgfsys@useobject{currentmarker}{}%
\end{pgfscope}%
\end{pgfscope}%
\begin{pgfscope}%
\pgfsetbuttcap%
\pgfsetroundjoin%
\definecolor{currentfill}{rgb}{0.000000,0.000000,0.000000}%
\pgfsetfillcolor{currentfill}%
\pgfsetlinewidth{0.501875pt}%
\definecolor{currentstroke}{rgb}{0.000000,0.000000,0.000000}%
\pgfsetstrokecolor{currentstroke}%
\pgfsetdash{}{0pt}%
\pgfsys@defobject{currentmarker}{\pgfqpoint{0.000000in}{-0.055556in}}{\pgfqpoint{0.000000in}{0.000000in}}{%
\pgfpathmoveto{\pgfqpoint{0.000000in}{0.000000in}}%
\pgfpathlineto{\pgfqpoint{0.000000in}{-0.055556in}}%
\pgfusepath{stroke,fill}%
}%
\begin{pgfscope}%
\pgfsys@transformshift{5.871429in}{5.400000in}%
\pgfsys@useobject{currentmarker}{}%
\end{pgfscope}%
\end{pgfscope}%
\begin{pgfscope}%
\pgftext[x=5.871429in,y=0.544444in,,top]{\sffamily\fontsize{12.000000}{14.400000}\selectfont 4}%
\end{pgfscope}%
\begin{pgfscope}%
\pgfsetbuttcap%
\pgfsetroundjoin%
\definecolor{currentfill}{rgb}{0.000000,0.000000,0.000000}%
\pgfsetfillcolor{currentfill}%
\pgfsetlinewidth{0.501875pt}%
\definecolor{currentstroke}{rgb}{0.000000,0.000000,0.000000}%
\pgfsetstrokecolor{currentstroke}%
\pgfsetdash{}{0pt}%
\pgfsys@defobject{currentmarker}{\pgfqpoint{0.000000in}{0.000000in}}{\pgfqpoint{0.000000in}{0.055556in}}{%
\pgfpathmoveto{\pgfqpoint{0.000000in}{0.000000in}}%
\pgfpathlineto{\pgfqpoint{0.000000in}{0.055556in}}%
\pgfusepath{stroke,fill}%
}%
\begin{pgfscope}%
\pgfsys@transformshift{6.978571in}{0.600000in}%
\pgfsys@useobject{currentmarker}{}%
\end{pgfscope}%
\end{pgfscope}%
\begin{pgfscope}%
\pgfsetbuttcap%
\pgfsetroundjoin%
\definecolor{currentfill}{rgb}{0.000000,0.000000,0.000000}%
\pgfsetfillcolor{currentfill}%
\pgfsetlinewidth{0.501875pt}%
\definecolor{currentstroke}{rgb}{0.000000,0.000000,0.000000}%
\pgfsetstrokecolor{currentstroke}%
\pgfsetdash{}{0pt}%
\pgfsys@defobject{currentmarker}{\pgfqpoint{0.000000in}{-0.055556in}}{\pgfqpoint{0.000000in}{0.000000in}}{%
\pgfpathmoveto{\pgfqpoint{0.000000in}{0.000000in}}%
\pgfpathlineto{\pgfqpoint{0.000000in}{-0.055556in}}%
\pgfusepath{stroke,fill}%
}%
\begin{pgfscope}%
\pgfsys@transformshift{6.978571in}{5.400000in}%
\pgfsys@useobject{currentmarker}{}%
\end{pgfscope}%
\end{pgfscope}%
\begin{pgfscope}%
\pgftext[x=6.978571in,y=0.544444in,,top]{\sffamily\fontsize{12.000000}{14.400000}\selectfont 5}%
\end{pgfscope}%
\begin{pgfscope}%
\pgftext[x=4.100000in,y=0.313705in,,top]{\sffamily\fontsize{12.000000}{14.400000}\selectfont x}%
\end{pgfscope}%
\begin{pgfscope}%
\pgfsetbuttcap%
\pgfsetroundjoin%
\definecolor{currentfill}{rgb}{0.000000,0.000000,0.000000}%
\pgfsetfillcolor{currentfill}%
\pgfsetlinewidth{0.501875pt}%
\definecolor{currentstroke}{rgb}{0.000000,0.000000,0.000000}%
\pgfsetstrokecolor{currentstroke}%
\pgfsetdash{}{0pt}%
\pgfsys@defobject{currentmarker}{\pgfqpoint{0.000000in}{0.000000in}}{\pgfqpoint{0.055556in}{0.000000in}}{%
\pgfpathmoveto{\pgfqpoint{0.000000in}{0.000000in}}%
\pgfpathlineto{\pgfqpoint{0.055556in}{0.000000in}}%
\pgfusepath{stroke,fill}%
}%
\begin{pgfscope}%
\pgfsys@transformshift{1.000000in}{0.660380in}%
\pgfsys@useobject{currentmarker}{}%
\end{pgfscope}%
\end{pgfscope}%
\begin{pgfscope}%
\pgfsetbuttcap%
\pgfsetroundjoin%
\definecolor{currentfill}{rgb}{0.000000,0.000000,0.000000}%
\pgfsetfillcolor{currentfill}%
\pgfsetlinewidth{0.501875pt}%
\definecolor{currentstroke}{rgb}{0.000000,0.000000,0.000000}%
\pgfsetstrokecolor{currentstroke}%
\pgfsetdash{}{0pt}%
\pgfsys@defobject{currentmarker}{\pgfqpoint{-0.055556in}{0.000000in}}{\pgfqpoint{0.000000in}{0.000000in}}{%
\pgfpathmoveto{\pgfqpoint{0.000000in}{0.000000in}}%
\pgfpathlineto{\pgfqpoint{-0.055556in}{0.000000in}}%
\pgfusepath{stroke,fill}%
}%
\begin{pgfscope}%
\pgfsys@transformshift{7.200000in}{0.660380in}%
\pgfsys@useobject{currentmarker}{}%
\end{pgfscope}%
\end{pgfscope}%
\begin{pgfscope}%
\pgftext[x=0.944444in,y=0.660380in,right,]{\sffamily\fontsize{12.000000}{14.400000}\selectfont 1}%
\end{pgfscope}%
\begin{pgfscope}%
\pgfsetbuttcap%
\pgfsetroundjoin%
\definecolor{currentfill}{rgb}{0.000000,0.000000,0.000000}%
\pgfsetfillcolor{currentfill}%
\pgfsetlinewidth{0.501875pt}%
\definecolor{currentstroke}{rgb}{0.000000,0.000000,0.000000}%
\pgfsetstrokecolor{currentstroke}%
\pgfsetdash{}{0pt}%
\pgfsys@defobject{currentmarker}{\pgfqpoint{0.000000in}{0.000000in}}{\pgfqpoint{0.055556in}{0.000000in}}{%
\pgfpathmoveto{\pgfqpoint{0.000000in}{0.000000in}}%
\pgfpathlineto{\pgfqpoint{0.055556in}{0.000000in}}%
\pgfusepath{stroke,fill}%
}%
\begin{pgfscope}%
\pgfsys@transformshift{1.000000in}{1.244650in}%
\pgfsys@useobject{currentmarker}{}%
\end{pgfscope}%
\end{pgfscope}%
\begin{pgfscope}%
\pgfsetbuttcap%
\pgfsetroundjoin%
\definecolor{currentfill}{rgb}{0.000000,0.000000,0.000000}%
\pgfsetfillcolor{currentfill}%
\pgfsetlinewidth{0.501875pt}%
\definecolor{currentstroke}{rgb}{0.000000,0.000000,0.000000}%
\pgfsetstrokecolor{currentstroke}%
\pgfsetdash{}{0pt}%
\pgfsys@defobject{currentmarker}{\pgfqpoint{-0.055556in}{0.000000in}}{\pgfqpoint{0.000000in}{0.000000in}}{%
\pgfpathmoveto{\pgfqpoint{0.000000in}{0.000000in}}%
\pgfpathlineto{\pgfqpoint{-0.055556in}{0.000000in}}%
\pgfusepath{stroke,fill}%
}%
\begin{pgfscope}%
\pgfsys@transformshift{7.200000in}{1.244650in}%
\pgfsys@useobject{currentmarker}{}%
\end{pgfscope}%
\end{pgfscope}%
\begin{pgfscope}%
\pgftext[x=0.944444in,y=1.244650in,right,]{\sffamily\fontsize{12.000000}{14.400000}\selectfont 2}%
\end{pgfscope}%
\begin{pgfscope}%
\pgfsetbuttcap%
\pgfsetroundjoin%
\definecolor{currentfill}{rgb}{0.000000,0.000000,0.000000}%
\pgfsetfillcolor{currentfill}%
\pgfsetlinewidth{0.501875pt}%
\definecolor{currentstroke}{rgb}{0.000000,0.000000,0.000000}%
\pgfsetstrokecolor{currentstroke}%
\pgfsetdash{}{0pt}%
\pgfsys@defobject{currentmarker}{\pgfqpoint{0.000000in}{0.000000in}}{\pgfqpoint{0.055556in}{0.000000in}}{%
\pgfpathmoveto{\pgfqpoint{0.000000in}{0.000000in}}%
\pgfpathlineto{\pgfqpoint{0.055556in}{0.000000in}}%
\pgfusepath{stroke,fill}%
}%
\begin{pgfscope}%
\pgfsys@transformshift{1.000000in}{1.828919in}%
\pgfsys@useobject{currentmarker}{}%
\end{pgfscope}%
\end{pgfscope}%
\begin{pgfscope}%
\pgfsetbuttcap%
\pgfsetroundjoin%
\definecolor{currentfill}{rgb}{0.000000,0.000000,0.000000}%
\pgfsetfillcolor{currentfill}%
\pgfsetlinewidth{0.501875pt}%
\definecolor{currentstroke}{rgb}{0.000000,0.000000,0.000000}%
\pgfsetstrokecolor{currentstroke}%
\pgfsetdash{}{0pt}%
\pgfsys@defobject{currentmarker}{\pgfqpoint{-0.055556in}{0.000000in}}{\pgfqpoint{0.000000in}{0.000000in}}{%
\pgfpathmoveto{\pgfqpoint{0.000000in}{0.000000in}}%
\pgfpathlineto{\pgfqpoint{-0.055556in}{0.000000in}}%
\pgfusepath{stroke,fill}%
}%
\begin{pgfscope}%
\pgfsys@transformshift{7.200000in}{1.828919in}%
\pgfsys@useobject{currentmarker}{}%
\end{pgfscope}%
\end{pgfscope}%
\begin{pgfscope}%
\pgftext[x=0.944444in,y=1.828919in,right,]{\sffamily\fontsize{12.000000}{14.400000}\selectfont 3}%
\end{pgfscope}%
\begin{pgfscope}%
\pgfsetbuttcap%
\pgfsetroundjoin%
\definecolor{currentfill}{rgb}{0.000000,0.000000,0.000000}%
\pgfsetfillcolor{currentfill}%
\pgfsetlinewidth{0.501875pt}%
\definecolor{currentstroke}{rgb}{0.000000,0.000000,0.000000}%
\pgfsetstrokecolor{currentstroke}%
\pgfsetdash{}{0pt}%
\pgfsys@defobject{currentmarker}{\pgfqpoint{0.000000in}{0.000000in}}{\pgfqpoint{0.055556in}{0.000000in}}{%
\pgfpathmoveto{\pgfqpoint{0.000000in}{0.000000in}}%
\pgfpathlineto{\pgfqpoint{0.055556in}{0.000000in}}%
\pgfusepath{stroke,fill}%
}%
\begin{pgfscope}%
\pgfsys@transformshift{1.000000in}{2.413188in}%
\pgfsys@useobject{currentmarker}{}%
\end{pgfscope}%
\end{pgfscope}%
\begin{pgfscope}%
\pgfsetbuttcap%
\pgfsetroundjoin%
\definecolor{currentfill}{rgb}{0.000000,0.000000,0.000000}%
\pgfsetfillcolor{currentfill}%
\pgfsetlinewidth{0.501875pt}%
\definecolor{currentstroke}{rgb}{0.000000,0.000000,0.000000}%
\pgfsetstrokecolor{currentstroke}%
\pgfsetdash{}{0pt}%
\pgfsys@defobject{currentmarker}{\pgfqpoint{-0.055556in}{0.000000in}}{\pgfqpoint{0.000000in}{0.000000in}}{%
\pgfpathmoveto{\pgfqpoint{0.000000in}{0.000000in}}%
\pgfpathlineto{\pgfqpoint{-0.055556in}{0.000000in}}%
\pgfusepath{stroke,fill}%
}%
\begin{pgfscope}%
\pgfsys@transformshift{7.200000in}{2.413188in}%
\pgfsys@useobject{currentmarker}{}%
\end{pgfscope}%
\end{pgfscope}%
\begin{pgfscope}%
\pgftext[x=0.944444in,y=2.413188in,right,]{\sffamily\fontsize{12.000000}{14.400000}\selectfont 4}%
\end{pgfscope}%
\begin{pgfscope}%
\pgfsetbuttcap%
\pgfsetroundjoin%
\definecolor{currentfill}{rgb}{0.000000,0.000000,0.000000}%
\pgfsetfillcolor{currentfill}%
\pgfsetlinewidth{0.501875pt}%
\definecolor{currentstroke}{rgb}{0.000000,0.000000,0.000000}%
\pgfsetstrokecolor{currentstroke}%
\pgfsetdash{}{0pt}%
\pgfsys@defobject{currentmarker}{\pgfqpoint{0.000000in}{0.000000in}}{\pgfqpoint{0.055556in}{0.000000in}}{%
\pgfpathmoveto{\pgfqpoint{0.000000in}{0.000000in}}%
\pgfpathlineto{\pgfqpoint{0.055556in}{0.000000in}}%
\pgfusepath{stroke,fill}%
}%
\begin{pgfscope}%
\pgfsys@transformshift{1.000000in}{2.997457in}%
\pgfsys@useobject{currentmarker}{}%
\end{pgfscope}%
\end{pgfscope}%
\begin{pgfscope}%
\pgfsetbuttcap%
\pgfsetroundjoin%
\definecolor{currentfill}{rgb}{0.000000,0.000000,0.000000}%
\pgfsetfillcolor{currentfill}%
\pgfsetlinewidth{0.501875pt}%
\definecolor{currentstroke}{rgb}{0.000000,0.000000,0.000000}%
\pgfsetstrokecolor{currentstroke}%
\pgfsetdash{}{0pt}%
\pgfsys@defobject{currentmarker}{\pgfqpoint{-0.055556in}{0.000000in}}{\pgfqpoint{0.000000in}{0.000000in}}{%
\pgfpathmoveto{\pgfqpoint{0.000000in}{0.000000in}}%
\pgfpathlineto{\pgfqpoint{-0.055556in}{0.000000in}}%
\pgfusepath{stroke,fill}%
}%
\begin{pgfscope}%
\pgfsys@transformshift{7.200000in}{2.997457in}%
\pgfsys@useobject{currentmarker}{}%
\end{pgfscope}%
\end{pgfscope}%
\begin{pgfscope}%
\pgftext[x=0.944444in,y=2.997457in,right,]{\sffamily\fontsize{12.000000}{14.400000}\selectfont 5}%
\end{pgfscope}%
\begin{pgfscope}%
\pgfsetbuttcap%
\pgfsetroundjoin%
\definecolor{currentfill}{rgb}{0.000000,0.000000,0.000000}%
\pgfsetfillcolor{currentfill}%
\pgfsetlinewidth{0.501875pt}%
\definecolor{currentstroke}{rgb}{0.000000,0.000000,0.000000}%
\pgfsetstrokecolor{currentstroke}%
\pgfsetdash{}{0pt}%
\pgfsys@defobject{currentmarker}{\pgfqpoint{0.000000in}{0.000000in}}{\pgfqpoint{0.055556in}{0.000000in}}{%
\pgfpathmoveto{\pgfqpoint{0.000000in}{0.000000in}}%
\pgfpathlineto{\pgfqpoint{0.055556in}{0.000000in}}%
\pgfusepath{stroke,fill}%
}%
\begin{pgfscope}%
\pgfsys@transformshift{1.000000in}{3.581726in}%
\pgfsys@useobject{currentmarker}{}%
\end{pgfscope}%
\end{pgfscope}%
\begin{pgfscope}%
\pgfsetbuttcap%
\pgfsetroundjoin%
\definecolor{currentfill}{rgb}{0.000000,0.000000,0.000000}%
\pgfsetfillcolor{currentfill}%
\pgfsetlinewidth{0.501875pt}%
\definecolor{currentstroke}{rgb}{0.000000,0.000000,0.000000}%
\pgfsetstrokecolor{currentstroke}%
\pgfsetdash{}{0pt}%
\pgfsys@defobject{currentmarker}{\pgfqpoint{-0.055556in}{0.000000in}}{\pgfqpoint{0.000000in}{0.000000in}}{%
\pgfpathmoveto{\pgfqpoint{0.000000in}{0.000000in}}%
\pgfpathlineto{\pgfqpoint{-0.055556in}{0.000000in}}%
\pgfusepath{stroke,fill}%
}%
\begin{pgfscope}%
\pgfsys@transformshift{7.200000in}{3.581726in}%
\pgfsys@useobject{currentmarker}{}%
\end{pgfscope}%
\end{pgfscope}%
\begin{pgfscope}%
\pgftext[x=0.944444in,y=3.581726in,right,]{\sffamily\fontsize{12.000000}{14.400000}\selectfont 6}%
\end{pgfscope}%
\begin{pgfscope}%
\pgfsetbuttcap%
\pgfsetroundjoin%
\definecolor{currentfill}{rgb}{0.000000,0.000000,0.000000}%
\pgfsetfillcolor{currentfill}%
\pgfsetlinewidth{0.501875pt}%
\definecolor{currentstroke}{rgb}{0.000000,0.000000,0.000000}%
\pgfsetstrokecolor{currentstroke}%
\pgfsetdash{}{0pt}%
\pgfsys@defobject{currentmarker}{\pgfqpoint{0.000000in}{0.000000in}}{\pgfqpoint{0.055556in}{0.000000in}}{%
\pgfpathmoveto{\pgfqpoint{0.000000in}{0.000000in}}%
\pgfpathlineto{\pgfqpoint{0.055556in}{0.000000in}}%
\pgfusepath{stroke,fill}%
}%
\begin{pgfscope}%
\pgfsys@transformshift{1.000000in}{4.165995in}%
\pgfsys@useobject{currentmarker}{}%
\end{pgfscope}%
\end{pgfscope}%
\begin{pgfscope}%
\pgfsetbuttcap%
\pgfsetroundjoin%
\definecolor{currentfill}{rgb}{0.000000,0.000000,0.000000}%
\pgfsetfillcolor{currentfill}%
\pgfsetlinewidth{0.501875pt}%
\definecolor{currentstroke}{rgb}{0.000000,0.000000,0.000000}%
\pgfsetstrokecolor{currentstroke}%
\pgfsetdash{}{0pt}%
\pgfsys@defobject{currentmarker}{\pgfqpoint{-0.055556in}{0.000000in}}{\pgfqpoint{0.000000in}{0.000000in}}{%
\pgfpathmoveto{\pgfqpoint{0.000000in}{0.000000in}}%
\pgfpathlineto{\pgfqpoint{-0.055556in}{0.000000in}}%
\pgfusepath{stroke,fill}%
}%
\begin{pgfscope}%
\pgfsys@transformshift{7.200000in}{4.165995in}%
\pgfsys@useobject{currentmarker}{}%
\end{pgfscope}%
\end{pgfscope}%
\begin{pgfscope}%
\pgftext[x=0.944444in,y=4.165995in,right,]{\sffamily\fontsize{12.000000}{14.400000}\selectfont 7}%
\end{pgfscope}%
\begin{pgfscope}%
\pgfsetbuttcap%
\pgfsetroundjoin%
\definecolor{currentfill}{rgb}{0.000000,0.000000,0.000000}%
\pgfsetfillcolor{currentfill}%
\pgfsetlinewidth{0.501875pt}%
\definecolor{currentstroke}{rgb}{0.000000,0.000000,0.000000}%
\pgfsetstrokecolor{currentstroke}%
\pgfsetdash{}{0pt}%
\pgfsys@defobject{currentmarker}{\pgfqpoint{0.000000in}{0.000000in}}{\pgfqpoint{0.055556in}{0.000000in}}{%
\pgfpathmoveto{\pgfqpoint{0.000000in}{0.000000in}}%
\pgfpathlineto{\pgfqpoint{0.055556in}{0.000000in}}%
\pgfusepath{stroke,fill}%
}%
\begin{pgfscope}%
\pgfsys@transformshift{1.000000in}{4.750264in}%
\pgfsys@useobject{currentmarker}{}%
\end{pgfscope}%
\end{pgfscope}%
\begin{pgfscope}%
\pgfsetbuttcap%
\pgfsetroundjoin%
\definecolor{currentfill}{rgb}{0.000000,0.000000,0.000000}%
\pgfsetfillcolor{currentfill}%
\pgfsetlinewidth{0.501875pt}%
\definecolor{currentstroke}{rgb}{0.000000,0.000000,0.000000}%
\pgfsetstrokecolor{currentstroke}%
\pgfsetdash{}{0pt}%
\pgfsys@defobject{currentmarker}{\pgfqpoint{-0.055556in}{0.000000in}}{\pgfqpoint{0.000000in}{0.000000in}}{%
\pgfpathmoveto{\pgfqpoint{0.000000in}{0.000000in}}%
\pgfpathlineto{\pgfqpoint{-0.055556in}{0.000000in}}%
\pgfusepath{stroke,fill}%
}%
\begin{pgfscope}%
\pgfsys@transformshift{7.200000in}{4.750264in}%
\pgfsys@useobject{currentmarker}{}%
\end{pgfscope}%
\end{pgfscope}%
\begin{pgfscope}%
\pgftext[x=0.944444in,y=4.750264in,right,]{\sffamily\fontsize{12.000000}{14.400000}\selectfont 8}%
\end{pgfscope}%
\begin{pgfscope}%
\pgfsetbuttcap%
\pgfsetroundjoin%
\definecolor{currentfill}{rgb}{0.000000,0.000000,0.000000}%
\pgfsetfillcolor{currentfill}%
\pgfsetlinewidth{0.501875pt}%
\definecolor{currentstroke}{rgb}{0.000000,0.000000,0.000000}%
\pgfsetstrokecolor{currentstroke}%
\pgfsetdash{}{0pt}%
\pgfsys@defobject{currentmarker}{\pgfqpoint{0.000000in}{0.000000in}}{\pgfqpoint{0.055556in}{0.000000in}}{%
\pgfpathmoveto{\pgfqpoint{0.000000in}{0.000000in}}%
\pgfpathlineto{\pgfqpoint{0.055556in}{0.000000in}}%
\pgfusepath{stroke,fill}%
}%
\begin{pgfscope}%
\pgfsys@transformshift{1.000000in}{5.334533in}%
\pgfsys@useobject{currentmarker}{}%
\end{pgfscope}%
\end{pgfscope}%
\begin{pgfscope}%
\pgfsetbuttcap%
\pgfsetroundjoin%
\definecolor{currentfill}{rgb}{0.000000,0.000000,0.000000}%
\pgfsetfillcolor{currentfill}%
\pgfsetlinewidth{0.501875pt}%
\definecolor{currentstroke}{rgb}{0.000000,0.000000,0.000000}%
\pgfsetstrokecolor{currentstroke}%
\pgfsetdash{}{0pt}%
\pgfsys@defobject{currentmarker}{\pgfqpoint{-0.055556in}{0.000000in}}{\pgfqpoint{0.000000in}{0.000000in}}{%
\pgfpathmoveto{\pgfqpoint{0.000000in}{0.000000in}}%
\pgfpathlineto{\pgfqpoint{-0.055556in}{0.000000in}}%
\pgfusepath{stroke,fill}%
}%
\begin{pgfscope}%
\pgfsys@transformshift{7.200000in}{5.334533in}%
\pgfsys@useobject{currentmarker}{}%
\end{pgfscope}%
\end{pgfscope}%
\begin{pgfscope}%
\pgftext[x=0.944444in,y=5.334533in,right,]{\sffamily\fontsize{12.000000}{14.400000}\selectfont 9}%
\end{pgfscope}%
\begin{pgfscope}%
\pgftext[x=0.768962in,y=3.000000in,,bottom]{\sffamily\fontsize{12.000000}{14.400000}\selectfont y}%
\end{pgfscope}%
\end{pgfpicture}%
\makeatother%
\endgroup%
}
	\resizebox{.45\linewidth}{!}{%% Creator: Matplotlib, PGF backend
%%
%% To include the figure in your LaTeX document, write
%%   \input{<filename>.pgf}
%%
%% Make sure the required packages are loaded in your preamble
%%   \usepackage{pgf}
%%
%% Figures using additional raster images can only be included by \input if
%% they are in the same directory as the main LaTeX file. For loading figures
%% from other directories you can use the `import` package
%%   \usepackage{import}
%% and then include the figures with
%%   \import{<path to file>}{<filename>.pgf}
%%
%% Matplotlib used the following preamble
%%   \usepackage{fontspec}
%%   \setmainfont{Bitstream Vera Serif}
%%   \setsansfont{Bitstream Vera Sans}
%%   \setmonofont{Bitstream Vera Sans Mono}
%%
\begingroup%
\makeatletter%
\begin{pgfpicture}%
\pgfpathrectangle{\pgfpointorigin}{\pgfqpoint{8.000000in}{6.000000in}}%
\pgfusepath{use as bounding box, clip}%
\begin{pgfscope}%
\pgfsetbuttcap%
\pgfsetmiterjoin%
\definecolor{currentfill}{rgb}{1.000000,1.000000,1.000000}%
\pgfsetfillcolor{currentfill}%
\pgfsetlinewidth{0.000000pt}%
\definecolor{currentstroke}{rgb}{1.000000,1.000000,1.000000}%
\pgfsetstrokecolor{currentstroke}%
\pgfsetdash{}{0pt}%
\pgfpathmoveto{\pgfqpoint{0.000000in}{0.000000in}}%
\pgfpathlineto{\pgfqpoint{8.000000in}{0.000000in}}%
\pgfpathlineto{\pgfqpoint{8.000000in}{6.000000in}}%
\pgfpathlineto{\pgfqpoint{0.000000in}{6.000000in}}%
\pgfpathclose%
\pgfusepath{fill}%
\end{pgfscope}%
\begin{pgfscope}%
\pgfsetbuttcap%
\pgfsetmiterjoin%
\definecolor{currentfill}{rgb}{1.000000,1.000000,1.000000}%
\pgfsetfillcolor{currentfill}%
\pgfsetlinewidth{0.000000pt}%
\definecolor{currentstroke}{rgb}{0.000000,0.000000,0.000000}%
\pgfsetstrokecolor{currentstroke}%
\pgfsetstrokeopacity{0.000000}%
\pgfsetdash{}{0pt}%
\pgfpathmoveto{\pgfqpoint{1.000000in}{0.600000in}}%
\pgfpathlineto{\pgfqpoint{7.200000in}{0.600000in}}%
\pgfpathlineto{\pgfqpoint{7.200000in}{5.400000in}}%
\pgfpathlineto{\pgfqpoint{1.000000in}{5.400000in}}%
\pgfpathclose%
\pgfusepath{fill}%
\end{pgfscope}%
\begin{pgfscope}%
\pgfpathrectangle{\pgfqpoint{1.000000in}{0.600000in}}{\pgfqpoint{6.200000in}{4.800000in}} %
\pgfusepath{clip}%
\pgfsetbuttcap%
\pgfsetroundjoin%
\definecolor{currentfill}{rgb}{0.000000,0.500000,0.000000}%
\pgfsetfillcolor{currentfill}%
\pgfsetlinewidth{1.003750pt}%
\definecolor{currentstroke}{rgb}{0.000000,0.500000,0.000000}%
\pgfsetstrokecolor{currentstroke}%
\pgfsetdash{}{0pt}%
\pgfpathmoveto{\pgfqpoint{1.442857in}{0.663119in}}%
\pgfpathcurveto{\pgfqpoint{1.459330in}{0.663119in}}{\pgfqpoint{1.475130in}{0.669664in}}{\pgfqpoint{1.486778in}{0.681312in}}%
\pgfpathcurveto{\pgfqpoint{1.498426in}{0.692960in}}{\pgfqpoint{1.504970in}{0.708760in}}{\pgfqpoint{1.504970in}{0.725232in}}%
\pgfpathcurveto{\pgfqpoint{1.504970in}{0.741705in}}{\pgfqpoint{1.498426in}{0.757505in}}{\pgfqpoint{1.486778in}{0.769153in}}%
\pgfpathcurveto{\pgfqpoint{1.475130in}{0.780801in}}{\pgfqpoint{1.459330in}{0.787345in}}{\pgfqpoint{1.442857in}{0.787345in}}%
\pgfpathcurveto{\pgfqpoint{1.426385in}{0.787345in}}{\pgfqpoint{1.410584in}{0.780801in}}{\pgfqpoint{1.398937in}{0.769153in}}%
\pgfpathcurveto{\pgfqpoint{1.387289in}{0.757505in}}{\pgfqpoint{1.380744in}{0.741705in}}{\pgfqpoint{1.380744in}{0.725232in}}%
\pgfpathcurveto{\pgfqpoint{1.380744in}{0.708760in}}{\pgfqpoint{1.387289in}{0.692960in}}{\pgfqpoint{1.398937in}{0.681312in}}%
\pgfpathcurveto{\pgfqpoint{1.410584in}{0.669664in}}{\pgfqpoint{1.426385in}{0.663119in}}{\pgfqpoint{1.442857in}{0.663119in}}%
\pgfpathclose%
\pgfusepath{stroke,fill}%
\end{pgfscope}%
\begin{pgfscope}%
\pgfpathrectangle{\pgfqpoint{1.000000in}{0.600000in}}{\pgfqpoint{6.200000in}{4.800000in}} %
\pgfusepath{clip}%
\pgfsetbuttcap%
\pgfsetroundjoin%
\definecolor{currentfill}{rgb}{0.000000,0.500000,0.000000}%
\pgfsetfillcolor{currentfill}%
\pgfsetlinewidth{1.003750pt}%
\definecolor{currentstroke}{rgb}{0.000000,0.500000,0.000000}%
\pgfsetstrokecolor{currentstroke}%
\pgfsetdash{}{0pt}%
\pgfpathmoveto{\pgfqpoint{2.057937in}{1.339529in}}%
\pgfpathcurveto{\pgfqpoint{2.074409in}{1.339529in}}{\pgfqpoint{2.090209in}{1.346074in}}{\pgfqpoint{2.101857in}{1.357722in}}%
\pgfpathcurveto{\pgfqpoint{2.113505in}{1.369370in}}{\pgfqpoint{2.120050in}{1.385170in}}{\pgfqpoint{2.120050in}{1.401642in}}%
\pgfpathcurveto{\pgfqpoint{2.120050in}{1.418115in}}{\pgfqpoint{2.113505in}{1.433915in}}{\pgfqpoint{2.101857in}{1.445563in}}%
\pgfpathcurveto{\pgfqpoint{2.090209in}{1.457211in}}{\pgfqpoint{2.074409in}{1.463755in}}{\pgfqpoint{2.057937in}{1.463755in}}%
\pgfpathcurveto{\pgfqpoint{2.041464in}{1.463755in}}{\pgfqpoint{2.025664in}{1.457211in}}{\pgfqpoint{2.014016in}{1.445563in}}%
\pgfpathcurveto{\pgfqpoint{2.002368in}{1.433915in}}{\pgfqpoint{1.995824in}{1.418115in}}{\pgfqpoint{1.995824in}{1.401642in}}%
\pgfpathcurveto{\pgfqpoint{1.995824in}{1.385170in}}{\pgfqpoint{2.002368in}{1.369370in}}{\pgfqpoint{2.014016in}{1.357722in}}%
\pgfpathcurveto{\pgfqpoint{2.025664in}{1.346074in}}{\pgfqpoint{2.041464in}{1.339529in}}{\pgfqpoint{2.057937in}{1.339529in}}%
\pgfpathclose%
\pgfusepath{stroke,fill}%
\end{pgfscope}%
\begin{pgfscope}%
\pgfpathrectangle{\pgfqpoint{1.000000in}{0.600000in}}{\pgfqpoint{6.200000in}{4.800000in}} %
\pgfusepath{clip}%
\pgfsetbuttcap%
\pgfsetroundjoin%
\definecolor{currentfill}{rgb}{0.000000,0.500000,0.000000}%
\pgfsetfillcolor{currentfill}%
\pgfsetlinewidth{1.003750pt}%
\definecolor{currentstroke}{rgb}{0.000000,0.500000,0.000000}%
\pgfsetstrokecolor{currentstroke}%
\pgfsetdash{}{0pt}%
\pgfpathmoveto{\pgfqpoint{2.673016in}{1.632439in}}%
\pgfpathcurveto{\pgfqpoint{2.689488in}{1.632439in}}{\pgfqpoint{2.705289in}{1.638984in}}{\pgfqpoint{2.716936in}{1.650632in}}%
\pgfpathcurveto{\pgfqpoint{2.728584in}{1.662279in}}{\pgfqpoint{2.735129in}{1.678080in}}{\pgfqpoint{2.735129in}{1.694552in}}%
\pgfpathcurveto{\pgfqpoint{2.735129in}{1.711025in}}{\pgfqpoint{2.728584in}{1.726825in}}{\pgfqpoint{2.716936in}{1.738473in}}%
\pgfpathcurveto{\pgfqpoint{2.705289in}{1.750120in}}{\pgfqpoint{2.689488in}{1.756665in}}{\pgfqpoint{2.673016in}{1.756665in}}%
\pgfpathcurveto{\pgfqpoint{2.656543in}{1.756665in}}{\pgfqpoint{2.640743in}{1.750120in}}{\pgfqpoint{2.629095in}{1.738473in}}%
\pgfpathcurveto{\pgfqpoint{2.617447in}{1.726825in}}{\pgfqpoint{2.610903in}{1.711025in}}{\pgfqpoint{2.610903in}{1.694552in}}%
\pgfpathcurveto{\pgfqpoint{2.610903in}{1.678080in}}{\pgfqpoint{2.617447in}{1.662279in}}{\pgfqpoint{2.629095in}{1.650632in}}%
\pgfpathcurveto{\pgfqpoint{2.640743in}{1.638984in}}{\pgfqpoint{2.656543in}{1.632439in}}{\pgfqpoint{2.673016in}{1.632439in}}%
\pgfpathclose%
\pgfusepath{stroke,fill}%
\end{pgfscope}%
\begin{pgfscope}%
\pgfpathrectangle{\pgfqpoint{1.000000in}{0.600000in}}{\pgfqpoint{6.200000in}{4.800000in}} %
\pgfusepath{clip}%
\pgfsetbuttcap%
\pgfsetroundjoin%
\definecolor{currentfill}{rgb}{0.000000,0.500000,0.000000}%
\pgfsetfillcolor{currentfill}%
\pgfsetlinewidth{1.003750pt}%
\definecolor{currentstroke}{rgb}{0.000000,0.500000,0.000000}%
\pgfsetstrokecolor{currentstroke}%
\pgfsetdash{}{0pt}%
\pgfpathmoveto{\pgfqpoint{3.288095in}{2.327298in}}%
\pgfpathcurveto{\pgfqpoint{3.304568in}{2.327298in}}{\pgfqpoint{3.320368in}{2.333843in}}{\pgfqpoint{3.332016in}{2.345491in}}%
\pgfpathcurveto{\pgfqpoint{3.343664in}{2.357138in}}{\pgfqpoint{3.350208in}{2.372939in}}{\pgfqpoint{3.350208in}{2.389411in}}%
\pgfpathcurveto{\pgfqpoint{3.350208in}{2.405884in}}{\pgfqpoint{3.343664in}{2.421684in}}{\pgfqpoint{3.332016in}{2.433332in}}%
\pgfpathcurveto{\pgfqpoint{3.320368in}{2.444979in}}{\pgfqpoint{3.304568in}{2.451524in}}{\pgfqpoint{3.288095in}{2.451524in}}%
\pgfpathcurveto{\pgfqpoint{3.271623in}{2.451524in}}{\pgfqpoint{3.255823in}{2.444979in}}{\pgfqpoint{3.244175in}{2.433332in}}%
\pgfpathcurveto{\pgfqpoint{3.232527in}{2.421684in}}{\pgfqpoint{3.225982in}{2.405884in}}{\pgfqpoint{3.225982in}{2.389411in}}%
\pgfpathcurveto{\pgfqpoint{3.225982in}{2.372939in}}{\pgfqpoint{3.232527in}{2.357138in}}{\pgfqpoint{3.244175in}{2.345491in}}%
\pgfpathcurveto{\pgfqpoint{3.255823in}{2.333843in}}{\pgfqpoint{3.271623in}{2.327298in}}{\pgfqpoint{3.288095in}{2.327298in}}%
\pgfpathclose%
\pgfusepath{stroke,fill}%
\end{pgfscope}%
\begin{pgfscope}%
\pgfpathrectangle{\pgfqpoint{1.000000in}{0.600000in}}{\pgfqpoint{6.200000in}{4.800000in}} %
\pgfusepath{clip}%
\pgfsetbuttcap%
\pgfsetroundjoin%
\definecolor{currentfill}{rgb}{0.000000,0.500000,0.000000}%
\pgfsetfillcolor{currentfill}%
\pgfsetlinewidth{1.003750pt}%
\definecolor{currentstroke}{rgb}{0.000000,0.500000,0.000000}%
\pgfsetstrokecolor{currentstroke}%
\pgfsetdash{}{0pt}%
\pgfpathmoveto{\pgfqpoint{3.903175in}{2.688513in}}%
\pgfpathcurveto{\pgfqpoint{3.919647in}{2.688513in}}{\pgfqpoint{3.935447in}{2.695057in}}{\pgfqpoint{3.947095in}{2.706705in}}%
\pgfpathcurveto{\pgfqpoint{3.958743in}{2.718353in}}{\pgfqpoint{3.965288in}{2.734153in}}{\pgfqpoint{3.965288in}{2.750626in}}%
\pgfpathcurveto{\pgfqpoint{3.965288in}{2.767098in}}{\pgfqpoint{3.958743in}{2.782898in}}{\pgfqpoint{3.947095in}{2.794546in}}%
\pgfpathcurveto{\pgfqpoint{3.935447in}{2.806194in}}{\pgfqpoint{3.919647in}{2.812739in}}{\pgfqpoint{3.903175in}{2.812739in}}%
\pgfpathcurveto{\pgfqpoint{3.886702in}{2.812739in}}{\pgfqpoint{3.870902in}{2.806194in}}{\pgfqpoint{3.859254in}{2.794546in}}%
\pgfpathcurveto{\pgfqpoint{3.847606in}{2.782898in}}{\pgfqpoint{3.841062in}{2.767098in}}{\pgfqpoint{3.841062in}{2.750626in}}%
\pgfpathcurveto{\pgfqpoint{3.841062in}{2.734153in}}{\pgfqpoint{3.847606in}{2.718353in}}{\pgfqpoint{3.859254in}{2.706705in}}%
\pgfpathcurveto{\pgfqpoint{3.870902in}{2.695057in}}{\pgfqpoint{3.886702in}{2.688513in}}{\pgfqpoint{3.903175in}{2.688513in}}%
\pgfpathclose%
\pgfusepath{stroke,fill}%
\end{pgfscope}%
\begin{pgfscope}%
\pgfpathrectangle{\pgfqpoint{1.000000in}{0.600000in}}{\pgfqpoint{6.200000in}{4.800000in}} %
\pgfusepath{clip}%
\pgfsetbuttcap%
\pgfsetroundjoin%
\definecolor{currentfill}{rgb}{0.000000,0.500000,0.000000}%
\pgfsetfillcolor{currentfill}%
\pgfsetlinewidth{1.003750pt}%
\definecolor{currentstroke}{rgb}{0.000000,0.500000,0.000000}%
\pgfsetstrokecolor{currentstroke}%
\pgfsetdash{}{0pt}%
\pgfpathmoveto{\pgfqpoint{4.518254in}{3.211433in}}%
\pgfpathcurveto{\pgfqpoint{4.534727in}{3.211433in}}{\pgfqpoint{4.550527in}{3.217978in}}{\pgfqpoint{4.562174in}{3.229626in}}%
\pgfpathcurveto{\pgfqpoint{4.573822in}{3.241274in}}{\pgfqpoint{4.580367in}{3.257074in}}{\pgfqpoint{4.580367in}{3.273546in}}%
\pgfpathcurveto{\pgfqpoint{4.580367in}{3.290019in}}{\pgfqpoint{4.573822in}{3.305819in}}{\pgfqpoint{4.562174in}{3.317467in}}%
\pgfpathcurveto{\pgfqpoint{4.550527in}{3.329115in}}{\pgfqpoint{4.534727in}{3.335659in}}{\pgfqpoint{4.518254in}{3.335659in}}%
\pgfpathcurveto{\pgfqpoint{4.501781in}{3.335659in}}{\pgfqpoint{4.485981in}{3.329115in}}{\pgfqpoint{4.474333in}{3.317467in}}%
\pgfpathcurveto{\pgfqpoint{4.462686in}{3.305819in}}{\pgfqpoint{4.456141in}{3.290019in}}{\pgfqpoint{4.456141in}{3.273546in}}%
\pgfpathcurveto{\pgfqpoint{4.456141in}{3.257074in}}{\pgfqpoint{4.462686in}{3.241274in}}{\pgfqpoint{4.474333in}{3.229626in}}%
\pgfpathcurveto{\pgfqpoint{4.485981in}{3.217978in}}{\pgfqpoint{4.501781in}{3.211433in}}{\pgfqpoint{4.518254in}{3.211433in}}%
\pgfpathclose%
\pgfusepath{stroke,fill}%
\end{pgfscope}%
\begin{pgfscope}%
\pgfpathrectangle{\pgfqpoint{1.000000in}{0.600000in}}{\pgfqpoint{6.200000in}{4.800000in}} %
\pgfusepath{clip}%
\pgfsetbuttcap%
\pgfsetroundjoin%
\definecolor{currentfill}{rgb}{0.000000,0.500000,0.000000}%
\pgfsetfillcolor{currentfill}%
\pgfsetlinewidth{1.003750pt}%
\definecolor{currentstroke}{rgb}{0.000000,0.500000,0.000000}%
\pgfsetstrokecolor{currentstroke}%
\pgfsetdash{}{0pt}%
\pgfpathmoveto{\pgfqpoint{5.133333in}{3.743361in}}%
\pgfpathcurveto{\pgfqpoint{5.149806in}{3.743361in}}{\pgfqpoint{5.165606in}{3.749906in}}{\pgfqpoint{5.177254in}{3.761553in}}%
\pgfpathcurveto{\pgfqpoint{5.188902in}{3.773201in}}{\pgfqpoint{5.195446in}{3.789001in}}{\pgfqpoint{5.195446in}{3.805474in}}%
\pgfpathcurveto{\pgfqpoint{5.195446in}{3.821946in}}{\pgfqpoint{5.188902in}{3.837747in}}{\pgfqpoint{5.177254in}{3.849394in}}%
\pgfpathcurveto{\pgfqpoint{5.165606in}{3.861042in}}{\pgfqpoint{5.149806in}{3.867587in}}{\pgfqpoint{5.133333in}{3.867587in}}%
\pgfpathcurveto{\pgfqpoint{5.116861in}{3.867587in}}{\pgfqpoint{5.101061in}{3.861042in}}{\pgfqpoint{5.089413in}{3.849394in}}%
\pgfpathcurveto{\pgfqpoint{5.077765in}{3.837747in}}{\pgfqpoint{5.071220in}{3.821946in}}{\pgfqpoint{5.071220in}{3.805474in}}%
\pgfpathcurveto{\pgfqpoint{5.071220in}{3.789001in}}{\pgfqpoint{5.077765in}{3.773201in}}{\pgfqpoint{5.089413in}{3.761553in}}%
\pgfpathcurveto{\pgfqpoint{5.101061in}{3.749906in}}{\pgfqpoint{5.116861in}{3.743361in}}{\pgfqpoint{5.133333in}{3.743361in}}%
\pgfpathclose%
\pgfusepath{stroke,fill}%
\end{pgfscope}%
\begin{pgfscope}%
\pgfpathrectangle{\pgfqpoint{1.000000in}{0.600000in}}{\pgfqpoint{6.200000in}{4.800000in}} %
\pgfusepath{clip}%
\pgfsetbuttcap%
\pgfsetroundjoin%
\definecolor{currentfill}{rgb}{0.000000,0.500000,0.000000}%
\pgfsetfillcolor{currentfill}%
\pgfsetlinewidth{1.003750pt}%
\definecolor{currentstroke}{rgb}{0.000000,0.500000,0.000000}%
\pgfsetstrokecolor{currentstroke}%
\pgfsetdash{}{0pt}%
\pgfpathmoveto{\pgfqpoint{5.748413in}{4.157951in}}%
\pgfpathcurveto{\pgfqpoint{5.764885in}{4.157951in}}{\pgfqpoint{5.780685in}{4.164496in}}{\pgfqpoint{5.792333in}{4.176144in}}%
\pgfpathcurveto{\pgfqpoint{5.803981in}{4.187792in}}{\pgfqpoint{5.810526in}{4.203592in}}{\pgfqpoint{5.810526in}{4.220064in}}%
\pgfpathcurveto{\pgfqpoint{5.810526in}{4.236537in}}{\pgfqpoint{5.803981in}{4.252337in}}{\pgfqpoint{5.792333in}{4.263985in}}%
\pgfpathcurveto{\pgfqpoint{5.780685in}{4.275633in}}{\pgfqpoint{5.764885in}{4.282177in}}{\pgfqpoint{5.748413in}{4.282177in}}%
\pgfpathcurveto{\pgfqpoint{5.731940in}{4.282177in}}{\pgfqpoint{5.716140in}{4.275633in}}{\pgfqpoint{5.704492in}{4.263985in}}%
\pgfpathcurveto{\pgfqpoint{5.692844in}{4.252337in}}{\pgfqpoint{5.686300in}{4.236537in}}{\pgfqpoint{5.686300in}{4.220064in}}%
\pgfpathcurveto{\pgfqpoint{5.686300in}{4.203592in}}{\pgfqpoint{5.692844in}{4.187792in}}{\pgfqpoint{5.704492in}{4.176144in}}%
\pgfpathcurveto{\pgfqpoint{5.716140in}{4.164496in}}{\pgfqpoint{5.731940in}{4.157951in}}{\pgfqpoint{5.748413in}{4.157951in}}%
\pgfpathclose%
\pgfusepath{stroke,fill}%
\end{pgfscope}%
\begin{pgfscope}%
\pgfpathrectangle{\pgfqpoint{1.000000in}{0.600000in}}{\pgfqpoint{6.200000in}{4.800000in}} %
\pgfusepath{clip}%
\pgfsetbuttcap%
\pgfsetroundjoin%
\definecolor{currentfill}{rgb}{0.000000,0.500000,0.000000}%
\pgfsetfillcolor{currentfill}%
\pgfsetlinewidth{1.003750pt}%
\definecolor{currentstroke}{rgb}{0.000000,0.500000,0.000000}%
\pgfsetstrokecolor{currentstroke}%
\pgfsetdash{}{0pt}%
\pgfpathmoveto{\pgfqpoint{6.363492in}{4.580100in}}%
\pgfpathcurveto{\pgfqpoint{6.379965in}{4.580100in}}{\pgfqpoint{6.395765in}{4.586645in}}{\pgfqpoint{6.407413in}{4.598293in}}%
\pgfpathcurveto{\pgfqpoint{6.419060in}{4.609941in}}{\pgfqpoint{6.425605in}{4.625741in}}{\pgfqpoint{6.425605in}{4.642213in}}%
\pgfpathcurveto{\pgfqpoint{6.425605in}{4.658686in}}{\pgfqpoint{6.419060in}{4.674486in}}{\pgfqpoint{6.407413in}{4.686134in}}%
\pgfpathcurveto{\pgfqpoint{6.395765in}{4.697782in}}{\pgfqpoint{6.379965in}{4.704326in}}{\pgfqpoint{6.363492in}{4.704326in}}%
\pgfpathcurveto{\pgfqpoint{6.347020in}{4.704326in}}{\pgfqpoint{6.331219in}{4.697782in}}{\pgfqpoint{6.319572in}{4.686134in}}%
\pgfpathcurveto{\pgfqpoint{6.307924in}{4.674486in}}{\pgfqpoint{6.301379in}{4.658686in}}{\pgfqpoint{6.301379in}{4.642213in}}%
\pgfpathcurveto{\pgfqpoint{6.301379in}{4.625741in}}{\pgfqpoint{6.307924in}{4.609941in}}{\pgfqpoint{6.319572in}{4.598293in}}%
\pgfpathcurveto{\pgfqpoint{6.331219in}{4.586645in}}{\pgfqpoint{6.347020in}{4.580100in}}{\pgfqpoint{6.363492in}{4.580100in}}%
\pgfpathclose%
\pgfusepath{stroke,fill}%
\end{pgfscope}%
\begin{pgfscope}%
\pgfpathrectangle{\pgfqpoint{1.000000in}{0.600000in}}{\pgfqpoint{6.200000in}{4.800000in}} %
\pgfusepath{clip}%
\pgfsetbuttcap%
\pgfsetroundjoin%
\definecolor{currentfill}{rgb}{0.000000,0.500000,0.000000}%
\pgfsetfillcolor{currentfill}%
\pgfsetlinewidth{1.003750pt}%
\definecolor{currentstroke}{rgb}{0.000000,0.500000,0.000000}%
\pgfsetstrokecolor{currentstroke}%
\pgfsetdash{}{0pt}%
\pgfpathmoveto{\pgfqpoint{6.978571in}{5.035125in}}%
\pgfpathcurveto{\pgfqpoint{6.995044in}{5.035125in}}{\pgfqpoint{7.010844in}{5.041670in}}{\pgfqpoint{7.022492in}{5.053318in}}%
\pgfpathcurveto{\pgfqpoint{7.034140in}{5.064966in}}{\pgfqpoint{7.040684in}{5.080766in}}{\pgfqpoint{7.040684in}{5.097238in}}%
\pgfpathcurveto{\pgfqpoint{7.040684in}{5.113711in}}{\pgfqpoint{7.034140in}{5.129511in}}{\pgfqpoint{7.022492in}{5.141159in}}%
\pgfpathcurveto{\pgfqpoint{7.010844in}{5.152807in}}{\pgfqpoint{6.995044in}{5.159351in}}{\pgfqpoint{6.978571in}{5.159351in}}%
\pgfpathcurveto{\pgfqpoint{6.962099in}{5.159351in}}{\pgfqpoint{6.946299in}{5.152807in}}{\pgfqpoint{6.934651in}{5.141159in}}%
\pgfpathcurveto{\pgfqpoint{6.923003in}{5.129511in}}{\pgfqpoint{6.916458in}{5.113711in}}{\pgfqpoint{6.916458in}{5.097238in}}%
\pgfpathcurveto{\pgfqpoint{6.916458in}{5.080766in}}{\pgfqpoint{6.923003in}{5.064966in}}{\pgfqpoint{6.934651in}{5.053318in}}%
\pgfpathcurveto{\pgfqpoint{6.946299in}{5.041670in}}{\pgfqpoint{6.962099in}{5.035125in}}{\pgfqpoint{6.978571in}{5.035125in}}%
\pgfpathclose%
\pgfusepath{stroke,fill}%
\end{pgfscope}%
\begin{pgfscope}%
\pgfpathrectangle{\pgfqpoint{1.000000in}{0.600000in}}{\pgfqpoint{6.200000in}{4.800000in}} %
\pgfusepath{clip}%
\pgfsetrectcap%
\pgfsetroundjoin%
\pgfsetlinewidth{2.509375pt}%
\definecolor{currentstroke}{rgb}{1.000000,0.000000,0.000000}%
\pgfsetstrokecolor{currentstroke}%
\pgfsetdash{}{0pt}%
\pgfpathmoveto{\pgfqpoint{1.442857in}{0.725232in}}%
\pgfpathlineto{\pgfqpoint{1.555831in}{1.416418in}}%
\pgfpathlineto{\pgfqpoint{1.668805in}{1.660708in}}%
\pgfpathlineto{\pgfqpoint{1.781778in}{1.662975in}}%
\pgfpathlineto{\pgfqpoint{1.894752in}{1.561820in}}%
\pgfpathlineto{\pgfqpoint{2.007726in}{1.444758in}}%
\pgfpathlineto{\pgfqpoint{2.120700in}{1.361175in}}%
\pgfpathlineto{\pgfqpoint{2.233673in}{1.333160in}}%
\pgfpathlineto{\pgfqpoint{2.346647in}{1.364382in}}%
\pgfpathlineto{\pgfqpoint{2.459621in}{1.447158in}}%
\pgfpathlineto{\pgfqpoint{2.572595in}{1.567893in}}%
\pgfpathlineto{\pgfqpoint{2.685569in}{1.711102in}}%
\pgfpathlineto{\pgfqpoint{2.798542in}{1.862194in}}%
\pgfpathlineto{\pgfqpoint{2.911516in}{2.009215in}}%
\pgfpathlineto{\pgfqpoint{3.024490in}{2.143744in}}%
\pgfpathlineto{\pgfqpoint{3.137464in}{2.261127in}}%
\pgfpathlineto{\pgfqpoint{3.250437in}{2.360215in}}%
\pgfpathlineto{\pgfqpoint{3.363411in}{2.442753in}}%
\pgfpathlineto{\pgfqpoint{3.476385in}{2.512587in}}%
\pgfpathlineto{\pgfqpoint{3.589359in}{2.574770in}}%
\pgfpathlineto{\pgfqpoint{3.702332in}{2.634707in}}%
\pgfpathlineto{\pgfqpoint{3.815306in}{2.697387in}}%
\pgfpathlineto{\pgfqpoint{3.928280in}{2.766783in}}%
\pgfpathlineto{\pgfqpoint{4.041254in}{2.845441in}}%
\pgfpathlineto{\pgfqpoint{4.154227in}{2.934293in}}%
\pgfpathlineto{\pgfqpoint{4.267201in}{3.032671in}}%
\pgfpathlineto{\pgfqpoint{4.380175in}{3.138526in}}%
\pgfpathlineto{\pgfqpoint{4.493149in}{3.248799in}}%
\pgfpathlineto{\pgfqpoint{4.606122in}{3.359905in}}%
\pgfpathlineto{\pgfqpoint{4.719096in}{3.468277in}}%
\pgfpathlineto{\pgfqpoint{4.832070in}{3.570897in}}%
\pgfpathlineto{\pgfqpoint{4.945044in}{3.665759in}}%
\pgfpathlineto{\pgfqpoint{5.058017in}{3.752194in}}%
\pgfpathlineto{\pgfqpoint{5.170991in}{3.831008in}}%
\pgfpathlineto{\pgfqpoint{5.283965in}{3.904397in}}%
\pgfpathlineto{\pgfqpoint{5.396939in}{3.975593in}}%
\pgfpathlineto{\pgfqpoint{5.509913in}{4.048286in}}%
\pgfpathlineto{\pgfqpoint{5.622886in}{4.125815in}}%
\pgfpathlineto{\pgfqpoint{5.735860in}{4.210239in}}%
\pgfpathlineto{\pgfqpoint{5.848834in}{4.301396in}}%
\pgfpathlineto{\pgfqpoint{5.961808in}{4.396140in}}%
\pgfpathlineto{\pgfqpoint{6.074781in}{4.488000in}}%
\pgfpathlineto{\pgfqpoint{6.187755in}{4.567590in}}%
\pgfpathlineto{\pgfqpoint{6.300729in}{4.624183in}}%
\pgfpathlineto{\pgfqpoint{6.413703in}{4.648937in}}%
\pgfpathlineto{\pgfqpoint{6.526676in}{4.640410in}}%
\pgfpathlineto{\pgfqpoint{6.639650in}{4.613081in}}%
\pgfpathlineto{\pgfqpoint{6.752624in}{4.609733in}}%
\pgfpathlineto{\pgfqpoint{6.865598in}{4.718715in}}%
\pgfpathlineto{\pgfqpoint{6.978571in}{5.097238in}}%
\pgfusepath{stroke}%
\end{pgfscope}%
\begin{pgfscope}%
\pgfsetrectcap%
\pgfsetmiterjoin%
\pgfsetlinewidth{1.003750pt}%
\definecolor{currentstroke}{rgb}{0.000000,0.000000,0.000000}%
\pgfsetstrokecolor{currentstroke}%
\pgfsetdash{}{0pt}%
\pgfpathmoveto{\pgfqpoint{1.000000in}{5.400000in}}%
\pgfpathlineto{\pgfqpoint{7.200000in}{5.400000in}}%
\pgfusepath{stroke}%
\end{pgfscope}%
\begin{pgfscope}%
\pgfsetrectcap%
\pgfsetmiterjoin%
\pgfsetlinewidth{1.003750pt}%
\definecolor{currentstroke}{rgb}{0.000000,0.000000,0.000000}%
\pgfsetstrokecolor{currentstroke}%
\pgfsetdash{}{0pt}%
\pgfpathmoveto{\pgfqpoint{1.000000in}{0.600000in}}%
\pgfpathlineto{\pgfqpoint{1.000000in}{5.400000in}}%
\pgfusepath{stroke}%
\end{pgfscope}%
\begin{pgfscope}%
\pgfsetrectcap%
\pgfsetmiterjoin%
\pgfsetlinewidth{1.003750pt}%
\definecolor{currentstroke}{rgb}{0.000000,0.000000,0.000000}%
\pgfsetstrokecolor{currentstroke}%
\pgfsetdash{}{0pt}%
\pgfpathmoveto{\pgfqpoint{1.000000in}{0.600000in}}%
\pgfpathlineto{\pgfqpoint{7.200000in}{0.600000in}}%
\pgfusepath{stroke}%
\end{pgfscope}%
\begin{pgfscope}%
\pgfsetrectcap%
\pgfsetmiterjoin%
\pgfsetlinewidth{1.003750pt}%
\definecolor{currentstroke}{rgb}{0.000000,0.000000,0.000000}%
\pgfsetstrokecolor{currentstroke}%
\pgfsetdash{}{0pt}%
\pgfpathmoveto{\pgfqpoint{7.200000in}{0.600000in}}%
\pgfpathlineto{\pgfqpoint{7.200000in}{5.400000in}}%
\pgfusepath{stroke}%
\end{pgfscope}%
\begin{pgfscope}%
\pgfsetbuttcap%
\pgfsetroundjoin%
\definecolor{currentfill}{rgb}{0.000000,0.000000,0.000000}%
\pgfsetfillcolor{currentfill}%
\pgfsetlinewidth{0.501875pt}%
\definecolor{currentstroke}{rgb}{0.000000,0.000000,0.000000}%
\pgfsetstrokecolor{currentstroke}%
\pgfsetdash{}{0pt}%
\pgfsys@defobject{currentmarker}{\pgfqpoint{0.000000in}{0.000000in}}{\pgfqpoint{0.000000in}{0.055556in}}{%
\pgfpathmoveto{\pgfqpoint{0.000000in}{0.000000in}}%
\pgfpathlineto{\pgfqpoint{0.000000in}{0.055556in}}%
\pgfusepath{stroke,fill}%
}%
\begin{pgfscope}%
\pgfsys@transformshift{1.442857in}{0.600000in}%
\pgfsys@useobject{currentmarker}{}%
\end{pgfscope}%
\end{pgfscope}%
\begin{pgfscope}%
\pgfsetbuttcap%
\pgfsetroundjoin%
\definecolor{currentfill}{rgb}{0.000000,0.000000,0.000000}%
\pgfsetfillcolor{currentfill}%
\pgfsetlinewidth{0.501875pt}%
\definecolor{currentstroke}{rgb}{0.000000,0.000000,0.000000}%
\pgfsetstrokecolor{currentstroke}%
\pgfsetdash{}{0pt}%
\pgfsys@defobject{currentmarker}{\pgfqpoint{0.000000in}{-0.055556in}}{\pgfqpoint{0.000000in}{0.000000in}}{%
\pgfpathmoveto{\pgfqpoint{0.000000in}{0.000000in}}%
\pgfpathlineto{\pgfqpoint{0.000000in}{-0.055556in}}%
\pgfusepath{stroke,fill}%
}%
\begin{pgfscope}%
\pgfsys@transformshift{1.442857in}{5.400000in}%
\pgfsys@useobject{currentmarker}{}%
\end{pgfscope}%
\end{pgfscope}%
\begin{pgfscope}%
\pgftext[x=1.442857in,y=0.544444in,,top]{\sffamily\fontsize{12.000000}{14.400000}\selectfont 0}%
\end{pgfscope}%
\begin{pgfscope}%
\pgfsetbuttcap%
\pgfsetroundjoin%
\definecolor{currentfill}{rgb}{0.000000,0.000000,0.000000}%
\pgfsetfillcolor{currentfill}%
\pgfsetlinewidth{0.501875pt}%
\definecolor{currentstroke}{rgb}{0.000000,0.000000,0.000000}%
\pgfsetstrokecolor{currentstroke}%
\pgfsetdash{}{0pt}%
\pgfsys@defobject{currentmarker}{\pgfqpoint{0.000000in}{0.000000in}}{\pgfqpoint{0.000000in}{0.055556in}}{%
\pgfpathmoveto{\pgfqpoint{0.000000in}{0.000000in}}%
\pgfpathlineto{\pgfqpoint{0.000000in}{0.055556in}}%
\pgfusepath{stroke,fill}%
}%
\begin{pgfscope}%
\pgfsys@transformshift{2.550000in}{0.600000in}%
\pgfsys@useobject{currentmarker}{}%
\end{pgfscope}%
\end{pgfscope}%
\begin{pgfscope}%
\pgfsetbuttcap%
\pgfsetroundjoin%
\definecolor{currentfill}{rgb}{0.000000,0.000000,0.000000}%
\pgfsetfillcolor{currentfill}%
\pgfsetlinewidth{0.501875pt}%
\definecolor{currentstroke}{rgb}{0.000000,0.000000,0.000000}%
\pgfsetstrokecolor{currentstroke}%
\pgfsetdash{}{0pt}%
\pgfsys@defobject{currentmarker}{\pgfqpoint{0.000000in}{-0.055556in}}{\pgfqpoint{0.000000in}{0.000000in}}{%
\pgfpathmoveto{\pgfqpoint{0.000000in}{0.000000in}}%
\pgfpathlineto{\pgfqpoint{0.000000in}{-0.055556in}}%
\pgfusepath{stroke,fill}%
}%
\begin{pgfscope}%
\pgfsys@transformshift{2.550000in}{5.400000in}%
\pgfsys@useobject{currentmarker}{}%
\end{pgfscope}%
\end{pgfscope}%
\begin{pgfscope}%
\pgftext[x=2.550000in,y=0.544444in,,top]{\sffamily\fontsize{12.000000}{14.400000}\selectfont 1}%
\end{pgfscope}%
\begin{pgfscope}%
\pgfsetbuttcap%
\pgfsetroundjoin%
\definecolor{currentfill}{rgb}{0.000000,0.000000,0.000000}%
\pgfsetfillcolor{currentfill}%
\pgfsetlinewidth{0.501875pt}%
\definecolor{currentstroke}{rgb}{0.000000,0.000000,0.000000}%
\pgfsetstrokecolor{currentstroke}%
\pgfsetdash{}{0pt}%
\pgfsys@defobject{currentmarker}{\pgfqpoint{0.000000in}{0.000000in}}{\pgfqpoint{0.000000in}{0.055556in}}{%
\pgfpathmoveto{\pgfqpoint{0.000000in}{0.000000in}}%
\pgfpathlineto{\pgfqpoint{0.000000in}{0.055556in}}%
\pgfusepath{stroke,fill}%
}%
\begin{pgfscope}%
\pgfsys@transformshift{3.657143in}{0.600000in}%
\pgfsys@useobject{currentmarker}{}%
\end{pgfscope}%
\end{pgfscope}%
\begin{pgfscope}%
\pgfsetbuttcap%
\pgfsetroundjoin%
\definecolor{currentfill}{rgb}{0.000000,0.000000,0.000000}%
\pgfsetfillcolor{currentfill}%
\pgfsetlinewidth{0.501875pt}%
\definecolor{currentstroke}{rgb}{0.000000,0.000000,0.000000}%
\pgfsetstrokecolor{currentstroke}%
\pgfsetdash{}{0pt}%
\pgfsys@defobject{currentmarker}{\pgfqpoint{0.000000in}{-0.055556in}}{\pgfqpoint{0.000000in}{0.000000in}}{%
\pgfpathmoveto{\pgfqpoint{0.000000in}{0.000000in}}%
\pgfpathlineto{\pgfqpoint{0.000000in}{-0.055556in}}%
\pgfusepath{stroke,fill}%
}%
\begin{pgfscope}%
\pgfsys@transformshift{3.657143in}{5.400000in}%
\pgfsys@useobject{currentmarker}{}%
\end{pgfscope}%
\end{pgfscope}%
\begin{pgfscope}%
\pgftext[x=3.657143in,y=0.544444in,,top]{\sffamily\fontsize{12.000000}{14.400000}\selectfont 2}%
\end{pgfscope}%
\begin{pgfscope}%
\pgfsetbuttcap%
\pgfsetroundjoin%
\definecolor{currentfill}{rgb}{0.000000,0.000000,0.000000}%
\pgfsetfillcolor{currentfill}%
\pgfsetlinewidth{0.501875pt}%
\definecolor{currentstroke}{rgb}{0.000000,0.000000,0.000000}%
\pgfsetstrokecolor{currentstroke}%
\pgfsetdash{}{0pt}%
\pgfsys@defobject{currentmarker}{\pgfqpoint{0.000000in}{0.000000in}}{\pgfqpoint{0.000000in}{0.055556in}}{%
\pgfpathmoveto{\pgfqpoint{0.000000in}{0.000000in}}%
\pgfpathlineto{\pgfqpoint{0.000000in}{0.055556in}}%
\pgfusepath{stroke,fill}%
}%
\begin{pgfscope}%
\pgfsys@transformshift{4.764286in}{0.600000in}%
\pgfsys@useobject{currentmarker}{}%
\end{pgfscope}%
\end{pgfscope}%
\begin{pgfscope}%
\pgfsetbuttcap%
\pgfsetroundjoin%
\definecolor{currentfill}{rgb}{0.000000,0.000000,0.000000}%
\pgfsetfillcolor{currentfill}%
\pgfsetlinewidth{0.501875pt}%
\definecolor{currentstroke}{rgb}{0.000000,0.000000,0.000000}%
\pgfsetstrokecolor{currentstroke}%
\pgfsetdash{}{0pt}%
\pgfsys@defobject{currentmarker}{\pgfqpoint{0.000000in}{-0.055556in}}{\pgfqpoint{0.000000in}{0.000000in}}{%
\pgfpathmoveto{\pgfqpoint{0.000000in}{0.000000in}}%
\pgfpathlineto{\pgfqpoint{0.000000in}{-0.055556in}}%
\pgfusepath{stroke,fill}%
}%
\begin{pgfscope}%
\pgfsys@transformshift{4.764286in}{5.400000in}%
\pgfsys@useobject{currentmarker}{}%
\end{pgfscope}%
\end{pgfscope}%
\begin{pgfscope}%
\pgftext[x=4.764286in,y=0.544444in,,top]{\sffamily\fontsize{12.000000}{14.400000}\selectfont 3}%
\end{pgfscope}%
\begin{pgfscope}%
\pgfsetbuttcap%
\pgfsetroundjoin%
\definecolor{currentfill}{rgb}{0.000000,0.000000,0.000000}%
\pgfsetfillcolor{currentfill}%
\pgfsetlinewidth{0.501875pt}%
\definecolor{currentstroke}{rgb}{0.000000,0.000000,0.000000}%
\pgfsetstrokecolor{currentstroke}%
\pgfsetdash{}{0pt}%
\pgfsys@defobject{currentmarker}{\pgfqpoint{0.000000in}{0.000000in}}{\pgfqpoint{0.000000in}{0.055556in}}{%
\pgfpathmoveto{\pgfqpoint{0.000000in}{0.000000in}}%
\pgfpathlineto{\pgfqpoint{0.000000in}{0.055556in}}%
\pgfusepath{stroke,fill}%
}%
\begin{pgfscope}%
\pgfsys@transformshift{5.871429in}{0.600000in}%
\pgfsys@useobject{currentmarker}{}%
\end{pgfscope}%
\end{pgfscope}%
\begin{pgfscope}%
\pgfsetbuttcap%
\pgfsetroundjoin%
\definecolor{currentfill}{rgb}{0.000000,0.000000,0.000000}%
\pgfsetfillcolor{currentfill}%
\pgfsetlinewidth{0.501875pt}%
\definecolor{currentstroke}{rgb}{0.000000,0.000000,0.000000}%
\pgfsetstrokecolor{currentstroke}%
\pgfsetdash{}{0pt}%
\pgfsys@defobject{currentmarker}{\pgfqpoint{0.000000in}{-0.055556in}}{\pgfqpoint{0.000000in}{0.000000in}}{%
\pgfpathmoveto{\pgfqpoint{0.000000in}{0.000000in}}%
\pgfpathlineto{\pgfqpoint{0.000000in}{-0.055556in}}%
\pgfusepath{stroke,fill}%
}%
\begin{pgfscope}%
\pgfsys@transformshift{5.871429in}{5.400000in}%
\pgfsys@useobject{currentmarker}{}%
\end{pgfscope}%
\end{pgfscope}%
\begin{pgfscope}%
\pgftext[x=5.871429in,y=0.544444in,,top]{\sffamily\fontsize{12.000000}{14.400000}\selectfont 4}%
\end{pgfscope}%
\begin{pgfscope}%
\pgfsetbuttcap%
\pgfsetroundjoin%
\definecolor{currentfill}{rgb}{0.000000,0.000000,0.000000}%
\pgfsetfillcolor{currentfill}%
\pgfsetlinewidth{0.501875pt}%
\definecolor{currentstroke}{rgb}{0.000000,0.000000,0.000000}%
\pgfsetstrokecolor{currentstroke}%
\pgfsetdash{}{0pt}%
\pgfsys@defobject{currentmarker}{\pgfqpoint{0.000000in}{0.000000in}}{\pgfqpoint{0.000000in}{0.055556in}}{%
\pgfpathmoveto{\pgfqpoint{0.000000in}{0.000000in}}%
\pgfpathlineto{\pgfqpoint{0.000000in}{0.055556in}}%
\pgfusepath{stroke,fill}%
}%
\begin{pgfscope}%
\pgfsys@transformshift{6.978571in}{0.600000in}%
\pgfsys@useobject{currentmarker}{}%
\end{pgfscope}%
\end{pgfscope}%
\begin{pgfscope}%
\pgfsetbuttcap%
\pgfsetroundjoin%
\definecolor{currentfill}{rgb}{0.000000,0.000000,0.000000}%
\pgfsetfillcolor{currentfill}%
\pgfsetlinewidth{0.501875pt}%
\definecolor{currentstroke}{rgb}{0.000000,0.000000,0.000000}%
\pgfsetstrokecolor{currentstroke}%
\pgfsetdash{}{0pt}%
\pgfsys@defobject{currentmarker}{\pgfqpoint{0.000000in}{-0.055556in}}{\pgfqpoint{0.000000in}{0.000000in}}{%
\pgfpathmoveto{\pgfqpoint{0.000000in}{0.000000in}}%
\pgfpathlineto{\pgfqpoint{0.000000in}{-0.055556in}}%
\pgfusepath{stroke,fill}%
}%
\begin{pgfscope}%
\pgfsys@transformshift{6.978571in}{5.400000in}%
\pgfsys@useobject{currentmarker}{}%
\end{pgfscope}%
\end{pgfscope}%
\begin{pgfscope}%
\pgftext[x=6.978571in,y=0.544444in,,top]{\sffamily\fontsize{12.000000}{14.400000}\selectfont 5}%
\end{pgfscope}%
\begin{pgfscope}%
\pgftext[x=4.100000in,y=0.313705in,,top]{\sffamily\fontsize{12.000000}{14.400000}\selectfont x}%
\end{pgfscope}%
\begin{pgfscope}%
\pgfsetbuttcap%
\pgfsetroundjoin%
\definecolor{currentfill}{rgb}{0.000000,0.000000,0.000000}%
\pgfsetfillcolor{currentfill}%
\pgfsetlinewidth{0.501875pt}%
\definecolor{currentstroke}{rgb}{0.000000,0.000000,0.000000}%
\pgfsetstrokecolor{currentstroke}%
\pgfsetdash{}{0pt}%
\pgfsys@defobject{currentmarker}{\pgfqpoint{0.000000in}{0.000000in}}{\pgfqpoint{0.055556in}{0.000000in}}{%
\pgfpathmoveto{\pgfqpoint{0.000000in}{0.000000in}}%
\pgfpathlineto{\pgfqpoint{0.055556in}{0.000000in}}%
\pgfusepath{stroke,fill}%
}%
\begin{pgfscope}%
\pgfsys@transformshift{1.000000in}{0.660380in}%
\pgfsys@useobject{currentmarker}{}%
\end{pgfscope}%
\end{pgfscope}%
\begin{pgfscope}%
\pgfsetbuttcap%
\pgfsetroundjoin%
\definecolor{currentfill}{rgb}{0.000000,0.000000,0.000000}%
\pgfsetfillcolor{currentfill}%
\pgfsetlinewidth{0.501875pt}%
\definecolor{currentstroke}{rgb}{0.000000,0.000000,0.000000}%
\pgfsetstrokecolor{currentstroke}%
\pgfsetdash{}{0pt}%
\pgfsys@defobject{currentmarker}{\pgfqpoint{-0.055556in}{0.000000in}}{\pgfqpoint{0.000000in}{0.000000in}}{%
\pgfpathmoveto{\pgfqpoint{0.000000in}{0.000000in}}%
\pgfpathlineto{\pgfqpoint{-0.055556in}{0.000000in}}%
\pgfusepath{stroke,fill}%
}%
\begin{pgfscope}%
\pgfsys@transformshift{7.200000in}{0.660380in}%
\pgfsys@useobject{currentmarker}{}%
\end{pgfscope}%
\end{pgfscope}%
\begin{pgfscope}%
\pgftext[x=0.944444in,y=0.660380in,right,]{\sffamily\fontsize{12.000000}{14.400000}\selectfont 1}%
\end{pgfscope}%
\begin{pgfscope}%
\pgfsetbuttcap%
\pgfsetroundjoin%
\definecolor{currentfill}{rgb}{0.000000,0.000000,0.000000}%
\pgfsetfillcolor{currentfill}%
\pgfsetlinewidth{0.501875pt}%
\definecolor{currentstroke}{rgb}{0.000000,0.000000,0.000000}%
\pgfsetstrokecolor{currentstroke}%
\pgfsetdash{}{0pt}%
\pgfsys@defobject{currentmarker}{\pgfqpoint{0.000000in}{0.000000in}}{\pgfqpoint{0.055556in}{0.000000in}}{%
\pgfpathmoveto{\pgfqpoint{0.000000in}{0.000000in}}%
\pgfpathlineto{\pgfqpoint{0.055556in}{0.000000in}}%
\pgfusepath{stroke,fill}%
}%
\begin{pgfscope}%
\pgfsys@transformshift{1.000000in}{1.244650in}%
\pgfsys@useobject{currentmarker}{}%
\end{pgfscope}%
\end{pgfscope}%
\begin{pgfscope}%
\pgfsetbuttcap%
\pgfsetroundjoin%
\definecolor{currentfill}{rgb}{0.000000,0.000000,0.000000}%
\pgfsetfillcolor{currentfill}%
\pgfsetlinewidth{0.501875pt}%
\definecolor{currentstroke}{rgb}{0.000000,0.000000,0.000000}%
\pgfsetstrokecolor{currentstroke}%
\pgfsetdash{}{0pt}%
\pgfsys@defobject{currentmarker}{\pgfqpoint{-0.055556in}{0.000000in}}{\pgfqpoint{0.000000in}{0.000000in}}{%
\pgfpathmoveto{\pgfqpoint{0.000000in}{0.000000in}}%
\pgfpathlineto{\pgfqpoint{-0.055556in}{0.000000in}}%
\pgfusepath{stroke,fill}%
}%
\begin{pgfscope}%
\pgfsys@transformshift{7.200000in}{1.244650in}%
\pgfsys@useobject{currentmarker}{}%
\end{pgfscope}%
\end{pgfscope}%
\begin{pgfscope}%
\pgftext[x=0.944444in,y=1.244650in,right,]{\sffamily\fontsize{12.000000}{14.400000}\selectfont 2}%
\end{pgfscope}%
\begin{pgfscope}%
\pgfsetbuttcap%
\pgfsetroundjoin%
\definecolor{currentfill}{rgb}{0.000000,0.000000,0.000000}%
\pgfsetfillcolor{currentfill}%
\pgfsetlinewidth{0.501875pt}%
\definecolor{currentstroke}{rgb}{0.000000,0.000000,0.000000}%
\pgfsetstrokecolor{currentstroke}%
\pgfsetdash{}{0pt}%
\pgfsys@defobject{currentmarker}{\pgfqpoint{0.000000in}{0.000000in}}{\pgfqpoint{0.055556in}{0.000000in}}{%
\pgfpathmoveto{\pgfqpoint{0.000000in}{0.000000in}}%
\pgfpathlineto{\pgfqpoint{0.055556in}{0.000000in}}%
\pgfusepath{stroke,fill}%
}%
\begin{pgfscope}%
\pgfsys@transformshift{1.000000in}{1.828919in}%
\pgfsys@useobject{currentmarker}{}%
\end{pgfscope}%
\end{pgfscope}%
\begin{pgfscope}%
\pgfsetbuttcap%
\pgfsetroundjoin%
\definecolor{currentfill}{rgb}{0.000000,0.000000,0.000000}%
\pgfsetfillcolor{currentfill}%
\pgfsetlinewidth{0.501875pt}%
\definecolor{currentstroke}{rgb}{0.000000,0.000000,0.000000}%
\pgfsetstrokecolor{currentstroke}%
\pgfsetdash{}{0pt}%
\pgfsys@defobject{currentmarker}{\pgfqpoint{-0.055556in}{0.000000in}}{\pgfqpoint{0.000000in}{0.000000in}}{%
\pgfpathmoveto{\pgfqpoint{0.000000in}{0.000000in}}%
\pgfpathlineto{\pgfqpoint{-0.055556in}{0.000000in}}%
\pgfusepath{stroke,fill}%
}%
\begin{pgfscope}%
\pgfsys@transformshift{7.200000in}{1.828919in}%
\pgfsys@useobject{currentmarker}{}%
\end{pgfscope}%
\end{pgfscope}%
\begin{pgfscope}%
\pgftext[x=0.944444in,y=1.828919in,right,]{\sffamily\fontsize{12.000000}{14.400000}\selectfont 3}%
\end{pgfscope}%
\begin{pgfscope}%
\pgfsetbuttcap%
\pgfsetroundjoin%
\definecolor{currentfill}{rgb}{0.000000,0.000000,0.000000}%
\pgfsetfillcolor{currentfill}%
\pgfsetlinewidth{0.501875pt}%
\definecolor{currentstroke}{rgb}{0.000000,0.000000,0.000000}%
\pgfsetstrokecolor{currentstroke}%
\pgfsetdash{}{0pt}%
\pgfsys@defobject{currentmarker}{\pgfqpoint{0.000000in}{0.000000in}}{\pgfqpoint{0.055556in}{0.000000in}}{%
\pgfpathmoveto{\pgfqpoint{0.000000in}{0.000000in}}%
\pgfpathlineto{\pgfqpoint{0.055556in}{0.000000in}}%
\pgfusepath{stroke,fill}%
}%
\begin{pgfscope}%
\pgfsys@transformshift{1.000000in}{2.413188in}%
\pgfsys@useobject{currentmarker}{}%
\end{pgfscope}%
\end{pgfscope}%
\begin{pgfscope}%
\pgfsetbuttcap%
\pgfsetroundjoin%
\definecolor{currentfill}{rgb}{0.000000,0.000000,0.000000}%
\pgfsetfillcolor{currentfill}%
\pgfsetlinewidth{0.501875pt}%
\definecolor{currentstroke}{rgb}{0.000000,0.000000,0.000000}%
\pgfsetstrokecolor{currentstroke}%
\pgfsetdash{}{0pt}%
\pgfsys@defobject{currentmarker}{\pgfqpoint{-0.055556in}{0.000000in}}{\pgfqpoint{0.000000in}{0.000000in}}{%
\pgfpathmoveto{\pgfqpoint{0.000000in}{0.000000in}}%
\pgfpathlineto{\pgfqpoint{-0.055556in}{0.000000in}}%
\pgfusepath{stroke,fill}%
}%
\begin{pgfscope}%
\pgfsys@transformshift{7.200000in}{2.413188in}%
\pgfsys@useobject{currentmarker}{}%
\end{pgfscope}%
\end{pgfscope}%
\begin{pgfscope}%
\pgftext[x=0.944444in,y=2.413188in,right,]{\sffamily\fontsize{12.000000}{14.400000}\selectfont 4}%
\end{pgfscope}%
\begin{pgfscope}%
\pgfsetbuttcap%
\pgfsetroundjoin%
\definecolor{currentfill}{rgb}{0.000000,0.000000,0.000000}%
\pgfsetfillcolor{currentfill}%
\pgfsetlinewidth{0.501875pt}%
\definecolor{currentstroke}{rgb}{0.000000,0.000000,0.000000}%
\pgfsetstrokecolor{currentstroke}%
\pgfsetdash{}{0pt}%
\pgfsys@defobject{currentmarker}{\pgfqpoint{0.000000in}{0.000000in}}{\pgfqpoint{0.055556in}{0.000000in}}{%
\pgfpathmoveto{\pgfqpoint{0.000000in}{0.000000in}}%
\pgfpathlineto{\pgfqpoint{0.055556in}{0.000000in}}%
\pgfusepath{stroke,fill}%
}%
\begin{pgfscope}%
\pgfsys@transformshift{1.000000in}{2.997457in}%
\pgfsys@useobject{currentmarker}{}%
\end{pgfscope}%
\end{pgfscope}%
\begin{pgfscope}%
\pgfsetbuttcap%
\pgfsetroundjoin%
\definecolor{currentfill}{rgb}{0.000000,0.000000,0.000000}%
\pgfsetfillcolor{currentfill}%
\pgfsetlinewidth{0.501875pt}%
\definecolor{currentstroke}{rgb}{0.000000,0.000000,0.000000}%
\pgfsetstrokecolor{currentstroke}%
\pgfsetdash{}{0pt}%
\pgfsys@defobject{currentmarker}{\pgfqpoint{-0.055556in}{0.000000in}}{\pgfqpoint{0.000000in}{0.000000in}}{%
\pgfpathmoveto{\pgfqpoint{0.000000in}{0.000000in}}%
\pgfpathlineto{\pgfqpoint{-0.055556in}{0.000000in}}%
\pgfusepath{stroke,fill}%
}%
\begin{pgfscope}%
\pgfsys@transformshift{7.200000in}{2.997457in}%
\pgfsys@useobject{currentmarker}{}%
\end{pgfscope}%
\end{pgfscope}%
\begin{pgfscope}%
\pgftext[x=0.944444in,y=2.997457in,right,]{\sffamily\fontsize{12.000000}{14.400000}\selectfont 5}%
\end{pgfscope}%
\begin{pgfscope}%
\pgfsetbuttcap%
\pgfsetroundjoin%
\definecolor{currentfill}{rgb}{0.000000,0.000000,0.000000}%
\pgfsetfillcolor{currentfill}%
\pgfsetlinewidth{0.501875pt}%
\definecolor{currentstroke}{rgb}{0.000000,0.000000,0.000000}%
\pgfsetstrokecolor{currentstroke}%
\pgfsetdash{}{0pt}%
\pgfsys@defobject{currentmarker}{\pgfqpoint{0.000000in}{0.000000in}}{\pgfqpoint{0.055556in}{0.000000in}}{%
\pgfpathmoveto{\pgfqpoint{0.000000in}{0.000000in}}%
\pgfpathlineto{\pgfqpoint{0.055556in}{0.000000in}}%
\pgfusepath{stroke,fill}%
}%
\begin{pgfscope}%
\pgfsys@transformshift{1.000000in}{3.581726in}%
\pgfsys@useobject{currentmarker}{}%
\end{pgfscope}%
\end{pgfscope}%
\begin{pgfscope}%
\pgfsetbuttcap%
\pgfsetroundjoin%
\definecolor{currentfill}{rgb}{0.000000,0.000000,0.000000}%
\pgfsetfillcolor{currentfill}%
\pgfsetlinewidth{0.501875pt}%
\definecolor{currentstroke}{rgb}{0.000000,0.000000,0.000000}%
\pgfsetstrokecolor{currentstroke}%
\pgfsetdash{}{0pt}%
\pgfsys@defobject{currentmarker}{\pgfqpoint{-0.055556in}{0.000000in}}{\pgfqpoint{0.000000in}{0.000000in}}{%
\pgfpathmoveto{\pgfqpoint{0.000000in}{0.000000in}}%
\pgfpathlineto{\pgfqpoint{-0.055556in}{0.000000in}}%
\pgfusepath{stroke,fill}%
}%
\begin{pgfscope}%
\pgfsys@transformshift{7.200000in}{3.581726in}%
\pgfsys@useobject{currentmarker}{}%
\end{pgfscope}%
\end{pgfscope}%
\begin{pgfscope}%
\pgftext[x=0.944444in,y=3.581726in,right,]{\sffamily\fontsize{12.000000}{14.400000}\selectfont 6}%
\end{pgfscope}%
\begin{pgfscope}%
\pgfsetbuttcap%
\pgfsetroundjoin%
\definecolor{currentfill}{rgb}{0.000000,0.000000,0.000000}%
\pgfsetfillcolor{currentfill}%
\pgfsetlinewidth{0.501875pt}%
\definecolor{currentstroke}{rgb}{0.000000,0.000000,0.000000}%
\pgfsetstrokecolor{currentstroke}%
\pgfsetdash{}{0pt}%
\pgfsys@defobject{currentmarker}{\pgfqpoint{0.000000in}{0.000000in}}{\pgfqpoint{0.055556in}{0.000000in}}{%
\pgfpathmoveto{\pgfqpoint{0.000000in}{0.000000in}}%
\pgfpathlineto{\pgfqpoint{0.055556in}{0.000000in}}%
\pgfusepath{stroke,fill}%
}%
\begin{pgfscope}%
\pgfsys@transformshift{1.000000in}{4.165995in}%
\pgfsys@useobject{currentmarker}{}%
\end{pgfscope}%
\end{pgfscope}%
\begin{pgfscope}%
\pgfsetbuttcap%
\pgfsetroundjoin%
\definecolor{currentfill}{rgb}{0.000000,0.000000,0.000000}%
\pgfsetfillcolor{currentfill}%
\pgfsetlinewidth{0.501875pt}%
\definecolor{currentstroke}{rgb}{0.000000,0.000000,0.000000}%
\pgfsetstrokecolor{currentstroke}%
\pgfsetdash{}{0pt}%
\pgfsys@defobject{currentmarker}{\pgfqpoint{-0.055556in}{0.000000in}}{\pgfqpoint{0.000000in}{0.000000in}}{%
\pgfpathmoveto{\pgfqpoint{0.000000in}{0.000000in}}%
\pgfpathlineto{\pgfqpoint{-0.055556in}{0.000000in}}%
\pgfusepath{stroke,fill}%
}%
\begin{pgfscope}%
\pgfsys@transformshift{7.200000in}{4.165995in}%
\pgfsys@useobject{currentmarker}{}%
\end{pgfscope}%
\end{pgfscope}%
\begin{pgfscope}%
\pgftext[x=0.944444in,y=4.165995in,right,]{\sffamily\fontsize{12.000000}{14.400000}\selectfont 7}%
\end{pgfscope}%
\begin{pgfscope}%
\pgfsetbuttcap%
\pgfsetroundjoin%
\definecolor{currentfill}{rgb}{0.000000,0.000000,0.000000}%
\pgfsetfillcolor{currentfill}%
\pgfsetlinewidth{0.501875pt}%
\definecolor{currentstroke}{rgb}{0.000000,0.000000,0.000000}%
\pgfsetstrokecolor{currentstroke}%
\pgfsetdash{}{0pt}%
\pgfsys@defobject{currentmarker}{\pgfqpoint{0.000000in}{0.000000in}}{\pgfqpoint{0.055556in}{0.000000in}}{%
\pgfpathmoveto{\pgfqpoint{0.000000in}{0.000000in}}%
\pgfpathlineto{\pgfqpoint{0.055556in}{0.000000in}}%
\pgfusepath{stroke,fill}%
}%
\begin{pgfscope}%
\pgfsys@transformshift{1.000000in}{4.750264in}%
\pgfsys@useobject{currentmarker}{}%
\end{pgfscope}%
\end{pgfscope}%
\begin{pgfscope}%
\pgfsetbuttcap%
\pgfsetroundjoin%
\definecolor{currentfill}{rgb}{0.000000,0.000000,0.000000}%
\pgfsetfillcolor{currentfill}%
\pgfsetlinewidth{0.501875pt}%
\definecolor{currentstroke}{rgb}{0.000000,0.000000,0.000000}%
\pgfsetstrokecolor{currentstroke}%
\pgfsetdash{}{0pt}%
\pgfsys@defobject{currentmarker}{\pgfqpoint{-0.055556in}{0.000000in}}{\pgfqpoint{0.000000in}{0.000000in}}{%
\pgfpathmoveto{\pgfqpoint{0.000000in}{0.000000in}}%
\pgfpathlineto{\pgfqpoint{-0.055556in}{0.000000in}}%
\pgfusepath{stroke,fill}%
}%
\begin{pgfscope}%
\pgfsys@transformshift{7.200000in}{4.750264in}%
\pgfsys@useobject{currentmarker}{}%
\end{pgfscope}%
\end{pgfscope}%
\begin{pgfscope}%
\pgftext[x=0.944444in,y=4.750264in,right,]{\sffamily\fontsize{12.000000}{14.400000}\selectfont 8}%
\end{pgfscope}%
\begin{pgfscope}%
\pgfsetbuttcap%
\pgfsetroundjoin%
\definecolor{currentfill}{rgb}{0.000000,0.000000,0.000000}%
\pgfsetfillcolor{currentfill}%
\pgfsetlinewidth{0.501875pt}%
\definecolor{currentstroke}{rgb}{0.000000,0.000000,0.000000}%
\pgfsetstrokecolor{currentstroke}%
\pgfsetdash{}{0pt}%
\pgfsys@defobject{currentmarker}{\pgfqpoint{0.000000in}{0.000000in}}{\pgfqpoint{0.055556in}{0.000000in}}{%
\pgfpathmoveto{\pgfqpoint{0.000000in}{0.000000in}}%
\pgfpathlineto{\pgfqpoint{0.055556in}{0.000000in}}%
\pgfusepath{stroke,fill}%
}%
\begin{pgfscope}%
\pgfsys@transformshift{1.000000in}{5.334533in}%
\pgfsys@useobject{currentmarker}{}%
\end{pgfscope}%
\end{pgfscope}%
\begin{pgfscope}%
\pgfsetbuttcap%
\pgfsetroundjoin%
\definecolor{currentfill}{rgb}{0.000000,0.000000,0.000000}%
\pgfsetfillcolor{currentfill}%
\pgfsetlinewidth{0.501875pt}%
\definecolor{currentstroke}{rgb}{0.000000,0.000000,0.000000}%
\pgfsetstrokecolor{currentstroke}%
\pgfsetdash{}{0pt}%
\pgfsys@defobject{currentmarker}{\pgfqpoint{-0.055556in}{0.000000in}}{\pgfqpoint{0.000000in}{0.000000in}}{%
\pgfpathmoveto{\pgfqpoint{0.000000in}{0.000000in}}%
\pgfpathlineto{\pgfqpoint{-0.055556in}{0.000000in}}%
\pgfusepath{stroke,fill}%
}%
\begin{pgfscope}%
\pgfsys@transformshift{7.200000in}{5.334533in}%
\pgfsys@useobject{currentmarker}{}%
\end{pgfscope}%
\end{pgfscope}%
\begin{pgfscope}%
\pgftext[x=0.944444in,y=5.334533in,right,]{\sffamily\fontsize{12.000000}{14.400000}\selectfont 9}%
\end{pgfscope}%
\begin{pgfscope}%
\pgftext[x=0.768962in,y=3.000000in,,bottom]{\sffamily\fontsize{12.000000}{14.400000}\selectfont y}%
\end{pgfscope}%
\end{pgfpicture}%
\makeatother%
\endgroup%
}
	\caption{Primer odabira modela pri linearnoj regresiji polinomom}
	\label{fig:odabir}
\end{figure}

\subsection{Podaci}

Mašinsko učenje bavi se generalizacijom nad nepoznatim objektima na osnovu već viđenih objekata. Pod pojmom objekta misli se na pojedinačni podatak koji sistem vidi. Koriste se još i izrazi primerak i instanca. Vrednosti podataka pripadaju nekom unapred zadatom skupu. Podaci mogu biti različitog tipa: numerički ili kategorički. Skupovi koji određuju vrednosti kojima se instance određuju nisu unapred zadati i neophodno ih je odrediti na način pogodan za rešavanje konkretnog problema. Na primer, ukoliko je neophodno razvrstati slike životinja i biljaka na te dve kategorije, informacija o količini zelene boje na slici može biti prilično korisna, dok pri razvrstavanju vrste biljaka u zavisnosti od lista ovaj podatak skoro nije upotrebljiv (ali podatak o nijansi zelene boje može biti). Dakle, dobar izbor atributa imaće veliki uticaj na kasnije korake učenja. Podaci se sistemu daju kao vektori atributa. \par

Podaci se neretko pre slanja sistemu obrađuju na neki način; ovaj postupak zove se pretprocesiranje. Postoje mnogi razlozi za pretprocesiranje a glavni cilj jeste da dobijemo objekte nad kojima učenje može da se desi. Međutim, i to zavisi od problema. Nekada će nepotpuni objekti, podaci koji ne sadrže sve informacije neophodne za učenje, biti izbačeni iz skupa podataka koji se razmatra, a u nekom drugom slučaju, i oni će biti korišćeni. Jedan primer pretprocesiranja jeste pretvaranje slike koja je u boji u crno beli zapis.

\subsection{Evaluacija modela}

Nakon obučavanja (treniranja), neophodno je izvršiti evaluaciju dobijenog modela. Na koji god način se ovo izvršava, podaci korišćeni za obučavanje ne smeju se koristiti za evaluaciju modela. Često se pribegava podeli podataka na skupove za obučavanje i za testiranje. Skup za obučavanje obično iznosi dve trećine skupa ukupnih podataka. No, kako različite podele skupa mogu izazvati dobijanje različitih modela, slučajno deljenje nije najbolji izbor.  Često korišćena tehnika jeste unakrsna validacija. Ovaj pristup podrazumeva podelu skupa podataka $D$ na $K$ podskupova približno jednake veličine, $S_i$ za $i=1,...,K$. Tada se za svako $i$ model trenira na skupu $D \setminus S_i$ a evaluacija se vrši pomoću podataka iz $S_i$. Posle izvedenog postupka za sve $i$, kao konačna ocena uzima se prosečna ocena svakog od $K$ treniranja i evaluacija modela. Za vrednosti $K$ uobičajeno se uzimaju vrednosti 5 ili 10. Ovaj metod vodi pouzdanijoj oceni kvaliteta modela.

\section{Problemi pri mašinskom učenju}


Kao što je podrazumevano pri pomenu pojma generalizacije, nije dovoljno odrediti funkciju koja dobro određuje izlazne vrednosti na osnovu promenljivih nad kojima se uči već je poželjno i novim ulaznim podacima dodeliti tačnu izlaznu vrednost. Odavde se može videti da će loš sistem za mašinsko učenje izuzetno dobro naučiti da preslikava ulazne vrednosti iz skupa za učenje u odgovarajuće izlazne vrednosti ali u situaciji kada se iz tog skupa izađe sistem neće davati zadovoljavajuće rezultate. Ovaj problem ima svoje ime: preprilagođavanje. Postoji i problem potprilagođavanja, koji podrazumeva da se sistem nije dovoljno prilagodio podacima. I preprilagođavanje i potprilagođavanje predstavljaju veliki problem ukoliko do njih dođe. Primer preprilagođavanja može se videti na slici \ref{fig:odabir}. Polinomom stepena 10 model se savršeno prilagodio podacima za trening ali neće biti u stanju da izvrši generalizaciju za nove podatke. \par
Na još jedan od mogućih problema nailazi se u slučaju neprikladnih podataka. Moguće je da ulazni atributi ne daju dovoljno informacija o izlaznim. Takođe je moguće da podataka jednostavno nema dovoljno. U ovom slučaju, sistem ne dobija dovoljno bogat skup informacija kako bi uspešno izvršio generalizaciju. S druge strane, moguće je da postoji prevelika količina podataka. Tada se pribegava pažljivom odabiru podataka koji se koriste za učenje ali ovo u opštem slučaju  treba izbegavati jer su podaci izuzetno vredan element procesa mašinskog učenja. Još jedan problem vezan za podatke može biti njihova nepotpunost. Na primer, moguće je da u nekim instancama postoje nedostajuće vrednosti atributa. \par
Kako je najčešče potrebno pretprocesirati podatke u sklopu procesa mašinskog učenja, moguće je da u ovom postupku dođe do greške. Primera radi, prilikom rada sa konvolutivnim neuronskim mrežama, o kojima će biti reči u jednom od narednih glava, nekada se slike  u boji pretvaraju u crno-bele. Ako se primeni transformacija koja onemogućuje razlikovanje objekata koji su različiti u početnoj slici a razlikovanje je neophodno za ispravno učenje sistema, tada proces treniranja neće teći kako je planirano. \par


% [ONO ZA RGB U GRAY U DQN BI BILO SOLIDNO OVDE UBACITI] \par
Problem može da nastane i ukoliko nije odabran pravi algoritam za učenje, ukoliko se loše pristupilo procesu optimizacije, prilikom lošeg procesa evaluacije i, naravno, prilikom loše implementacije algoritma. Sve ove prepreke moguće je prevazići ali je jasno da je neophodno biti izuzetno pažljiv prilikom celog procesa mašinskog učenja.