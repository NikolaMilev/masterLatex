\chapter{Uvod}



U maju 1997. godine, Gari Kasparov, tadašnji svetski šampion u šahu, izgubio je meč protiv računarskog sistem pod nazivom "Deep Blue". Skoro dvadeset godina kasnije, program pod nazivom "AlphaGo" pobedio je profesionalnog ljudskog igrača u igri go. Iako su obe igre strateške i igraju se na tabli, između šaha i igre go postoji ogromna razlika. Pravila igre go dosta su jednostavna u odnosu na šah ali je prostor koji opisuje poteze igre go više od $10^{100}$ puta veći od prostora koji opisuje poteze šaha. Programi koji igraju šah često se zasnivaju na korišćenju stabala pretrage i ovaj pristup jednostavno nije primenljiv na igru go. \par

Na čemu je onda zasnovan "AlphaGo"? U pitanju je učenje potkrepljivanjem (eng. reinforcement learning). Ovo je vrsta mašinskog učenja koja počiva na sistemu kazne i nagrade. Podrazumeva se da se sistem sastoji od agenta i okruženja u kom agent dela (vrši akcije) i dobija povratne informacije o svom učinku. Poput dresiranja psa, nagradama se ohrabruje poželjno ponašanje dok se nepoželjno kažnjava. Cilj jeste ostvariti što veću dugoročnu nagradu. Međutim, agent mora sam kroz istraživanje da shvati kako da dostigne najveću nagradu tako što isprobava različite akcije. Takođe, preduzete akcije mogu da utiču i na nagradu koja se pojavljuje dugo nakon što je sama akcija preduzeta. Ovo zahteva da se uvede pojam buduće nagrade. Pojmovi istraživanja i buduće nagrade su ključni pri učenju potkrepljivanjem. \par
Pri učenju potkrepljivanjem, najčešće se pretpostavlja da je skup svih mogućih stanja okruženja diskretan. Ovo omogućuje primenu Markovljevih procesa odlučivanja i omogućuje formalan opis problema koji se rešava, kao i pristupa njegovog rešavanja. Formalno predstavljanje problema i rešenja dato je u poglavlju [POGLAVLJE O UCENJU POTKREPLJIVANJEM].

Učenje potkrepljivanjem jedna je od tri vrste mašinskog učenja, pored nadgledanog i nenadgledanog učenja. Pri nadgledanom učenju, sistem dobija skup ulaznih i izlaznih podataka s ciljem da izvrši generalizaciju nad tim podacima i uspešno generiše izlazne podatke od do sada nevidjenih ulaznih podataka. Pri učenju potkrepljivanjem, ne postoje unapred poznate akcije koje treba preduzeti već sistem na osnovu nagrade mora zaključi koji je optimalni sled akcija. Iako široko korišćeno, nadgledano učenje nije prikladno za učenje iz novih iskustava, kada ciljni rezultati nisu dostupni.  Kod nenadgledanog učenja, često je neophodno pronaći neku strukturu u podacima nad kojima se uči bez ikakvog pređašnjeg znanja o njima. Iako učenje potkrepljivanjem liči i na nadgledano i na nenadgledano učenje, agent ne traži strukturu niti postoji unapred određeno optimalno ponašanje već teži maksimizaciji nagrade koju dobija od okruženja. \par

Učenje potkrepljivanjem ima primene u raznim poljima kao što su industrija, istraživanje podataka, mašinsko učenje (kompanije Gugl (eng. Google) koristi učenje potkrepljivanjem radi automatskog dizajna neuralnih mreža), obrazovanje, medicina i finansije. Ovaj vid mašinskog učenja pokazao se kao dobar i za igranje video igara.  U radu objavljenom 2015. godine u časopisu "Nature", "DeepMind" predstavlja sistem koji uči da igra video igre sa konzole Atari 2600, neke čak i daleko bolje od ljudi\footnote{UBACI NEKU REFERENCU}. U avgustu 2017. godine, OpenAI predstavlja agenta koji isključivo kroz igranje igre i bez pređašnjeg znanja o igri stiče nivo umeća dovoljan da pobedi i neke od najboljih ljudskih takmičara u video igri "Dota 2"\footnote{https://blog.openai.com/dota-2/}. 

U naučnom radu koji je objavila kompanija "DeepMind" u časopisu "Nature" predložen je novi algoritam, DQN (deep Q - network), koji koristi spoj učenja uslovljavanjem i duboke neuronske mreže i uspesno savladava razne igre za Atari 2600 konzolu. Sve informacije dostupne agentu jesu pikseli sa ekrana, kao i trenutni rezultat u igri. Algoritam skladišti prethodna iskustva i umesto učenja neuronske mreže na osnovu samo poslednjih akcija i nagrada, prethodno iskustvo se periodično koristi radi treniranja mreže, nasumičnim uzorkovanjem, smanjujući korelaciju između ulaznih podataka. 
U sklopu ovog rada, ispitana je struktura algoritma DQN i data je implementacija čije su performanse ispitane na manjoj skali od one date u radu, zbog ograničenih resursa. Takodje je eksperimentisano sa elementima samog algoritma i opisano kako oni utiču na njegovo ponašanje.


[MOZDA NESTO O REZULTATIMA KADA IH BUDE]


U sklopu rada opisani su osnovni pojmovi mašinskog učenja (glava 2), zadržavajući se na neuronskim mrežama uopšte (glava 3) i na konvolutivnim neuronskim mrežama (glava 4). Glava 5 posvećena je učenju potkrepljivanjem dok je algoritam DQN u celosti opisan u glavi 6. U glavi 7 data je implementacija kao i njena evaluacija, dok su eksperimenti i njihovi rezultati opisani u glavi 8. 