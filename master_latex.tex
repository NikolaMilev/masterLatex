\documentclass{article}
\usepackage[utf8]{inputenc}
\usepackage[T1]{fontenc}
\begin{document}
\section{Uvod}
\section{Mašinsko učenje}

Čovek kroz istoriju teži da posao uradi na najjednostavniji i najlakši način. Ova težnja dovela je do velikih otkrića i preokreta u razvoju civilizacije. Točak je omogućio nastanak kopnenih vozila. Gutenbergova presa ubrzala je umnožavanje knjiga i time ubrzala proširivanje znanja. Otkriće električne struje i njeno eksploatisanje oslobodilo nas je zavisnosti od dnevnog svetla i dovelo do mnogih otkrića bez kojih danas ne možemo da zamislimo život. Jedno od tih otkrića jesu elektronski računari. Idejno, računari su mašine koje mogu da rade prilično jednostavne poslove kao što su osnovne aritmetičke operacije. Daljim razvojem računara, oni su mogli da rešavaju te jednostavne probleme brže. Otkrili smo način da budu manji, jeftiniji i efikasniji i time su postali široko dostupni ali se sama suština računara kao mašine nije promenila. Računari izuzetno brzo rešavaju probleme koje čovek ume formalno da postavi i da algoritam za njegovo rešavanje. Međutim, računari ne umeju da razmišljaju na način na koji čovek razmišlja. Već godinama, naučna fantastika puna je priča o veštačkoj inteligenciji, ljudskoj tvorevini koja parira čoveku po intelektu, mašini koja misli. Ono što danas u računarstvu zovemo veštačkom inteligencijom nije ni blizu tih priča ali se neprestano razvijaju tehnike kojima možemo da implementiramo sistem koji uči da rešava neki relativno uzak opseg problema. 
Jedna od grana veštačke inteligencije, mašinsko učenje, bavi se baš korišćenjem statističkih i numeričkih metoda u cilju učenja tj. poboljšavanja uspeha pri obavljanju određenog zadatka. 

Probleme koje rešavamo korišćenjem tehnika veštačke inteligencije uglavnom i čovek može da uradi, s tim što bi čoveku to bilo naporno i često bi rešavanje bilo izuzetno sporo.
Kada se u problem uvedu apstrakcija i intuicija, nastaje problem. Na primer, prepoznavanje pisanog teksta ili lica na slikama nije nešto što se tek tako može formalizovati. 
Danas postoje tehnike kojima 



\subsection{Vrste mašinskog učenja}

\section{Neuronske mreže}
\section{Konvolutivne neuronske mreže}
\section{Učenje potkrepljivanjem}
\section{DQN}
\section{Detalji implementacije}
\end{document}

